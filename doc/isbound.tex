\subsection{isbound}
\label{labisbound}
\noindent Name: \textbf{isbound}\\
\phantom{aaa}indicates whether a variable is bound or not.\\[0.2cm]
\noindent Usage: 
\begin{center}
\textbf{isbound}(\emph{ident}) : \textsf{boolean}\\
\end{center}
Parameters: 
\begin{itemize}
\item \emph{ident} is a name.
\end{itemize}
\noindent Description: \begin{itemize}

\item \textbf{isbound}(\emph{ident}) returns a boolean value indicating whether the name \emph{ident}
   is used or not to represent a variable. It returns true when \emph{ident} is the 
   name used to represent the global variable or if the name is currently used
   to refer to a (possibly local) variable.

\item When a variable is defined in a block and has not been defined outside, 
   \textbf{isbound} returns true when called inside the block, and false outside.
   Note that \textbf{isbound} returns true as soon as a variable has been declared with 
   \textbf{var}, even if no value is actually stored in it.

\item If \emph{ident1} is bound to a variable and if \emph{ident2} refers to the global 
   variable, the command \textbf{rename}(\emph{ident2}, \emph{ident1}) hides the value of \emph{ident1}
   which becomes the global variable. However, if the global variable is again
   renamed, \emph{ident1} gets its value back. In this case, \textbf{isbound}(\emph{ident1}) returns
   true. If \emph{ident1} was not bound before, \textbf{isbound}(\emph{ident1}) returns false after
   that \emph{ident1} has been renamed.
\end{itemize}
\noindent Example 1: 
\begin{center}\begin{minipage}{15cm}\begin{Verbatim}[frame=single,commandchars=\\\|\~]
> isbound(x);
false
> isbound(f);
false
> isbound(g);
false
> f=sin(x);
> isbound(x);
true
> isbound(f);
true
> isbound(g);
false
\end{Verbatim}
\end{minipage}\end{center}
\noindent Example 2: 
\begin{center}\begin{minipage}{15cm}\begin{Verbatim}[frame=single,commandchars=\\\|\~]
> isbound(a);
false
> { var a; isbound(a); };
true
> isbound(a);
false
\end{Verbatim}
\end{minipage}\end{center}
\noindent Example 3: 
\begin{center}\begin{minipage}{15cm}\begin{Verbatim}[frame=single,commandchars=\\\|\~]
> f=sin(x);
> isbound(x);
true
> rename(x,y);
> isbound(x);
false
\end{Verbatim}
\end{minipage}\end{center}
\noindent Example 4: 
\begin{center}\begin{minipage}{15cm}\begin{Verbatim}[frame=single,commandchars=\\\|\~]
> x=1;
> f=sin(y);
> rename(y,x);
> f;
sin(x)
> x;
x
> isbound(x);
true
> rename(x,y);
> isbound(x);
true
> x;
1
\end{Verbatim}
\end{minipage}\end{center}
See also: \textbf{rename} (\ref{labrename})
