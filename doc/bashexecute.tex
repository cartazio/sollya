\subsection{ bashexecute }
\noindent Name: \textbf{bashexecute}\\
executes a shell command.\\

\noindent Usage: 
\begin{center}
\textbf{bashexecute}(\emph{command}) : \textsf{string} $\rightarrow$ \textsf{void}\\
\end{center}
Parameters: 
\emph{command} is a command to be interpreted by the shell.\\

\noindent Description: \begin{itemize}

\item \textbf{bashexecute}(\emph{command}) lets the shell interpret \emph{command}. It is useful to execute
   some external code within Sollya.

\item \textbf{bashexecute} does not return anything. It just executes its argument. However, if
   \emph{command} produces an output in a file, this result can be imported in Sollya
   with help of commands like \textbf{execute}, \textbf{readfile} and \textbf{parse}.
\end{itemize}
\noindent Example 1: 
\begin{center}\begin{minipage}{14.8cm}\begin{Verbatim}[frame=single]
   > bashexecute("ls /");
   bin
   boot
   cdrom
   dev
   etc
   home
   lib
   lib64
   lost+found
   media
   mnt
   opt
   proc
   root
   sbin
   srv
   sys
   tmp
   usr
   var
\end{Verbatim}
\end{minipage}\end{center}
See also: \textbf{execute}, \textbf{readfile}, \textbf{parse}
