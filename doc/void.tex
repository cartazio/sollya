\subsection{void}
\label{labvoid}
\noindent Name: \textbf{void}\\
the functional result of a side-effect or empty argument resp. the corresponding type\\
\noindent Usage: 
\begin{center}
\textbf{void} : \textsf{void} $|$ \textsf{type type}\\
\end{center}
\noindent Description: \begin{itemize}

\item The variable \\textbf{void} represents the functional result of a\n   side-effect or an empty argument.  It is used only in combination with\n   the applications of procedures or identifiers bound through\n   \\textbf{externalproc} to external procedures.\n    \n   The \\textbf{void} result produced by a procedure or an external procedure\n   is not printed at the prompt. However, it is possible to print it out\n   in a print statement or in complex data types such as lists.\n    \n   The \\textbf{void} argument is implicit when giving no argument to a\n   procedure or an external procedure when applied. It can nevertheless be given\n   explicitly.  For example, suppose that foo is a procedure or an\n   external procedure with a void argument. Then foo() and foo(void) are\n   correct calls to foo. Here, a distinction must be made for procedures having an\n   arbitrary number of arguments. In this case, an implicit \\textbf{void}\n   as the only parameter to a call of such a procedure gets converted into \n   an empty list of arguments, an explicit \\textbf{void} gets passed as-is in the\n   formal list of parameters the procedure receives.\n
\item \\textbf{void} is used also as a type identifier for\n   \\textbf{externalproc}. Typically, an external procedure taking \\textbf{void} as an\n   argument or returning \\textbf{void} is bound with a signature \\textbf{void} $->$\n   some type or some type $->$ \\textbf{void}. See \\textbf{externalproc} for more\n   details.\n\end{itemize}
\noindent Example 1: 
\begin{center}\begin{minipage}{15cm}\begin{Verbatim}[frame=single]
\end{Verbatim}
\end{minipage}\end{center}
\noindent Example 2: 
\begin{center}\begin{minipage}{15cm}\begin{Verbatim}[frame=single]
\end{Verbatim}
\end{minipage}\end{center}
\noindent Example 3: 
\begin{center}\begin{minipage}{15cm}\begin{Verbatim}[frame=single]
\end{Verbatim}
\end{minipage}\end{center}
\noindent Example 4: 
\begin{center}\begin{minipage}{15cm}\begin{Verbatim}[frame=single]
\end{Verbatim}
\end{minipage}\end{center}
See also: \textbf{error} (\ref{laberror}), \textbf{proc} (\ref{labproc}), \textbf{externalproc} (\ref{labexternalproc})
