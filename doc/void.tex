\subsection{void}
\label{labvoid}
\noindent Name: \textbf{void}\\
the functional result of a side-effect or empty argument resp. the corresponding type\\
\noindent Usage: 
\begin{center}
\textbf{void} : \textsf{void} $|$ \textsf{type type}\\
\end{center}
\noindent Description: \begin{itemize}

\item The variable \textbf{void} represents the functional result of a
   side-effect or an empty argument.  It is used only in combination with
   the applications of procedures or identifiers bound through
   \textbf{externalproc} to external procedures.
    
   The \textbf{void} result produced by a procedure or an external procedure
   is not printed at the prompt. However, it is possible to print it out
   in a print statement or in complex data types such as lists.
    
   The \textbf{void} argument is implicit when giving no argument to a
   procedure or an external procedure when applied. It can nevertheless be given
   explicitly.  For example, suppose that foo is a procedure or an
   external procedure with a void argument. Then foo() and foo(void) are
   correct calls to foo. Here, a distinction must be made for procedures having an
   arbitrary number of arguments. In this case, an implicit \textbf{void}
   as the only parameter to a call of such a procedure gets converted into 
   an empty list of arguments, an explicit \textbf{void} gets passed as-is in the
   formal list of parameters the procedure receives.

\item \textbf{void} is used also as a type identifier for
   \textbf{externalproc}. Typically, an external procedure taking \textbf{void} as an
   argument or returning \textbf{void} is bound with a signature \textbf{void} $->$
   some type or some type $->$ \textbf{void}. See \textbf{externalproc} for more
   details.
\end{itemize}
\noindent Example 1: 
\begin{center}\begin{minipage}{15cm}\begin{Verbatim}[frame=single]
> print(void);
void
> void;
\end{Verbatim}
\end{minipage}\end{center}
\noindent Example 2: 
\begin{center}\begin{minipage}{15cm}\begin{Verbatim}[frame=single]
> hey = proc() { print("Hello world."); };
> hey;
proc()
{
print("Hello world.");
return void;
}
> hey();
Hello world.
> hey(void);
Hello world.
> print(hey());
Hello world.
void
\end{Verbatim}
\end{minipage}\end{center}
\noindent Example 3: 
\begin{center}\begin{minipage}{15cm}\begin{Verbatim}[frame=single]
> bashexecute("gcc -fPIC -Wall -c externalprocvoidexample.c");
> bashexecute("gcc -fPIC -shared -o externalprocvoidexample externalprocvoidexam
ple.o");
> externalproc(foo, "./externalprocvoidexample", void -> void);
> foo;
foo(void) -> void
> foo();
Hello from the external world.
> foo(void);
Hello from the external world.
> print(foo());
Hello from the external world.
void
\end{Verbatim}
\end{minipage}\end{center}
\noindent Example 4: 
\begin{center}\begin{minipage}{15cm}\begin{Verbatim}[frame=single]
> procedure blub(L = ...) { print("Argument list:", L); };
> blub(1);
Argument list: [|1|]
> blub();
Argument list: [| |]
> blub(void); 
Argument list: [|void|]
\end{Verbatim}
\end{minipage}\end{center}
See also: \textbf{error} (\ref{laberror}), \textbf{proc} (\ref{labproc}), \textbf{externalproc} (\ref{labexternalproc})
