\subsection{findzeros}
\label{labfindzeros}
\noindent Name: \textbf{findzeros}\\
gives a list of intervals containing all zeros of a function on an interval.\\
\noindent Usage: 
\begin{center}
\textbf{findzeros}(\emph{f},\emph{I}) : (\textsf{function}, \textsf{range}) $\rightarrow$ \textsf{list}\\
\end{center}
Parameters: 
\begin{itemize}
\item \emph{f} is a function.
\item \emph{I} is an interval.
\end{itemize}
\noindent Description: \begin{itemize}

\item \\textbf{findzeros}(\\emph{f},\\emph{I}) returns a list of intervals $I_1, \\dots, I_n$ such that, for \n   every zero $z$ of $f$, there exists some $k$ such that $z \\in I_k$.\n
\item The list may contain intervals $I_k$ that do not contain any zero of \\emph{f}.\n   An interval \\emph{Ik} may contain many zeros of \\emph{f}.\n
\item This command is meant for cases when safety is critical. If you want to be sure\n   not to forget any zero, use \\textbf{findzeros}. However, if you just want to know \n   numerical values for the zeros of \\emph{f}, \\textbf{dirtyfindzeros} should be quite \n   satisfactory and a lot faster.\n
\item If $\\delta$ denotes the value of global variable \\textbf{diam}, the algorithm ensures\n   that for each $k$, $|I_k| \\le \\delta \\cdot |I|$.\n
\item The algorithm used is basically a bisection algorithm. It is the same algorithm\n   that the one used for \\textbf{infnorm}. See the help page of this command for more \n   details. In short, the behavior of the algorithm depends on global variables\n   \\textbf{prec}, \\textbf{diam}, \\textbf{taylorrecursions} and \\textbf{hopitalrecursions}.\n\end{itemize}
\noindent Example 1: 
\begin{center}\begin{minipage}{15cm}\begin{Verbatim}[frame=single]
\end{Verbatim}
\end{minipage}\end{center}
See also: \textbf{dirtyfindzeros} (\ref{labdirtyfindzeros}), \textbf{infnorm} (\ref{labinfnorm}), \textbf{prec} (\ref{labprec}), \textbf{diam} (\ref{labdiam}), \textbf{taylorrecursions} (\ref{labtaylorrecursions}), \textbf{hopitalrecursions} (\ref{labhopitalrecursions})
