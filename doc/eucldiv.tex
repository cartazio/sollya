\subsection{div}
\label{labeucldiv}
\noindent Name: \textbf{div}\\
\phantom{aaa}Computes the euclidian division of polynomials or numbers and returns the quotient\\[0.2cm]
\noindent Library name:\\
\verb|   sollya_obj_t sollya_lib_euclidian_div(sollya_obj_t, sollya_obj_t)|\\[0.2cm]
\noindent Usage: 
\begin{center}
\textbf{div}(\emph{a}, \emph{b}) : (\textsf{function}, \textsf{function}) $\rightarrow$ \textsf{function}\\
\end{center}
Parameters: 
\begin{itemize}
\item \emph{a} is a polynomial.
\item \emph{b} is a polynomial.
\end{itemize}
\noindent Description: \begin{itemize}

\item When both \emph{a} and \emph{b} are constants, \textbf{div}(\emph{a},\emph{b}) computes the
   largest integer less than or equal to \emph{a} divided by \emph{b}. In other
   words, it returns the quotient of the Euclidian division of \emph{a} by
   \emph{b}.

\item When at least one of \emph{a} or \emph{b} is a polynomial of degree at least
   $1$, \textbf{div}(\emph{a},\emph{b}) computes two polynomials \emph{q} and \emph{r} such
   that \emph{a} is equal to the product of \emph{q} and \emph{b} plus \emph{r}. The
   polynomial \emph{r} is of least degree possible. The \textbf{div} command
   returns \emph{q}. In order to recover \emph{r}, use the \textbf{mod} command.

\item When at least one of \emph{a} or \emph{b} is a function that is no polynomial,
   \textbf{div}(\emph{a},\emph{b}) returns $0$.
\end{itemize}
\noindent Example 1: 
\begin{center}\begin{minipage}{15cm}\begin{Verbatim}[frame=single]
> div(1001, 231);
4
> div(13, 17);
0
> div(-14, 15);
-1
> div(-213, -5);
42
> div(23/13, 11/17);
2
> div(exp(13),-sin(17));
460177
\end{Verbatim}
\end{minipage}\end{center}
\noindent Example 2: 
\begin{center}\begin{minipage}{15cm}\begin{Verbatim}[frame=single]
> div(24 + 68 * x + 74 * x^2 + 39 * x^3 + 10 * x^4 + x^5, 4 + 4 * x + x^2);
6 + x * (11 + x * (6 + x))
> div(24 + 68 * x + 74 * x^2 + 39 * x^3 + 10 * x^4 + x^5, 2 * x^3);
19.5 + x * (5 + x * 0.5)
> div(x^2, x^3);
0
\end{Verbatim}
\end{minipage}\end{center}
\noindent Example 3: 
\begin{center}\begin{minipage}{15cm}\begin{Verbatim}[frame=single]
> div(exp(x), x^2);
0
\end{Verbatim}
\end{minipage}\end{center}
See also: \textbf{gcd} (\ref{labgcd}), \textbf{mod} (\ref{labeuclmod}), \textbf{numberroots} (\ref{labnumberroots})
