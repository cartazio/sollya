\subsection{readfile}
\label{labreadfile}
\noindent Name: \textbf{readfile}\\
reads the content of a file into a string variable\\
\noindent Usage: 
\begin{center}
\textbf{readfile}(\emph{filename}) : \textsf{string} $\rightarrow$ \textsf{string}\\
\end{center}
Parameters: 
\begin{itemize}
\item \emph{filename} represents a character sequence indicating a file name
\end{itemize}
\noindent Description: \begin{itemize}

\item \\textbf{readfile} opens the file indicated by \\emph{filename}, reads it and puts its\n   contents in a character sequence of type \\textsf{string} that is returned.\n    \n   If the file indicated by \\emph{filename} cannot be opened for reading, a\n   warning is displayed and \\textbf{readfile} evaluates to an \\textbf{error} variable of\n   type \\textsf{error}.\n\end{itemize}
\noindent Example 1: 
\begin{center}\begin{minipage}{15cm}\begin{Verbatim}[frame=single]
\end{Verbatim}
\end{minipage}\end{center}
\noindent Example 2: 
\begin{center}\begin{minipage}{15cm}\begin{Verbatim}[frame=single]
\end{Verbatim}
\end{minipage}\end{center}
See also: \textbf{parse} (\ref{labparse}), \textbf{execute} (\ref{labexecute}), \textbf{write} (\ref{labwrite}), \textbf{print} (\ref{labprint})
