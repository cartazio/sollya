\subsection{readfile}
\label{labreadfile}
\noindent Name: \textbf{readfile}\\
\phantom{aaa}reads the content of a file into a string variable\\[0.2cm]
\noindent Usage: 
\begin{center}
\textbf{readfile}(\emph{filename}) : \textsf{string} $\rightarrow$ \textsf{string}\\
\end{center}
Parameters: 
\begin{itemize}
\item \emph{filename} represents a character sequence indicating a file name
\end{itemize}
\noindent Description: \begin{itemize}

\item \textbf{readfile} opens the file indicated by \emph{filename}, reads it and puts its
   contents in a character sequence of type \textsf{string} that is returned.
    
   If the file indicated by \emph{filename} cannot be opened for reading, a
   warning is displayed and \textbf{readfile} evaluates to an \textbf{error} variable of
   type \textsf{error}.
\end{itemize}
\noindent Example 1: 
\begin{center}\begin{minipage}{15cm}\begin{Verbatim}[frame=single]
> print("Hello world") > "myfile.txt";
> t = readfile("myfile.txt"); 
> t;
Hello world

\end{Verbatim}
\end{minipage}\end{center}
\noindent Example 2: 
\begin{center}\begin{minipage}{15cm}\begin{Verbatim}[frame=single]
> verbosity=1!;
> readfile("afile.txt");
Warning: the file "(null)" could not be opened for reading.
Warning: at least one of the given expressions or a subexpression is not correct
ly typed
or its evaluation has failed because of some error on a side-effect.
error
\end{Verbatim}
\end{minipage}\end{center}
See also: \textbf{parse} (\ref{labparse}), \textbf{execute} (\ref{labexecute}), \textbf{write} (\ref{labwrite}), \textbf{print} (\ref{labprint}), \textbf{bashexecute} (\ref{labbashexecute}), \textbf{error} (\ref{laberror})
