\subsection{write}
\label{labwrite}
\noindent Name: \textbf{write}\\
prints an expression without separators\\
\noindent Usage: 
\begin{center}
\textbf{write}(\emph{expr1},...,\emph{exprn}) : (\textsf{any type},..., \textsf{any type}) $\rightarrow$ \textsf{void}\\
\textbf{write}(\emph{expr1},...,\emph{exprn}) $>$ \emph{filename} : (\textsf{any type},..., \textsf{any type}, \textsf{string}) $\rightarrow$ \textsf{void}\\
\textbf{write}(\emph{expr1},...,\emph{exprn}) $>>$ \emph{filename} : (\textsf{any type},...,\textsf{any type}, \textsf{string}) $\rightarrow$ \textsf{void}\\
\end{center}
Parameters: 
\begin{itemize}
\item \emph{expr} represents an expression
\item \emph{filename} represents a character sequence indicating a file name
\end{itemize}
\noindent Description: \begin{itemize}

\item \\textbf{write}(\\emph{expr1},...,\\emph{exprn}) prints the expressions \\emph{expr1} through\n   \\emph{exprn}. The character sequences corresponding to the expressions are\n   concatenated without any separator. No newline is displayed at the\n   end.  In contrast to \\textbf{print}, \\textbf{write} expects the user to give all\n   separators and newlines explicitly.\n    \n   If a second argument \\emph{filename} is given after a single "$>$", the\n   displaying is not output on the standard output of \\sollya but if in\n   the file \\emph{filename} that get newly created or overwritten. If a double\n    "$>>$" is given, the output will be appended to the file \\emph{filename}.\n    \n   The global variables \\textbf{display}, \\textbf{midpointmode} and \\textbf{fullparentheses} have\n   some influence on the formatting of the output (see \\textbf{display},\n   \\textbf{midpointmode} and \\textbf{fullparentheses}).\n    \n   Remark that if one of the expressions \\emph{expri} given in argument is of\n   type \\textsf{string}, the character sequence \\emph{expri} evaluates to is\n   displayed. However, if \\emph{expri} is of type \\textsf{list} and this list\n   contains a variable of type \\textsf{string}, the expression for the list\n   is displayed, i.e.  all character sequences get displayed surrounded\n   by quotes ("). Nevertheless, escape sequences used upon defining\n   character sequences are interpreted immediately.\n\end{itemize}
\noindent Example 1: 
\begin{center}\begin{minipage}{15cm}\begin{Verbatim}[frame=single]
\end{Verbatim}
\end{minipage}\end{center}
\noindent Example 2: 
\begin{center}\begin{minipage}{15cm}\begin{Verbatim}[frame=single]
\end{Verbatim}
\end{minipage}\end{center}
\noindent Example 3: 
\begin{center}\begin{minipage}{15cm}\begin{Verbatim}[frame=single]
\end{Verbatim}
\end{minipage}\end{center}
\noindent Example 4: 
\begin{center}\begin{minipage}{15cm}\begin{Verbatim}[frame=single]
\end{Verbatim}
\end{minipage}\end{center}
See also: \textbf{print} (\ref{labprint}), \textbf{printexpansion} (\ref{labprintexpansion}), \textbf{printhexa} (\ref{labprinthexa}), \textbf{printfloat} (\ref{labprintfloat}), \textbf{printxml} (\ref{labprintxml}), \textbf{readfile} (\ref{labreadfile}), \textbf{autosimplify} (\ref{labautosimplify}), \textbf{display} (\ref{labdisplay}), \textbf{midpointmode} (\ref{labmidpointmode}), \textbf{fullparentheses} (\ref{labfullparentheses}), \textbf{evaluate} (\ref{labevaluate})
