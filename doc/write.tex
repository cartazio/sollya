\subsection{write}
\label{labwrite}
\noindent Name: \textbf{write}\\
prints an expression without separators\\
\noindent Usage: 
\begin{center}
\textbf{write}(\emph{expr1},...,\emph{exprn}) : (\textsf{any type},..., \textsf{any type}) $\rightarrow$ \textsf{void}\\
\textbf{write}(\emph{expr1},...,\emph{exprn}) $>$ \emph{filename} : (\textsf{any type},..., \textsf{any type}, \textsf{string}) $\rightarrow$ \textsf{void}\\
\textbf{write}(\emph{expr1},...,\emph{exprn}) $>>$ \emph{filename} : (\textsf{any type},...,\textsf{any type}, \textsf{string}) $\rightarrow$ \textsf{void}\\
\end{center}
Parameters: 
\begin{itemize}
\item \emph{expr} represents an expression
\item \emph{filename} represents a character sequence indicating a file name
\end{itemize}
\noindent Description: \begin{itemize}

\item \textbf{write}(\emph{expr1},...,\emph{exprn}) prints the expressions \emph{expr1} through
   \emph{exprn}. The character sequences corresponding to the expressions are
   concatenated without any separator. No newline is displayed at the
   end.  In contrast to \textbf{print}, \textbf{write} expects the user to give all
   separators and newlines explicitly.
    
   If a second argument \emph{filename} is given after a single "$>$", the
   displaying is not output on the standard output of \sollya but if in
   the file \emph{filename} that get newly created or overwritten. If a double
    "$>>$" is given, the output will be appended to the file \emph{filename}.
    
   The global variables \textbf{display}, \textbf{midpointmode} and \textbf{fullparentheses} have
   some influence on the formatting of the output (see \textbf{display},
   \textbf{midpointmode} and \textbf{fullparentheses}).
    
   Remark that if one of the expressions \emph{expri} given in argument is of
   type \textsf{string}, the character sequence \emph{expri} evaluates to is
   displayed. However, if \emph{expri} is of type \textsf{list} and this list
   contains a variable of type \textsf{string}, the expression for the list
   is displayed, i.e.  all character sequences get displayed surrounded
   by quotes ("). Nevertheless, escape sequences used upon defining
   character sequences are interpreted immediately.
\end{itemize}
\noindent Example 1: 
\begin{center}\begin{minipage}{15cm}\begin{Verbatim}[frame=single]
> write(x + 2 + exp(sin(x))); 
> write("Hello\n");
x + 2 + exp(sin(x))Hello
> write("Hello","world\n");
Helloworld
> write("Hello","you", 4 + 3, "other persons.\n");
Helloyou7other persons.
\end{Verbatim}
\end{minipage}\end{center}
\noindent Example 2: 
\begin{center}\begin{minipage}{15cm}\begin{Verbatim}[frame=single]
> write("Hello","\n");
Hello
> write([|"Hello"|],"\n");
[|"Hello"|]
> s = "Hello";
> write(s,[|s|],"\n");
Hello[|"Hello"|]
> t = "Hello\tyou";
> write(t,[|t|],"\n");
Hello    you[|"Hello\tyou"|]
\end{Verbatim}
\end{minipage}\end{center}
\noindent Example 3: 
\begin{center}\begin{minipage}{15cm}\begin{Verbatim}[frame=single]
> write(x + 2 + exp(sin(x))) > "foo.sol";
> readfile("foo.sol");
x + 2 + exp(sin(x))
\end{Verbatim}
\end{minipage}\end{center}
\noindent Example 4: 
\begin{center}\begin{minipage}{15cm}\begin{Verbatim}[frame=single]
> write(x + 2 + exp(sin(x))) >> "foo.sol";
\end{Verbatim}
\end{minipage}\end{center}
See also: \textbf{print} (\ref{labprint}), \textbf{printexpansion} (\ref{labprintexpansion}), \textbf{printdouble} (\ref{labprintdouble}), \textbf{printsingle} (\ref{labprintsingle}), \textbf{printxml} (\ref{labprintxml}), \textbf{readfile} (\ref{labreadfile}), \textbf{autosimplify} (\ref{labautosimplify}), \textbf{display} (\ref{labdisplay}), \textbf{midpointmode} (\ref{labmidpointmode}), \textbf{fullparentheses} (\ref{labfullparentheses}), \textbf{evaluate} (\ref{labevaluate}), \textbf{roundingwarnings} (\ref{labroundingwarnings}), \textbf{autosimplify} (\ref{labautosimplify})
