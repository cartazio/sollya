\subsection{externalplot}
\label{labexternalplot}
\noindent Name: \textbf{externalplot}\\
plots the error of an external code with regard to a function\\
\noindent Usage: 
\begin{center}
\textbf{externalplot}(\emph{filename}, \emph{mode}, \emph{function}, \emph{range}, \emph{precision}) : (\textsf{string}, \textsf{absolute$|$relative}, \textsf{function}, \textsf{range}, \textsf{integer}) $\rightarrow$ \textsf{void}\\
\textbf{externalplot}(\emph{filename}, \emph{mode}, \emph{function}, \emph{range}, \emph{precision}, \emph{perturb}) : (\textsf{string}, \textsf{absolute$|$relative}, \textsf{function}, \textsf{range}, \textsf{integer}, \textsf{perturb}) $\rightarrow$ \textsf{void}\\
\textbf{externalplot}(\emph{filename}, \emph{mode}, \emph{function}, \emph{range}, \emph{precision}, \emph{plot mode}, \emph{result filename}) : (\textsf{string}, \textsf{absolute$|$relative}, \textsf{function}, \textsf{range}, \textsf{integer}, \textsf{file$|$postscript$|$postscriptfile}, \textsf{string}) $\rightarrow$ \textsf{void}\\
\textbf{externalplot}(\emph{filename}, \emph{mode}, \emph{function}, \emph{range}, \emph{precision}, \emph{perturb}, \emph{plot mode}, \emph{result filename}) : (\textsf{string}, \textsf{absolute$|$relative}, \textsf{function}, \textsf{range}, \textsf{integer}, \textsf{perturb}, \textsf{file$|$postscript$|$postscriptfile}, \textsf{string}) $\rightarrow$ \textsf{void}\\
\end{center}
\noindent Description: \begin{itemize}

\item The command \\textbf{externalplot} plots the error of an external function\n   evaluation code sequence implemented in the object file named\n   \\emph{filename} with regard to the function \\emph{function}.  If \\emph{mode}\n   evaluates to \\emph{absolute}, the difference of both functions is\n   considered as an error function; if \\emph{mode} evaluates to \\emph{relative},\n   the difference is divided by the function \\emph{function}. The resulting\n   error function is plotted on all floating-point numbers with\n   \\emph{precision} significant mantissa bits in the range \\emph{range}. \n    \n   If the sixth argument of the command \\textbf{externalplot} is given and evaluates to\n   \\textbf{perturb}, each of the floating-point numbers the function is evaluated at gets perturbed by a\n   random value that is uniformly distributed in $\\pm1$ ulp\n   around the original \\emph{precision} bit floating-point variable.\n    \n   If a sixth and seventh argument, respectively a seventh and eighth\n   argument in the presence of \\textbf{perturb} as a sixth argument, are given\n   that evaluate to a variable of type \\textsf{file$|$postscript$|$postscriptfile} respectively to a\n   character sequence of type \\textsf{string}, \\textbf{externalplot} will plot\n   (additionally) to a file in the same way as the command \\textbf{plot}\n   does. See \\textbf{plot} for details.\n    \n   The external function evaluation code given in the object file name\n   \\emph{filename} is supposed to define a function name \\texttt{f} as\n   follows (here in C syntax): \\texttt{void f(mpfr\\_t rop, mpfr\\_ op)}. \n   This function is supposed to evaluate \\texttt{op} with an accuracy corresponding\n   to the precision of \\texttt{rop} and assign this value to\n   \\texttt{rop}.\n\end{itemize}
\noindent Example 1: 
\begin{center}\begin{minipage}{15cm}\begin{Verbatim}[frame=single]
\end{Verbatim}
\end{minipage}\end{center}
See also: \textbf{plot} (\ref{labplot}), \textbf{asciiplot} (\ref{labasciiplot}), \textbf{perturb} (\ref{labperturb}), \textbf{absolute} (\ref{lababsolute}), \textbf{relative} (\ref{labrelative}), \textbf{file} (\ref{labfile}), \textbf{postscript} (\ref{labpostscript}), \textbf{postscriptfile} (\ref{labpostscriptfile}), \textbf{bashexecute} (\ref{labbashexecute}), \textbf{externalproc} (\ref{labexternalproc}), \textbf{library} (\ref{lablibrary})
