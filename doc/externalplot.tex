\subsection{ externalplot }
\noindent Name: \textbf{externalplot}\\
plots the error of an external code with regard to a function\\

\noindent Usage: 
\begin{center}
\textbf{externalplot}(\emph{filename}, \emph{mode}, \emph{function}, \emph{range}, \emph{precision}) : (\textsf{string}, \textsf{absolute|relative}, \textsf{function}, \textsf{range}, \textsf{integer}) $\rightarrow$ \textsf{void}\\
\textbf{externalplot}(\emph{filename}, \emph{mode}, \emph{function}, \emph{range}, \emph{precision}, \emph{perturb}) : (\textsf{string}, \textsf{absolute|relative}, \textsf{function}, \textsf{range}, \textsf{integer}, \textsf{perturb}) $\rightarrow$ \textsf{void}\\
\textbf{externalplot}(\emph{filename}, \emph{mode}, \emph{function}, \emph{range}, \emph{precision}, \emph{plot mode}, \emph{result filename}) : (\textsf{string}, \textsf{absolute|relative}, \textsf{function}, \textsf{range}, \textsf{integer}, \textsf{file|postscript|postscriptfile}, \textsf{string}) $\rightarrow$ \textsf{void}\\
\textbf{externalplot}(\emph{filename}, \emph{mode}, \emph{function}, \emph{range}, \emph{precision}, \emph{perturb}, \emph{plot mode}, \emph{result filename}) : (\textsf{string}, \textsf{absolute|relative}, \textsf{function}, \textsf{range}, \textsf{integer}, \textsf{perturb}, \textsf{file|postscript|postscriptfile}, \textsf{string}) $\rightarrow$ \textsf{void}\\
\end{center}
\noindent Description: \begin{itemize}

\item The command \textbf{externalplot} plots the error of an external function
   evaluation code sequence implemented in the object file named
   \emph{filename} with regard to the function \emph{function}.  If \emph{mode}
   evaluates to \emph{absolute}, the difference of both functions is
   considered as an error function; if \emph{mode} evaluates to \emph{relative},
   the difference is quotiented by the function \emph{function}. The resulting
   error function is plotted on all floating-point numbers with
   \emph{precision} significant mantissa bits in the range \emph{range}. 
   If the sixth argument of the command \textbf{externalplot} is given an evaluates to
   \textbf{perturb}, each of these floating-point numbers is perturbed by a
   random value that is uniformly distributed in $\pm1$ ulp
   around the original \emph{precision} bit floating-point variable.
   If a sixth and seventh argument, respectively a seventh and eighth
   argument in the presence of \textbf{perturb} as a sixth argument, are given
   that evaluate to a variable of type \textsf{file|postscript|postscriptfile} respectively to a
   character sequence of type \textsf{string}, \textbf{externalplot} will plot
   (additionally) to a file in the same way as the command \textbf{plot}
   does. See \textbf{plot} for details.
   The external function evaluation code given in the object file name
   \emph{filename} is supposed to define a function name \texttt{f} as
   follows (here in C syntax): \texttt{void f(mpfr\_t rop, mpfr\_ op)}. 
   This function is supposed to evaluate \texttt{op} with an accuracy corresponding
   to the precision of \texttt{rop} and assign this value to
   \texttt{rop}.
\end{itemize}
\noindent Example 1: 
\begin{center}\begin{minipage}{14.8cm}\begin{Verbatim}[frame=single]
   > bashexecute("gcc -fPIC -c externalplotexample.c");
   > bashexecute("gcc -shared -o externalplotexample externalplotexample.o -lgmp -lmpfr");
   > externalplot("./externalplotexample",relative,exp(x),[-1/2;1/2],12,perturb);
\end{Verbatim}
\end{minipage}\end{center}
See also: \textbf{plot}, \textbf{asciiplot}, \textbf{perturb}, \textbf{absolute}, \textbf{relative}, \textbf{file}, \textbf{postscript}, \textbf{postscriptfile}, \textbf{bashexecute}, \textbf{externalproc}, \textbf{library}
