\subsection{floating}
\label{labfloating}
\noindent Name: \textbf{floating}\\
indicates that floating-point formats should be used for \textbf{fpminimax}\\

\noindent Usage: 
\begin{center}
\textbf{floating} : \textsf{fixed$|$floating}\\
\end{center}
\noindent Description: \begin{itemize}

\item The use of \textbf{floating} in the command \textbf{fpminimax} indicates that the list of
   formats given as argument is to be considered as a list of floating-point
   formats.
   See \textbf{fpminimax} for details.
\end{itemize}
\noindent Example 1: 
\begin{center}\begin{minipage}{15cm}\begin{Verbatim}[frame=single]
> fpminimax(cos(x),6,[|D...|],[-1;1],floating);
0.99999974816012948686250183527590706944465637207031 + x * (5.521004406122249513
1782035802443168321913900126185e-14 + x * (-0.4999928698019768802396356477402150
630950927734375 + x * (-3.95371609372064761555136192612768146546591008227978e-13
 + x * (4.16335155285858099505347240665287245064973831176758e-2 + x * (5.2492670
395835122748014980938834327670386437070249e-13 + x * (-1.33822408807599468535953
768366653093835338950157166e-3))))))
\end{Verbatim}
\end{minipage}\end{center}
See also: \textbf{fpminimax} (\ref{labfpminimax}), \textbf{fixed} (\ref{labfixed})
