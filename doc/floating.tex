\subsection{floating}
\label{labfloating}
\noindent Name: \textbf{floating}\\
\phantom{aaa}indicates that floating-point formats should be used for \textbf{fpminimax}\\[0.2cm]
\noindent Library names:\\
\verb|   sollya_obj_t sollya_lib_floating()|\\
\verb|   int sollya_lib_is_floating(sollya_obj_t)|\\[0.2cm]
\noindent Usage: 
\begin{center}
\textbf{floating} : \textsf{fixed$|$floating}\\
\end{center}
\noindent Description: \begin{itemize}

\item The use of \textbf{floating} in the command \textbf{fpminimax} indicates that the list of
   formats given as argument is to be considered to be a list of floating-point
   formats.
   See \textbf{fpminimax} for details.
\end{itemize}
\noindent Example 1: 
\begin{center}\begin{minipage}{15cm}\begin{Verbatim}[frame=single]
> fpminimax(cos(x),6,[|D...|],[-1;1],floating);
0.99999974816012215939053930924274027347564697265625 + x * (-2.79593179695850233
4440230695107655659202089892465e-15 + x * (-0.4999928698020140171998093592264922
3357439041137695 + x * (4.0484539189054105169841244454207387920433372507922e-14 
+ x * (4.1633515528919168291466235132247675210237503051758e-2 + x * (-4.01585881
8743733758578949218474363725507386355118e-14 + x * (-1.3382240885483781024645200
119493892998434603214264e-3))))))
\end{Verbatim}
\end{minipage}\end{center}
See also: \textbf{fpminimax} (\ref{labfpminimax}), \textbf{fixed} (\ref{labfixed})
