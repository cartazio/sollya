\subsection{doubleextended}
\label{labdoubleextended}
\noindent Names: \textbf{doubleextended}, \textbf{DE}\\
computes the nearest number with 64 bits of mantissa.\\
\noindent Description: \begin{itemize}

\item \\textbf{doubleextended} is a function that computes the nearest floating-point number with\n   64 bits of mantissa to a given number. Since it is a function, it can be\n   composed with other \\sollya functions such as \\textbf{exp}, \\textbf{sin}, etc.\n
\item It does not handle subnormal numbers. The range of possible exponents is the \n   range used for all numbers represented in \\sollya (e.g. basically the range \n   used in the library MPFR).\n
\item Since it is a function and not a command, its behavior is a bit different from \n   the behavior of \\textbf{round}(x,64,RN) even if the result is exactly the same.\n   \\textbf{round}(x,64,RN) is immediately evaluated whereas \\textbf{doubleextended}(x) can be composed \n   with other functions (and thus be plotted and so on).\n
\item Be aware that \\textbf{doubleextended} cannot be used as a constant to represent a format in the\n   commands \\textbf{roundcoefficients} and \\textbf{implementpoly} (contrary to \\textbf{D}, \\textbf{DD},and \\textbf{TD}). However, it\n   can be used in \\textbf{round}.\n\end{itemize}
\noindent Example 1: 
\begin{center}\begin{minipage}{15cm}\begin{Verbatim}[frame=single]
\end{Verbatim}
\end{minipage}\end{center}
\noindent Example 2: 
\begin{center}\begin{minipage}{15cm}\begin{Verbatim}[frame=single]
\end{Verbatim}
\end{minipage}\end{center}
\noindent Example 3: 
\begin{center}\begin{minipage}{15cm}\begin{Verbatim}[frame=single]
\end{Verbatim}
\end{minipage}\end{center}
See also: \textbf{roundcoefficients} (\ref{labroundcoefficients}), \textbf{single} (\ref{labsingle}), \textbf{double} (\ref{labdouble}), \textbf{doubledouble} (\ref{labdoubledouble}), \textbf{tripledouble} (\ref{labtripledouble}), \textbf{round} (\ref{labround})
