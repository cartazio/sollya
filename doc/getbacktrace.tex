\subsection{getbacktrace}
\label{labgetbacktrace}
\noindent Name: \textbf{getbacktrace}\\
\phantom{aaa}returns the list of \sollya procedures currently run\\[0.2cm]
\noindent Library name:\\
\verb|   sollya_obj_t sollya_lib_getbacktrace();|\\[0.2cm]
\noindent Usage: 
\begin{center}
\textbf{getbacktrace}() : \textsf{void} $\rightarrow$ \textsf{list}\\
\end{center}
\noindent Description: \begin{itemize}

\item The \textbf{getbacktrace} command allows the stack of \sollya procedures that are
   currently run to be inspected. When called, \textbf{getbacktrace} returns an
   ordered list of structures, each of which contains an element
   passed\_args and an element called\_proc. The element called\_proc
   contains the \sollya object representing the procedure being run. The
   element passed\_args contains an ordered list of all effective
   arguments passed to the procedure when it was called. The procedure called
   last (i.e. on top of the stack) comes first in the list returned
   by \textbf{getbacktrace}. When any of the procedure called takes no arguments, the
   passed\_args element of the corresponding structure evaluates to an empty
   list.

\item When called from outside any procedure (at toplevel), \textbf{getbacktrace} returns
   an empty list.

\item When called for a stack containing a call to a variadic procedure that was
   called with an infinite number of effective arguments, the corresponding
   passed\_args element evaluates to an end-elliptic list.
\end{itemize}
\noindent Example 1: 
\begin{center}\begin{minipage}{15cm}\begin{Verbatim}[frame=single]
> procedure testA() {
        "Current backtrace:";
        getbacktrace();
  };
> procedure testB(X) {
        "X = ", X;
        testA();
  };
> procedure testC(X, Y) {
        "X = ", X, ", Y = ", Y;
        testB(Y);
  };
> testC(17, 42);
X = 17, Y = 42
X = 42
Current backtrace:
[|{ .passed_args = [| |], .called_proc = proc()
{
"Current backtrace:";
getbacktrace();
return void;
} }, { .passed_args = [|42|], .called_proc = proc(X)
{
"X = ", X;
testA();
return void;
} }, { .passed_args = [|17, 42|], .called_proc = proc(X, Y)
{
"X = ", X, ", Y = ", Y;
testB(Y);
return void;
} }|]
\end{Verbatim}
\end{minipage}\end{center}
\noindent Example 2: 
\begin{center}\begin{minipage}{15cm}\begin{Verbatim}[frame=single]
> getbacktrace();
[| |]
\end{Verbatim}
\end{minipage}\end{center}
\noindent Example 3: 
\begin{center}\begin{minipage}{15cm}\begin{Verbatim}[frame=single]
> procedure printnumargs(X) {
        "number of arguments: ", X;
        getbacktrace();
  };
> procedure numargs(l = ...) {
        "l[17] = ", l[17];
        printnumargs(length(l));
  };
> procedure test() {
        numargs @ [|25 ...|];
  };
> test();
l[17] = 42
number of arguments: infty
[|{ .passed_args = [|infty|], .called_proc = proc(X)
{
"number of arguments: ", X;
getbacktrace();
return void;
} }, { .passed_args = [|25|], .called_proc = proc(l = ...)
{
"l[17] = ", l[17];
printnumargs(length(l));
return void;
} }, { .passed_args = [| |], .called_proc = proc()
{
(numargs) @ ([|25...|]);
return void;
} }|]
\end{Verbatim}
\end{minipage}\end{center}
See also: \textbf{proc} (\ref{labproc}), \textbf{procedure} (\ref{labprocedure}), \textbf{bind} (\ref{labbind}), \textbf{@} (\ref{labconcat})
