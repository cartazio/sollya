\subsection{tripledouble}
\label{labtripledouble}
\noindent Names: \textbf{tripledouble}, \textbf{TD}\\
represents a number as the sum of three IEEE doubles.\\
\noindent Description: \begin{itemize}

\item \textbf{tripledouble} is both a function and a constant.

\item As a function, it rounds its argument to the nearest number that can be written
   as the sum of three double precision numbers.

\item The algorithm used to compute \textbf{tripledouble}(x) is the following: let xh = \textbf{double}(x)
   and let xl = \textbf{doubledouble}(x-xh). Return the number xh+xl. Note that if the
   current precision is not sufficient to represent exactly xh+xl, a rounding will
   occur and the result of \textbf{tripledouble}(x) will be useless.

\item As a constant, it symbolizes the triple-double precision format. It is used in 
   contexts when a precision format is necessary, e.g. in the commands 
   \textbf{roundcoefficients} and \textbf{implementpoly}.
   See the corresponding help pages for examples.
\end{itemize}
\noindent Example 1: 
\begin{center}\begin{minipage}{15cm}\begin{Verbatim}[frame=single]
> verbosity=1!;
> a = 1+ 2^(-55)+2^(-115);
> TD(a);
1.00000000000000002775557561562891353466491600711096
> prec=110!;
> TD(a);
Warning: double rounding occurred on invoking the triple-double rounding operato
r.
Try to increase the working precision.
1.000000000000000027755575615628913
\end{Verbatim}
\end{minipage}\end{center}
See also: \textbf{double} (\ref{labdouble}), \textbf{doubleextended} (\ref{labdoubleextended}), \textbf{doubledouble} (\ref{labdoubledouble}), \textbf{roundcoefficients} (\ref{labroundcoefficients}), \textbf{implementpoly} (\ref{labimplementpoly})
