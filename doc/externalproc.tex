\subsection{externalproc}
\label{labexternalproc}
\noindent Name: \textbf{externalproc}\\
binds an external code to a \sollya procedure\\
\noindent Usage: 
\begin{center}
\textbf{externalproc}(\emph{identifier}, \emph{filename}, \emph{argumenttype} $->$ \emph{resulttype}) : (\textsf{identifier type}, \textsf{string}, \textsf{type type}, \textsf{type type}) $\rightarrow$ \textsf{void}\\
\end{center}
Parameters: 
\begin{itemize}
\item \emph{identifier} represents the identifier the code is to be bound to
\item \emph{filename} of type \textsf{string} represents the name of the object file where the code of procedure can be found
\item \emph{argumenttype} represents a definition of the types of the arguments of the \sollya procedure and the external code
\item \emph{resulttype} represents a definition of the result type of the external code
\end{itemize}
\noindent Description: \begin{itemize}

\item \textbf{externalproc} allows for binding the \sollya identifier
   \emph{identifier} to an external code.  After this binding, when \sollya
   encounters \emph{identifier} applied to a list of actual parameters, it
   will evaluate these parameters and call the external code with these
   parameters. If the external code indicated success, it will receive
   the result produced by the external code, transform it to \sollya's
   internal representation and return it.
    
   In order to allow correct evaluation and typing of the data in
   parameter and in result to be passed to and received from the external
   code, \textbf{externalproc} has a third parameter \emph{argumenttype} $->$ \emph{resulttype}.
   Both \emph{argumenttype} and \emph{resulttype} are one of \textbf{void}, \textbf{constant},
   \textbf{function}, \textbf{range}, \textbf{integer}, \textbf{string}, \textbf{boolean}, \textbf{list of} \textbf{constant}, \textbf{list of} \textbf{function}, 
   \textbf{list of} \textbf{range}, \textbf{list of} \textbf{integer}, \textbf{list of} \textbf{string}, \textbf{list of} \textbf{boolean}.
    
   If upon a usage of a procedure bound to an external procedure the type
   of the actual parameters given or its number is not correct, \sollya
   produces a type error. An external function not applied to arguments
   represents itself and prints out with its argument and result types.
    
   The external function is supposed to return an integer indicating
   success.  It returns its result depending on its \sollya result type
   as follows. Here, the external procedure is assumed to be implemented
   as a C function.
    
   If the \sollya result type is void, the C function has no pointer
   argument for the result.  If the \sollya result type is \textbf{constant}, the
   first argument of the C function is of C type \texttt{mpfr\_t *}, the result is
   returned by affecting the MPFR variable.  If the \sollya result type
   is \textbf{function}, the first argument of the C function is of C type \texttt{node **},
   the result is returned by the \texttt{node *} pointed with a new \texttt{node *}.
   If the \sollya result type is \textbf{range}, the first argument of the C
   function is of C type \texttt{mpfi\_t *}, the result is returned by affecting
   the MPFI variable.  If the \sollya result type is \textbf{integer}, the first
   argument of the C function is of C type \texttt{int *}, the result is returned
   by affecting the int variable.  If the \sollya result type is \textbf{string},
   the first argument of the C function is of C type \texttt{char **}, the result
   is returned by the \texttt{char *} pointed with a new \texttt{char *}.  If the \sollya
   result type is \textbf{boolean}, the first argument of the C function is of C
   type \texttt{int *}, the result is returned by affecting the int variable with
   a boolean value.  If the \sollya result type is \textbf{list of} type, the
   first argument of the C function is of C type \texttt{chain **}, the result is
   returned by the \texttt{chain *} pointed with a new \texttt{chain *}.  This chain
   contains for \sollya type \textbf{constant} pointers \texttt{mpfr\_t *} to new MPFR
   variables, for \sollya type \textbf{function} pointers \texttt{node *} to new nodes, for
   \sollya type \textbf{range} pointers \texttt{mpfi\_t *}  to new MPFI variables, for
   \sollya type \textbf{integer} pointers \texttt{int *} to new int variables for \sollya
   type \textbf{string} pointers \texttt{char *} to new \texttt{char *} variables and for \sollya
   type \textbf{boolean} pointers \texttt{int *} to new int variables representing boolean
   values.
    	       
   The external procedure affects its possible pointer argument if and
   only if it succeeds.  This means, if the function returns an integer
   indicating failure, it does not leak any memory to the encompassing
   environment.
    
   The external procedure receives its arguments as follows: If the
   \sollya argument type is \textbf{void}, no argument array is given.  Otherwise
   the C function receives a C \texttt{void **} argument representing an array of
   size equal to the arity of the function where each entry (of C type
   \texttt{void *}) represents a value with a C type depending on the
   corresponding \sollya type. If the \sollya type is \textbf{constant}, the C
   type the \texttt{void *} is to be casted to is \texttt{mpfr\_t *}.  If the \sollya type
   is \textbf{function}, the C type the \texttt{void *} is to be casted to is \texttt{node *}.  If
   the \sollya type is \textbf{range}, the C type the \texttt{void *} is to be casted to is
   \texttt{mpfi\_t *}.  If the \sollya type is \textbf{integer}, the C type the \texttt{void *} is to
   be casted to is \texttt{int *}.  If the \sollya type is \textbf{string}, the C type the
   \texttt{void *} is to be casted to is \texttt{char *}.  If the \sollya type is \textbf{boolean},
   the C type the \texttt{void *} is to be casted to is \texttt{int *}.  If the \sollya
   type is \textbf{list of} type, the C type the \texttt{void *} is to be casted to is
   \texttt{chain *}.  Here depending on type, the values in the chain are to be
   casted to \texttt{mpfr\_t *}  for \sollya type \textbf{constant}, \texttt{node *} for \sollya type
   \textbf{function}, \texttt{mpfi\_t *} for \sollya type \textbf{range}, \texttt{int *} for \sollya type
   \textbf{integer}, \texttt{char *} for \sollya type \textbf{string} and \texttt{int *} for \sollya type
   \textbf{boolean}.
    
   The external procedure is not supposed to alter the memory pointed by
   its array argument \texttt{void **}.
    
   In both directions (argument and result values), empty lists are
   represented by \texttt{chain * NULL} pointers.
    
   In contrast to internal procedures, externally bounded procedures can
   be considered to be objects inside \sollya that can be assigned to other
   variables, stored in list etc.
\end{itemize}
\noindent Example 1: 
\begin{center}\begin{minipage}{15cm}\begin{Verbatim}[frame=single]
> bashexecute("gcc -fPIC -Wall -c externalprocexample.c");
> bashexecute("gcc -fPIC -shared -o externalprocexample externalprocexample.o");

> externalproc(foo, "./externalprocexample", (integer, integer) -> integer);
> foo;
foo(integer, integer) -> integer
> foo(5, 6);
11
> verbosity = 1!;
> foo();
Warning: at least one of the given expressions or a subexpression is not correct
ly typed
or its evaluation has failed because of some error on a side-effect.
error
> a = foo;
> a(5,6);
11
\end{Verbatim}
\end{minipage}\end{center}
See also: \textbf{library} (\ref{lablibrary}), \textbf{externalplot} (\ref{labexternalplot}), \textbf{bashexecute} (\ref{labbashexecute}), \textbf{void} (\ref{labvoid}), \textbf{constant} (\ref{labconstant}), \textbf{function} (\ref{labfunction}), \textbf{range} (\ref{labrange}), \textbf{integer} (\ref{labinteger}), \textbf{string} (\ref{labstring}), \textbf{boolean} (\ref{labboolean}), \textbf{list of} (\ref{lablistof})
