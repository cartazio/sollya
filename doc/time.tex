\subsection{time}
\label{labtime}
\noindent Name: \textbf{time}\\
procedure for timing \sollya code.\\
\noindent Usage: 
\begin{center}
\textbf{time}(\emph{code}) : \textsf{code} $\rightarrow$ \textsf{constant}\\
\end{center}
Parameters: 
\begin{itemize}
\item \emph{code} is the code to be timed.
\end{itemize}
\noindent Description: \begin{itemize}

\item \textbf{time} permits timing a \sollya instruction, resp. a begin-end block
   of \sollya instructions. The timing value, measured in seconds, is returned
   as a \sollya constant (and not merely displayed as for \textbf{timing}). This 
   permits performing computations of the timing measurement value inside \sollya.

\item The extended \textbf{nop} command permits executing a defined number of
   useless instructions. Taking the ratio of the time needed to execute a
   certain \sollya instruction and the time for executing a \textbf{nop}
   therefore gives a way to abstract from the speed of a particular 
   machine when evaluating an algorithm's performance.
\end{itemize}
\noindent Example 1: 
\begin{center}\begin{minipage}{15cm}\begin{Verbatim}[frame=single]
> t = time(p=remez(sin(x),10,[-1;1]));
> write(t,"s were spent computing p = ",p,"\n");
0.22568900000000000001824582152032405701902462169528s were spent computing p = -
3.3426550293345171908513995127407122194691200059639e-17 + x * (0.999999999736283
59955372011464713121003442988167693 + x * (7.88027518773027866844993437990477324
95568873819693e-16 + x * (-0.166666661386013037032912982196741385680498698107285
 + x * (-5.3734444911159112186289355138557504839692987221233e-15 + x * (8.333303
7186548537651002133031675072810009327877148e-3 + x * (1.337972213892188158841123
41005509831429347230871284e-14 + x * (-1.983448630182774164932681551541589244220
04290239026e-4 + x * (-1.3789116451286674170531616441916183417598709732816e-14 +
 x * (2.6876259495430304684251822024896210963401672262005e-6 + x * 5.02823783500
10211058128384123578805586173782863605e-15)))))))))
\end{Verbatim}
\end{minipage}\end{center}
\noindent Example 2: 
\begin{center}\begin{minipage}{15cm}\begin{Verbatim}[frame=single]
> write(time({ p=remez(sin(x),10,[-1;1]); write("The error is 2^(", log2(dirtyin
fnorm(p-sin(x),[-1;1])), ")\n"); }), " s were spent\n");
The error is 2^(log2(2.39602467695631727848641768186659313738474584992648e-11))
0.40015999999999999996429939086439730999700259417295 s were spent
\end{Verbatim}
\end{minipage}\end{center}
\noindent Example 3: 
\begin{center}\begin{minipage}{15cm}\begin{Verbatim}[frame=single]
> t = time(bashexecute("sleep 10"));
> write(~(t-10),"s of execution overhead.\n");
3.87399999999999897215552380203007487580180168151855e-3s of execution overhead.
\end{Verbatim}
\end{minipage}\end{center}
\noindent Example 4: 
\begin{center}\begin{minipage}{15cm}\begin{Verbatim}[frame=single]
> ratio := time(p=remez(sin(x),10,[-1;1]))/time(nop(10));
> write("This ratio = ", ratio, " should somehow be independent of the type of m
achine.\n");
This ratio = 6.6980001788215658820378916265155049132984203512334 should somehow 
be independent of the type of machine.
\end{Verbatim}
\end{minipage}\end{center}
See also: \textbf{timing} (\ref{labtiming}), \textbf{nop} (\ref{labnop})
