\subsection{gt}
\label{labgt}
\noindent Name: \textbf{$>$}\\
greater-than operator\\
\noindent Usage: 
\begin{center}
\emph{expr1} \textbf{$>$} \emph{expr2} : (\textsf{constant}, \textsf{constant}) $\rightarrow$ \textsf{boolean}
\end{center}
Parameters: 
\begin{itemize}
\item \emph{expr1} and \emph{expr2} represent constant expressions
\end{itemize}
\noindent Description: \begin{itemize}

\item The operator \textbf{$>$} evaluates to true iff its operands \emph{expr1} and
   \emph{expr2} evaluate to two floating-point numbers $a_1$
   respectively $a_2$ with the global precision \textbf{prec} and
   $a_1$ is greater than $a_2$. The user should
   be aware of the fact that because of floating-point evaluation, the
   operator \textbf{$>$} is not exactly the same as the mathematical
   operation \emph{greater-than}.
\end{itemize}
\noindent Example 1: 
\begin{center}\begin{minipage}{15cm}\begin{Verbatim}[frame=single]
> 5 > 4;
true
> 5 > 5;
false
> 5 > 6;
false
> exp(2) > exp(1);
true
> log(1) > exp(2);
false
\end{Verbatim}
\end{minipage}\end{center}
\noindent Example 2: 
\begin{center}\begin{minipage}{15cm}\begin{Verbatim}[frame=single]
> prec = 12;
The precision has been set to 12 bits.
> 16385.1 > 16384.1;
false
\end{Verbatim}
\end{minipage}\end{center}
See also: \textbf{$==$} (\ref{labequal}), \textbf{!$=$} (\ref{labneq}), \textbf{$>=$} (\ref{labge}), \textbf{$<=$} (\ref{lable}), \textbf{$<$} (\ref{lablt}), \textbf{!} (\ref{labnot}), \textbf{$\&\&$} (\ref{laband}), \textbf{$||$} (\ref{labor}), \textbf{prec} (\ref{labprec})
