\subsection{honorcoeffprec}
\label{labhonorcoeffprec}
\noindent Name: \textbf{honorcoeffprec}\\
indicates the (forced) honoring the precision of the coefficients in \textbf{implementpoly}\\
\noindent Usage: 
\begin{center}
\textbf{honorcoeffprec} : \textsf{honorcoeffprec}\\
\end{center}
\noindent Description: \begin{itemize}

\item Used with command \\textbf{implementpoly}, \\textbf{honorcoeffprec} makes \\textbf{implementpoly} honor\n   the precision of the given polynomial. This means if a coefficient\n   needs a double-double or a triple-double to be exactly stored,\n   \\textbf{implementpoly} will allocate appropriate space and use a double-double\n   or triple-double operation even if the automatic (heuristic)\n   determination implemented in command \\textbf{implementpoly} indicates that the\n   coefficient could be stored on less precision or, respectively, the\n   operation could be performed with less precision. See \\textbf{implementpoly}\n   for details.\n\end{itemize}
\noindent Example 1: 
\begin{center}\begin{minipage}{15cm}\begin{Verbatim}[frame=single]
\end{Verbatim}
\end{minipage}\end{center}
See also: \textbf{implementpoly} (\ref{labimplementpoly}), \textbf{printexpansion} (\ref{labprintexpansion})
