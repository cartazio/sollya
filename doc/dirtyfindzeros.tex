\subsection{dirtyfindzeros}
\label{labdirtyfindzeros}
\noindent Name: \textbf{dirtyfindzeros}\\
gives a list of numerical values listing the zeros of a function on an interval.\\
\noindent Usage: 
\begin{center}
\textbf{dirtyfindzeros}(\emph{f},\emph{I}) : (\textsf{function}, \textsf{range}) $\rightarrow$ \textsf{list}\\
\end{center}
Parameters: 
\begin{itemize}
\item \emph{f} is a function.
\item \emph{I} is an interval.
\end{itemize}
\noindent Description: \begin{itemize}

\item \\textbf{dirtyfindzeros}(\\emph{f},\\emph{I}) returns a list containing some zeros of \\emph{f} in the\n   interval \\emph{I}. The values in the list are numerical approximation of the exact\n   zeros. The precision of these approximations is approximately the precision\n   stored in \\textbf{prec}. If \\emph{f} does not have two zeros very close to each other, it \n   can be expected that all zeros are listed. However, some zeros may be\n   forgotten. This command should be considered as a numerical algorithm and\n   should not be used if safety is critical.\n
\item More precisely, the algorithm relies on global variables \\textbf{prec} and \\textbf{points} and it performs the following steps: \n   let $n$ be the value of variable \\textbf{points} and $t$ be the value\n   of variable \\textbf{prec}.\n   \\begin{itemize}\n   \\item Evaluate $|f|$ at $n$ evenly distributed points in the interval $I$.\n     The working precision to be used is automatically chosen in order to ensure that the sign\n     is correct.\n   \\item Whenever $f$ changes its sign for two consecutive points,\n     find an approximation $x$ of its zero with precision $t$ using\n     Newton's algorithm. The number of steps in Newton's iteration depends on $t$:\n     the precision of the approximation is supposed to be doubled at each step.\n   \\item Add this value to the list.\n   \\end{itemize}\n\end{itemize}
\noindent Example 1: 
\begin{center}\begin{minipage}{15cm}\begin{Verbatim}[frame=single]
\end{Verbatim}
\end{minipage}\end{center}
\noindent Example 2: 
\begin{center}\begin{minipage}{15cm}\begin{Verbatim}[frame=single]
\end{Verbatim}
\end{minipage}\end{center}
See also: \textbf{prec} (\ref{labprec}), \textbf{points} (\ref{labpoints}), \textbf{findzeros} (\ref{labfindzeros})
