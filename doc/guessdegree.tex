\subsection{guessdegree}
\label{labguessdegree}
\noindent Name: \textbf{guessdegree}\\
returns the minimal degree needed for a polynomial to approximate a function with a certain error on an interval.\\
\noindent Usage: 
\begin{center}
\textbf{guessdegree}(\emph{f},\emph{I},\emph{eps},\emph{w}) : (\textsf{function}, \textsf{range}, \textsf{constant}, \textsf{function}) $\rightarrow$ \textsf{range}
\end{center}
Parameters: 
\begin{itemize}
\item \emph{f} is the function to be approximated.
\item \emph{I} is the interval where the function must be approximated.
\item \emph{eps} is the maximal acceptable error.
\item \emph{w} (optional) is a weight function. Default is 1.
\end{itemize}
\noindent Description: \begin{itemize}

\item \textbf{guessdegree} tries to find the minimal degree needed to approximate \emph{f}
   on \emph{I} by a polynomial with an infinity norm not greater than \emph{eps}.
   More precisely, it finds $n$ minimal such that there exists a
   polynomial $p$ of degree $n$ such that $\|pw-f\|_{\infty} < \mathrm{eps}$.

\item \textbf{guessdegree} returns an interval: for common cases, this interval is reduced to a 
   single number (e.g. the minimal degree). But in certain cases, \textbf{guessdegree} does
   not succeed in finding the minimal degree. In such cases the returned interval
   is of the form $[n,\,p]$ such that:
   \begin{itemize}
   \item no polynomial of degree $n-1$ gives an error less than \emph{eps}.
   \item there exists a polynomial of degree $p$ giving an error less than \emph{eps}. 
   \end{itemize}
\end{itemize}
\noindent Example 1: 
\begin{center}\begin{minipage}{15cm}\begin{Verbatim}[frame=single]
> guessdegree(exp(x),[-1;1],1e-10);
[10;10]
\end{Verbatim}
\end{minipage}\end{center}
\noindent Example 2: 
\begin{center}\begin{minipage}{15cm}\begin{Verbatim}[frame=single]
> guessdegree(1, [-1;1], 1e-8, 1/exp(x));
[8;9]
\end{Verbatim}
\end{minipage}\end{center}
See also: \textbf{dirtyinfnorm} (\ref{labdirtyinfnorm}), \textbf{remez} (\ref{labremez})
