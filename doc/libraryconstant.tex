\subsection{libraryconstant}
\label{lablibraryconstant}
\noindent Name: \textbf{libraryconstant}\\
\phantom{aaa}binds an external mathematical constant to a variable in \sollya\\[0.2cm]
\noindent Library names:\\
\verb|   sollya_obj_t sollya_lib_libraryconstant(char *, void (*)(mpfr_t, mp_prec_t))|\\
\verb|   sollya_obj_t sollya_lib_build_function_libraryconstant(char *,|\\
\verb|                                                          void (*)(mpfr_t,|\\
\verb|                                                                   mp_prec_t))|\\
\verb|   sollya_obj_t sollya_lib_libraryconstant_with_data(char *, void (*)(mpfr_t, mp_prec_t, void *), void *)|\\
\verb|   sollya_obj_t sollya_lib_build_function_libraryconstant_with_data(char *,|\\
\verb|                                                                    void (*)(mpfr_t,|\\
\verb|                                                                    mp_prec_t, void *), void *)|\\[0.2cm]
\noindent Usage: 
\begin{center}
\textbf{libraryconstant}(\emph{path}) : \textsf{string} $\rightarrow$ \textsf{function}\\
\end{center}
\noindent Description: \begin{itemize}

\item The command \textbf{libraryconstant} lets you extend the set of mathematical
   constants known to \sollya.
   By default, the only mathematical constant known by \sollya is \textbf{pi}.
   For particular applications, one may want to
   manipulate other constants, such as Euler's gamma constant, for instance.

\item \textbf{libraryconstant} makes it possible to let \sollya know about new constants.
   In order to let it know, you have to provide an implementation of the
   constant you are interested in. This implementation is a C file containing
   a function of the form:
   \begin{verbatim} void my_ident(mpfr_t result, mp_prec_t prec)\end{verbatim}
   The semantic of this function is the following: it is an implementation of
   the constant in arbitrary precision.
   \verb|my_ident(result, prec)| shall set the
   precision of the variable result to a suitable precision (the variable is
   assumed to be already initialized) and store in result an approximate value
   of the constant with a relative error not greater than $2^{1-\mathrm{prec}}$.
   More precisely, if $c$ is the exact value of the constant, the value stored
   in result should satisfy $$|\mathrm{result}-c| \le |c|\,2^{1-\mathrm{prec}}.$$

\item You can include sollya.h in your implementation and use library 
   functionnalities of \sollya for your implementation. However, this requires to
   have compiled \sollya with \texttt{-fPIC} in order to make the \sollya executable
   code position independent and to use a system on with programs, using \texttt{dlopen}
   to open dynamic routines can dynamically open themselves.

\item To bind your constant into \sollya, you must use the same identifier as the
   function name used in your implementation file (\verb|my_ident| in the previous
   example). Once the function code has been bound to an identifier, you can use
   a simple assignment to assign the bound identifier to yet another identifier.
   This way, you may use convenient names inside \sollya even if your
   implementation environment requires you to use a less convenient name.

\item Once your constant is bound, it is considered by \sollya as an infinitely
   accurate constant (i.e. a 0-ary function, exactly like \textbf{pi}).

\item The dynamic object file whose name is given to \textbf{libraryconstant} for binding of an
   external library constant may also define a destructor function \verb|int sollya_external_lib_close(void)|.
   If \sollya finds such a destructor function in the dynamic object file, it will call 
   that function when closing the dynamic object file again. This happens when \sollya
   is terminated or when the current \sollya session is restarted using \textbf{restart}.
   The purpose of the destructor function is to allow the dynamically bound code
   to free any memory that it might have allocated before \sollya is terminated 
   or restarted. 
   The dynamic object file is not necessarily needed to define a destructor
   function. This ensure backward compatibility with older \sollya external 
   library function object files.
   When defined, the destructor function is supposed to return an integer
   value indicating if an error has happened. Upon success, the destructor
   functions is to return a zero value, upon error a non-zero value.
\end{itemize}
\noindent Example 1: 
\begin{center}\begin{minipage}{15cm}\begin{Verbatim}[frame=single]
> bashexecute("gcc -fPIC -Wall -c libraryconstantexample.c -I$HOME/.local/includ
e");
> bashexecute("gcc -shared -o libraryconstantexample libraryconstantexample.o -l
gmp -lmpfr");
> euler_gamma = libraryconstant("./libraryconstantexample");
> prec = 20!;
> euler_gamma;
0.577215
> prec = 100!;
> euler_gamma;
0.577215664901532860606512090082
> midpointmode = on;
Midpoint mode has been activated.
> [euler_gamma];
0.57721566490153286060651209008~2/4~
\end{Verbatim}
\end{minipage}\end{center}
See also: \textbf{bashexecute} (\ref{labbashexecute}), \textbf{externalproc} (\ref{labexternalproc}), \textbf{externalplot} (\ref{labexternalplot}), \textbf{pi} (\ref{labpi}), \textbf{library} (\ref{lablibrary}), \textbf{evaluate} (\ref{labevaluate}), \textbf{implementconstant} (\ref{labimplementconstant})
