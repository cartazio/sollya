\subsection{RU}
\label{labru}
\noindent Name: \textbf{RU}\\
\phantom{aaa}constant representing rounding-upwards mode.\\[0.2cm]
\noindent Library names:\\
\verb|   sollya_obj_t sollya_lib_round_up()|\\
\verb|   int sollya_lib_is_round_up(sollya_obj_t)|\\[0.2cm]
\noindent Description: \begin{itemize}

\item \textbf{RU} is used in command \textbf{round} to specify that the value $x$ must be rounded
   to the smallest floating-point number $y$ such that $x \le y$.
\end{itemize}
\noindent Example 1: 
\begin{center}\begin{minipage}{15cm}\begin{Verbatim}[frame=single]
> display=binary!;
> round(Pi,20,RU);
1.100100100001111111_2 * 2^(1)
\end{Verbatim}
\end{minipage}\end{center}
See also: \textbf{RZ} (\ref{labrz}), \textbf{RD} (\ref{labrd}), \textbf{RN} (\ref{labrn}), \textbf{round} (\ref{labround}), \textbf{ceil} (\ref{labceil})
