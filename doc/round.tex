\subsection{ round }
\noindent Name: \textbf{round}\\
rounds a number to a floating-point format.\\

\noindent Usage: 
\begin{center}
\textbf{round}(\emph{x},\emph{n},\emph{mode}) : (\textsf{constant}, \textsf{integer}, \textbf{RD}|\textbf{RU}|\textbf{RN}|\textbf{RZ}) $\rightarrow$ \textsf{constant}\\
\end{center}
Parameters: 
\emph{x} is a constant to be rounded.\\
\emph{n} is the precision of the target format.\\
\emph{mode} is the desired rounding mode.\\

\noindent Description: \begin{itemize}

\item \textbf{round}(\emph{x},\emph{n},\emph{mode}) rounds \emph{x} to a floating-point number with 
   precision \emph{n}, according to rounding-mode \emph{mode}. 

\item Subnormal numbers are not handled. The range of possible exponents is the 
   range used for all numbers represented in Sollya (e.g. basically the range 
   used in the library MPFR). Please use the functions \textbf{double}, \textbf{doubleextended},
   \textbf{doubledouble} and \textbf{tripledouble} for roundings to classical formats with their
   range of exponents.
\end{itemize}
\noindent Example 1: 
\begin{center}\begin{minipage}{14.8cm}\begin{Verbatim}[frame=single]
   > display=binary!;
   > round(Pi,20,RN);
   1.100100100001111111_2 * 2^(1)
\end{Verbatim}
\end{minipage}\end{center}
\noindent Example 2: 
\begin{center}\begin{minipage}{14.8cm}\begin{Verbatim}[frame=single]
   > display=binary!;
   > a=2^(-1100);
   > round(a,53,RN);
   1._2 * 2^(-1100)
   > double(a);
   0
\end{Verbatim}
\end{minipage}\end{center}
See also: \textbf{RN}, \textbf{RD}, \textbf{RU}, \textbf{RZ}, \textbf{double}, \textbf{doubleextended}, \textbf{doubledouble}, \textbf{tripledouble}, \textbf{roundcoefficients}, \textbf{roundcorrectly}
