\subsection{inf}
\label{labinf}
\noindent Name: \textbf{inf}\\
gives the lower bound of an interval.\\
\noindent Usage: 
\begin{center}
\textbf{inf}(\emph{I}) : \textsf{range} $\rightarrow$ \textsf{constant}
\textbf{inf}(\emph{x}) : \textsf{constant} $\rightarrow$ \textsf{constant}
\end{center}
Parameters: 
\begin{itemize}
\item \emph{I} is an interval.
\item \emph{x} is a real number.
\end{itemize}
\noindent Description: \begin{itemize}

\item Returns the lower bound of the interval \emph{I}. Each bound of an interval has its 
   own precision, so this command is exact, even if the current precision is too 
   small to represent the bound.

\item When called on a real number \emph{x}, \textbf{inf} considers it as an interval formed
   of a single point: [x, x]. In other words, \textbf{inf} behaves like the identity.
\end{itemize}
\noindent Example 1: 
\begin{center}\begin{minipage}{15cm}\begin{Verbatim}[frame=single]
> inf([1;3]);
1
> inf(0);
0
\end{Verbatim}
\end{minipage}\end{center}
\noindent Example 2: 
\begin{center}\begin{minipage}{15cm}\begin{Verbatim}[frame=single]
> display=binary!;
> I=[0.111110000011111_2; 1];
> inf(I);
1.11110000011111_2 * 2^(-1)
> prec=12!;
> inf(I);
1.11110000011111_2 * 2^(-1)
\end{Verbatim}
\end{minipage}\end{center}
See also: \textbf{mid} (\ref{labmid}), \textbf{sup} (\ref{labsup})
