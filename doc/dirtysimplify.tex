\subsection{dirtysimplify}
\label{labdirtysimplify}
\noindent Name: \textbf{dirtysimplify}\\
\phantom{aaa}simplifies an expression representing a function\\[0.2cm]
\noindent Library name:\\
\verb|   sollya_obj_t sollya_lib_dirtysimplify(sollya_obj_t)|\\[0.2cm]
\noindent Usage: 
\begin{center}
\textbf{dirtysimplify}(\emph{function}) : \textsf{function} $\rightarrow$ \textsf{function}\\
\end{center}
Parameters: 
\begin{itemize}
\item \emph{function} represents the expression to be simplified
\end{itemize}
\noindent Description: \begin{itemize}

\item The command \textbf{dirtysimplify} simplifies constant subexpressions of the
   expression given in argument representing the function
   \emph{function}. Those constant subexpressions are evaluated using
   floating-point arithmetic with the global precision \textbf{prec}.
\end{itemize}
\noindent Example 1: 
\begin{center}\begin{minipage}{15cm}\begin{Verbatim}[frame=single]
> print(dirtysimplify(sin(pi * x)));
sin(3.1415926535897932384626433832795028841971693993751 * x)
> print(dirtysimplify(erf(exp(3) + x * log(4))));
erf(20.0855369231876677409285296545817178969879078385544 + x * 1.386294361119890
6188344642429163531361510002687205)
\end{Verbatim}
\end{minipage}\end{center}
\noindent Example 2: 
\begin{center}\begin{minipage}{15cm}\begin{Verbatim}[frame=single]
> prec = 20!;
> t = erf(0.5);
> s = dirtysimplify(erf(0.5));
> prec = 200!;
> t;
0.520499877813046537682746653891964528736451575757963700058806
> s;
0.52050018310546875
\end{Verbatim}
\end{minipage}\end{center}
See also: \textbf{simplify} (\ref{labsimplify}), \textbf{autosimplify} (\ref{labautosimplify}), \textbf{prec} (\ref{labprec}), \textbf{evaluate} (\ref{labevaluate}), \textbf{horner} (\ref{labhorner}), \textbf{rationalmode} (\ref{labrationalmode})
