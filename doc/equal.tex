\subsection{$==$}
\label{labequal}
\noindent Name: \textbf{$==$}\\
\phantom{aaa}equality test operator\\[0.2cm]
\noindent Library name:\\
\verb|   sollya_obj_t sollya_lib_cmp_equal(sollya_obj_t, sollya_obj_t)|\\[0.2cm]
\noindent Usage: 
\begin{center}
\emph{expr1} \textbf{$==$} \emph{expr2} : (\textsf{any type}, \textsf{any type}) $\rightarrow$ \textsf{boolean}\\
\end{center}
Parameters: 
\begin{itemize}
\item \emph{expr1} and \emph{expr2} represent expressions
\end{itemize}
\noindent Description: \begin{itemize}

\item The test \emph{expr1} \textbf{$==$} \emph{expr2} returns \textbf{true} when \emph{expr1} and \emph{expr2} are
   syntactically equal and different from \textbf{error} and @NaN@. Conversely if \emph{expr1}
   and \emph{expr2} are objects that are mathematically different and \sollya manages
   to figure it out, the test returns \textbf{false}. In between these two cases, there
   is the grey zone of expressions that are not syntactically equal but are
   mathematically equal. In such a case, \sollya normally tries to determine if
   the expressions are mathematically equal and if it manages to prove it, it
   returns \textbf{true}, without a warning. In the case when \emph{expr1} and \emph{expr2} are
   two constant expressions, \sollya will in particular try to evaluate their
   difference: in the case when the difference is 0 or is so small that \sollya
   does not manage to obtain a faithful rounding of the real value, it will
   return \textbf{true} (with a warning if it has not been possible to actually prove
   that the real value is 0). In any other case, when both expressions are not
   syntactically equal and \sollya has not been able to prove that they are
   mathematically equal, it returns \textbf{false}.

\item The level of simplifications performed by \sollya to determine if
   expressions are mathematically equal depends on the value of \textbf{autosimplify}.
   If it is \textbf{off}, no formal simplification is performed, hence expression trees
   as simple as x+1 and 1+x will be considered not equal. Conversely, if
   \textbf{autosimplify} is set to \textbf{on}, polynomial subexpressions that are mathematically
   equal will in general be recognized as being equal.

\item The user should always keep in mind that a litteral constant written in
   decimal arithmetic (such as 0.1 for instance) is not considered as an exact
   constant by \sollya (unless it is exactly representable in binary without
   requiring too much precision) and is first correctly rounded at precision
   \textbf{prec}, prior to any other operation. Of course, this leads to a rounding
   warning, but it is important to remember that this is done before the
   expression trees are compared, possibly leading to two expressions comparing
   equal, while they are obviously mathematically different, just because they
   contain different constants that have been rounded to the same value at
   precision \textbf{prec}. As a general rule, to avoid this behavior, the user should
   represent constants in an exact format such as hexadecimal or represent
   decimal constants as integer fractions (e.g., 0.1 represented by the constant
   expression 1/10).

\item Notice that @NaN@ and \textbf{error} share the property that they both compare equal
   and different to anything, i.e., if the variable \emph{a} contains @NaN@ or \textbf{error}
   and whatever the content of variable \emph{b} is, the tests \emph{a} \textbf{$==$} \emph{b} and
   \emph{a} \textbf{!$=$} \emph{b} both return \textbf{false}. The standard way of testing if \emph{a} contains
   @NaN@ or \textbf{error} is indeed to check if \emph{a} \textbf{$==$} \emph{a} returns false. In such a
   case, it is however impossible to determine what is the actual value of \emph{a}
   amongst both possibilities using only \textbf{$==$} or \textbf{!$=$}. The standard way to
   discriminate this situation is to use the match ... with ... construct.
\end{itemize}
\noindent Example 1: 
\begin{center}\begin{minipage}{15cm}\begin{Verbatim}[frame=single]
> "Hello" == "Hello";
true
> "Hello" == "Salut";
false
> "Hello" == 5;
false
> 5 + x == 5 + x;
true
\end{Verbatim}
\end{minipage}\end{center}
\noindent Example 2: 
\begin{center}\begin{minipage}{15cm}\begin{Verbatim}[frame=single]
> verbosity = 1!;
> asin(1) * 2 == pi;
true
> cos(3)^2 == 1 - sin(3)^2;
Warning: the tool is unable to decide an equality test by evaluation even though
 faithful evaluation of the terms has been possible. The terms will be considere
d to be equal.
true
> exp(5) == log(4);
false
\end{Verbatim}
\end{minipage}\end{center}
\noindent Example 3: 
\begin{center}\begin{minipage}{15cm}\begin{Verbatim}[frame=single]
> autosimplify=off;
Automatic pure tree simplification has been deactivated.
> exp(1+x) == exp(x+1);
false
> autosimplify=on;
Automatic pure tree simplification has been activated.
> exp(1+x) == exp(x+1);
false
> (1/3+x)^2 == x^2 + 1/9 + (5-3)*x/3;
true
> log(x)/log(10) == log10(x);
false
\end{Verbatim}
\end{minipage}\end{center}
\noindent Example 4: 
\begin{center}\begin{minipage}{15cm}\begin{Verbatim}[frame=single]
> prec = 12;
The precision has been set to 12 bits.
> verbosity = 1!;
> 16384.1 == 16385.1;
Warning: Rounding occurred when converting the constant "16384.1" to floating-po
int with 12 bits.
If safe computation is needed, try to increase the precision.
Warning: Rounding occurred when converting the constant "16385.1" to floating-po
int with 12 bits.
If safe computation is needed, try to increase the precision.
true
> 16384 == 16384.25;
false
> 0.1 == 1/10;
Warning: Rounding occurred when converting the constant "0.1" to floating-point 
with 12 bits.
If safe computation is needed, try to increase the precision.
false
> 0.1 == round(1/10, prec, RN);
Warning: Rounding occurred when converting the constant "0.1" to floating-point 
with 12 bits.
If safe computation is needed, try to increase the precision.
true
\end{Verbatim}
\end{minipage}\end{center}
\noindent Example 5: 
\begin{center}\begin{minipage}{15cm}\begin{Verbatim}[frame=single]
> error == error;
false
> error != error;
false
> @NaN@ == @NaN@;
false
> @NaN@ != @NaN@;
false
> error == @NaN@;
false
> error != @NaN@;
false
> a = error;
> match a with
   @NaN@ : ("a contains @NaN@")
   default:("a contains something else");
error
> a = @NaN@;
> match a with
   @NaN@ : ("a contains @NaN@")
   default:("a contains something else");
a contains @NaN@
\end{Verbatim}
\end{minipage}\end{center}
See also: \textbf{!$=$} (\ref{labneq}), \textbf{$>$} (\ref{labgt}), \textbf{$>=$} (\ref{labge}), \textbf{$<=$} (\ref{lable}), \textbf{$<$} (\ref{lablt}), \textbf{in} (\ref{labin}), \textbf{!} (\ref{labnot}), \textbf{$\&\&$} (\ref{laband}), \textbf{$||$} (\ref{labor}), \textbf{error} (\ref{laberror}), \textbf{prec} (\ref{labprec}), \textbf{autosimplify} (\ref{labautosimplify})
