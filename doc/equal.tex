\subsection{$==$}
\label{labequal}
\noindent Name: \textbf{$==$}\\
equality test operator\\
\noindent Usage: 
\begin{center}
\emph{expr1} \textbf{$==$} \emph{expr2} : (\textsf{any type}, \textsf{any type}) $\rightarrow$ \textsf{boolean}\\
\end{center}
Parameters: 
\begin{itemize}
\item \emph{expr1} and \emph{expr2} represent expressions
\end{itemize}
\noindent Description: \begin{itemize}

\item The operator \\textbf{$==$} evaluates to true iff its operands \\emph{expr1} and\n   \\emph{expr2} are syntactically equal and different from \\textbf{error} or constant\n   expressions that are not constants and that evaluate to the same\n   floating-point number with the global precision \\textbf{prec}. The user should\n   be aware of the fact that because of floating-point evaluation, the\n   operator \\textbf{$==$} is not exactly the same as the mathematical\n   equality. Further remark that according to IEEE 754-2008 floating-point\n   rules, which \\sollya emulates, floating-point data which are NaN do not\n   compare equal to any other floating-point datum, including NaN. \n\end{itemize}
\noindent Example 1: 
\begin{center}\begin{minipage}{15cm}\begin{Verbatim}[frame=single]
\end{Verbatim}
\end{minipage}\end{center}
\noindent Example 2: 
\begin{center}\begin{minipage}{15cm}\begin{Verbatim}[frame=single]
\end{Verbatim}
\end{minipage}\end{center}
\noindent Example 3: 
\begin{center}\begin{minipage}{15cm}\begin{Verbatim}[frame=single]
\end{Verbatim}
\end{minipage}\end{center}
\noindent Example 4: 
\begin{center}\begin{minipage}{15cm}\begin{Verbatim}[frame=single]
\end{Verbatim}
\end{minipage}\end{center}
\noindent Example 5: 
\begin{center}\begin{minipage}{15cm}\begin{Verbatim}[frame=single]
\end{Verbatim}
\end{minipage}\end{center}
\noindent Example 6: 
\begin{center}\begin{minipage}{15cm}\begin{Verbatim}[frame=single]
\end{Verbatim}
\end{minipage}\end{center}
See also: \textbf{!$=$} (\ref{labneq}), \textbf{$>$} (\ref{labgt}), \textbf{$>=$} (\ref{labge}), \textbf{$<=$} (\ref{lable}), \textbf{$<$} (\ref{lablt}), \textbf{!} (\ref{labnot}), \textbf{$\&\&$} (\ref{laband}), \textbf{$||$} (\ref{labor}), \textbf{error} (\ref{laberror}), \textbf{prec} (\ref{labprec})
