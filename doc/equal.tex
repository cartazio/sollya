\subsection{ equal }
\noindent Name: \textbf{$==$}\\
equality test operator\\

\noindent Usage: 
\begin{center}
\emph{expr1} \textbf{$==$} \emph{expr2} : (\textsf{any type}, \textsf{any type}) $\rightarrow$ \textsf{boolean}\\
\end{center}
Parameters: 
\begin{itemize}
\item \emph{expr1} and \emph{expr2} represent expressions
\end{itemize}
\noindent Description: \begin{itemize}

\item The operator \textbf{$==$} evaluates to true iff its operands \emph{expr1} and
   \emph{expr2} are syntactically equal and different from \textbf{error} or constant
   expressions that evaluate to the same floating-point number with the
   global precision \textbf{prec}. The user should be aware of the fact that
   because of floating-point evaluation, the operator \textbf{$==$} is not
   exactly the same as the mathematical equality.
\end{itemize}
\noindent Example 1: 
\begin{center}\begin{minipage}{15cm}\begin{Verbatim}[frame=single]
> "Hello" == "Hello";
true
> "Hello" == "Salut";
false
> "Hello" == 5;
false
> 5 + x == 5 + x;
true
\end{Verbatim}
\end{minipage}\end{center}
\noindent Example 2: 
\begin{center}\begin{minipage}{15cm}\begin{Verbatim}[frame=single]
> 1 == exp(0);
true
> asin(1) * 2 == pi;
true
> exp(5) == log(4);
false
\end{Verbatim}
\end{minipage}\end{center}
\noindent Example 3: 
\begin{center}\begin{minipage}{15cm}\begin{Verbatim}[frame=single]
> prec = 12;
The precision has been set to 12 bits.
> 16384 == 16385;
true
\end{Verbatim}
\end{minipage}\end{center}
\noindent Example 4: 
\begin{center}\begin{minipage}{15cm}\begin{Verbatim}[frame=single]
> error == error;
false
\end{Verbatim}
\end{minipage}\end{center}
See also: \textbf{!$=$}, \textbf{$>$}, \textbf{$>=$}, \textbf{$<=$}, \textbf{$<$}, \textbf{!}, \textbf{$\&\&$}, \textbf{$||$}, \textbf{error}, \textbf{prec}
