\subsection{ printhexa }
\noindent Name: \textbf{printhexa}\\
prints a constant value as a hexadecimal double precision number\\

\noindent Usage: 
\begin{center}
\textbf{printhexa}(\emph{constant}) : \textsf{constant} $\rightarrow$ \textsf{void}\\
\end{center}
Parameters: 
\emph{constant} represents a constant\\

\noindent Description: \begin{itemize}

\item Prints a constant value as a hexadecimal number on 16 hexadecimal
   digits. The hexadecimal number represents the integer equivalent to
   the 64 bit memory representation of the constant considered as a
   double precision number.
   If the constant value does not hold on a double precision number, it
   is first rounded to the nearest double precision number before
   displayed. A warning is displayed in this case.
\end{itemize}
\noindent Example 1: 
\begin{center}\begin{minipage}{14.8cm}\begin{Verbatim}[frame=single]
   > printhexa(3);
   0x4008000000000000
\end{Verbatim}
\end{minipage}\end{center}
\noindent Example 2: 
\begin{center}\begin{minipage}{14.8cm}\begin{Verbatim}[frame=single]
   > prec=100!;
   > verbosity = 1!;
   > printhexa(exp(5));
   Warning: the given expression is not a constant but an expression to evaluate.
   Warning: rounding occurred before printing a value as a double.
   0x40628d389970338f
\end{Verbatim}
\end{minipage}\end{center}
See also: \textbf{printfloat}, \textbf{printexpansion}
