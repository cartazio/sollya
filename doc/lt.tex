\subsection{lt}
\label{lablt}
\noindent Name: \textbf{$<$}\\
less-than operator\\

\noindent Usage: 
\begin{center}
\emph{expr1} \textbf{$<$} \emph{expr2} : (\textsf{constant}, \textsf{constant}) $\rightarrow$ \textsf{boolean}\\
\end{center}
Parameters: 
\begin{itemize}
\item \emph{expr1} and \emph{expr2} represent constant expressions
\end{itemize}
\noindent Description: \begin{itemize}

\item The operator \textbf{$<$} evaluates to true iff its operands \emph{expr1} and
   \emph{expr2} evaluate to two floating-point numbers $a_1$
   respectively $a_2$ with the global precision \textbf{prec} and
   $a_1$ is less than $a_2$. The user should
   be aware of the fact that because of floating-point evaluation, the
   operator \textbf{$<$} is not exactly the same as the mathematical
   operation \emph{less-than}.
\end{itemize}
\noindent Example 1: 
\begin{center}\begin{minipage}{15cm}\begin{Verbatim}[frame=single]
> 5 < 4;
false
> 5 < 5;
false
> 5 < 6;
true
> exp(2) < exp(1);
false
> log(1) < exp(2);
true
\end{Verbatim}
\end{minipage}\end{center}
\noindent Example 2: 
\begin{center}\begin{minipage}{15cm}\begin{Verbatim}[frame=single]
> prec = 12;
The precision has been set to 12 bits.
> 16384 < 16385;
false
\end{Verbatim}
\end{minipage}\end{center}
See also: \textbf{$==$} (\ref{labequal}), \textbf{!$=$} (\ref{labneq}), \textbf{$>=$} (\ref{labge}), \textbf{$>$} (\ref{labgt}), \textbf{$<=$} (\ref{lable}), \textbf{!} (\ref{labnot}), \textbf{$\&\&$} (\ref{laband}), \textbf{$||$} (\ref{labor}), \textbf{prec} (\ref{labprec})
