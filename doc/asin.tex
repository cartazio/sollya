\subsection{asin}
\label{labasin}
\noindent Name: \textbf{asin}\\
the arcsine function.\\
\noindent Description: \begin{itemize}

\item \\textbf{asin} is the inverse of the function \\textbf{sin}: \\textbf{asin}($y$) is the unique number \n   $x \\in [-\\pi/2; \\pi/2]$ such that \\textbf{sin}($x$)=$y$.\n
\item It is defined only for $y \\in [-1;1]$.\n\end{itemize}
See also: \textbf{sin} (\ref{labsin})
