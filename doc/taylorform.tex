\subsection{taylorform}
\label{labtaylorform}
\noindent Name: \textbf{taylorform}\\
computes a rigorous polynomial approximation (polynomial, interval error bound) for a function, based on Taylor expansions\\
\noindent Usage: 
\begin{center}
\textbf{taylorform}(\emph{f}, \emph{n}, \emph{$x_0$} \emph{I}, \emph{errorType}) : (\textsf{function}, \textsf{integer}, \textsf{constant}, \textsf{range}, \textsf{absolute$|$relative}) $\rightarrow$ \textsf{list}\\
\textbf{taylorform}(\emph{f}, \emph{n}, \emph{$x_0$} \emph{I}, \emph{errorType}) : (\textsf{function}, \textsf{integer}, \textsf{range}, \textsf{range}, \textsf{absolute$|$relative}) $\rightarrow$ \textsf{list}\\
\end{center}
Parameters: 
\begin{itemize}
\item \emph{f} is the function to be approximated
\item \emph{n} is the order of the Taylor form, meaning $\emph{n}-1$ is the degree of the polynomial that must approximate \emph{f}
\item \emph{$x_0$} is the point (it can be a real number or an interval) where the Taylor exansion of the function is to be considered
\item \emph{I} is the interval over which the function is to be approximated
\item \emph{errorType} is the type of error to be considered. See the detailed description below.
\end{itemize}
\noindent Description: \begin{itemize}

\item \textbf{taylorform} computes an approximation polynomial and an interval error bound for function $f$. More precisely, it 
   returns a list $L = \left[p, \textrm{coeffErrors}, \Delta \right]$ where:
   \begin{itemize}
   \item $p$ is an approximation polynomial of degree $n-1$ which is roughly speaking a numerical Taylor expansion of $f$ at the point $x_0$.
   \item coeffsErrors is a list of $n$ intervals. Each interval coeffsErrors[$i$] contains an enclosure of all the errors accumulated when computing the $i$-th coefficient of $p$.
   \item $\Delta$ is an interval that provides a bound for the approximation error between $p$ and $f$. Its significance depends on the \emph{errorType} considered.
   \end{itemize}

\item Please note that $x_0$ can be an interval. In general, it is meant to be a small interval approximating a non representable value. For instance, if one desires to compute a Taylor approximation at point $\pi$, it is possible to set $x_0$ to the (almost) point-interval $[\pi]$. It is also possible to use a large interval for $x_0$, though it is not obvious to give an intuitive sense to the result of \textbf{taylorform} in that case.

\item More formally, the mathematical property ensured by the algorithm may be stated as follows. For all $xi_0$ in $x_0$, there exist (small) values $\varepsilon_i \in \textrm{coeffsErrors}[i]$ such that:
   \\
   If \emph{errorType} is \textbf{absolute}, $\forall x \in I, \exists \delta \in \Delta,\, f(x)-p(x-xi_0) = \sum\limits_{i=0}^{n-1} \varepsilon_i\, (x-xi_0)^i + \delta$.
   \\
   If \emph{errorType} is \textbf{relative}, $\forall x \in I, \exists \delta \in \Delta,\, f(x)-p(x-xi_0) = \sum\limits_{i=0}^{n-1} \varepsilon_i\, (x-xi_0)^i + \delta\,(x-xi_0)^n$.

\item The polynomial $p$ and the bound  $\Delta$ are obtained using Taylor Models principles.

\item Note: The relative case is especially useful when functions with removable singularities are considered. In such a case, this routine is able to compute a finite remainder bound, provided that the expansion point given is the problematic removable singularity point.

\item Note: the algorithm does not guarantee that by increasing the degree of the approximation, the remainder bound will become smaller. Moreover, it may 
   even become larger due to the dependecy phenomenon present with interval arithmetic. In order to reduce this phenomenon, a possible solution is to split the definition domain $I$ into several smaller intervals. 
\end{itemize}
\noindent Example 1: 
\begin{center}\begin{minipage}{15cm}\begin{Verbatim}[frame=single]
> TL=taylorform(exp(x), 10, 0, [-1,1], absolute);
> p=TL[0];
> Delta=TL[2];
> errors=TL[1];
> p; Delta;
1 + x * (1 + x * (0.5 + x * (0.1666666666666666666666666666666666666666666666666
7 + x * (4.1666666666666666666666666666666666666666666666667e-2 + x * (8.3333333
333333333333333333333333333333333333333333e-3 + x * (1.3888888888888888888888888
8888888888888888888888889e-3 + x * (1.984126984126984126984126984126984126984126
98412698e-4 + x * (2.4801587301587301587301587301587301587301587301587e-5 + x * 
(2.75573192239858906525573192239858906525573192239859e-6 + x * 2.755731922398589
0652557319223985890652557319223986e-7)))))))))
[-2.31142719641187619441242534182684745832539555102969e-8;2.73126607556424744202
06278018039434042553645532164e-8]
\end{Verbatim}
\end{minipage}\end{center}
\noindent Example 2: 
\begin{center}\begin{minipage}{15cm}\begin{Verbatim}[frame=single]
> TL=taylorform(sin(x)/x, 10, 0, [-1,1], relative);
> p=TL[0];
> Delta=TL[2];
> errors=TL[1];
> p; Delta;
1 + x^2 * (-0.16666666666666666666666666666666666666666666666667 + x^2 * (8.3333
333333333333333333333333333333333333333333333e-3 + x^2 * (-1.9841269841269841269
8412698412698412698412698412698e-4 + x^2 * (2.7557319223985890652557319223985890
6525573192239859e-6 + x^2 * (-2.505210838544171877505210838544171877505210838544
19e-8)))))
[-1.6135797443886066084999806203254010793747502812764e-10;1.61357974438860660849
99806203254010793747502812764e-10]
\end{Verbatim}
\end{minipage}\end{center}
