\subsection{taylorform}
\label{labtaylorform}
\noindent Name: \textbf{taylorform}\\
computes a rigorous polynomial approximation (polynomial, interval error bound) for a function, based on Taylor expansions.\\
\noindent Usage: 
\begin{center}
\textbf{taylorform}(\emph{f}, \emph{n}, \emph{$x_0$}, \emph{I}, \emph{errorType}) : (\textsf{function}, \textsf{integer}, \textsf{constant}, \textsf{range}, \textsf{absolute$|$relative}) $\rightarrow$ \textsf{list}\\
\textbf{taylorform}(\emph{f}, \emph{n}, \emph{$x_0$}, \emph{I}, \emph{errorType}) : (\textsf{function}, \textsf{integer}, \textsf{range}, \textsf{range}, \textsf{absolute$|$relative}) $\rightarrow$ \textsf{list}\\
\textbf{taylorform}(\emph{f}, \emph{n}, \emph{$x_0$}, \emph{errorType}) : (\textsf{function}, \textsf{integer}, \textsf{constant}, \textsf{absolute$|$relative}) $\rightarrow$ \textsf{list}\\
\textbf{taylorform}(\emph{f}, \emph{n}, \emph{$x_0$}, \emph{errorType}) : (\textsf{function}, \textsf{integer}, \textsf{range}, \textsf{absolute$|$relative}) $\rightarrow$ \textsf{list}\\
\end{center}
Parameters: 
\begin{itemize}
\item \emph{f} is the function to be approximated.
\item \emph{n} is the order of the Taylor form, meaning $\emph{n}-1$ is the degree of the polynomial that must approximate \emph{f}.
\item \emph{$x_0$} is the point (it can be a real number or an interval) where the Taylor exansion of the function is to be considered.
\item \emph{I} is the interval over which the function is to be approximated. If this parameter is omitted, the behavior is changed (see detailed description below).
\item \emph{errorType} (optional) is the type of error to be considered. See the detailed description below. Default is \textbf{absolute}.
\end{itemize}
\noindent Description: \begin{itemize}

\item \textbf{taylorform} computes an approximation polynomial and an interval error bound
   for function $f$. More precisely, it returns a list $L = \left[p, \textrm{coeffErrors},\Delta \right]$
   where:
   \begin{itemize}
   \item $p$ is an approximation polynomial of degree $n-1$ which is roughly speaking
   a numerical Taylor expansion of $f$ at the point $x_0$.
   \item coeffsErrors is a list of $n$ intervals. Each interval coeffsErrors[$i$]
   contains an enclosure of all the errors accumulated when computing the $i$-th
   coefficient of $p$.
   \item $\Delta$ is an interval that provides a bound for the approximation error
   between $p$ and $f$. Its significance depends on the \emph{errorType} considered.
   \end{itemize}

\item The polynomial $p$ and the bound  $\Delta$ are obtained using Taylor Models
   principles.

\item Please note that $x_0$ can be an interval. In general, it is meant to be a
   small interval approximating a non representable value. If $x_0$ is given as a
   constant expression, it is first numerically evaluated (leading to a faithful
   rounding $\tilde{x_0}$ at precision \textbf{prec}), and it is then replaced by the (exactly
   representable) point-interval $[\tilde{x_0},\,\tilde{x_0}]$. In particular, it is not
   the same to call \textbf{taylorform} with $x_0 = \textbf{pi}$ and with $x_0 = [\textbf{pi}]$, for instance.
   In general, if the point around which one desires to compute the polynomial
   is not exactly representable, one should preferably use a small interval
   for~$x_0$.

\item More formally, the mathematical property ensured by the algorithm may be
   stated as follows. For all $\csi_0$ in $x_0$, there exist (small) values
   $\varepsilon_i \in \textrm{coeffsErrors}[i]$ such that:
   \\
   If \emph{errorType} is \textbf{absolute}, $\forall x \in I, \exists \delta \in \Delta,\,$
   $f(x)-p(x-\csi_0) = \sum\limits_{i=0}^{n-1} \varepsilon_i\, (x-\csi_0)^i + \delta$.
   \\
   If \emph{errorType} is \textbf{relative}, $\forall x \in I, \exists \delta \in \Delta,\,$
   $f(x)-p(x-\csi_0) = \sum\limits_{i=0}^{n-1} \varepsilon_i\, (x-\csi_0)^i + \delta\,(x-\csi_0)^n$.

\item It is also possible to use a large interval for $x_0$, though it is not
   obvious to give an intuitive sense to the result of \textbf{taylorform} in that case.
   A particular case that might be interesting is when $x_0 = I$ in relative mode.
   In that case, denoting by $p_i$ the coefficient of $p$ of order $i$, the interval
   $p_i + \textrm{coeffsError}[i]$ gives an enclosure of $f^{(i)}(I)/i!$.
   However, the command \textbf{autodiff} is more convenient for computing such
   enclosures.

\item When the interval $I$ is not given, the approximated Taylor polynomial is
   computed but no remainder is produced. In that case the returned list
   is $L = \left[p, \textrm{coeffErrors}\right]$.

\item The relative case is especially useful when functions with removable
   singularities are considered. In such a case, this routine is able to compute
   a finite remainder bound, provided that the expansion point given is the
   problematic removable singularity point.

\item The algorithm does not guarantee that by increasing the degree of the
   approximation, the remainder bound will become smaller. Moreover, it may 
   even become larger due to the dependecy phenomenon present with interval
   arithmetic. In order to reduce this phenomenon, a possible solution is to
   split the definition domain $I$ into several smaller intervals. 

\item The command \textbf{taylor} also computes a Taylor polynomial of a function. However
   it does not provide a bound on the remainder. Besides, \textbf{taylor} is a somehow
   symbolic command: each coefficient of the Taylor polynomial is computed
   exactly and returned as an expression tree exactly equal to theoretical
   value. It is henceforth much more inefficient than \textbf{taylorform} and \textbf{taylorform}
   should be prefered if only numercial (yet safe) computations are required.
   The same difference exists between commands \textbf{diff} and \textbf{autodiff}.
\end{itemize}
\noindent Example 1: 
\begin{center}\begin{minipage}{15cm}\begin{Verbatim}[frame=single]
> TL=taylorform(sin(x)/x, 10, 0, [-1,1], relative);
> p=TL[0];
> Delta=TL[2];
> errors=TL[1];
> for epsi in errors do epsi;
[0;0]
[0;0]
[0;5.3455294201843912922810729343029637576303937602101e-51]
[0;0]
[-3.3409558876152445576756705839393523485189961001313e-52;3.34095588761524455767
56705839393523485189961001313e-52]
[0;0]
[-1.04404871487976392427364705748104760891218628129103e-53;1.0440487148797639242
7364705748104760891218628129103e-53]
[0;0]
[-1.63132611699963113167757352731413688892529106451724e-55;1.6313261169996311316
7757352731413688892529106451724e-55]
[0;0]
[-1.91171029335894273243465647732125416670932546623114e-57;1.9117102933589427324
3465647732125416670932546623114e-57]
> p; Delta;
1 + x^2 * (-0.16666666666666666666666666666666666666666666666667 + x^2 * (8.3333
333333333333333333333333333333333333333333333e-3 + x^2 * (-1.9841269841269841269
8412698412698412698412698412698e-4 + x^2 * (2.7557319223985890652557319223985890
6525573192239859e-6 + x^2 * (-2.505210838544171877505210838544171877505210838544
19e-8)))))
[-1.6135797443886066084999806203254010793747502812764e-10;1.61357974438860660849
99806203254010793747502812764e-10]
\end{Verbatim}
\end{minipage}\end{center}
\noindent Example 2: 
\begin{center}\begin{minipage}{15cm}\begin{Verbatim}[frame=single]
> TL=taylorform(exp(x), 10, 0, [-1,1], absolute);
> p=TL[0];
> Delta=TL[2];
> p; Delta;
1 + x * (1 + x * (0.5 + x * (0.1666666666666666666666666666666666666666666666666
7 + x * (4.1666666666666666666666666666666666666666666666667e-2 + x * (8.3333333
333333333333333333333333333333333333333333e-3 + x * (1.3888888888888888888888888
8888888888888888888888889e-3 + x * (1.984126984126984126984126984126984126984126
98412698e-4 + x * (2.4801587301587301587301587301587301587301587301587e-5 + x * 
(2.75573192239858906525573192239858906525573192239859e-6 + x * 2.755731922398589
0652557319223985890652557319223986e-7)))))))))
[-2.31142719641187619441242534182684745832539555102969e-8;2.73126607556424744202
06278018039434042553645532164e-8]
\end{Verbatim}
\end{minipage}\end{center}
\noindent Example 3: 
\begin{center}\begin{minipage}{15cm}\begin{Verbatim}[frame=single]
> TL1 = taylorform(exp(x), 10, log2(10), [-1,1], absolute);
> TL2 = taylorform(exp(x), 10, [log2(10)], [-1,1], absolute);
> TL1==TL2;
false
\end{Verbatim}
\end{minipage}\end{center}
\noindent Example 4: 
\begin{center}\begin{minipage}{15cm}\begin{Verbatim}[frame=single]
> TL1 = taylorform(exp(x), 3, 0, [0,1], relative);
> TL2 = taylorform(exp(x), 3, 0, relative);
> TL1[0]==TL2[0];
true
> TL1[1]==TL2[1];
true
> length(TL1);
3
> length(TL2);
2
\end{Verbatim}
\end{minipage}\end{center}
\noindent Example 5: 
\begin{center}\begin{minipage}{15cm}\begin{Verbatim}[frame=single]
> f = exp(cos(x)); x0 = 0;
> TL = taylorform(f, 3, x0);
> T1 = TL[0];
> T2 = taylor(f, 3, x0);
> print(coeff(T1, 2));
-1.35914091422952261768014373567633124887862354684999
> print(coeff(T2, 2));
-(0.5 * exp(1))
\end{Verbatim}
\end{minipage}\end{center}
See also: \textbf{diff} (\ref{labdiff}), \textbf{autodiff} (\ref{labautodiff}), \textbf{taylor} (\ref{labtaylor}), \textbf{remez} (\ref{labremez})
