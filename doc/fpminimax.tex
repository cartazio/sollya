\subsection{fpminimax}
\label{labfpminimax}
\noindent Name: \textbf{fpminimax}\\
computes a good polynomial approximation with fixed-point or floating-point coefficients\\
\noindent Usage: 
\begin{center}
\textbf{fpminimax}(\emph{f}, \emph{n}, \emph{formats}, \emph{range}, \emph{indic1}, \emph{indic2}, \emph{indic3}, \emph{P}) : (\textsf{function}, \textsf{integer}, \textsf{list}, \textsf{range}, \textsf{absolute$|$relative} $|$ \textsf{fixed$|$floating} $|$ \textsf{function}, \textsf{absolute$|$relative} $|$ \textsf{fixed$|$floating} $|$ \textsf{function}, \textsf{absolute$|$relative} $|$ \textsf{fixed$|$floating} $|$ \textsf{function}, \textsf{function}) $\rightarrow$ \textsf{function}\\
\textbf{fpminimax}(\emph{f}, \emph{monomials}, \emph{formats}, \emph{range}, \emph{indic1}, \emph{indic2}, \emph{indic3}, \emph{P}) : (\textsf{function}, \textsf{list}, \textsf{list}, \textsf{range},  \textsf{absolute$|$relative} $|$ \textsf{fixed$|$floating} $|$ \textsf{function}, \textsf{absolute$|$relative} $|$ \textsf{fixed$|$floating} $|$ \textsf{function}, \textsf{absolute$|$relative} $|$ \textsf{fixed$|$floating} $|$ \textsf{function}, \textsf{function}) $\rightarrow$ \textsf{function}\\
\textbf{fpminimax}(\emph{f}, \emph{n}, \emph{formats}, \emph{L}, \emph{indic1}, \emph{indic2}, \emph{indic3}, \emph{P}) : (\textsf{function}, \textsf{integer}, \textsf{list}, \textsf{list},  \textsf{absolute$|$relative} $|$ \textsf{fixed$|$floating} $|$ \textsf{function}, \textsf{absolute$|$relative} $|$ \textsf{fixed$|$floating} $|$ \textsf{function}, \textsf{absolute$|$relative} $|$ \textsf{fixed$|$floating} $|$ \textsf{function}, \textsf{function}) $\rightarrow$ \textsf{function}\\
\textbf{fpminimax}(\emph{f}, \emph{monomials}, \emph{formats}, \emph{L}, \emph{indic1}, \emph{indic2}, \emph{indic3}, \emph{P}) : (\textsf{function}, \textsf{list}, \textsf{list}, \textsf{list},  \textsf{absolute$|$relative} $|$ \textsf{fixed$|$floating} $|$ \textsf{function}, \textsf{absolute$|$relative} $|$ \textsf{fixed$|$floating} $|$ \textsf{function}, \textsf{absolute$|$relative} $|$ \textsf{fixed$|$floating} $|$ \textsf{function}, \textsf{function}) $\rightarrow$ \textsf{function}\\
\end{center}
Parameters: 
\begin{itemize}
\item \emph{f} is the function to be approximated
\item \emph{n} is the degree of the polynomial that must approximate \emph{f}
\item \emph{monomials} is the list of monomials that must be used to represent the polynomial that approximates~\emph{f}
\item \emph{formats} is a list indicating the formats that the coefficients of the polynomial must have
\item \emph{range} is the interval where the function must be approximated
\item \emph{L} is a list of interpolation points used by the method
\item \emph{indic1} (optional) is one of the optional indication parameters. See the detailed description below.
\item \emph{indic2} (optional) is one of the optional indication parameters. See the detailed description below.
\item \emph{indic3} (optional) is one of the optional indication parameters. See the detailed description below.
\item \emph{P} (optional) is the minimax polynomial to be considered for solving the problem.
\end{itemize}
\noindent Description: \begin{itemize}

\item \\textbf{fpminimax} uses a heuristic (but practically efficient) method to find a good\n   polynomial approximation of a function \\emph{f} on an interval \\emph{range}. It \n   implements the method published in the article:\\\\\n   Efficient polynomial $L^\\infty$-approximations\\\\ \n   Nicolas Brisebarre and Sylvain Chevillard\\\\\n   Proceedings of the 18th IEEE Symposium on Computer Arithmetic (ARITH 18)\\\\\n   pp. 169-176\n
\item The basic usage of this command is \\textbf{fpminimax}(\\emph{f}, \\emph{n}, \\emph{formats}, \\emph{range}).\n   It computes a polynomial approximation of $f$ with degree at most $n$\n   on the interval \\emph{range}. \\emph{formats} is a list of integers or format types \n   (such as \\textbf{double}, \\textbf{doubledouble}, etc.). The polynomial returned by the\n   command has its coefficients that fit the formats indications. For \n   instance, if formats[0] is 35, the coefficient of degree 0 of the \n   polynomial will fit a floating-point format of 35 bits. If formats[1] \n   is D, the coefficient of degree 1 will be representable by a floating-point\n   number with a precision of 53 bits (which is not necessarily an IEEE 754 double\n   precision number. See the remark below), etc.\n
\item The second argument may be either an integer or a list of integers\n   interpreted as the list of desired monomials. For instance, the list\n   $[|0,\\,2,\\,4,\\,6|]$ indicates that the polynomial must be even and of\n   degree at most 6. Giving an integer $n$ as second argument is equivalent\n   as giving $[|0,\\,\\dots,\\,n|]$.\\\\\n   The list of formats is interpreted with respect to the list of monomials. For\n   instance, if the list of monomials is $[|0,\\,2,\\,4,\\,6|]$ and the list\n   of formats is $[|161,\\,107,\\,53,\\,24|]$, the coefficients of degree 0 is \n   searched as a floating-point number with precision 161, the coefficient of \n   degree 2 is searched as a number of precision 107, and so on.\n
\item The list of formats may contain either integers or format types (\\textbf{double},\n   \\textbf{doubledouble}, \\textbf{tripledouble} and \\textbf{doubleextended}). The list may be too large\n   or even infinite. Only the first indications will be considered. For \n   instance, for a degree $n$ polynomial, $\\mathrm{formats}[n+1]$ and above will\n   be discarded. This lets one use elliptical indications for the last\n   coefficients.\n
\item The floating-point coefficients considered by \\textbf{fpminimax} do not have an\n   exponent range. In particular, in the format list, \\textbf{double} or 53 does not\n   imply that the corresponding coefficient is an IEEE-754 double.\n
\item By default, the list of formats is interpreted as a list of floating-point\n   formats. This may be changed by passing \\textbf{fixed} as an optional argument (see\n   below). Let us take an example: \\textbf{fpminimax}(f, 2, [107, DD, 53], [0;1]).\n   Here the optional argument is missing (we could have set it to \\textbf{floating}).\n   Thus, \\textbf{fpminimax} will search for a polynomial of degree 2 with a constant \n   coefficient that is a 107 bits floating-point number, etc.\\\\\n   Currently, \\textbf{doubledouble} is just a synonym for 107 and \\textbf{tripledouble} a \n   synonym for 161. This behavior may change in the future (taking into\n   account the fact that some double-doubles are not representable with\n   107 bits).\\\\\n   Second example: \\textbf{fpminimax}(f, 2, [25, 18, 30], [0;1], \\textbf{fixed}).\n   In this case, \\textbf{fpminimax} will search for a polynomial of degree 2 with a\n   constant coefficient of the form $m/2^{25}$ where $m$ is an\n   integer. In other words, it is a fixed-point number with 25 bits after\n   the point. Note that even with argument \\textbf{fixed}, the formats list is \n   allowed to contain \\textbf{double}, \\textbf{doubledouble} or \\textbf{tripledouble}. In this this\n   case, it is just a synonym for 53, 107 or 161. This is deprecated and may\n   change in the future.\n
\item The fourth argument may be a range or a list. Lists are for advanced users\n   that know what they are doing. The core of the  method is a kind of\n   approximated interpolation. The list given here is a list of points that\n   must be considered for the interpolation. It must contain at least as \n   many points as unknown coefficients. If you give a list, it is also \n   recommended that you provide the minimax polynomial as last argument.\n   If you give a range, the list of points will be automatically computed.\n
\item The fifth, sixth and seventh arguments are optional. By default, \\textbf{fpminimax}\n   will approximate $f$ while optimizing the relative error, and interpreting\n   the list of formats as a list of floating-point formats.\\\\\n   This default behavior may be changed with these optional arguments. You\n   may provide zero, one, two or three of the arguments in any order.\n   This lets the user indicate only the non-default arguments.\\\\\n   The three possible arguments are: \\begin{itemize}\n   \\item \\textbf{relative} or \\textbf{absolute}: the error to be optimized;\n   \\item \\textbf{floating} or \\textbf{fixed}: formats of the coefficients;\n   \\item a constrained part $q$.\n   \\end{itemize}\n   The constrained part lets the user assign in advance some of the\n   coefficients. For instance, for approximating $\\exp(x)$, it may\n   be interesting to search for a polynomial $p$ of the form\n                   $$p = 1 + x + \\frac{x^2}{2} + a_3 x^3 + a_4 x^4.$$\n   Thus, there is a constrained part $q = 1 + x + x^2/2$ and the unknown\n   polynomial should be considered in the monomial basis $[|3, 4|]$.\n   Calling \\textbf{fpminimax} with monomial basis $[|3,\\,4|]$ and constrained\n   part $q$, will return a polynomial with the right form.\n
\item The last argument is for advanced users. It is the minimax polynomial that\n   approximates the function $f$ in the monomial basis. If it is not given\n   this polynomial will be automatically computed by \\textbf{fpminimax}.\n   \\\\\n   This minimax polynomial is used to compute the list of interpolation\n   points required by the method. In general, you do not have to provide this\n   argument. But if you want to obtain several polynomials of the same degree\n   that approximate the same function on the same range, just changing the\n   formats, you should probably consider computing only once the minimax\n   polynomial and the list of points instead of letting \\textbf{fpminimax} recompute\n   them each time.\n   \\\\\n   Note that in the case when a constrained part is given, the minimax \n   polynomial must take that into account. For instance, in the previous\n   example, the minimax would be obtained by the following command:\n          \\begin{center}\\verb~P = remez(1-(1+x+x^2/2)/exp(x), [|3,4|], range, 1/exp(x));~\\end{center}\n   Note that the constrained part is not to be added to $P$.\n
\item Note that \\textbf{fpminimax} internally computes a minimax polynomial (using\n   the same algorithm as \\textbf{remez} command). Thus \\textbf{fpminimax} may encounter\n   the same problems as \\textbf{remez}. In particular, it may be very slow \n   when Haar condition is not fulfilled. Another consequence is that\n   currently \\textbf{fpminimax} has to be run with a sufficiently high working precision.\n\end{itemize}
\noindent Example 1: 
\begin{center}\begin{minipage}{15cm}\begin{Verbatim}[frame=single]
\end{Verbatim}
\end{minipage}\end{center}
\noindent Example 2: 
\begin{center}\begin{minipage}{15cm}\begin{Verbatim}[frame=single]
\end{Verbatim}
\end{minipage}\end{center}
\noindent Example 3: 
\begin{center}\begin{minipage}{15cm}\begin{Verbatim}[frame=single]
\end{Verbatim}
\end{minipage}\end{center}
\noindent Example 4: 
\begin{center}\begin{minipage}{15cm}\begin{Verbatim}[frame=single]
\end{Verbatim}
\end{minipage}\end{center}
See also: \textbf{remez} (\ref{labremez}), \textbf{dirtyfindzeros} (\ref{labdirtyfindzeros}), \textbf{absolute} (\ref{lababsolute}), \textbf{relative} (\ref{labrelative}), \textbf{fixed} (\ref{labfixed}), \textbf{floating} (\ref{labfloating}), \textbf{default} (\ref{labdefault})
