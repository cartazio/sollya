\subsection{/}
\label{labdivide}
\noindent Name: \textbf{/}\\
\phantom{aaa}division function\\[0.2cm]
\noindent Library names:\\
\verb|   sollya_obj_t sollya_lib_div(sollya_obj_t, sollya_obj_t)|\\
\verb|   sollya_obj_t sollya_lib_build_function_div(sollya_obj_t, sollya_obj_t)|\\
\verb|   #define SOLLYA_DIV(x,y) sollya_lib_build_function_div((x), (y))|\\[0.2cm]
\noindent Usage: 
\begin{center}
\emph{function1} \textbf{/} \emph{function2} : (\textsf{function}, \textsf{function}) $\rightarrow$ \textsf{function}\\
\emph{interval1} \textbf{/} \emph{interval2} : (\textsf{range}, \textsf{range}) $\rightarrow$ \textsf{range}\\
\emph{interval1} \textbf{/} \emph{constant} : (\textsf{range}, \textsf{constant}) $\rightarrow$ \textsf{range}\\
\emph{interval1} \textbf{/} \emph{constant} : (\textsf{constant}, \textsf{range}) $\rightarrow$ \textsf{range}\\
\end{center}
Parameters: 
\begin{itemize}
\item \emph{function1} and \emph{function2} represent functions
\item \emph{interval1} and \emph{interval2} represent intervals (ranges)
\item \emph{constant} represents a constant or constant expression
\end{itemize}
\noindent Description: \begin{itemize}

\item \textbf{/} represents the division (function) on reals. 
   The expression \emph{function1} \textbf{/} \emph{function2} stands for
   the function composed of the division function and the two
   functions \emph{function1} and \emph{function2}, where \emph{function1} is
   the numerator and \emph{function2} the denominator.

\item \textbf{/} can be used for interval arithmetic on intervals
   (ranges). \textbf{/} will evaluate to an interval that safely
   encompasses all images of the division function with arguments
   varying in the given intervals. If the intervals given contain points
   where the division function is not defined, infinities and NaNs will be
   produced in the output interval.  Any combination of intervals with
   intervals or constants (resp. constant expressions) is
   supported. However, it is not possible to represent families of
   functions using an interval as one argument and a function (varying in
   the free variable) as the other one.
\end{itemize}
\noindent Example 1: 
\begin{center}\begin{minipage}{15cm}\begin{Verbatim}[frame=single]
> 5 / 2;
2.5
\end{Verbatim}
\end{minipage}\end{center}
\noindent Example 2: 
\begin{center}\begin{minipage}{15cm}\begin{Verbatim}[frame=single]
> x / 2;
x * 0.5
\end{Verbatim}
\end{minipage}\end{center}
\noindent Example 3: 
\begin{center}\begin{minipage}{15cm}\begin{Verbatim}[frame=single]
> x / x;
1
\end{Verbatim}
\end{minipage}\end{center}
\noindent Example 4: 
\begin{center}\begin{minipage}{15cm}\begin{Verbatim}[frame=single]
> 3 / 0;
NaN
\end{Verbatim}
\end{minipage}\end{center}
\noindent Example 5: 
\begin{center}\begin{minipage}{15cm}\begin{Verbatim}[frame=single]
> diff(sin(x) / exp(x));
(exp(x) * cos(x) - sin(x) * exp(x)) / exp(x)^2
\end{Verbatim}
\end{minipage}\end{center}
\noindent Example 6: 
\begin{center}\begin{minipage}{15cm}\begin{Verbatim}[frame=single]
> [1;2] / [3;4];
[0.25;0.66666666666666666666666666666666666666666666666668]
> [1;2] / 17;
[5.8823529411764705882352941176470588235294117647059e-2;0.1176470588235294117647
0588235294117647058823529412]
> -13 / [4;17];
[-3.25;-0.76470588235294117647058823529411764705882352941175]
\end{Verbatim}
\end{minipage}\end{center}
See also: \textbf{$+$} (\ref{labplus}), \textbf{$-$} (\ref{labminus}), \textbf{$*$} (\ref{labmult}), \textbf{$\mathbf{\hat{~}}$} (\ref{labpower})
