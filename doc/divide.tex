\subsection{\/}
\label{labdivide}
\noindent Name: \textbf{/}\\
division function\\
\noindent Usage: 
\begin{center}
\emph{function1} \textbf{/} \emph{function2} : (\textsf{function}, \textsf{function}) $\rightarrow$ \textsf{function}\\
\emph{interval1} \textbf{/} \emph{interval2} : (\textsf{range}, \textsf{range}) $\rightarrow$ \textsf{range}\\
\emph{interval1} \textbf{/} \emph{constant} : (\textsf{range}, \textsf{constant}) $\rightarrow$ \textsf{range}\\
\emph{interval1} \textbf{/} \emph{constant} : (\textsf{constant}, \textsf{range}) $\rightarrow$ \textsf{range}\\
\end{center}
Parameters: 
\begin{itemize}
\item \emph{function1} and \emph{function2} represent functions
\item \emph{interval1} and \emph{interval2} represent intervals (ranges)
\item \emph{constant} represents a constant or constant expression
\end{itemize}
\noindent Description: \begin{itemize}

\item \\textbf{/} represents the division (function) on reals. \n   The expression \\emph{function1} \\textbf{/} \\emph{function2} stands for\n   the function composed of the division function and the two\n   functions \\emph{function1} and \\emph{function2}, where \\emph{function1} is\n   the numerator and \\emph{function2} the denominator.\n
\item \\textbf{/} can be used for interval arithmetic on intervals\n   (ranges). \\textbf{/} will evaluate to an interval that safely\n   encompasses all images of the division function with arguments\n   varying in the given intervals. If the intervals given contain points\n   where the division function is not defined, infinities and NaNs will be\n   produced in the output interval.  Any combination of intervals with\n   intervals or constants (resp. constant expressions) is\n   supported. However, it is not possible to represent families of\n   functions using an interval as one argument and a function (varying in\n   the free variable) as the other one.\n\end{itemize}
\noindent Example 1: 
\begin{center}\begin{minipage}{15cm}\begin{Verbatim}[frame=single]
\end{Verbatim}
\end{minipage}\end{center}
\noindent Example 2: 
\begin{center}\begin{minipage}{15cm}\begin{Verbatim}[frame=single]
\end{Verbatim}
\end{minipage}\end{center}
\noindent Example 3: 
\begin{center}\begin{minipage}{15cm}\begin{Verbatim}[frame=single]
\end{Verbatim}
\end{minipage}\end{center}
\noindent Example 4: 
\begin{center}\begin{minipage}{15cm}\begin{Verbatim}[frame=single]
\end{Verbatim}
\end{minipage}\end{center}
\noindent Example 5: 
\begin{center}\begin{minipage}{15cm}\begin{Verbatim}[frame=single]
\end{Verbatim}
\end{minipage}\end{center}
\noindent Example 6: 
\begin{center}\begin{minipage}{15cm}\begin{Verbatim}[frame=single]
\end{Verbatim}
\end{minipage}\end{center}
See also: \textbf{$+$} (\ref{labplus}), \textbf{$-$} (\ref{labminus}), \textbf{$*$} (\ref{labmult}), \textbf{$\mathbf{\hat{~}}$} (\ref{labpower})
