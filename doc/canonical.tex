\subsection{canonical}
\label{labcanonical}
\noindent Name: \textbf{canonical}\\
brings all polynomial subexpressions of an expression to canonical form or activates, deactivates or checks canonical form printing\\
\noindent Usage: 
\begin{center}
\textbf{canonical}(\emph{function}) : \textsf{function} $\rightarrow$ \textsf{function}\\
\textbf{canonical} = \emph{activation value} : \textsf{on$|$off} $\rightarrow$ \textsf{void}\\
\textbf{canonical} = \emph{activation value} ! : \textsf{on$|$off} $\rightarrow$ \textsf{void}\\
\end{center}
Parameters: 
\begin{itemize}
\item \emph{function} represents the expression to be rewritten in canonical form
\item \emph{activation value} represents \textbf{on} or \textbf{off}, i.e. activation or deactivation
\end{itemize}
\noindent Description: \begin{itemize}

\item The command \\textbf{canonical} rewrites the expression representing the function\n   \\emph{function} in a way such that all polynomial subexpressions (or the\n   whole expression itself, if it is a polynomial) are written in\n   canonical form, i.e. as a sum of monomials in the canonical base. The\n   canonical base is the base of the integer powers of the global free\n   variable. The command \\textbf{canonical} does not endanger the safety of\n   computations even in \\sollya's floating-point environment: the\n   function returned is mathematically equal to the function \\emph{function}.\n
\item An assignment \\textbf{canonical} = \\emph{activation value}, where \\emph{activation value}\n   is one of \\textbf{on} or \\textbf{off}, activates respectively deactivates the\n   automatic printing of polynomial expressions in canonical form,\n   i.e. as a sum of monomials in the canonical base. If automatic\n   printing in canonical form is deactivated, automatic printing yields to\n   displaying polynomial subexpressions in Horner form.\n    \n   If the assignment \\textbf{canonical} = \\emph{activation value} is followed by an\n   exclamation mark, no message indicating the new state is\n   displayed. Otherwise the user is informed of the new state of the\n   global mode by an indication.\n\end{itemize}
\noindent Example 1: 
\begin{center}\begin{minipage}{15cm}\begin{Verbatim}[frame=single]
\end{Verbatim}
\end{minipage}\end{center}
\noindent Example 2: 
\begin{center}\begin{minipage}{15cm}\begin{Verbatim}[frame=single]
\end{Verbatim}
\end{minipage}\end{center}
\noindent Example 3: 
\begin{center}\begin{minipage}{15cm}\begin{Verbatim}[frame=single]
\end{Verbatim}
\end{minipage}\end{center}
See also: \textbf{horner} (\ref{labhorner}), \textbf{print} (\ref{labprint})
