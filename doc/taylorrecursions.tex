\subsection{taylorrecursions}
\label{labtaylorrecursions}
\noindent Name: \textbf{taylorrecursions}\\
controls the number of recursion steps when applying Taylor's rule.\\
\noindent Usage: 
\begin{center}
\textbf{taylorrecursions} = \emph{n} : \textsf{integer} $\rightarrow$ \textsf{void}\\
\textbf{taylorrecursions} = \emph{n} ! : \textsf{integer} $\rightarrow$ \textsf{void}\\
\textbf{taylorrecursions} : \textsf{integer}\\
\end{center}
Parameters: 
\begin{itemize}
\item \emph{n} represents the number of recursions
\end{itemize}
\noindent Description: \begin{itemize}

\item \\textbf{taylorrecursions} is a global variable. Its value represents the number of steps\n   of recursion that are used when applying Taylor's rule. This rule is applied\n   by the interval evaluator present in the core of \\sollya (and particularly\n   visible in commands like \\textbf{infnorm}).\n
\item To improve the quality of an interval evaluation of a function $f$, in \n   particular when there are problems of decorrelation), the evaluator of \\sollya\n   uses Taylor's rule:  $f([a,b]) \\subseteq f(m) + [a-m,\\,b-m] \\cdot f'([a,\\,b])$ where $m=\\frac{a+b}{2}$.\n   This rule can be applied recursively.\n   The number of step in this recursion process is controlled by \\textbf{taylorrecursions}.\n
\item Setting \\textbf{taylorrecursions} to 0 makes \\sollya use this rule only once;\n   setting it to 1 makes \\sollya use the rule twice, and so on.\n   In particular: the rule is always applied at least once.\n\end{itemize}
\noindent Example 1: 
\begin{center}\begin{minipage}{15cm}\begin{Verbatim}[frame=single]
\end{Verbatim}
\end{minipage}\end{center}
