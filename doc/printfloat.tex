\subsection{printfloat}
\label{labprintfloat}
\noindent Name: \textbf{printfloat}\\
prints a constant value as a hexadecimal single precision number\\

\noindent Usage: 
\begin{center}
\textbf{printfloat}(\emph{constant}) : \textsf{constant} $\rightarrow$ \textsf{void}\\
\end{center}
Parameters: 
\begin{itemize}
\item \emph{constant} represents a constant
\end{itemize}
\noindent Description: \begin{itemize}

\item Prints a constant value as a hexadecimal number on 8 hexadecimal
   digits. The hexadecimal number represents the integer equivalent to
   the 32 bit memory representation of the constant considered as a
   single precision number.
    
   If the constant value does not hold on a single precision number, it
   is first rounded to the nearest single precision number before
   displayed. A warning is displayed in this case.
\end{itemize}
\noindent Example 1: 
\begin{center}\begin{minipage}{15cm}\begin{Verbatim}[frame=single]
> printfloat(3);
0x40400000
\end{Verbatim}
\end{minipage}\end{center}
\noindent Example 2: 
\begin{center}\begin{minipage}{15cm}\begin{Verbatim}[frame=single]
> prec=100!;
> verbosity = 1!;
> printfloat(exp(5));
Warning: the given expression is not a constant but an expression to evaluate.
Warning: rounding occurred before printing a value as a simple.
0x431469c5
\end{Verbatim}
\end{minipage}\end{center}
See also: \textbf{printhexa} (\ref{labprinthexa})
