\subsection{suppressmessage}
\label{labsuppressmessage}
\noindent Name: \textbf{suppressmessage}\\
\phantom{aaa}suppresses the displaying of messages with a certain number\\[0.2cm]
\noindent Library name:\\
\verb|   void sollya_lib_suppressmessage(sollya_obj_t, ...);|\\[0.2cm]
\noindent Usage: 
\begin{center}
\textbf{suppressmessage}(\emph{msg num 1}, ..., \emph{msg num n})   : (\textsf{integer}, ..., \textsf{integer}) $\rightarrow$ \textsf{void}\\
\textbf{suppressmessage}(\emph{msg list})   : \textsf{list} $\rightarrow$ \textsf{void}\\
\end{center}
Parameters: 
\begin{itemize}
\item \emph{msg num 1} thru \emph{msg num n} represent the numbers of $n$ messages to be suppressed
\item \emph{msg list} represents a list with numbers of messages to be suppressed
\end{itemize}
\noindent Description: \begin{itemize}

\item The \textbf{suppressmessage} command allows particular warning and information
   messages to be suppressed from message output, while maintaining
   global verbosity levels (see \textbf{verbosity}) high. Every \sollya warning
   or information message (that is not fatal to the tool's execution) has
   a message number. When these message numbers \emph{msg num 1} thru \emph{msg num n} 
   are given to \textbf{suppressmessage}, the corresponding message are no longer
   displayed.  The \textbf{unsuppressmessage} command reverts this suppression
   from output for a particular message.

\item Instead of giving \textbf{suppressmessage} several message numbers \emph{msg num 1} thru
   \emph{msg num n} or calling \textbf{suppressmessage} several times, it is possible to give
   a whole list \emph{msg list} of message numbers to \textbf{suppressmessage}.

\item The user should be aware that \textbf{suppressmessage} presents sticky behavior for
   the warning and information messages suppressed from output. This
   means that even if subsequent calls to \textbf{suppressmessage} occur, a message
   suppressed by a call to \textbf{suppressmessage} stays suppressed until it is
   unsuppressed using \textbf{unsuppressmessage} or the tool is restarted. This
   behavior distinguishes message suppression from other global states of
   the \sollya tool. The user may use \textbf{getsuppressedmessages} to obtain a
   list of currently suppressed messages.

\item When \textbf{suppressmessage} is used on message numbers that do not exist in the
   current version of the tool, a warning is displayed. The call has no
   other effect though.
\end{itemize}
\noindent Example 1: 
\begin{center}\begin{minipage}{15cm}\begin{Verbatim}[frame=single]
> verbosity = 1;
The verbosity level has been set to 1.
> 0.1;
Warning: Rounding occurred when converting the constant "0.1" to floating-point 
with 165 bits.
If safe computation is needed, try to increase the precision.
0.1
> suppressmessage(174);
> 0.1;
0.1
> suppressmessage(407);
> 0.1;
0.1
> verbosity = 12;
The verbosity level has been set to 12.
> showmessagenumbers = on;
Displaying of message numbers has been activated.
> diff(exp(x * 0.1));
Information (196): formally differentiating a function.
Information (197): differentiating the expression 'exp(x * 0.1)'
Information (207): no Horner simplification will be performed because the given 
tree is already in Horner form.
exp(x * 0.1) * 0.1
> suppressmessage(207, 196);
> diff(exp(x * 0.1));
Information (197): differentiating the expression 'exp(x * 0.1)'
exp(x * 0.1) * 0.1
> unsuppressmessage(174);
> 0.1;
Warning (174): Rounding occurred when converting the constant "0.1" to floating-
point with 165 bits.
If safe computation is needed, try to increase the precision.
0.1
\end{Verbatim}
\end{minipage}\end{center}
\noindent Example 2: 
\begin{center}\begin{minipage}{15cm}\begin{Verbatim}[frame=single]
> verbosity = 12;
The verbosity level has been set to 12.
> showmessagenumbers = on; 
Displaying of message numbers has been activated.
> diff(exp(x * 0.1));
Warning (174): Rounding occurred when converting the constant "0.1" to floating-
point with 165 bits.
If safe computation is needed, try to increase the precision.
Information (196): formally differentiating a function.
Information (197): differentiating the expression 'exp(x * 0.1)'
Information (207): no Horner simplification will be performed because the given 
tree is already in Horner form.
exp(x * 0.1) * 0.1
> suppressmessage([| 174, 207, 196 |]);
> diff(exp(x * 0.1));
Information (197): differentiating the expression 'exp(x * 0.1)'
exp(x * 0.1) * 0.1
\end{Verbatim}
\end{minipage}\end{center}
See also: \textbf{getsuppressedmessages} (\ref{labgetsuppressedmessages}), \textbf{suppressmessage} (\ref{labsuppressmessage}), \textbf{unsuppressmessage} (\ref{labunsuppressmessage}), \textbf{verbosity} (\ref{labverbosity}), \textbf{roundingwarnings} (\ref{labroundingwarnings})
