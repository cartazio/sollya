\subsection{ integral }
\noindent Name: \textbf{integral}\\
computes an interval bounding the integral of a function on an interval.\\

\noindent Usage: 
\begin{center}
\textbf{integral}(\emph{f},\emph{I}) : (\textsf{function}, \textsf{range}) $\rightarrow$ \textsf{range}\\
\end{center}
Parameters: 
\begin{itemize}
\item \emph{f} is a function.
\item \emph{I} is an interval.
\end{itemize}
\noindent Description: \begin{itemize}

\item \textbf{integral}(\emph{f},\emph{I}) returns an interval $J$ such that the exact value of 
   the integral of \emph{f} on \emph{I} lies in $J$.

\item This command is safe but very unefficient. Use \textbf{dirtyintegral} if you just want
   an approximate value.

\item The result of this command depends on the global variable \textbf{diam}.
   The method used is the following: \emph{I} is cut into intervals of length not 
   greater then $\delta \cdot |I|$ where $\delta$ is the value
   of global variable \textbf{diam}.
   On each small interval \emph{J}, an evaluation of \emph{f} by interval is
   performed. The result is multiplied by the length of \emph{J}. Finally all values 
   are summed.
\end{itemize}
\noindent Example 1: 
\begin{center}\begin{minipage}{15cm}\begin{Verbatim}[frame=single]
> sin(10);
-0.544021110889369813404747661851377281683643012916219
> integral(cos(x),[0;10]);
[-0.547101979835796902240976371635259430756985992573329;-0.540940151300131838481
505408813733707440537411917285]
> diam=1e-5!;
> integral(cos(x),[0;10]);
[-0.544329156859554271018577802959369567752938763827772;-0.543713064012499695080
396442219274890104258031735553]
\end{Verbatim}
\end{minipage}\end{center}
See also: \textbf{points}, \textbf{dirtyintegral}
