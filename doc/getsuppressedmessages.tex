\subsection{getsuppressedmessages}
\label{labgetsuppressedmessages}
\noindent Name: \textbf{getsuppressedmessages}\\
\phantom{aaa}returns a list of numbers of messages that have been suppressed from message output\\[0.2cm]
\noindent Library name:\\
\verb|   sollya_obj_t sollya_lib_getsuppressedmessages();|\\[0.2cm]
\noindent Usage: 
\begin{center}
\textbf{getsuppressedmessages}()   : \textsf{void} $\rightarrow$ \textsf{list}\\
\end{center}
\noindent Description: \begin{itemize}

\item The \textbf{getsuppressedmessages} command allows the user to inspect the state of warning
   and information message suppression. When called, \textbf{getsuppressedmessages} returns a
   list of integers numbers that stand for the warning and information
   messages that have been suppressed.  If no message is suppressed,
   \textbf{getsuppressedmessages} returns an empty list.

\item Every \sollya warning or information message (that is not fatal to the
   tool's execution) has a message number. By default, these numbers are
   not displayed when a message is output. When message number displaying
   is activated using \textbf{showmessagenumbers}, the message numbers are
   displayed together with the message. This allows the user to match the
   numbers returned in a list by \textbf{getsuppressedmessages} with the actual warning and
   information messages.

\item The list of message numbers returned by \textbf{getsuppressedmessages} is suitable to be fed
   into the \textbf{unsuppressmessage} command. This way, the user may unsuppress
   all warning and information messages that have been suppressed.
\end{itemize}
\noindent Example 1: 
\begin{center}\begin{minipage}{15cm}\begin{Verbatim}[frame=single,commandchars=\\\|\~]
> verbosity = 1;
The verbosity level has been set to 1.
> 0.1;
Warning: Rounding occurred when converting the constant "0.1" to floating-point 
with 165 bits.
If safe computation is needed, try to increase the precision.
0.1
> suppressmessage(174);
> 0.1;
0.1
> suppressmessage(407);
> 0.1;
0.1
> getsuppressedmessages();
[|174, 407|]
> suppressmessage(207, 196);
> getsuppressedmessages();
[|174, 196, 207, 407|]
\end{Verbatim}
\end{minipage}\end{center}
\noindent Example 2: 
\begin{center}\begin{minipage}{15cm}\begin{Verbatim}[frame=single,commandchars=\\\|\~]
> suppressmessage(174, 209, 13, 24, 196);
> suppressmessage([| 16, 17 |]);
> suppressmessage(19);
> unsuppressmessage([| 13, 17 |]);
> getsuppressedmessages();
[|16, 19, 24, 174, 196, 209|]
> unsuppressmessage(getsuppressedmessages());
> getsuppressedmessages();
[| |]
\end{Verbatim}
\end{minipage}\end{center}
\noindent Example 3: 
\begin{center}\begin{minipage}{15cm}\begin{Verbatim}[frame=single,commandchars=\\\|\~]
> verbosity = 12;
The verbosity level has been set to 12.
> suppressmessage(174);
> exp(x * 0.1);
Information: no Horner simplification will be performed because the given tree i
s already in Horner form.
exp(x * 0.1)
> getsuppressedmessages();
[|174|]
> verbosity = 0;
The verbosity level has been set to 0.
> exp(x * 0.1);
exp(x * 0.1)
> getsuppressedmessages();
[|174|]
\end{Verbatim}
\end{minipage}\end{center}
See also: \textbf{getsuppressedmessages} (\ref{labgetsuppressedmessages}), \textbf{suppressmessage} (\ref{labsuppressmessage}), \textbf{unsuppressmessage} (\ref{labunsuppressmessage}), \textbf{verbosity} (\ref{labverbosity}), \textbf{roundingwarnings} (\ref{labroundingwarnings})
