\subsection{library}
\label{lablibrary}
\noindent Name: \textbf{library}\\
\phantom{aaa}binds an external mathematical function to a variable in \sollya\\[0.2cm]
\noindent Library names:\\
\verb|   sollya_obj_t sollya_lib_libraryfunction(sollya_obj_t, char *,|\\
\verb|                                           int (*)(mpfi_t, mpfi_t, int))|\\
\verb|   sollya_obj_t sollya_lib_build_function_libraryfunction(sollya_obj_t, char *,|\\
\verb|                                                          int (*)(mpfi_t,|\\
\verb|                                                                  mpfi_t, int))|\\[0.2cm]
\noindent Usage: 
\begin{center}
\textbf{library}(\emph{path}) : \textsf{string} $\rightarrow$ \textsf{function}\\
\end{center}
\noindent Description: \begin{itemize}

\item The command \textbf{library} lets you extend the set of mathematical
   functions known to \sollya.
   By default, \sollya knows the most common mathematical functions such
   as \textbf{exp}, \textbf{sin}, \textbf{erf}, etc. Within \sollya, these functions may be
   composed. This way, \sollya should satisfy the needs of a lot of
   users. However, for particular applications, one may want to
   manipulate other functions such as Bessel functions, or functions
   defined by an integral or even a particular solution of an ODE.

\item \textbf{library} makes it possible to let \sollya know about new functions. In
   order to let it know, you have to provide an implementation of the
   function you are interested in. This implementation is a C file containing
   a function of the form:
   \begin{verbatim} int my_ident(sollya_mpfi_t result, sollya_mpfi_t op, int n)\end{verbatim}
   The semantic of this function is the following: it is an implementation of
   the function and its derivatives in interval arithmetic.
   \verb|my_ident(result, I, n)| shall store in \verb|result| an enclosure 
   of the image set of the $n$-th derivative
   of the function f over \verb|I|: $f^{(n)}(I) \subseteq \mathrm{result}$.

\item The integer value returned by the function implementation currently has no meaning.

\item You do not need to provide a working implementation for any \verb|n|. Most functions
   of \sollya requires a relevant implementation only for $f$, $f'$ and $f''$. For higher 
   derivatives, its is not so critical and the implementation may just store 
   $[-\infty,\,+\infty]$ in result whenever $n>2$.

\item Note that you should respect somehow interval-arithmetic standards in your implementation:
   \verb|result| has its own precision and you should perform the 
   intermediate computations so that \verb|result| is as tight as possible.

\item You can include sollya.h in your implementation and use library 
   functionnalities of \sollya for your implementation. However, this requires to have compiled
   \sollya with \texttt{-fPIC} in order to make the \sollya executable code position 
   independent and to use a system on with programs, using \texttt{dlopen} to open
   dynamic routines can dynamically open themselves.

\item To bind your function into \sollya, you must use the same identifier as the
   function name used in your implementation file (\verb|my_ident| in the previous
   example). Once the function code has been bound to an identifier, you can use a simple assignment
   to assign the bound identifier to yet another identifier. This way, you may use convenient
   names inside \sollya even if your implementation environment requires you to use a less
   convenient name.

\item The dynamic object file whose name is given to \textbf{library} for binding of an
   external library function may also define a destructor function \verb|int sollya_external_lib_close(void)|.
   If \sollya finds such a destructor function in the dynamic object file, it will call 
   that function when closing the dynamic object file again. This happens when \sollya
   is terminated or when the current \sollya session is restarted using \textbf{restart}.
   The purpose of the destructor function is to allow the dynamically bound code
   to free any memory that it might have allocated before \sollya is terminated 
   or restarted. 
   The dynamic object file is not necessarily needed to define a destructor
   function. This ensure backward compatibility with older \sollya external 
   library function object files.
   When defined, the destructor function is supposed to return an integer
   value indicating if an error has happened. Upon success, the destructor
   functions is to return a zero value, upon error a non-zero value.
\end{itemize}
\noindent Example 1: 
\begin{center}\begin{minipage}{15cm}\begin{Verbatim}[frame=single]
> bashexecute("gcc -fPIC -Wall -c libraryexample.c -I$HOME/.local/include");
> bashexecute("gcc -shared -o libraryexample libraryexample.o -lgmp -lmpfr");
> myownlog = library("./libraryexample");
> evaluate(log(x), 2);
0.69314718055994530941723212145817656807550013436024
> evaluate(myownlog(x), 2);
0.69314718055994530941723212145817656807550013436024
\end{Verbatim}
\end{minipage}\end{center}
See also: \textbf{function} (\ref{labfunction}), \textbf{bashexecute} (\ref{labbashexecute}), \textbf{externalproc} (\ref{labexternalproc}), \textbf{externalplot} (\ref{labexternalplot}), \textbf{diff} (\ref{labdiff}), \textbf{evaluate} (\ref{labevaluate}), \textbf{libraryconstant} (\ref{lablibraryconstant})
