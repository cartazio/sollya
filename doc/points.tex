\subsection{points}
\label{labpoints}
\noindent Name: \textbf{points}\\
\phantom{aaa}controls the number of points chosen by \sollya in certain commands.\\[0.2cm]
\noindent Library names:\\
\verb|   void sollya_lib_set_points_and_print(sollya_obj_t)|\\
\verb|   void sollya_lib_set_points(sollya_obj_t)|\\
\verb|   sollya_obj_t sollya_lib_get_points()|\\[0.2cm]
\noindent Usage: 
\begin{center}
\textbf{points} = \emph{n} : \textsf{integer} $\rightarrow$ \textsf{void}\\
\textbf{points} = \emph{n} ! : \textsf{integer} $\rightarrow$ \textsf{void}\\
\textbf{points} : \textsf{constant}\\
\end{center}
Parameters: 
\begin{itemize}
\item \emph{n} represents the number of points
\end{itemize}
\noindent Description: \begin{itemize}

\item \textbf{points} is a global variable. Its value represents the number of points
   used in numerical algorithms of \sollya (namely \textbf{dirtyinfnorm},
   \textbf{dirtyintegral}, \textbf{dirtyfindzeros}, \textbf{plot}).
\end{itemize}
\noindent Example 1: 
\begin{center}\begin{minipage}{15cm}\begin{Verbatim}[frame=single,commandchars=\\\|\~]
> f=x^2*sin(1/x);
> points=10;
The number of points has been set to 10.
> dirtyfindzeros(f, [0;1]);
[|0, 0.31830988618379067153776752674502872406891929148092|]
> points=100;
The number of points has been set to 100.
> dirtyfindzeros(f, [0;1]);
[|0, 2.4485375860291590118289809749617594159147637806224e-2, 3.97887357729738339
42220940843128590508614911435115e-2, 4.54728408833986673625382181064326748669884
70211559e-2, 5.3051647697298445256294587790838120678153215246819e-2, 6.366197723
6758134307553505349005744813783858296184e-2, 7.957747154594766788444188168625718
101722982287023e-2, 0.106103295394596890512589175581676241356306430493638, 0.159
15494309189533576888376337251436203445964574046, 0.31830988618379067153776752674
502872406891929148092|]
\end{Verbatim}
\end{minipage}\end{center}
See also: \textbf{dirtyinfnorm} (\ref{labdirtyinfnorm}), \textbf{dirtyintegral} (\ref{labdirtyintegral}), \textbf{dirtyfindzeros} (\ref{labdirtyfindzeros}), \textbf{plot} (\ref{labplot}), \textbf{diam} (\ref{labdiam}), \textbf{prec} (\ref{labprec})
