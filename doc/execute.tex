\subsection{execute}
\label{labexecute}
\noindent Name: \textbf{execute}\\
executes the content of a file\\
\noindent Usage: 
\begin{center}
\textbf{execute}(\emph{filename}) : \textsf{string} $\rightarrow$ \textsf{void}\\
\end{center}
Parameters: 
\begin{itemize}
\item \emph{filename} is a string representing a file name
\end{itemize}
\noindent Description: \begin{itemize}

\item \\textbf{execute} opens the file indicated by \\emph{filename}, and executes the sequence of \n   commands it contains. This command is evaluated at execution time: this way you\n   can modify the file \\emph{filename} (for instance using \\textbf{bashexecute}) and execute it\n   just after.\n
\item If \\emph{filename} contains a command \\textbf{execute}, it will be executed recursively.\n
\item If \\emph{filename} contains a call to \\textbf{restart}, it will be neglected.\n
\item If \\emph{filename} contains a call to \\textbf{quit}, the commands following \\textbf{quit}\n   in \\emph{filename} will be neglected.\n\end{itemize}
\noindent Example 1: 
\begin{center}\begin{minipage}{15cm}\begin{Verbatim}[frame=single]
\end{Verbatim}
\end{minipage}\end{center}
\noindent Example 2: 
\begin{center}\begin{minipage}{15cm}\begin{Verbatim}[frame=single]
\end{Verbatim}
\end{minipage}\end{center}
\noindent Example 3: 
\begin{center}\begin{minipage}{15cm}\begin{Verbatim}[frame=single]
\end{Verbatim}
\end{minipage}\end{center}
See also: \textbf{parse} (\ref{labparse}), \textbf{readfile} (\ref{labreadfile}), \textbf{write} (\ref{labwrite}), \textbf{print} (\ref{labprint}), \textbf{bashexecute} (\ref{labbashexecute})
