\subsection{\_x\_}
\label{labxfreevariable}
\noindent Name: \textbf{\_x\_}\\
\phantom{aaa}universal name for the mathematical free variable.\\[0.2cm]
\noindent Library names:\\
\verb|   sollya_obj_t sollya_lib_free_variable()|\\
\verb|   sollya_obj_t sollya_lib_build_function_free_variable()|\\
\verb|   #define SOLLYA_X_ (sollya_lib_build_function_free_variable())|\\[0.2cm]
\noindent Description: \begin{itemize}

\item \textbf{\_x\_} is an identifier that always denotes the mathematical free variable.
   It cannot be assigned.

\item \sollya manipulates mathematical functions of a single variable. The first
   time that a variable name is used without having been assigned before, this
   variable name is automatically considered by \sollya as the name of the
   free variable. Subsequently, any other unassigned variable name will be
   considered as the free variable with a warning making this conversion
   explicit. This is convenient for an every-day use of the interactive tool,
   but it has the drawback that the free variable name can change from a
   session to another. There are contexts (\emph{e.g.}, within a procedure, or for
   doing pattern matching) when one might want to refer to the free variable
   regardless of its name in the current session. For this purpose \textbf{\_x\_} is
   a universal identifier, always available and always denoting the free
   variable, whatever its name is in the current context.
\end{itemize}
\noindent Example 1: 
\begin{center}\begin{minipage}{15cm}\begin{Verbatim}[frame=single]
> verbosity=1!;
> sin(a);
sin(a)
> b;
Warning: the identifier "b" is neither assigned to, nor bound to a library funct
ion nor external procedure, nor equal to the current free variable.
Will interpret "b" as "a".
a
> _x_;
a
\end{Verbatim}
\end{minipage}\end{center}
\noindent Example 2: 
\begin{center}\begin{minipage}{15cm}\begin{Verbatim}[frame=single]
> verbosity=1!;
> sin(y);
sin(y)
> f = proc(a) {
    return sin(a + _x_);
  };
> rename(y,z);
Information: the free variable has been renamed from "y" to "z".
> f(1);
sin(1 + z)
\end{Verbatim}
\end{minipage}\end{center}
\noindent Example 3: 
\begin{center}\begin{minipage}{15cm}\begin{Verbatim}[frame=single]
> f = sin(y);
> match f with
    sin(a) : { print("sin of a with a =", a);
               match a with
                 _x_ : { print("a turns out to be the free variable"); }
             default : { print("a is some expression"); };
             }
    _x_ : { print("Free variable") ; }
    default: { print("Something else"); };
sin of a with a = y
a turns out to be the free variable
\end{Verbatim}
\end{minipage}\end{center}
See also: \textbf{rename} (\ref{labrename}), \textbf{isbound} (\ref{labisbound}), \textbf{proc} (\ref{labproc})
