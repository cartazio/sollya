\subsection{horner}
\label{labhorner}
\noindent Name: \textbf{horner}\\
brings all polynomial subexpressions of an expression to Horner form\\
\noindent Usage: 
\begin{center}
\textbf{horner}(\emph{function}) : \textsf{function} $\rightarrow$ \textsf{function}
\\ 
\end{center}
Parameters: 
\begin{itemize}
\item \emph{function} represents the expression to be rewritten in Horner form
\end{itemize}
\noindent Description: \begin{itemize}

\item The command \textbf{horner} rewrites the expression representing the function
   \emph{function} in a way such that all polynomial subexpressions (or the
   whole expression itself, if it is a polynomial) are written in Horner
   form.  The command \textbf{horner} does not endanger the safety of
   computations even in \sollya's floating-point environment: the
   function returned is mathematically equal to the function \emph{function}.
\end{itemize}
\noindent Example 1: 
\begin{center}\begin{minipage}{15cm}\begin{Verbatim}[frame=single]
> print(horner(1 + 2 * x + 3 * x^2));
1 + x * (2 + x * 3)
> print(horner((x + 1)^7));
1 + x * (7 + x * (21 + x * (35 + x * (35 + x * (21 + x * (7 + x))))))
\end{Verbatim}
\end{minipage}\end{center}
\noindent Example 2: 
\begin{center}\begin{minipage}{15cm}\begin{Verbatim}[frame=single]
> print(horner(exp((x + 1)^5) - log(asin(x + x^3) + x)));
exp(1 + x * (5 + x * (10 + x * (10 + x * (5 + x))))) - log(asin(x * (1 + x^2)) +
 x)
\end{Verbatim}
\end{minipage}\end{center}
See also: \textbf{canonical} (\ref{labcanonical}), \textbf{print} (\ref{labprint})
