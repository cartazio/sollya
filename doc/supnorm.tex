\subsection{supnorm}
\label{labsupnorm}
\noindent Name: \textbf{supnorm}\\
computes an interval bounding the supremum norm of an approximation error (absolute or relative) between a given polynomial and a function.\\
\noindent Usage: 
\begin{center}
\textbf{supnorm}(\emph{p}, \emph{f}, \emph{I}, \emph{errorType}, \emph{accuracy}) : (\textsf{function}, \textsf{function}, \textsf{range}, \textsf{absolute$|$relative}, \textsf{constant}) $\rightarrow$ \textsf{range}\\
\end{center}
Parameters: 
\begin{itemize}
\item \emph{p} is a polynomial.
\item \emph{f} is a function.
\item \emph{I} is an interval.
\item \emph{errorType} is the type of error to be considered: \textbf{absolute} or \textbf{relative} (see details below).
\item \emph{accuracy} is a constant that controls the relative tightness of the interval returned. 
\end{itemize}
\noindent Description: \begin{itemize}

\item \textbf{supnorm}(\emph{p}, \emph{f}, \emph{I}, \emph{errorType}, \emph{accuracy}) tries to compute an
   interval bound $r = [\ell,\,u]$ for the supremum norm of the error function
   $\varepsilon_{\textrm{absolute}}=p-f$ (when \emph{errorType} evaluates to \textbf{absolute})
   or $\varepsilon_{\textrm{relative}}=p/f-1$ (when \emph{errorType} evaluates to \textbf{relative}),
   over the interval $I$, such that $\sup_{x \in I} \{|\varepsilon(x)|\}\subseteq r$
   and $0 \le \left \vert\frac{u}{\ell} - 1 \right \vert \le$ \emph{accuracy}. If \textbf{supnorm} succeeds in computing a suitable
   interval $r$, which it returns, that interval is guaranteed to contain the
   supremum norm value and to satisfy the required quality. Otherwise, \textbf{supnorm}
   evaluates to \textbf{error}, displaying a corresponding error message. These failure
   cases are rare and basically happen only for functions which are too
   complicated.

\item Roughly speaking, \textbf{supnorm} is based on \textbf{taylorform} to obtain a higher
   degree polynomial approximation for \emph{f}. This process is coupled with
   an a posteriori validation of a potential supremum norm upper bound. 
   The validation is based on showing a certain polynomial the problem gets
   reduced to does not vanish. In cases when this process alone does not 
   succeed, for instance because \textbf{taylorform} is unable to compute a 
   sufficiently good approximation to \emph{f}, \textbf{supnorm} falls back to
   bisecting the working interval until safe supremum norm bounds can be
   computed with the required accuracy or until the width of the
   subintervals becomes less than \textbf{diam} times the original interval \emph{I}, 
   in which case \textbf{supnorm} fails.

\item The algorithm used for \textbf{supnorm} is quite complex, which makes it impossible 
   to explain it here in further detail. 
   Please find a complete description in the following article:\\
   ~\\
      Sylvain Chevillard, John Harrison, Mioara Joldes, Christoph Lauter\\
      Efficient and accurate computation of upper bounds of approximation errors\\
      Journal of Theoretical Computer Science (TCS), 2010\\
      LIP Research Report number RR LIP2010-2\\
      http://prunel.ccsd.cnrs.fr/ensl-00445343/fr/\\

\item In practical cases, \textbf{supnorm} should be able to automatically handle
   removable discontinuities that relative errors might have. This means
   that usually, if \emph{f} vanishes at a point $x_0$ in the interval
   considered, the approximation polynomial \emph{p} is designed such that it
   also vanishes at the same point with a multiplicity large
   enough. Hence, although \emph{f} vanishes, $\varepsilon_{\textrm{relative}}=p/f-1$ may be 
   defined by continuous extension at such points $x_0$, called 
   removable discontinuities (see Example~$3$).
\end{itemize}
\noindent Example 1: 
\begin{center}\begin{minipage}{15cm}\begin{Verbatim}[frame=single]
> p = remez(exp(x), 5, [-1;1]);
> midpointmode=on!;
> supnorm(p, exp(x), [-1;1], absolute, 2^(-40));
0.452055210438~2/7~e-4
\end{Verbatim}
\end{minipage}\end{center}
\noindent Example 2: 
\begin{center}\begin{minipage}{15cm}\begin{Verbatim}[frame=single]
> prec=200!;
> midpointmode=on!;
> d = [1;2];
> f = exp(cos(x)^2 + 1);
> p = remez(1,15,d,1/f,1e-40);
> theta=1b-60;
> prec=default!;
> mode=relative;
> supnorm(p,f,d,mode,theta);
0.30893006200251428~5/6~e-13
\end{Verbatim}
\end{minipage}\end{center}
\noindent Example 3: 
\begin{center}\begin{minipage}{15cm}\begin{Verbatim}[frame=single]
> midpointmode=on!;
> mode=relative;
> theta=1b-135;
> d = [-1b-2;1b-2];
> f = expm1(x);
> p = x * (1 +  x * ( 2097145 * 2^(-22) + x * ( 349527 * 2^(-21) + x * (87609 * 
2^(-21) + x * 4369 * 2^(-19))))); 
> theta=1b-40;
> supnorm(p,f,d,mode,theta);
0.98349131972~2/3~e-7
\end{Verbatim}
\end{minipage}\end{center}
See also: \textbf{dirtyinfnorm} (\ref{labdirtyinfnorm}), \textbf{infnorm} (\ref{labinfnorm}), \textbf{checkinfnorm} (\ref{labcheckinfnorm}), \textbf{absolute} (\ref{lababsolute}), \textbf{relative} (\ref{labrelative}), \textbf{taylorform} (\ref{labtaylorform}), \textbf{autodiff} (\ref{labautodiff}), \textbf{numberroots} (\ref{labnumberroots}), \textbf{diam} (\ref{labdiam})
