\subsection{implementconstant}
\label{labimplementconstant}
\noindent Name: \textbf{implementconstant}\\
implements a constant in arbitrary precision\\
\noindent Usage: 
\begin{center}
\textbf{implementconstant}(\emph{expr}) : \textsf{constant} $\rightarrow$ \textsf{void}\\
\textbf{implementconstant}(\emph{expr},\emph{filename}) : (\textsf{constant}, \textsf{string}) $\rightarrow$ \textsf{void}\\
\textbf{implementconstant}(\emph{expr},\emph{filename},\emph{functionname}) : (\textsf{constant}, \textsf{string}, \textsf{string}) $\rightarrow$ \textsf{void}\\
\end{center}
\noindent Description: \begin{itemize}

\item The command \textbf{implementconstant} implements the constant expression \emph{expr} in 
   arbitrary precision. More precisely, it generates the source code (written
   in C, and using MPFR) of a C function \texttt{const\_something} with the following
   signature:
   \begin{center}
   \texttt{void const\_something (mpfr\_ptr y, mp\_prec\_t prec)}
   \end{center}
   Let us denote by $c$ the exact mathematical value of the constant defined by
   the expression \emph{expr}. When called with arguments $y$ and prec (where the
   variable $y$ is supposed to be already initialized), the function
   \texttt{mpfr\_const\_something} sets the precision of $y$ to a suitable precision and
   stores in it an approximate value of $c$ such that
   $$|y-c| \le |c|\,2^{1-\mathrm{prec}}.$$

\item When no filename \emph{filename} is given or if \textbf{default} is given as
   \emph{filename}, the source code produced by \textbf{implementconstant} is printed on
   standard output. Otherwise, when \emph{filename} is given as a 
   string of characters, the source code is output to a file 
   named \emph{filename}. If that file cannot be opened and/or 
   written to, \textbf{implementconstant} fails and has no other effect.

\item When \emph{functionname} is given as an argument to \textbf{implementconstant} and
   \emph{functionname} evaluates to a string of characters, the default name
   for the C function \texttt{const\_something} is
   replaced by \emph{functionname}. When \textbf{default} is given as \emph{functionname},
   the default name is used nevertheless, as if no \emph{functionname}
   argument were given.  When choosing a character sequence for
   \emph{functionname}, the user should keep attention to the fact that
   \emph{functionname} must be a valid C identifier in order to enable
   error-free compilation of the produced code.

\item If \emph{expr} refers to a constant defined with \textbf{libraryconstant}, the produced
   code uses the external code implementing this constant. The user should
   keep in mind that it is up to them to make sure the symbol for that 
   external code can get resolved when the newly generated code is to 
   be loaded.

\item If a subexpression of \emph{expr} evaluates to $0$, \textbf{implementconstant} will most
   likely fail with an error message.

\item \textbf{implementconstant} is unable to implement constant expressions \emph{expr} that
   contain procedure-based functions, i.e. functions created from \sollya
   procedures using the \textbf{function} construct. If \emph{expr} contains such a
   procedure-based function, \textbf{implementconstant} prints a warning and fails
   silently. The reason for this lack of functionality is that the
   produced C source code, which is supposed to be compiled, would have
   to call back to the \sollya interpreter in order to evaluate the
   procedure-based function.

\item Similarily, \textbf{implementconstant} is currently unable to implement constant
   expressions \emph{expr} that contain library-based functions, i.e.
   functions dynamically bound to \sollya using the \textbf{library} construct.
   If \emph{expr} contains such a library-based function, \textbf{implementconstant} prints
   a warning and fails silently. Support for this feature is in principle
   feasible from a technical standpoint and might be added in a future
   release of \sollya.

\item Currently, non-differentiable functions such as \textbf{double}, \textbf{doubledouble},
   \textbf{tripledouble}, \textbf{single}, \textbf{halfprecision}, \textbf{quad}, \textbf{doubleextended}, 
   \textbf{floor}, \textbf{ceil}, \textbf{nearestint} are not supported by \textbf{implementconstant}. 
   If \textbf{implementconstant} encounters one of them, a warning message is displayed 
   and no code is produced. However, if \textbf{autosimplify} equals on, it is 
   possible that \sollya silently simplifies subexpressions of \emph{expr} 
   containing such functions and that \textbf{implementconstant} successfully produces 
   code for evaluating \emph{expr}.

\item While it produces an MPFR-based C source code for \emph{expr}, \textbf{implementconstant}
   takes architectural and system-dependent parameters into account.  For
   example, it checks whether literal constants figuring in \emph{expr} can be
   represented on a C \texttt{long int} type or if they must
   be stored in a different manner not to affect their accuracy. These
   tests, performed by \sollya during execution of \textbf{implementconstant}, depend
   themselves on the architecture \sollya is running on. Users should
   keep this matter in mind, especially when trying to compile source
   code on one machine whilst it has been produced on another.
\end{itemize}
\noindent Example 1: 
\begin{center}\begin{minipage}{15cm}\begin{Verbatim}[frame=single]
> implementconstant(exp(1)+log(2)/sqrt(1/10));
#include <mpfr.h>

void
const_something (mpfr_ptr y, mp_prec_t prec)
{
  /* Declarations */
  mpfr_t tmp1;
  mpfr_t tmp2;
  mpfr_t tmp3;
  mpfr_t tmp4;
  mpfr_t tmp5;
  mpfr_t tmp6;
  mpfr_t tmp7;

  /* Initializations */
  mpfr_init2 (tmp2, prec+5);
  mpfr_init2 (tmp1, prec+3);
  mpfr_init2 (tmp4, prec+8);
  mpfr_init2 (tmp3, prec+7);
  mpfr_init2 (tmp6, prec+11);
  mpfr_init2 (tmp7, prec+11);
  mpfr_init2 (tmp5, prec+11);

  /* Core */
  mpfr_set_prec (tmp2, prec+4);
  mpfr_set_ui (tmp2, 1, MPFR_RNDN);
  mpfr_set_prec (tmp1, prec+3);
  mpfr_exp (tmp1, tmp2, MPFR_RNDN);
  mpfr_set_prec (tmp4, prec+8);
  mpfr_set_ui (tmp4, 2, MPFR_RNDN);
  mpfr_set_prec (tmp3, prec+7);
  mpfr_log (tmp3, tmp4, MPFR_RNDN);
  mpfr_set_prec (tmp6, prec+11);
  mpfr_set_ui (tmp6, 1, MPFR_RNDN);
  mpfr_set_prec (tmp7, prec+11);
  mpfr_set_ui (tmp7, 10, MPFR_RNDN);
  mpfr_set_prec (tmp5, prec+11);
  mpfr_div (tmp5, tmp6, tmp7, MPFR_RNDN);
  mpfr_set_prec (tmp4, prec+7);
  mpfr_sqrt (tmp4, tmp5, MPFR_RNDN);
  mpfr_set_prec (tmp2, prec+5);
  mpfr_div (tmp2, tmp3, tmp4, MPFR_RNDN);
  mpfr_set_prec (y, prec+3);
  mpfr_add (y, tmp1, tmp2, MPFR_RNDN);

  /* Cleaning stuff */
  mpfr_clear(tmp1);
  mpfr_clear(tmp2);
  mpfr_clear(tmp3);
  mpfr_clear(tmp4);
  mpfr_clear(tmp5);
  mpfr_clear(tmp6);
  mpfr_clear(tmp7);
}
\end{Verbatim}
\end{minipage}\end{center}
\noindent Example 2: 
\begin{center}\begin{minipage}{15cm}\begin{Verbatim}[frame=single]
> implementconstant(sin(13/17),"sine_of_thirteen_seventeenth.c");
> readfile("sine_of_thirteen_seventeenth.c");
#include <mpfr.h>

void
const_something (mpfr_ptr y, mp_prec_t prec)
{
  /* Declarations */
  mpfr_t tmp1;
  mpfr_t tmp2;
  mpfr_t tmp3;

  /* Initializations */
  mpfr_init2 (tmp2, prec+6);
  mpfr_init2 (tmp3, prec+6);
  mpfr_init2 (tmp1, prec+6);

  /* Core */
  mpfr_set_prec (tmp2, prec+6);
  mpfr_set_ui (tmp2, 13, MPFR_RNDN);
  mpfr_set_prec (tmp3, prec+6);
  mpfr_set_ui (tmp3, 17, MPFR_RNDN);
  mpfr_set_prec (tmp1, prec+6);
  mpfr_div (tmp1, tmp2, tmp3, MPFR_RNDN);
  mpfr_set_prec (y, prec+2);
  mpfr_sin (y, tmp1, MPFR_RNDN);

  /* Cleaning stuff */
  mpfr_clear(tmp1);
  mpfr_clear(tmp2);
  mpfr_clear(tmp3);
}

\end{Verbatim}
\end{minipage}\end{center}
\noindent Example 3: 
\begin{center}\begin{minipage}{15cm}\begin{Verbatim}[frame=single]
> implementconstant(asin(1/3 * pi),default,"arcsin_of_one_third_pi");
#include <mpfr.h>

void
arcsin_of_one_third_pi (mpfr_ptr y, mp_prec_t prec)
{
  /* Declarations */
  mpfr_t tmp1;
  mpfr_t tmp2;
  mpfr_t tmp3;

  /* Initializations */
  if (prec >= 2147483640+MPFR_PREC_MIN)
  {
    mpfr_init2 (tmp2, prec-2147483640);
  }
  else
  {
    mpfr_init2 (tmp2, MPFR_PREC_MIN);
  }
  if (prec >= 2147483640+MPFR_PREC_MIN)
  {
    mpfr_init2 (tmp3, prec-2147483640);
  }
  else
  {
    mpfr_init2 (tmp3, MPFR_PREC_MIN);
  }
  if (prec >= 2147483640+MPFR_PREC_MIN)
  {
    mpfr_init2 (tmp1, prec-2147483640);
  }
  else
  {
    mpfr_init2 (tmp1, MPFR_PREC_MIN);
  }

  /* Core */
  if (prec >= 2147483640+MPFR_PREC_MIN)
  {
    mpfr_set_prec (tmp2, prec-2147483640);
  }
  else
  {
    mpfr_set_prec (tmp2, MPFR_PREC_MIN);
  }
  mpfr_const_pi (tmp2, MPFR_RNDN);
  if (prec >= 2147483640+MPFR_PREC_MIN)
  {
    mpfr_set_prec (tmp3, prec-2147483640);
  }
  else
  {
    mpfr_set_prec (tmp3, MPFR_PREC_MIN);
  }
  mpfr_set_ui (tmp3, 3, MPFR_RNDN);
  if (prec >= 2147483640+MPFR_PREC_MIN)
  {
    mpfr_set_prec (tmp1, prec-2147483640);
  }
  else
  {
    mpfr_set_prec (tmp1, MPFR_PREC_MIN);
  }
  mpfr_div (tmp1, tmp2, tmp3, MPFR_RNDN);
  mpfr_set_prec (y, prec+2);
  mpfr_asin (y, tmp1, MPFR_RNDN);

  /* Cleaning stuff */
  mpfr_clear(tmp1);
  mpfr_clear(tmp2);
  mpfr_clear(tmp3);
}
\end{Verbatim}
\end{minipage}\end{center}
\noindent Example 4: 
\begin{center}\begin{minipage}{15cm}\begin{Verbatim}[frame=single]
> implementconstant(ceil(log(19 + 1/3)),"constant_code.c","magic_constant");
> readfile("constant_code.c");
#include <mpfr.h>

void
magic_constant (mpfr_ptr y, mp_prec_t prec)
{
  /* Initializations */

  /* Core */
  mpfr_set_prec (y, prec);
  mpfr_set_ui (y, 3, MPFR_RNDN);
}

\end{Verbatim}
\end{minipage}\end{center}
\noindent Example 5: 
\begin{center}\begin{minipage}{15cm}\begin{Verbatim}[frame=single]
> bashexecute("gcc -fPIC -Wall -c libraryconstantexample.c -I$HOME/.local/includ
e");
> bashexecute("gcc -shared -o libraryconstantexample libraryconstantexample.o -l
gmp -lmpfr");
> euler_gamma = libraryconstant("./libraryconstantexample");
> implementconstant(euler_gamma^(1/3));
#include <mpfr.h>

void
const_something (mpfr_ptr y, mp_prec_t prec)
{
  /* Declarations */
  mpfr_t tmp1;

  /* Initializations */
  mpfr_init2 (tmp1, prec+1);

  /* Core */
  euler_gamma (tmp1, prec+1);
  mpfr_set_prec (y, prec+2);
  mpfr_root (y, tmp1, 3, MPFR_RNDN);

  /* Cleaning stuff */
  mpfr_clear(tmp1);
}
\end{Verbatim}
\end{minipage}\end{center}
See also: \textbf{implementpoly} (\ref{labimplementpoly}), \textbf{libraryconstant} (\ref{lablibraryconstant}), \textbf{library} (\ref{lablibrary}), \textbf{function} (\ref{labfunction})
