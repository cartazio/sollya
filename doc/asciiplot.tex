\subsection{asciiplot}
\label{labasciiplot}
\noindent Name: \textbf{asciiplot}\\
plots a function in a range using ASCII characters\\
\noindent Usage: 
\begin{center}
\textbf{asciiplot}(\emph{function}, \emph{range}) : (\textsf{function}, \textsf{range}) $\rightarrow$ \textsf{void}\\
\end{center}
Parameters: 
\begin{itemize}
\item \emph{function} represents a function to be plotted
\item \emph{range} represents a range the function is to be plotted in 
\end{itemize}
\noindent Description: \begin{itemize}

\item \\textbf{asciiplot} plots the function \\emph{function} in range \\emph{range} using ASCII\n   characters.  On systems that provide the necessary \n   \\texttt{TIOCGWINSZ ioctl}, \\sollya determines the size of the\n   terminal for the plot size if connected to a terminal. If it is not\n   connected to a terminal or if the test is not possible, the plot is of\n   fixed size $77\\times25$ characters.  The function is\n   evaluated on a number of points equal to the number of columns\n   available. Its value is rounded to the next integer in the range of\n   lines available. A letter \\texttt{x} is written at this place. If zero is in\n   the hull of the image domain of the function, an x-axis is\n   displayed. If zero is in range, a y-axis is displayed.  If the\n   function is constant or if the range is reduced to one point, the\n   function is evaluated to a constant and the constant is displayed\n   instead of a plot.\n\end{itemize}
\noindent Example 1: 
\begin{center}\begin{minipage}{15cm}\begin{Verbatim}[frame=single]
\end{Verbatim}
\end{minipage}\end{center}
\noindent Example 2: 
\begin{center}\begin{minipage}{15cm}\begin{Verbatim}[frame=single]
\end{Verbatim}
\end{minipage}\end{center}
\noindent Example 3: 
\begin{center}\begin{minipage}{15cm}\begin{Verbatim}[frame=single]
\end{Verbatim}
\end{minipage}\end{center}
\noindent Example 4: 
\begin{center}\begin{minipage}{15cm}\begin{Verbatim}[frame=single]
\end{Verbatim}
\end{minipage}\end{center}
See also: \textbf{plot} (\ref{labplot})
