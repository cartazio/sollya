\subsection{remez}
\label{labremez}
\noindent Name: \textbf{remez}\\
computes the minimax of a function on an interval.\\
\noindent Usage: 
\begin{center}
\textbf{remez}(\emph{f}, \emph{n}, \emph{range}, \emph{w}, \emph{quality}) : (\textsf{function}, \textsf{integer}, \textsf{range}, \textsf{function}, \textsf{constant}) $\rightarrow$ \textsf{function}\\
\textbf{remez}(\emph{f}, \emph{L}, \emph{range}, \emph{w}, \emph{quality}) : (\textsf{function}, \textsf{list}, \textsf{range}, \textsf{function}, \textsf{constant}) $\rightarrow$ \textsf{function}\\
\end{center}
Parameters: 
\begin{itemize}
\item \emph{f} is the function to be approximated
\item \emph{n} is the degree of the polynomial that must approximate \emph{f}
\item \emph{L} is a list of integers or a list of functions and indicates the basis for the approximation of \emph{f}
\item \emph{range} is the interval where the function must be approximated
\item \emph{w} (optional) is a weight function. Default is 1.
\item \emph{quality} (optional) is a parameter that controls the quality of the returned polynomial \emph{p}, with respect to the exact minimax $p^\star$. Default is 1e-5.
\end{itemize}
\noindent Description: \begin{itemize}

\item \textbf{remez} computes an approximation of the function $f$ with respect to
   the weight function $w$ on the interval \emph{range}. More precisely, it
   searches $p$ such that $\|pw-f\|_{\infty}$ is
   (almost) minimal among all $p$ of a certain form. The norm is
   the infinity norm, e.g. $\|g\|_{\infty} = \max \{|g(x)|, x \in \mathrm{range}\}.$

\item If $w=1$ (the default case), it consists in searching the best
   polynomial approximation of $f$ with respect to the absolute error.
   If $f=1$ and $w$ is of the form $1/g$, it consists in
   searching the best polynomial approximation of $g$ with respect to the
   relative error.

\item If $n$ is given, $p$ is searched among the polynomials with degree not
   greater than $n$.
   If \emph{L} is given and is a list of integers, $p$ is searched as a linear
   combination of monomials $X^k$ where $k$ belongs to \emph{L}.
   In the case when \emph{L} is a list of integers, it may contain ellipses but cannot be
   end-elliptic.
   If \emph{L} is given and is a list of functions $g_k$, $p$ is searched as a
   linear combination of the $g_k$. In that case \emph{L} cannot contain ellipses.
   It is the user responsability to check that the $g_k$ are linearly independent
   over the interval \emph{range}. Moreover, the functions $w\cdot g_k$ must be at least
   twice differentiable over \emph{range}. If these conditions are not fulfilled, the
   algorithm might fail or even silently return a result as if it successfully
   found the minimax, though the returned $p$ is not optimal.

\item The polynomial is obtained by a convergent iteration called Remez' algorithm
   (and an extension of this algorithm, due to Stiefel).
   The algorithm computes a sequence $p_1,\dots ,p_k,\dots$
   such that $e_k = \|p_k w-f\|_{\infty}$ converges towards
   the optimal value $e$. The algorithm is stopped when the relative error
   between $e_k$ and $e$ is less than \emph{quality}.
\end{itemize}
\noindent Example 1: 
\begin{center}\begin{minipage}{15cm}\begin{Verbatim}[frame=single]
> p = remez(exp(x),5,[0;1]);
> degree(p);
5
> dirtyinfnorm(p-exp(x),[0;1]);
1.12956984638214536849843017679626063762687501534126e-6
\end{Verbatim}
\end{minipage}\end{center}
\noindent Example 2: 
\begin{center}\begin{minipage}{15cm}\begin{Verbatim}[frame=single]
> p = remez(1,[|0,2,4,6,8|],[0,Pi/4],1/cos(x));
> canonical=on!;
> p;
0.99999999994393749280444571988532724907643631727381 + -0.4999999957155746773720
4931630836834563663039748203 * x^2 + 4.16666132335010905188253972212748718651775
241902969e-2 * x^4 + -1.38865291475286141707180658383176799662601691348739e-3 * 
x^6 + 2.437267919111162694221738667927916761689966804242e-5 * x^8
\end{Verbatim}
\end{minipage}\end{center}
\noindent Example 3: 
\begin{center}\begin{minipage}{15cm}\begin{Verbatim}[frame=single]
> p1 = remez(exp(x),5,[0;1],default,1e-5);
> p2 = remez(exp(x),5,[0;1],default,1e-10);
> p3 = remez(exp(x),5,[0;1],default,1e-15);
> dirtyinfnorm(p1-exp(x),[0;1]);
1.12956984638214536849843017679626063762687501534126e-6
> dirtyinfnorm(p2-exp(x),[0;1]);
1.12956980227478687332174207517728389861926659249056e-6
> dirtyinfnorm(p3-exp(x),[0;1]);
1.12956980227478687332174207517728389861926659249056e-6
\end{Verbatim}
\end{minipage}\end{center}
\noindent Example 4: 
\begin{center}\begin{minipage}{15cm}\begin{Verbatim}[frame=single]
> L = [|exp(x), sin(x), cos(x)-1, sin(x^3)|];
> g = (2^x-1)/x;
> p1 = remez(g, L, [-1/16;1/16]);
> p2 = remez(g, 3, [-1/16;1/16]);
> dirtyinfnorm(p1 - g, [-1/16;1/16]);
9.8841323805554845968308959691395564355375312205068e-8
> dirtyinfnorm(p2 - g, [-1/16;1/16]);
2.5433780105975429703888838928671089431318650967714e-9
\end{Verbatim}
\end{minipage}\end{center}
See also: \textbf{dirtyinfnorm} (\ref{labdirtyinfnorm}), \textbf{infnorm} (\ref{labinfnorm}), \textbf{fpminimax} (\ref{labfpminimax}), \textbf{guessdegree} (\ref{labguessdegree}), \textbf{taylorform} (\ref{labtaylorform}), \textbf{taylor} (\ref{labtaylor})
