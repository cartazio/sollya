\subsection{remez}
\label{labremez}
\noindent Name: \textbf{remez}\\
computes the minimax of a function on an interval.\\

\noindent Usage: 
\begin{center}
\textbf{remez}(\emph{f}, \emph{n}, \emph{range}, \emph{w}, \emph{quality}) : (\textsf{function}, \textsf{integer}, \textsf{range}, \textsf{function}, \textsf{constant}) $\rightarrow$ \textsf{function}\\
\textbf{remez}(\emph{f}, \emph{L}, \emph{range}, \emph{w}, \emph{quality}) : (\textsf{function}, \textsf{list}, \textsf{range}, \textsf{function}, \textsf{constant}) $\rightarrow$ \textsf{function}\\
\end{center}
Parameters: 
\begin{itemize}
\item \emph{f} is the function to be approximated
\item \emph{n} is the degree of the polynomial that must approximate \emph{f}
\item \emph{L} is a list of monomials that can be used to represent the polynomial that must approximate \emph{f}
\item \emph{range} is the interval where the function must be approximated
\item \emph{w} (optional) is a weight function. Default is 1.
\item \emph{quality} (optional) is a parameter that controls the quality of the returned polynomial \emph{p}, with respect to the exact minimax $p^\star$. Default is 1e-5.
\end{itemize}
\noindent Description: \begin{itemize}

\item \textbf{remez} computes an approximation of the function $f$ with respect to 
   the weight function $w$ on the interval \emph{range}. More precisely, it 
   searches a polynomial $p$ such that $\|pw-f\|_{\infty}$ is 
   (almost minimal) among all polynomials $p$ of a certain form. The norm is
   the infinity norm, e.g. $\|g\|_{\infty} = \max \{|g(x)|, x \in \mathrm{range}\}.$

\item If $w=1$ (the default case), it consists in searching the best 
   polynomial approximation of $f$ with respect to the absolute error.
   If $f=1$ and $w$ is of the form $1/g$, it consists in 
   searching the best polynomial approximation of $g$ with respect to the 
   relative error.

\item If $n$ is given, the polynomial $p$ is searched among the 
   polynomials with degree not greater than $n$.
   If \emph{L} is given, the polynomial $p$ is searched as a linear combination 
   of monomials $X^k$ where $k$ belongs to \emph{L}.
   \emph{L} may contain ellipses but cannot be end-elliptic.

\item The polynomial is obtained by a convergent iteration called Remez' algorithm. 
   The algorithm computes a sequence $p_1,\dots ,p_k,\dots$ 
   such that $e_k = \|p_k w-f\|_{\infty}$ converges towards 
   the optimal value $e$. The algorithm is stopped when the relative error 
   between $e_k$ and $e$ is less than \emph{quality}.

\item Note: the algorithm may not converge in certain cases. Moreover, it may 
   converge towards a polynomial that is not optimal. These cases correspond to 
   the cases when the Haar condition is not fulfilled.
   See [Cheney - Approximation theory] for details.
\end{itemize}
\noindent Example 1: 
\begin{center}\begin{minipage}{15cm}\begin{Verbatim}[frame=single]
> p = remez(exp(x),5,[0;1]);
> degree(p);
5
> dirtyinfnorm(p-exp(x),[0;1]);
1.12956984638214536849843017679626063762687503980789e-6
\end{Verbatim}
\end{minipage}\end{center}
\noindent Example 2: 
\begin{center}\begin{minipage}{15cm}\begin{Verbatim}[frame=single]
> p = remez(1,[|0,2,4,6,8|],[0,Pi/4],1/cos(x));
> canonical=on!;
> p;
0.99999999994393749280444571988532724907643631727379 + -0.4999999957155746773720
49316308368345636630397481628 * x^2 + 4.1666613233501090518825397221274871865177
52418561e-2 * x^4 + -1.38865291475286141707180658383176799662601690152622e-3 * x
^6 + 2.43726791911116269422173866792791676168996590663655e-5 * x^8
\end{Verbatim}
\end{minipage}\end{center}
\noindent Example 3: 
\begin{center}\begin{minipage}{15cm}\begin{Verbatim}[frame=single]
> p1 = remez(exp(x),5,[0;1],default,1e-5);
> p2 = remez(exp(x),5,[0;1],default,1e-10);
> p3 = remez(exp(x),5,[0;1],default,1e-15);
> dirtyinfnorm(p1-exp(x),[0;1]);
1.12956984638214536849843017679626063762687503980789e-6
> dirtyinfnorm(p2-exp(x),[0;1]);
1.1295698022747868733217420751772838986192666255395e-6
> dirtyinfnorm(p3-exp(x),[0;1]);
1.1295698022747868733217420751772838986192666255395e-6
\end{Verbatim}
\end{minipage}\end{center}
See also: \textbf{dirtyinfnorm} (\ref{labdirtyinfnorm}), \textbf{infnorm} (\ref{labinfnorm})
