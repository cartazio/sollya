\subsection{ simplifysafe }
\noindent Name: \textbf{simplifysafe}\\
simplifies an expression representing a function\\

\noindent Usage: 
\begin{center}
\textbf{simplifysafe}(\emph{function}) : \textsf{function} $\rightarrow$ \textsf{function}\\
\end{center}
Parameters: 
\begin{itemize}
\item \emph{function} represents the expression to be simplified
\end{itemize}
\noindent Description: \begin{itemize}

\item The command \textbf{simplifysafe} simplifies the expression given in argument
   representing the function \emph{function}.  The command \textbf{simplifysafe} does not
   endanger the safety of computations even in Sollya's floating-point
   environment: the function returned is mathematically equal to the
   function \emph{function}. 
   Remark that the simplification provided by \textbf{simplifysafe} is not perfect:
   they may exist simpler equivalent expressions for expressions returned
   by \textbf{simplifysafe}.
\end{itemize}
\noindent Example 1: 
\begin{center}\begin{minipage}{15cm}\begin{Verbatim}[frame=single]
> print(simplifysafe((6 + 2) + (5 + exp(0)) * x));
8 + 6 * x
\end{Verbatim}
\end{minipage}\end{center}
\noindent Example 2: 
\begin{center}\begin{minipage}{15cm}\begin{Verbatim}[frame=single]
> print(simplifysafe((log(x - x + 1) + asin(1))));
(pi) / 2
\end{Verbatim}
\end{minipage}\end{center}
\noindent Example 3: 
\begin{center}\begin{minipage}{15cm}\begin{Verbatim}[frame=single]
> print(simplifysafe((log(x - x + 1) + asin(1)) - (atan(1) * 2)));
(pi) / 2 - (pi) / 4 * 2
\end{Verbatim}
\end{minipage}\end{center}
See also: \textbf{simplify}, \textbf{autosimplify}
