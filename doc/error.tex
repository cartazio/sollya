\subsection{error}
\label{laberror}
\noindent Name: \textbf{error}\\
expression representing an input that is wrongly typed or that cannot be executed\\
\noindent Usage: 
\begin{center}
\textbf{error} : \textsf{error}\\
\end{center}
\noindent Description: \begin{itemize}

\item The variable \textbf{error} represents an input during the evaluation of
   which a type or execution error has been detected or is to be
   detected. Inputs that are syntactically correct but wrongly typed
   evaluate to \textbf{error} at some stage.  Inputs that are correctly typed
   but containing commands that depend on side-effects that cannot be
   performed or inputs that are wrongly typed at meta-level (cf. \textbf{parse}),
   evaluate to \textbf{error}.
    
   Remark that in contrast to all other elements of the \sollya language,
   \textbf{error} compares neither equal nor unequal to itself. This provides a
   means of detecting syntax errors inside the \sollya language itself
   without introducing issues of two different wrongly typed inputs being
   equal.
\end{itemize}
\noindent Example 1: 
\begin{center}\begin{minipage}{15cm}\begin{Verbatim}[frame=single]
> print(5 + "foo");
error
\end{Verbatim}
\end{minipage}\end{center}
\noindent Example 2: 
\begin{center}\begin{minipage}{15cm}\begin{Verbatim}[frame=single]
> error;
error
\end{Verbatim}
\end{minipage}\end{center}
\noindent Example 3: 
\begin{center}\begin{minipage}{15cm}\begin{Verbatim}[frame=single]
> error == error;
false
> error != error;
false
\end{Verbatim}
\end{minipage}\end{center}
\noindent Example 4: 
\begin{center}\begin{minipage}{15cm}\begin{Verbatim}[frame=single]
> correct = 5 + 6;
> incorrect = 5 + "foo";
> (correct == error || correct != error);
true
> (incorrect == error || incorrect != error);
false
\end{Verbatim}
\end{minipage}\end{center}
See also: \textbf{void} (\ref{labvoid}), \textbf{parse} (\ref{labparse})
