\subsection{error}
\label{laberror}
\noindent Name: \textbf{error}\\
expression representing an input that is wrongly typed or that cannot be executed\\
\noindent Usage: 
\begin{center}
\textbf{error} : \textsf{error}\\
\end{center}
\noindent Description: \begin{itemize}

\item The variable \\textbf{error} represents an input during the evaluation of\n   which a type or execution error has been detected or is to be\n   detected. Inputs that are syntactically correct but wrongly typed\n   evaluate to \\textbf{error} at some stage.  Inputs that are correctly typed\n   but containing commands that depend on side-effects that cannot be\n   performed or inputs that are wrongly typed at meta-level (cf. \\textbf{parse}),\n   evaluate to \\textbf{error}.\n    \n   Remark that in contrast to all other elements of the \\sollya language,\n   \\textbf{error} compares neither equal nor unequal to itself. This provides a\n   means of detecting syntax errors inside the \\sollya language itself\n   without introducing issues of two different wrongly typed inputs being\n   equal.\n\end{itemize}
\noindent Example 1: 
\begin{center}\begin{minipage}{15cm}\begin{Verbatim}[frame=single]
\end{Verbatim}
\end{minipage}\end{center}
\noindent Example 2: 
\begin{center}\begin{minipage}{15cm}\begin{Verbatim}[frame=single]
\end{Verbatim}
\end{minipage}\end{center}
\noindent Example 3: 
\begin{center}\begin{minipage}{15cm}\begin{Verbatim}[frame=single]
\end{Verbatim}
\end{minipage}\end{center}
\noindent Example 4: 
\begin{center}\begin{minipage}{15cm}\begin{Verbatim}[frame=single]
\end{Verbatim}
\end{minipage}\end{center}
See also: \textbf{void} (\ref{labvoid}), \textbf{parse} (\ref{labparse})
