\subsection{evaluate}
\label{labevaluate}
\noindent Name: \textbf{evaluate}\\
evaluates a function at a constant point or in a range\\
\noindent Usage: 
\begin{center}
\textbf{evaluate}(\emph{function}, \emph{constant}) : (\textsf{function}, \textsf{constant}) $\rightarrow$ \textsf{constant} $|$ \textsf{range}\\
\textbf{evaluate}(\emph{function}, \emph{range}) : (\textsf{function}, \textsf{range}) $\rightarrow$ \textsf{range}\\
\textbf{evaluate}(\emph{function}, \emph{function2}) : (\textsf{function}, \textsf{function}) $\rightarrow$ \textsf{function}\\
\end{center}
Parameters: 
\begin{itemize}
\item \emph{function} represents a function
\item \emph{constant} represents a constant point
\item \emph{range} represents a range
\item \emph{function2} represents a function that is not constant
\end{itemize}
\noindent Description: \begin{itemize}

\item If its second argument is a constant \\emph{constant}, \\textbf{evaluate} evaluates\n   its first argument \\emph{function} at the point indicated by\n   \\emph{constant}. This evaluation is performed in a way that the result is a\n   faithful rounding of the real value of the \\emph{function} at \\emph{constant} to\n   the current global precision. If such a faithful rounding is not\n   possible, \\textbf{evaluate} returns a range surely encompassing the real value\n   of the function \\emph{function} at \\emph{constant}. If even interval evaluation\n   is not possible because the expression is undefined or numerically\n   unstable, NaN will be produced.\n
\item If its second argument is a range \\emph{range}, \\textbf{evaluate} evaluates its\n   first argument \\emph{function} by interval evaluation on this range\n   \\emph{range}. This ensures that the image domain of the function \\emph{function}\n   on the preimage domain \\emph{range} is surely enclosed in the returned\n   range.\n
\item In the case when the second argument is a range that is reduced to a\n   single point (such that $[1;\\,1]$ for instance), the evaluation\n   is performed in the same way as when the second argument is a constant but\n   it produces a range as a result: \\textbf{evaluate} automatically adjusts the precision\n   of the intern computations and returns a range that contains at most three floating-point\n   consecutive numbers in precision \\textbf{prec}. This correponds to the same accuracy\n   as a faithful rounding of the actual result. If such a faithful rounding\n   is not possible, \\textbf{evaluate} has the same behavior as in the case when the\n   second argument is a constant.\n
\item If its second argument is a function \\emph{function2} that is not a\n   constant, \\textbf{evaluate} replaces all occurrences of the free variable in\n   function \\emph{function} by function \\emph{function2}.\n\end{itemize}
\noindent Example 1: 
\begin{center}\begin{minipage}{15cm}\begin{Verbatim}[frame=single]
\end{Verbatim}
\end{minipage}\end{center}
\noindent Example 2: 
\begin{center}\begin{minipage}{15cm}\begin{Verbatim}[frame=single]
\end{Verbatim}
\end{minipage}\end{center}
\noindent Example 3: 
\begin{center}\begin{minipage}{15cm}\begin{Verbatim}[frame=single]
\end{Verbatim}
\end{minipage}\end{center}
\noindent Example 4: 
\begin{center}\begin{minipage}{15cm}\begin{Verbatim}[frame=single]
\end{Verbatim}
\end{minipage}\end{center}
\noindent Example 5: 
\begin{center}\begin{minipage}{15cm}\begin{Verbatim}[frame=single]
\end{Verbatim}
\end{minipage}\end{center}
See also: \textbf{isevaluable} (\ref{labisevaluable})
