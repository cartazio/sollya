\subsection{ coeff }
\noindent Name: \textbf{coeff}\\
gives the coefficient of degree n of a polynomial\\

\noindent Usage: 
\begin{center}
\textbf{coeff}(\emph{f},\emph{n}) : (\textsf{function}, \textsf{integer}) $\rightarrow$ \textsf{constant}\\
\end{center}
Parameters: 
\emph{f} is a function (usually a polynomial).\\
\emph{n} is an integer\\

\noindent Description: \begin{itemize}

\item If \emph{f} is a polynomial, \textbf{coeff}(\emph{f}, \emph{n}) returns the coefficient of
   degree \emph{n} in \emph{f}.

\item If \emph{f} is a function that is not a polynomial, \textbf{coeff}(\emph{f}, \emph{n}) returns 0.
\end{itemize}
\noindent Example 1: 
\begin{center}\begin{minipage}{14.8cm}\begin{Verbatim}[frame=single]
   > coeff((1+x)^5,3);
   10
\end{Verbatim}
\end{minipage}\end{center}
\noindent Example 2: 
\begin{center}\begin{minipage}{14.8cm}\begin{Verbatim}[frame=single]
   > coeff(sin(x),0);
   0
\end{Verbatim}
\end{minipage}\end{center}
See also: \textbf{degree}
