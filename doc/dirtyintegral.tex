\subsection{dirtyintegral}
\label{labdirtyintegral}
\noindent Name: \textbf{dirtyintegral}\\
computes a numerical approximation of the integral of a function on an interval.\\

\noindent Usage: 
\begin{center}
\textbf{dirtyintegral}(\emph{f},\emph{I}) : (\textsf{function}, \textsf{range}) $\rightarrow$ \textsf{constant}\\
\end{center}
Parameters: 
\begin{itemize}
\item \emph{f} is a function.
\item \emph{I} is an interval.
\end{itemize}
\noindent Description: \begin{itemize}

\item \textbf{dirtyintegral}(\emph{f},\emph{I}) computes an approximation of the integral of \emph{f} on \emph{I}.

\item The interval must be bound. If the interval contains one of -Inf or +Inf, the 
   result of \textbf{dirtyintegral} is NaN, even if the integral has a meaning.

\item The result of this command depends on the global variables \textbf{prec} and \textbf{points}.
   The method used is the trapezium rule applied at $n$ evenly distributed
   points in the interval, where $n$ is the value of global variable \textbf{points}.

\item This command computes a numerical approximation of the exact value of the 
   integral. It should not be used if safety is critical. In this case, use
   command \textbf{integral} instead.

\item Warning: this command is known to be currently unsatisfactory. If you really
   need to compute integrals, think of using an other tool or report a feature
   request to sylvain.chevillard@ens-lyon.fr.
\end{itemize}
\noindent Example 1: 
\begin{center}\begin{minipage}{15cm}\begin{Verbatim}[frame=single]
> sin(10);
-0.54402111088936981340474766185137728168364301291624
> dirtyintegral(cos(x),[0;10]);
-0.544003049051526298224480588824753820365362983562797
> points=2000!;
> dirtyintegral(cos(x),[0;10]);
-0.54401997751158321972222697312583199035995837926892
\end{Verbatim}
\end{minipage}\end{center}
See also: \textbf{prec} (\ref{labprec}), \textbf{points} (\ref{labpoints}), \textbf{integral} (\ref{labintegral})
