\subsection{revert}
\label{labrevert}
\noindent Name: \textbf{revert}\\
reverts a list.\\
\noindent Usage: 
\begin{center}
\textbf{revert}(\emph{L}) : \textsf{list} $\rightarrow$ \textsf{list}
\end{center}
Parameters: 
\begin{itemize}
\item \emph{L} is a list.
\end{itemize}
\noindent Description: \begin{itemize}

\item \textbf{revert}(\emph{L}) returns the same list, but with its elements in reverse order.

\item If \emph{L} is an end-elliptic list, \textbf{revert} will fail with an error.
\end{itemize}
\noindent Example 1: 
\begin{center}\begin{minipage}{15cm}\begin{Verbatim}[frame=single]
> revert([| |]);
[| |]
\end{Verbatim}
\end{minipage}\end{center}
\noindent Example 2: 
\begin{center}\begin{minipage}{15cm}\begin{Verbatim}[frame=single]
> revert([|2,3,5,2,1,4|]);
[|4, 1, 2, 5, 3, 2|]
\end{Verbatim}
\end{minipage}\end{center}
