\subsection{double}
\label{labdouble}
\noindent Names: \textbf{double}, \textbf{D}\\
rounding to the nearest IEEE double.\\

\noindent Description: \begin{itemize}

\item \textbf{double} is both a function and a constant.

\item As a function, it rounds its argument to the nearest double precision number.
   Subnormal numbers are supported as well as standard numbers: it is the real
   rounding described in the standard.

\item As a constant, it symbolizes the double precision format. It is used in 
   contexts when a precision format is necessary, e.g. in the commands 
   \textbf{roundcoefficients} and \textbf{implementpoly}.
   See the corresponding help pages for examples.
\end{itemize}
\noindent Example 1: 
\begin{center}\begin{minipage}{15cm}\begin{Verbatim}[frame=single]
> display=binary!;
> D(0.1);
1.100110011001100110011001100110011001100110011001101_2 * 2^(-4)
> D(4.17);
1.000010101110000101000111101011100001010001111010111_2 * 2^(2)
> D(1.011_2 * 2^(-1073));
1.1_2 * 2^(-1073)
\end{Verbatim}
\end{minipage}\end{center}
See also: \textbf{doubleextended} (\ref{labdoubleextended}), \textbf{doubledouble} (\ref{labdoubledouble}), \textbf{tripledouble} (\ref{labtripledouble}), \textbf{roundcoefficients} (\ref{labroundcoefficients}), \textbf{implementpoly} (\ref{labimplementpoly})
