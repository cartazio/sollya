\subsection{annotatefunction}
\label{labannotatefunction}
\noindent Name: \textbf{annotatefunction}\\
\phantom{aaa}Annotates a \sollya function object with an approximation that is faster to evaluate\\[0.2cm]
\noindent Library name:\\
\verb|   sollya_obj_t sollya_lib_annotatefunction(sollya_obj_t, sollya_obj_t, |\\
\verb|                                            sollya_obj_t, sollya_obj_t, ...);|\\[0.2cm]
\noindent Usage: 
\begin{center}
\textbf{annotatefunction}(\emph{f}, \emph{g}, \emph{I}, \emph{d}) : (\textsf{function}, \textsf{function}, \textsf{range}, \textsf{range}) $\rightarrow$ \textsf{function}\\
\textbf{annotatefunction}(\emph{f}, \emph{g}, \emph{I}, \emph{d}, \emph{t}) : (\textsf{function}, \textsf{function}, \textsf{range}, \textsf{range}, \textsf{constant}) $\rightarrow$ \textsf{function}\\
\end{center}
Parameters: 
\begin{itemize}
\item \emph{f} is a function.
\item \emph{g} is a function, in most cases a polynomial.
\item \emph{I} is an interval.
\item \emph{d} is an interval.
\item \emph{t} is a constant.
\end{itemize}
\noindent Description: \begin{itemize}

\item \textbf{annotatefunction}(\emph{f}, \emph{g}, \emph{I}, \emph{delta}) TODO TODO
\end{itemize}
\noindent Example 1: 
\begin{center}\begin{minipage}{15cm}\begin{Verbatim}[frame=single]
> procedure EXP(X,n,p) {
            var res, oldPrec;
            oldPrec = prec;
            prec = p!;
            
        "Using procedure function exponential";
            res = exp(X);
            
            prec = oldPrec!;
            return res;
       };
> g = function(EXP);
> p = 46768052394588893382516870161332864698044514954899b-165 + x * (23384026197
294446691258465802074096632225783601255b-164 + x * (58460065493236116729484266
13035653821819225877423b-163 + x * (389733769954907444862769649080681513731982
1946501b-164 + x * (7794675399098148717422744621371434831048848817417b-167 + x
 * (24942961277114075921122941174178849425809856036737b-171 + x * (83143204257
04876115613838900105097456456371179471b-172 + x * (190041609730397013715793569
91645932289422670402995b-176 + x * (190041609726693241489121222544499121560039
26801563b-179 + x * (33785175062542597526738679493857229456702396042255b-183 +
 x * (6757035113643674378393625988264926886191860669891b-184 + x * (9828414707
511252769908089206114262766633532289937b-188 + x * (26208861108003813314724515
233584738706961162212965b-193 + x * (32257064253325954315953742396999456577223
350602741b-197 + x * (578429089657689569703509185903214676926704485495b-195 + 
x * 2467888542176675658523627105540996778984959471957b-201))))))))))))));
> h = annotatefunction(g, p, [-1/2;1/2], [-475294848522543b-124;475294848522543b
-124]);
> prec = 24;
The precision has been set to 24 bits.
> "h(0.25) = ";
h(0.25) = 
> h(0.25);
1.2840254
> prec = 72;
The precision has been set to 72 bits.
> "h(0.25) = ";
h(0.25) = 
> h(0.25);
Using procedure function exponential
Using procedure function exponential
1.2840254166877414840735
\end{Verbatim}
\end{minipage}\end{center}
See also: \textbf{taylorform} (\ref{labtaylorform}), \textbf{remez} (\ref{labremez}), \textbf{supnorm} (\ref{labsupnorm}), \textbf{infnorm} (\ref{labinfnorm})
