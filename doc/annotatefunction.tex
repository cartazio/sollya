\subsection{annotatefunction}
\label{labannotatefunction}
\noindent Name: \textbf{annotatefunction}\\
\phantom{aaa}Annotates a \sollya function object with an approximation that is faster to evaluate\\[0.2cm]
\noindent Library name:\\
\verb|   sollya_obj_t sollya_lib_annotatefunction(sollya_obj_t, sollya_obj_t, |\\
\verb|                                            sollya_obj_t, sollya_obj_t, ...);|\\[0.2cm]
\noindent Usage: 
\begin{center}
\textbf{annotatefunction}(\emph{f}, \emph{g}, \emph{I}, \emph{d}) : (\textsf{function}, \textsf{function}, \textsf{range}, \textsf{range}) $\rightarrow$ \textsf{function}\\
\textbf{annotatefunction}(\emph{f}, \emph{g}, \emph{I}, \emph{d}, \emph{t}) : (\textsf{function}, \textsf{function}, \textsf{range}, \textsf{range}, \textsf{constant}) $\rightarrow$ \textsf{function}\\
\end{center}
Parameters: 
\begin{itemize}
\item \emph{f} is a function.
\item \emph{g} is a function, in most cases a polynomial.
\item \emph{I} is an interval.
\item \emph{d} is an interval.
\item \emph{t} is a constant.
\end{itemize}
\noindent Description: \begin{itemize}

\item \textbf{annotatefunction}(\emph{f}, \emph{g}, \emph{I}, \emph{d}, \emph{t}) resp. \textbf{annotatefunction}(\emph{f}, \emph{g}, \emph{I}, \emph{d}) 
   (where \emph{t} is assumed to be zero) annotates the given \sollya
   function object \emph{f} with an approximation \emph{g} in such a way that upon
   evaluation of \emph{f} at a point in \emph{I}, \emph{g} is used instead of \emph{f} as
   long as the error between \emph{f} and \emph{g}, which must be bounded over all
   \emph{I} by \emph{d}, is not too large. The approximation \emph{g} is not evaluated
   at the same point as \emph{f} would be evaluated but after subtraction of 
   \emph{t} from that point.

\item In order not to provoke any incorrect behavior, the given \emph{f}, \emph{g},
   \emph{I}, \emph{d} and \emph{t} must satisfy the following property:
   $$\forall x \in I, f(x) - g(x - t) \in d.$$

\item The approximation \emph{g} can be any \sollya function but particular
   performance is expected when \emph{g} is a polynomial. Upon annotation with
   a polynomial, precomputations are performed to analyze certain
   properties of the given approximation polynomial.

\item The \textbf{annotatefunction} command returns the annotated \sollya function
   object. This object is indistinguishable from the original \emph{f} with
   respect to \sollya comparisons, printing and so on. Unless \sollya has
   been compiled in some particular debug mode, the \textbf{annotatefunction} command does
   not deeply copy the function object \emph{f}, so the original function
   object \emph{f} also bears the annotation.

\item \sollya function objects can be annotated more than once with
   different approximations on different domains, that do not need to be
   disjoint. Upon evaluation of the annotated function object, \sollya
   chooses the approximation annotation that provides for sufficient
   accuracy at the evaluation point.

\item As an absolute error bound \emph{d} is given for annotation, \sollya cannot
   reuse an approximation annotation for the derivative of the annotated
   function. If any (higher) derivative of a function needs to bear an
   annotation, that derivative needs to be annotated using \textbf{annotatefunction}.
\end{itemize}
\noindent Example 1: 
\begin{center}\begin{minipage}{15cm}\begin{Verbatim}[frame=single]
> procedure EXP(X,n,p) {
            var res, oldPrec;
            oldPrec = prec;
            prec = p!;
            
            "Using procedure function exponential";
            res = exp(X);
            
            prec = oldPrec!;
            return res;
       };
> g = function(EXP);
> p = 46768052394588893382516870161332864698044514954899b-165 + x * (23384026197
294446691258465802074096632225783601255b-164 + x * (58460065493236116729484266
13035653821819225877423b-163 + x * (389733769954907444862769649080681513731982
1946501b-164 + x * (7794675399098148717422744621371434831048848817417b-167 + x
 * (24942961277114075921122941174178849425809856036737b-171 + x * (83143204257
04876115613838900105097456456371179471b-172 + x * (190041609730397013715793569
91645932289422670402995b-176 + x * (190041609726693241489121222544499121560039
26801563b-179 + x * (33785175062542597526738679493857229456702396042255b-183 +
 x * (6757035113643674378393625988264926886191860669891b-184 + x * (9828414707
511252769908089206114262766633532289937b-188 + x * (26208861108003813314724515
233584738706961162212965b-193 + x * (32257064253325954315953742396999456577223
350602741b-197 + x * (578429089657689569703509185903214676926704485495b-195 + 
x * 2467888542176675658523627105540996778984959471957b-201))))))))))))));
> h = annotatefunction(g, p, [-1/2;1/2], [-475294848522543b-124;475294848522543b
-124]);
> prec = 24;
The precision has been set to 24 bits.
> "h(0.25) = ";
h(0.25) = 
> h(0.25);
1.2840254
> prec = 72;
The precision has been set to 72 bits.
> "h(0.25) = ";
h(0.25) = 
> h(0.25);
1.2840254166877414840735
> prec = 165;
The precision has been set to 165 bits.
> "h(0.25) = ";
h(0.25) = 
> h(0.25);
Using procedure function exponential
1.28402541668774148407342056806243645833628086528147
\end{Verbatim}
\end{minipage}\end{center}
See also: \textbf{taylorform} (\ref{labtaylorform}), \textbf{remez} (\ref{labremez}), \textbf{supnorm} (\ref{labsupnorm}), \textbf{infnorm} (\ref{labinfnorm})
