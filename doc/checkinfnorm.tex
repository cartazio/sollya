\subsection{checkinfnorm}
\label{labcheckinfnorm}
\noindent Name: \textbf{checkinfnorm}\\
checks whether the infinity norm of a function is bounded by a value\\
\noindent Usage: 
\begin{center}
\textbf{checkinfnorm}(\emph{function},\emph{range},\emph{constant}) : (\textsf{function}, \textsf{range}, \textsf{constant}) $\rightarrow$ \textsf{boolean}\\
\end{center}
Parameters: 
\begin{itemize}
\item \emph{function} represents the function whose infinity norm is to be checked
\item \emph{range} represents the infinity norm is to be considered on
\item \emph{constant} represents the upper bound the infinity norm is to be checked to
\end{itemize}
\noindent Description: \begin{itemize}

\item The command \textbf{checkinfnorm} checks whether the infinity norm of the given
   function \emph{function} in the range \emph{range} can be proven (by \sollya) to
   be less than the given bound \emph{bound}. This means, if \textbf{checkinfnorm}
   evaluates to \textbf{true}, the infinity norm has been proven (by \sollya's
   interval arithmetic) to be less than the bound. If \textbf{checkinfnorm} evaluates
   to \textbf{false}, there are two possibilities: either the bound is less than
   or equal to the infinity norm of the function or the bound is greater
   than the infinity norm but \sollya could not conclude using its
   internal interval arithmetic.
    
   \textbf{checkinfnorm} is sensitive to the global variable \textbf{diam}. The smaller \textbf{diam},
   the more time \sollya will spend on the evaluation of \textbf{checkinfnorm} in
   order to prove the bound before returning \textbf{false} although the infinity
   norm is bounded by the bound. If \textbf{diam} is equal to $0$, \sollya will
   eventually spend infinite time on instances where the given bound
   \emph{bound} is less or equal to the infinity norm of the function
   \emph{function} in range \emph{range}. In contrast, with \textbf{diam} being zero,
   \textbf{checkinfnorm} evaluates to \textbf{true} iff the infinity norm of the function in
   the range is bounded by the given bound.
\end{itemize}
\noindent Example 1: 
\begin{center}\begin{minipage}{15cm}\begin{Verbatim}[frame=single]
> checkinfnorm(sin(x),[0;1.75], 1);
true
> checkinfnorm(sin(x),[0;1.75], 1/2); checkinfnorm(sin(x),[0;20/39],1/2);
false
true
\end{Verbatim}
\end{minipage}\end{center}
\noindent Example 2: 
\begin{center}\begin{minipage}{15cm}\begin{Verbatim}[frame=single]
> p = remez(exp(x), 5, [-1;1]);
> b = dirtyinfnorm(p - exp(x), [-1;1]);
> checkinfnorm(p - exp(x), [-1;1], b);
false
> b1 = round(b, 15, RU);
> checkinfnorm(p - exp(x), [-1;1], b1);
true
> b2 = round(b, 25, RU);
> checkinfnorm(p - exp(x), [-1;1], b2);
false
> diam = 1b-20!;
> checkinfnorm(p - exp(x), [-1;1], b2);
true
\end{Verbatim}
\end{minipage}\end{center}
See also: \textbf{infnorm} (\ref{labinfnorm}), \textbf{dirtyinfnorm} (\ref{labdirtyinfnorm}), \textbf{supnorm} (\ref{labsupnorm}), \textbf{accurateinfnorm} (\ref{labaccurateinfnorm}), \textbf{remez} (\ref{labremez}), \textbf{diam} (\ref{labdiam})
