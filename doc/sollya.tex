\documentclass[a4paper]{article}
 
\usepackage[english]{babel}
\usepackage[naturalnames]{hyperref}
\usepackage{fullpage}
\usepackage{xspace}
\usepackage{amssymb}
\usepackage{fancyvrb}

\newcommand{\com}[1]{\texttt{#1}}
\newcommand{\key}[1]{\texttt{#1}}
\newcommand{\sollya}{\texttt{Sollya}\xspace}
\newcommand{\rlwrap}{\texttt{rlwrap}\xspace}

\newcommand{\code}[1]{
\begin{center}
\begin{tabular}{|p{14.8cm}|}
\hline
#1
\hline
\end{tabular}
\end{center}
}

\newcommand{\ligne}[1]{\texttt{#1}\\}

\title{Users' manual for the \sollya tool \\ {\large Release 1.0.1} \\ ~ \\ {\large Laboratoire de l'Informatique du Parall\'elisme \\ UMR CNRS - ENS Lyon - UCB Lyon 1 - INRIA 5668}}

\author{Sylvain Chevillard \\ \small{\url{sylvain.chevillard@ens-lyon.fr}} \and Christoph Lauter \\ \small{\url{christoph.lauter@ens-lyon.fr}} \and Nicolas Jourdan \\ \small{\url{nicolas.jourdan@ens-lyon.fr}}}

\date{\today}


\begin{document}

\maketitle

\section*{License}

The \sollya tool is copyright \copyright~ 2008 Laboratoire de
l'Informatique du Parall\'elisme - UMR CNRS - ENS Lyon - UCB Lyon 1 -
INRIA 5668.  

The \sollya tool is open software. It is distributed and can be used,
modified and redistributed under the terms of the CeCILL-C licence
available at \url{http://www.cecill.info/} and reproduced in the
\texttt{COPYING} file of the distribution. The distribution contains
parts of other libraries as a support for but not integral part of
\sollya. These libraries are reigned by the GNU Lesser General Public
License that is available at \url{http://www.gnu.org/licenses/} and
reproduced in the \texttt{COPYING} file of the distribution.

\tableofcontents

\section{Compilation and installation of the \sollya tool}

The \sollya distribution can be compiled and installed using the usual
\texttt{./configure}, \texttt{make}, \texttt{make install}
procedure. Besides a \texttt{C} compiler, \sollya needs the following
software libraries and tools to be installed. The \texttt{./configure}
script checks for the installation of the libraries. However \sollya
will build without error if some of its external tools is not
installed. In this case an error will be produced at runtime.
\begin{itemize}
\item \texttt{GMP}
\item \texttt{MPFR}
\item \texttt{MPFI}
\item \texttt{fplll}
\item \texttt{libxml2}
\item \texttt{gnuplot}
\end{itemize}
The use of the external tool \texttt{rlwrap} is highly recommended but
not indispensable.


\section{Introduction}
\sollya is an interactive tool for handling numerical functions and working with arbitrary precision. It can evaluate functions accurately, compute polynomial approximations of functions, automatically implement polynomials for use in math libraries, plot functions, compute infinite norms, etc. The language \sollya comes with is a full-featured script programming language with support for procedures etc. 

Let us begin this manual with an example. \sollya does not allow command line edition; since that may quickly become uncomfortable, we highly suggest to use the software \rlwrap with \sollya:

\begin{center}\begin{minipage}{15cm}\begin{Verbatim}[frame=single]
~/% rlwrap sollya
>
\end{Verbatim}
\end{minipage}\end{center}

\sollya manipulates only univariate functions. The first time that an unbound variable is used, this name is fixed. It will be used to refer to the free variable. For instance, try

\begin{center}\begin{minipage}{15cm}\begin{Verbatim}[frame=single]
> f = sin(x)/x;
> g = cos(y)-1;
Warning: the identifier "y" is neither assigned to, nor bound to a library funct
ion nor equal to the current free variable.
Will interpret "y" as "x".
> g;
cos(x) - 1
\end{Verbatim}
\end{minipage}\end{center}


Now, the name $x$ can only be used to refer to the free variable:

\begin{center}\begin{minipage}{15cm}\begin{Verbatim}[frame=single]
> x = 3;
Warning: the identifier "x" is already bound to the free variable, to a library 
function or to an external procedure.
The command will have no effect.
Warning: the last assignment will have no effect.
\end{Verbatim}
\end{minipage}\end{center}


If you really want to unbound $x$, you can use the \com{rename} command and change the name of the free variable:

\begin{center}\begin{minipage}{15cm}\begin{Verbatim}[frame=single]
> rename(x,y);
Information: the free variable has been renamed from "x" to "y".
> g;
cos(y) - 1
> x=3;
Warning: syntax error, unexpected IDENTIFIERTOKEN, expecting SEMICOLONTOKEN.
The last symbol read has been "x".
Will skip input until next semicolon after the unexpected token. May leak memory
.
> x;
Warning: the identifier "x" is neither assigned to, nor bound to a library funct
ion nor equal to the current free variable.
Will interpret "x" as "y".
y
\end{Verbatim}
\end{minipage}\end{center}


As you have seen, you can name functions and easily work with. The basic thing to do with a function is to evaluate it at some point:

\begin{center}\begin{minipage}{15cm}\begin{Verbatim}[frame=single]
> f(-2);
Warning: rounding has happened. The value displayed is a faithful rounding of th
e true result.
0.45464871341284084769800993295587242135112748572394
> evaluate(f,-2);
0.45464871341284084769800993295587242135112748572394
\end{Verbatim}
\end{minipage}\end{center}


The printed value is generally a faithful rounding of the exact value at the working precision. The working precision is controlled by the global variable \com{prec}:

\begin{center}\begin{minipage}{15cm}\begin{Verbatim}[frame=single]
> prec;
165
> prec=200;
The precision has been set to 200 bits.
> prec;
200
> f(-2);
Warning: rounding has happened. The value displayed is a faithful rounding of th
e true result.
0.4546487134128408476980099329558724213511274857239451341894865
\end{Verbatim}
\end{minipage}\end{center}


Sometimes, a faithful rounding cannot easily be computed. In such a case, an approximated value is printed:

\begin{center}\begin{minipage}{15cm}\begin{Verbatim}[frame=single]
> sin(pi);
Warning: rounding has happened. The value displayed is *NOT* a faithful rounding
 of the true result.
-3.79705991005939815725347821572628308530195421950339e-12716
\end{Verbatim}
\end{minipage}\end{center}


The philosophy of \sollya is: whenever something is not exact, print a warning. This explains the warnings in the previous examples. If the result can be shown to be exact, there is no warning:

\begin{center}\begin{minipage}{15cm}\begin{Verbatim}[frame=single]
> sin(0);
0
\end{Verbatim}
\end{minipage}\end{center}


Let us finish this Section with a small complete example that shows a bit of what can be done with \sollya:

\begin{center}\begin{minipage}{15cm}\begin{Verbatim}[frame=single]
> restart;
The tool has been restarted.
> prec=50;
The precision has been set to 50 bits.
> f=cos(2*exp(x));
> d=[-1/8;1/8];
> p=remez(f,2,d);
> derivativeZeros = dirtyfindzeros(diff(p-f),d);
> derivativeZeros = inf(d).:derivativeZeros:.sup(d);
> max=0;
> for t in derivativeZeros do {
     r = evaluate(abs(p-f), t);
     if r > max then { max=r; argmax=t; };
  };
> print("The infinite norm of", p-f, "is", max, "and is reached at", argmax);
The infinite norm of (-0.416265572875373) + x * ((-1.798067209218835) + x * (-3.
89710727747639e-2)) - cos(2 * exp(x)) is 8.630659443624325e-4 and is reached at 
-5.801672331417684e-2
\end{Verbatim}
\end{minipage}\end{center}


In this example, we define a function $f$, an interval $d$ and we compute the best degree-4 polynomial approximation of $f$ on $d$ with respect to the infinite norm. In other words, $\max_{x \in d} \{|p(x)-f(x)|\}$ is minimal amongst polynomials with degree not greater than $4$. Then, we compute the list of the zeros of the derivative of $p-f$ and add the bounds of $d$ to this list. Finally, we evaluate $|p-f|$ for each point in the list and store the maximum and the point where it is reached. We conclude by printing the result in a formatted way.

Note that you do not really need to use such a script for computing infinite norm; as we will see, the command \com{dirtyinfnorm} does this for you.

\section{General principles}
The first goal of \sollya is to help people to use numerical functions and numerical algorithms in a safe way. It is first designed to be used interactively but it can also be used in scripts\footnote{Remark: some of the behaviors of \sollya slightly change when used in scripts. For example, no prompt is printed.}.

One of the originalities of \sollya is to work with multi-precision arithmetic (it uses the \texttt{MPFR} library). For safety purposes, \sollya knows how to use interval arithmetic. It uses the interval arithmetic to produce tight and safe results with the precision required by the user.

The general philosophy of \sollya is: \emph{When you can make a computation exactly and sufficiently quickly, do it; when you cannot, do not, unless you have been explicitly asked for.}

The precision of the tools is set by the global variable \key{prec}. It indicates the number of bits used to represent the constants in \sollya. In general, the variable \key{prec} determines the precision of the outputs of commands: more precisely, the command will internally determine what precision should be used during the computations in order to ensure that the output is a faithful result with \key{prec} bits.

For decidability and efficiency reasons, this general principle cannot be applied every time, so be careful. Moreover certain commands are known to be unsafe: they give in general excellent results and give almost \key{prec} correct bits in output for everyday examples. However they are just heuristic and should not be used when the result must be safe. See the documentation of each command to know precisely how confident you can be with its result.

A second principle (that comes together with the first one) is: \emph{When a computation leads to inexact results, inform the user with a warning}. This can be quite irritating in some circumstances: in particular if you are using \sollya within other scripts. The global variable \key{verbosity} lets you change the level of verbosity of \sollya. When set to $0$, \sollya becomes completely silent on stdout and prints only very important messages on stderr. Increase \key{verbosity} if you want more informations about what \sollya is doing. Note that when you affect a value to a global variable, a message is always printed even if \com{verbosity} is set to $0$. In order to silently affect a global variable, use \texttt{!}:

\begin{center}\begin{minipage}{15cm}\begin{Verbatim}[frame=single]
> prec=30;
The precision has been set to 30 bits.
> prec=30!;
>  
\end{Verbatim}
\end{minipage}\end{center}


For conviviality reasons, values are displayed in decimal by default. This lets a normal human being understand the numbers he or she manipulates. But since constants are internally represented in binary, this causes permanent conversions that are sources of roundings. Thus you are loosing in accuracy and \sollya is always complaining about inexact results. If you just want to store or communicate your results (to another tools for instance) you can use bit-exact representations available in \sollya. The global variable \key{display} defines the way constants are displayed. Here is an example of the five available modes:

\begin{center}\begin{minipage}{15cm}\begin{Verbatim}[frame=single]
> prec=30!;
> a = 17.25;
> display=decimal;
Display mode is decimal numbers.
> a;
1.725e1
> display=binary;
Display mode is binary numbers.
> a;
1.000101_2 * 2^(4)
> display=powers;
Display mode is dyadic numbers in integer-power-of-2 notation.
> a;
69 * 2^(-2)
> display=dyadic;
Display mode is dyadic numbers.
> a;
69b-2
> display=hexadecimal;
Display mode is hexadecimal numbers.
> a;
0x1.14p4
\end{Verbatim}
\end{minipage}\end{center}


As always, the symbol \texttt{e} means $\times 10^\square $. The same way the symbol \texttt{b} means  $\times 2^\square $. The symbol \texttt{p} means $\times 16^\square$ and is used only with the \texttt{0x} prefix. The prefix \texttt{0x} indicates that the digits of the following number until 
a symbol \texttt{p} or whitespace are hexadecimal. The suffix \texttt{\_2} indicates to \sollya that the previous number has been written in binary. \sollya can parse these notations even if you are not in the corresponding \key{display} mode, so you can always use them.

You can also use memory-dump hexadecimal notation frequently used to represent IEEE 754 \texttt{double} and \texttt{single} precision numbers. Since this notation does not allow for exactly representing numbers with arbitrary precision, there is no corresponding \key{display} mode. However, the commands \com{printhexa} respectively \com{printfloat} round the value to the nearest \texttt{double} respectively \texttt{single}. The number is then printed in hexadecimal as the integer number corresponding to the memory representation of the IEEE 754 \texttt{double} or \texttt{single} number:

\begin{center}\begin{minipage}{15cm}\begin{Verbatim}[frame=single]
> printhexa(a);
0x4031400000000000
> printfloat(a);
0x418a0000
\end{Verbatim}
\end{minipage}\end{center}


\sollya can parse these memory-dump hexadecimal notation back in any \key{display} mode.

\section{Variables}\label{variables}

As already explained, \sollya can manipulate univariate functional
expressions. These expressions contain a unique free variable the name
of which is fixed by its first usage in an expression that is not a
left-hand-side of an assignment. This global and unique free variable is 
a variable in the mathematical sense of the term. 

\sollya also provides variables in the sense programming languages
give to the term.  These variables, that must be different in their
name from the global free variable, may be global or declared and
attached to a block of statements, i.e. a begin-end-block. These
programming language variables may hold any object of the \sollya
language, as for example functional expressions, strings, intervals,
constant values, procedures, external functions and procedures, etc.

Global variables need not to be declared. They start existing,
i.e. can be correctly used in expressions that are not left-hand-sides
of assignments, when they are assigned a value in an assignment. Since
they are global, this kind of variables is recommended only for small
\sollya scripts.  Larger scripts with code reutilization should use
declared variables in order to avoid name clashes for example in loop
variables.

Declared variables are attached to a begin-end-block. The block
structure builds scopes for declared variables. Declared variables in
inner scopes shadow variables (global and declared) of outer
scopes. The global free variable, i.e. the mathematical variable for
univariate functional expressions, cannot be shadowed. Variables are
declared using \key{var} keyword. See section \ref{labvar} for details
on its usage and semantic.

The following code examples illustrate the usage of variables.


\begin{center}\begin{minipage}{15cm}\begin{Verbatim}[frame=single]
> f = exp(x);
> f;
exp(x)
> a = "Hello world";
> a;
Hello world
> b = 5;
> f(b);
Warning: rounding has happened. The value displayed is a faithful rounding of th
e true result.
1.48413159102576603421115580040552279623487667593878e2
> {var b; b = 4; f(b); };
Warning: rounding has happened. The value displayed is a faithful rounding of th
e true result.
5.45981500331442390781102612028608784027907370386137e1
> {var x; x = 3; };
Warning: the identifier "x" is already bound to the current free variable.
It cannot be declared as a local variable. The declaration of "x" will have no e
ffect.
Warning: the identifier "x" is already bound to the free variable, to a library 
function or to an external procedure.
The command will have no effect.
Warning: the last assignment will have no effect.
> {var a, b; a=5; b=3; {var a; var b; b = true; a = 1; a; b;}; a; b; };
1
true
5
3
> a;
Hello world
\end{Verbatim}
\end{minipage}\end{center}


\section{Data types}
\sollya has a (very) basic system of types. If you try to perform an illicit operation (such as adding a number and a string, for instance), you will get a type error. Let us see the available data types.

\subsection{Booleans}
There are two special values \key{true} and \key{false}. Boolean expressions can be constructed using the boolean connectors \key{\&\&} (and), \key{||} (or), \key{!} (not), and comparisons.

The comparison operators \key{<}, \key{<=}, \key{>} and \key{>=} can only be used between two numbers or constant expressions.

The comparison operators \key{==} and \key{!=} are polymorphic. You can use it to compare any two objects, like two strings, two intervals, etc. Note that testing the equality between two functions will return \key{true} if and only if the expression trees representing the two functions are exactly the same. See \ref{laberror} for an exception concerning the special object \key{error}. Example:

\begin{center}\begin{minipage}{15cm}\begin{Verbatim}[frame=single]
> 1+x==1+x;
true
\end{Verbatim}
\end{minipage}\end{center}


\subsection{Numbers}
\sollya represents numbers as floating-point values. For integer values and values in dyadic, binary, hexadecimal or memory dump notation, it 
automatically uses a precision needed for representing the value exactly. Otherwise the values are represented with the current precision \com{prec}. A number in an expression is rounded to the precision \com{prec} when the expression gets evaluated:

\begin{center}\begin{minipage}{15cm}\begin{Verbatim}[frame=single]
> prec=12!;
> 4097.1;
Warning: Rounding occurred when converting the constant "4097.1" to floating-poi
nt with 12 bits.
If safe computation is needed, try to increase the precision.
4098
> 4097.1+1;
Warning: Rounding occurred when converting the constant "4097.1" to floating-poi
nt with 12 bits.
If safe computation is needed, try to increase the precision.
4099
\end{Verbatim}
\end{minipage}\end{center}


Note that each variable has its own precision that corresponds to its intrinsic precision or, if it cannot be represented, to the value of \com{prec} when the variable was set. Thus you can work with variables having a precision bigger than the current precision.

The same way, if you define a function that refers to some constant, this constant is stored in the function with the current precision and will keep this value in the future, even if \com{prec} becomes smaller.

If you define a function that refers to some variable, the precision of the variable is kept, independently of the current precision:

\begin{center}\begin{minipage}{15cm}\begin{Verbatim}[frame=single]
> prec = 50!;
> a = 4097.1;
Warning: Rounding occurred when converting the constant "4097.1" to floating-poi
nt with 50 bits.
If safe computation is needed, try to increase the precision.
> prec = 12!;
> f = x + a;
> g = x + 4097.1;
Warning: Rounding occurred when converting the constant "4097.1" to floating-poi
nt with 12 bits.
If safe computation is needed, try to increase the precision.
> prec = 120;
The precision has been set to 120 bits.
> f;
4.097099999999998544808477163314819335e3 + x
> g;
4098 + x
\end{Verbatim}
\end{minipage}\end{center}


\subsection{Intervals}
Intervals are composed of two numbers or constant expressions representing the lower and the upper bound. These values are separated either by commas or semi-colons:

\begin{center}\begin{minipage}{15cm}\begin{Verbatim}[frame=single]
> d=[1;2];
> d2=[1,1+1];
> d==d2;
true
\end{Verbatim}
\end{minipage}\end{center}


If bounds are defined by constant expressions, these are evaluated to floating-point numbers using the current precision. Numbers or variables containing numbers keep their precision for the interval bounds. Interval bound evaluation is performed in a way that ensures the inclusion property: all points
in the original, unevaluated interval will be contained in the interval with its bounds evaluated to floating-point numbers. Remark that 
evaluation bounds defined by constant expressions includes $\pi$:

\begin{center}\begin{minipage}{15cm}\begin{Verbatim}[frame=single]
> prec = 300!;
> a = 4097.1;
Warning: Rounding occurred when converting the constant "4097.1" to floating-poi
nt with 300 bits.
If safe computation is needed, try to increase the precision.
> prec = 12!;
> d = [4097.1; a];
Warning: Rounding occurred when converting the constant "4097.1" to floating-poi
nt with 12 bits.
If safe computation is needed, try to increase the precision.
Warning: the bounds of the given range are in wrong order. Will reverse them.
> prec = 300!;
> d;
[4.0971e3;4098]
> prec = 30!;
> [-pi;pi];
Warning: the given expression is not a constant but an expression to evaluate.
Warning: the given expression is not a constant but an expression to evaluate.
[-3.141592659;3.141592659]
\end{Verbatim}
\end{minipage}\end{center}


You can get the upper-bound (respectively the lower-bound)) of an interval with the function \com{sup} (respectively \com{inf}). The middle of the interval is got with the function \com{mid}. Note that these functions can also be used on numbers (in that case, the number is interpreted as an interval containing only one single point. Thus the functions \com{inf}, \com{mid} and \com{sup} are just the identity):

\begin{center}\begin{minipage}{15cm}\begin{Verbatim}[frame=single]
> d=[1;3];
> inf(d);
1
> mid(d);
2
> sup(4);
4
\end{Verbatim}
\end{minipage}\end{center}


\subsection{Functions}
\sollya knows only functions with one single variable. The first time in a session that an unbound name is used (without being assigned) it determines the name used to refer to the free variable.

The basic functions available in \sollya are the following:
\begin{itemize}
\item \com{+}, \com{-}, \com{*}, \com{/}, \com{\^{}}
\item \com{sqrt}
\item \com{abs}
\item \com{sin}, \com{cos}, \com{tan}, \com{sinh}, \com{cosh}, \com{tanh}
\item \com{asin}, \com{acos}, \com{atan}, \com{asinh}, \com{atanh}
\item \com{exp}, \com{expm1} (defined as $\mathrm{expm1}(x) = \exp(x)-1$)
\item \com{log} (neperian logarithm), \com{log2} (binary logarithm), \com{log10} (decimal logarithm), \com{log1p} (defined as $\mathrm{log1p}(x) = \log(1+x)$)
\item \com{erf}, \com{erfc}
\end{itemize}

The constant $\pi$ is available through the keyword \key{pi} as a $0$-ary function: its behavior is exactly the same as if it were a constant with an infinite precision:

\begin{center}\begin{minipage}{15cm}\begin{Verbatim}[frame=single]
> display=binary!;
> prec=12!;
> a=pi;
> a;
Warning: rounding has happened. The value displayed is a faithful rounding of th
e true result.
1.10010010001_2 * 2^(1)
> prec=30!;
> a;
Warning: rounding has happened. The value displayed is a faithful rounding of th
e true result.
1.10010010000111111011010101001_2 * 2^(1)
\end{Verbatim}
\end{minipage}\end{center}


\subsection{Strings}
Anything written between quotes is interpreted as a string. The infix operator \com{@} concatenates two strings. To get the length of a string, use the \com{length} function. You can access the $i$-th character of a string using brackets (see the example below). There is no character type in \sollya: the $i$-th character of a string is returned as a string itself.

\begin{center}\begin{minipage}{15cm}\begin{Verbatim}[frame=single]
> s1 = "Hello "; s2 = "World!";
> s = s1@s2;
> length(s);
12
> s[0];
H
> s[11];
!
\end{Verbatim}
\end{minipage}\end{center}


Strings may contain the following escape sequences:
\texttt{$\backslash\backslash$}, \texttt{$\backslash$\"},
\texttt{$\backslash$?}, \texttt{$\backslash$\'},
\texttt{$\backslash$n}, \texttt{$\backslash$t},
\texttt{$\backslash$a}, \texttt{$\backslash$b},
\texttt{$\backslash$f}, \texttt{$\backslash$r},
\texttt{$\backslash$v}, \texttt{$\backslash$x}[hexadecimal number] and
\texttt{$\backslash$}[octal number]. Refer to the C99 standard for their
meaning.

\subsection{Particular values}
\sollya knows some particular values. These values do not really have a type but they can be stored in variables and in lists. A (possibly not exhaustive) list of such values is the following:

\begin{itemize}
\item \com{on}, \com{off} (see sections \ref{labon} and \ref{laboff})
\item \com{dyadic}, \com{powers}, \com{binary}, \com{decimal}, \com{hexadecimal} (see sections \ref{labdyadic}, \ref{labpowers}, \ref{labbinary}, \ref{labdecimal} and \ref{labhexadecimal})
\item \com{file}, \com{postscript}, \com{postscriptfile} (see sections \ref{labfile}, \ref{labpostscript} and \ref{labpostscriptfile})
\item \com{RU}, \com{RD}, \com{RN}, \com{RZ} (see sections \ref{labru}, \ref{labrd}, \ref{labrn} and \ref{labrz})
\item \com{absolute}, \com{relative} (see sections \ref{lababsolute} and \ref{labrelative})
\item \com{floating}, \com{fixed} (see sections \ref{labfloating} and \ref{labfixed})
\item \com{double}, \com{doubleextended}, \com{doubledouble}, \com{tripledouble} (see sections \ref{labdouble}, \ref{labdoubleextended}, \ref{labdoubledouble} and \ref{labtripledouble})
\item \com{D}, \com{DE}, \com{DD}, \com{TD} (see sections \ref{labdouble}, \ref{labdoubleextended}, \ref{labdoubledouble} and \ref{labtripledouble})
\item \com{perturb} (see section \ref{labperturb})
\item \com{honorcoeffprec} (see section \ref{labhonorcoeffprec})
\item \com{default} (see section \ref{labdefault})
\item \com{error} (see section \ref{laberror})
\item \com{void} (see section \ref{labvoid})
\end{itemize}

\subsection{Lists}
Objects can be grouped into lists. A list can contain elements with different types. As for strings, you can concatenate two lists with \com{@}. The function \com{length} gives also the length of a list.

You can prepend an element to a list using \com{.:} (in $\mathcal{O}(1)$) and you can append an element to a list using \com{:.} (in $\mathcal{O}(n)$). The following example illustrates some features:

\begin{center}\begin{minipage}{15cm}\begin{Verbatim}[frame=single]
> l = [| "foo" |];
> l = l:.1;
> l = "bar".:l;
> l;
[|"bar", "foo", 1|]
> l[1];
foo
> l@l;
[|"bar", "foo", 1, "bar", "foo", 1|]
\end{Verbatim}
\end{minipage}\end{center}


Lists can be considered as arrays and elements of lists can be
referenced using brackets. Possible indices start at $0$. The
following example illustrates this point:

\begin{center}\begin{minipage}{15cm}\begin{Verbatim}[frame=single]
> l = [|1,2,3,4,5|];
> l;
[|1, 2, 3, 4, 5|]
> l[3];
4
\end{Verbatim}
\end{minipage}\end{center}


Remark that the complexity for accessing an element of the list using
indices is $\mathcal{O}(n)$.

Lists may contain ellipses indicated by \texttt{,...,} between
elements that are constant and evaluate to integers that are
incrementally ordered. \sollya translates such ellipses to the full
list upon evaluation. Using ellipses between elements that are not
constants is not allowed. This feature is provided for ease of
programming; remark that the complexity of expanding such lists is
high. For illustration, see the following example:

\begin{center}\begin{minipage}{15cm}\begin{Verbatim}[frame=single]
> [|1,...,5|];
[|1, 2, 3, 4, 5|]
> [|-5,...,5|];
[|-5, -4, -3, -2, -1, 0, 1, 2, 3, 4, 5|]
> [|3,...,1|];
Warning: at least one of the given expressions or a subexpression is not correct
ly typed
or its evaluation has failed because of some error on a side-effect.
error
> [|true,...,false|];
Warning: at least one of the given expressions or a subexpression is not correct
ly typed
or its evaluation has failed because of some error on a side-effect.
error
\end{Verbatim}
\end{minipage}\end{center}


Lists may be continued to infinity by means of the \texttt{...}
indicator after the last element given. At least one element must
explicitly be given. If the last element given is a constant
expression that evaluates to an integer, the list is considered as
continued to infinity by all integers greater than that last
element. If the last element is another object, the list is considered
as continued to infinity by re-duplicating this last element. Remark
that bracket notation is supported for such end-elliptic lists even
for implicitly given elements. However, evaluation complexity is
high. Combinations of ellipses inside a list and in its end are
possible. The usage of lists described here is best illustrated by the
following examples:

\begin{center}\begin{minipage}{15cm}\begin{Verbatim}[frame=single]
> l = [|1,2,true,3...|];
> l;
[|1, 2, true, 3...|]
> l[2];
true
> l[3];
3
> l[4];
4
> l[1200];
1200
> l = [|1,...,5,true...|];
> l;
[|1, 2, 3, 4, 5, true...|]
> l[1200];
true
\end{Verbatim}
\end{minipage}\end{center}


\section{Iterative language elements: assignments, conditional statements and loops}

\subsection{Blocks}

Statements in \sollya can be regrouped in blocks, so-called
begin-end-blocks.  This can be done using the keywords \key{begin} and
\key{end} or their shorter variants \key{$\lbrace$} and
\key{$\rbrace$}. Blocks declared this way are considered as one single
statement. As already explained in section \ref{variables}, using
begin-end-blocks also opens the possibility of declaring variables
through the keyword \key{var}. 

\subsection{Assignments}

\sollya has two different assignment operators, \texttt{=} and
\texttt{:=}. The assignment operator \texttt{=} assigns its
right-hand-object ``as is'', i.e. without evaluating functional
expressions. For instance, \texttt{i = i + 1;} will dereferentiate the
identifier \texttt{i} with some content, notate it $y$, build up the
expression (function) $y + 1$ and assign this expression back to
\texttt{i}. In the example, if \texttt{i} stood for the value $1000$,
the statement \texttt{i = i + 1;} will assign $1000 + 1$ -- and not
$1001$ -- to \texttt{i}. The assignment operator \texttt{:=} evaluates
constant functional expressions before assigning them. On other
expressions it behaves like \texttt{=}. Still in the example, the
statement \texttt{i := i + 1;} really assigns $1001$ to \texttt{i}.

Both \sollya assignment operators support indexing of lists or strings
elements using brackets on the left-hand-side of the assignment
operator. The indexed element of the list or string gets replaced by
the right-hand-side of the assignment operator.  When indexing strings
this way, that right-hand side must evaluate to a string of length
$1$. End-elliptic lists are supported with their usual semantic for
this kind of assignment.  When referencing and assigning a value in
the implicit part of the end-elliptic list, the list gets expanded to
the corresponding length. The indexing of lists on left-hand sides of
assignments is reduced to the first order. Multiple indexing of lists
of lists is not supported for complexity reasons. 

The following examples well illustrate the behavior of assignment
statements:

\begin{center}\begin{minipage}{15cm}\begin{Verbatim}[frame=single]
> autosimplify = off;
Automatic pure tree simplification has been deactivated.
> i = 1000;
> i = i + 1;
> print(i);
1000 + 1
> i := i + 1;
> print(i);
1002
> l = [|1,...,5|];
> print(l);
[|1, 2, 3, 4, 5|]
> l[3] = l[3] + 1;
> l[4] := l[4] + 1;
> print(l);
[|1, 2, 3, 4 + 1, 6|]
> l[5] = true;
> l;
[|1, 2, 3, 5, 6, true|]
> s = "Hello world";
> s;
Hello world
> s[1] = "a";
> s;
Hallo world
> l = [|true,1,...,5,9...|];
> l;
[|true, 1, 2, 3, 4, 5, 9...|]
> l[13] = "Hello";
> l;
[|true, 1, 2, 3, 4, 5, 9, 10, 11, 12, 13, 14, 15, "Hello"...|]
\end{Verbatim}
\end{minipage}\end{center}


\subsection{Conditional statements}

\sollya supports conditional statements expressed with the keywords
\key{if}, \key{then} and optionally \key{else}. Remark that only
conditional statements are supported not conditional expressions. 

The following examples illustrate both syntax and semantic of
conditional statements in \sollya. Concerning syntax, consider also
section \ref{grammar} and remark that there must not be any semicolon
before the \key{else} keyword.

\begin{center}\begin{minipage}{15cm}\begin{Verbatim}[frame=single]
> a = 3;
> b = 4;
> if (a == b) then print("Hello world");
> b = 3;
> if (a == b) then print("Hello world");
Hello world
> if (a == b) then print("You are telling the truth") else print("Liar!");
You are telling the truth
\end{Verbatim}
\end{minipage}\end{center}


\subsection{Loops}

\sollya supports three kinds of loops. General \emph{while-condition}
loops can be expressed using the keywords \key{while} and
\key{do}. Remark that the condition test is executed always before the
loop, there is no \emph{do-until-condition} loop. Consider the following 
examples for both syntax and semantic:

\begin{center}\begin{minipage}{15cm}\begin{Verbatim}[frame=single]
> verbosity = 0!;
> prec = 30!;
> i = 5;
> while (expm1(i) > 0) do { expm1(i); i := i - 1; };
1.474131591e2
5.359815e1
1.908553692e1
6.3890561
1.718281827
> print(i);
0
\end{Verbatim}
\end{minipage}\end{center}


The second kind of loops are loops on a variable ranging from a
numerical start value and a end value. These kind of loops can be
expressed using the keywords \key{for}, \key{from}, \key{to}, \key{do}
and optionally \key{by}. The \key{by} statement indicates the width of
the steps on the variable from the start value to the end value. Once
again, syntax and semantic are best explained with an example:

\begin{center}\begin{minipage}{15cm}\begin{Verbatim}[frame=single]
> for i from 1 to 5 do print ("Hello world",i);
Hello world 1
Hello world 2
Hello world 3
Hello world 4
Hello world 5
> for i from 2 to 1 by -0.5 do print("Hello world",i);
Hello world 2
Hello world 1.5
Hello world 1
\end{Verbatim}
\end{minipage}\end{center}


The third kind of loops are loops on a variables ranging on values
contained in a list. In order to ensure the termination of the loop,
that list must not be end-elliptic. The loop is expressed using the
keywords \key{for}, \key{in} and \key{do} as in the following
examples:

\begin{center}\begin{minipage}{15cm}\begin{Verbatim}[frame=single]
> l = [|true, false, 1,...,4, "Hello", exp(x)|];
> for i in l do i;
true
false
1
2
3
4
Hello
exp(x)
\end{Verbatim}
\end{minipage}\end{center}


For both types of \key{for} loops, assigning the loop variable is
allowed and possible. If the loop terminates, the loop variable will
contain the value that made the loop condition fail. Consider the
following examples:

\begin{center}\begin{minipage}{15cm}\begin{Verbatim}[frame=single]
> for i from 1 to 5 do { if (i == 3) then i = 4 else i; };
1
2
5
> i;
6
\end{Verbatim}
\end{minipage}\end{center}


\section{Functional language elements: procedures}

\sollya has some elements of functional languages. In order to 
avoid confusion with mathematical functions, the associated 
programming objects are called procedures in \sollya. 

\sollya procedures are common objects that can be, for example,
assigned to variables or stored in lists. Procedures are declared by
the \key{proc} keyword; see section \ref{labproc} for details. The
returned procedure object must then be assigned to a variable and can
hence be applied to arguments with common application syntax. The
\key{procedure} keyword provides an abbreviation for declaring and
assigning a procedure; see section \ref{labprocedure} for details.

\sollya procedures can return objects using the \key{return} keyword
at the end of the begin-end-block of the procedure. Section
\ref{labreturn} gives details on the usage of \key{return}. Procedures
further can take any type of object in argument, in particular also
other procedures that are then applied to arguments. Procedures can
be declared inside other procedures. 

Remark that declaring a procedure does not involve any evaluation or
other interpretation of the procedure body. In particular, this means
that constants are evaluated to floating-point values inside \sollya
when the procedure is applied to actual parameters and the global
precision valid at this moment.

\sollya procedures are well illustrated by the following examples:

\begin{center}\begin{minipage}{15cm}\begin{Verbatim}[frame=single]
> succ = proc(n) { return n + 1; };
> succ(5);
6
> 3 + succ(0);
4
> succ;
proc(n)
begin
nop;
return (n) + (1);
end
> add = proc(m,n) { var res; res := m + n; return res; };
> add(5,6);
11
> hey = proc() { print("Hello world."); };
> hey();
Hello world.
> print(hey());
Hello world.
void
> hey;
proc()
begin
print("Hello world.");
return void;
end
> fac = proc(n) { var res; if (n == 0) then res := 1 else res := n * fac(n - 1);
 return res; };
> fac(5);
120
> fac(11);
39916800
> fac;
proc(n)
begin
var res;
if (n) == (0) then
res := 1
else
res := (n) * (fac((n) - (1)));
return res;
end
\end{Verbatim}
\end{minipage}\end{center}


\sollya also supports external procedures, i.e. procedures written in
\texttt{C} (or some other language) and dynamically bound to \sollya
identifiers. See \ref{labexternalproc} for details.

\section{Commands and functions}

\subsection{absolute}
\label{lababsolute}
\noindent Name: \textbf{absolute}\\
indicates an absolute error for \textbf{externalplot} or \textbf{fpminimax}\\
\noindent Usage: 
\begin{center}
\textbf{absolute} : \textsf{absolute$|$relative}
\\ 
\end{center}
\noindent Description: \begin{itemize}

\item The use of \textbf{absolute} in the command \textbf{externalplot} indicates that during
   plotting in \textbf{externalplot} an absolute error is to be considered.
    
   See \textbf{externalplot} for details.
   Used with \textbf{fpminimax}, \textbf{absolute} indicates that \textbf{fpminimax} must try to minimize
   the absolute error.
    
   See \textbf{fpminimax} for details.
\end{itemize}
\noindent Example 1: 
\begin{center}\begin{minipage}{15cm}\begin{Verbatim}[frame=single]
> bashexecute("gcc -fPIC -c externalplotexample.c");
> bashexecute("gcc -shared -o externalplotexample externalplotexample.o -lgmp -l
mpfr");
> externalplot("./externalplotexample",absolute,exp(x),[-1/2;1/2],12,perturb);
\end{Verbatim}
\end{minipage}\end{center}
See also: \textbf{externalplot} (\ref{labexternalplot}), \textbf{fpminimax} (\ref{labfpminimax}), \textbf{relative} (\ref{labrelative}), \textbf{bashexecute} (\ref{labbashexecute})

\subsection{abs}
\label{lababs}
\noindent Name: \textbf{abs}\\
the absolute value.\\
\noindent Description: \begin{itemize}

\item \\textbf{abs} is the absolute value function. \\textbf{abs}(x)=$\\left \\lbrace \\begin{array}{rl} x & x > 0 \\\\ -x & x \\leq 0 \\end{array}  \\right.$.\n\end{itemize}

\subsection{accurateinfnorm}
\label{labaccurateinfnorm}
\noindent Name: \textbf{accurateinfnorm}\\
computes a faithful rounding of the infinity norm of a function \\
\noindent Usage: 
\begin{center}
\textbf{accurateinfnorm}(\emph{function},\emph{range},\emph{constant}) : (\textsf{function}, \textsf{range}, \textsf{constant}) $\rightarrow$ \textsf{constant}
\textbf{accurateinfnorm}(\emph{function},\emph{range},\emph{constant},\emph{exclusion range 1},...,\emph{exclusion range n}) : (\textsf{function}, \textsf{range}, \textsf{constant}, \textsf{range}, ..., \textsf{range}) $\rightarrow$ \textsf{constant}
\end{center}
Parameters: 
\begin{itemize}
\item \emph{function} represents the function whose infinity norm is to be computed
\item \emph{range} represents the infinity norm is to be considered on
\item \emph{constant} represents the number of bits in the significant of the result
\item \emph{exclusion range 1} through \emph{exclusion range n} represent ranges to be excluded 
\end{itemize}
\noindent Description: \begin{itemize}

\item The command \textbf{accurateinfnorm} computes an upper bound to the infinity norm of
   function \emph{function} in \emph{range}. This upper bound is the least
   floating-point number greater than the value of the infinity norm that
   lies in the set of dyadic floating point numbers having \emph{constant}
   significant mantissa bits. This means the value \textbf{accurateinfnorm} evaluates to
   is at the time an upper bound and a faithful rounding to \emph{constant}
   bits of the infinity norm of function \emph{function} on range \emph{range}.
    
   If given, the fourth and further arguments of the command \textbf{accurateinfnorm},
   \emph{exclusion range 1} through \emph{exclusion range n} the infinity norm of
   the function \emph{function} is not to be considered on.
\end{itemize}
\noindent Example 1: 
\begin{center}\begin{minipage}{15cm}\begin{Verbatim}[frame=single]
> p = remez(exp(x), 5, [-1;1]);
> accurateinfnorm(p - exp(x), [-1;1], 20);
4.52055246569216251373291015625e-5
> accurateinfnorm(p - exp(x), [-1;1], 30);
4.520552107578623690642416477203369140625e-5
> accurateinfnorm(p - exp(x), [-1;1], 40);
4.5205521044089369553375945542939007282257080078125e-5
\end{Verbatim}
\end{minipage}\end{center}
\noindent Example 2: 
\begin{center}\begin{minipage}{15cm}\begin{Verbatim}[frame=single]
> p = remez(exp(x), 5, [-1;1]);
> midpointmode = on!;
> infnorm(p - exp(x), [-1;1]);
0.45205~5/7~e-4
> accurateinfnorm(p - exp(x), [-1;1], 40);
4.5205521044089369553375945542939007282257080078125e-5
\end{Verbatim}
\end{minipage}\end{center}
See also: \textbf{infnorm} (\ref{labinfnorm}), \textbf{dirtyinfnorm} (\ref{labdirtyinfnorm}), \textbf{checkinfnorm} (\ref{labcheckinfnorm}), \textbf{remez} (\ref{labremez}), \textbf{diam} (\ref{labdiam})

\subsection{acosh}
\label{labacosh}
\noindent Name: \textbf{acosh}\\
the arg-hyperbolic cosine function.\\

\noindent Description: \begin{itemize}

\item \textbf{acosh} is the inverse of the function \textbf{cosh}: \textbf{acosh}(y) is the unique number 
   $x \in [0; +\infty]$ such that \textbf{cosh}(x)=y.

\item It is defined only for $y \in [0;+\infty]$.
\end{itemize}
See also: \textbf{cosh} (\ref{labcosh})

\subsection{ acos }
\noindent Name: \textbf{acos}\\
the arccosine function.\\

\noindent Description: \begin{itemize}

\item \textbf{acos} is the inverse of the function \textbf{cos}: \textbf{acos}(y) is the unique number 
   $x \in [0; \pi]$ such that \textbf{cos}(x)=y.

\item It is defined only for $y \in [-1;1]$.
\end{itemize}
See also: \textbf{cos}

\subsection{and}
\label{laband}
\noindent Name: \textbf{$\&\&$}\\
boolean AND operator\\
\noindent Usage: 
\begin{center}
\emph{expr1} \textbf{$\&\&$} \emph{expr2} : (\textsf{boolean}, \textsf{boolean}) $\rightarrow$ \textsf{boolean}
\\ 
\end{center}
Parameters: 
\begin{itemize}
\item \emph{expr1} and \emph{expr2} represent boolean expressions
\end{itemize}
\noindent Description: \begin{itemize}

\item \textbf{$\&\&$} evaluates to the boolean AND of the two
   boolean expressions \emph{expr1} and \emph{expr2}. \textbf{$\&\&$} evaluates to 
   true iff both \emph{expr1} and \emph{expr2} evaluate to true.
\end{itemize}
\noindent Example 1: 
\begin{center}\begin{minipage}{15cm}\begin{Verbatim}[frame=single]
> true && false;
false
\end{Verbatim}
\end{minipage}\end{center}
\noindent Example 2: 
\begin{center}\begin{minipage}{15cm}\begin{Verbatim}[frame=single]
> (1 == exp(0)) && (0 == log(1));
true
\end{Verbatim}
\end{minipage}\end{center}
See also: \textbf{$||$} (\ref{labor}), \textbf{!} (\ref{labnot})

\subsection{:.}
\label{labappend}
\noindent Name: \textbf{:.}\\
add an element at the end of a list.\\
\noindent Usage: 
\begin{center}
\emph{L}\textbf{:.}\emph{x} : (\textsf{list}, \textsf{any type}) $\rightarrow$ \textsf{list}\\
\end{center}
Parameters: 
\begin{itemize}
\item \emph{L} is a list (possibly empty).
\item \emph{x} is an object of any type.
\end{itemize}
\noindent Description: \begin{itemize}

\item \\textbf{:.} adds the element \\emph{x} at the end of the list \\emph{L}.\n
\item Note that since \\emph{x} may be of any type, it can in particular be a list.\n\end{itemize}
\noindent Example 1: 
\begin{center}\begin{minipage}{15cm}\begin{Verbatim}[frame=single]
\end{Verbatim}
\end{minipage}\end{center}
\noindent Example 2: 
\begin{center}\begin{minipage}{15cm}\begin{Verbatim}[frame=single]
\end{Verbatim}
\end{minipage}\end{center}
\noindent Example 3: 
\begin{center}\begin{minipage}{15cm}\begin{Verbatim}[frame=single]
\end{Verbatim}
\end{minipage}\end{center}
See also: \textbf{.:} (\ref{labprepend}), \textbf{@} (\ref{labconcat})

\subsection{$\sim$}
\label{labapprox}
\noindent Name: \textbf{$\sim$}\\
\phantom{aaa}floating-point evaluation of a constant expression\\[0.2cm]
\noindent Library name:\\
\verb|   sollya_obj_t sollya_lib_approx(sollya_obj_t)|\\[0.2cm]
\noindent Usage: 
\begin{center}
\textbf{$\sim$} \emph{expression} : \textsf{function} $\rightarrow$ \textsf{constant}\\
\textbf{$\sim$} \emph{something} : \textsf{any type} $\rightarrow$ \textsf{any type}\\
\end{center}
Parameters: 
\begin{itemize}
\item \emph{expression} stands for an expression that is a constant
\item \emph{something} stands for some language element that is not a constant expression
\end{itemize}
\noindent Description: \begin{itemize}

\item \textbf{$\sim$} \emph{expression} evaluates the \emph{expression} that is a constant
   term to a floating-point constant. The evaluation may involve a
   rounding. If \emph{expression} is not a constant, the evaluated constant is
   a faithful rounding of \emph{expression} with \textbf{precision} bits, unless the
   \emph{expression} is exactly $0$ as a result of cancellation. In the
   latter case, a floating-point approximation of some (unknown) accuracy
   is returned.

\item \textbf{$\sim$} does not do anything on all language elements that are not a
   constant expression.  In other words, it behaves like the identity
   function on any type that is not a constant expression. It can hence
   be used in any place where one wants to be sure that expressions are
   simplified using floating-point computations to constants of a known
   precision, regardless of the type of actual language elements.

\item \textbf{$\sim$} \textbf{error} evaluates to error and provokes a warning.

\item \textbf{$\sim$} is a prefix operator not requiring parentheses. Its
   precedence is the same as for the unary $+$ and $-$
   operators. It cannot be repeatedly used without brackets.
\end{itemize}
\noindent Example 1: 
\begin{center}\begin{minipage}{15cm}\begin{Verbatim}[frame=single,commandchars=\\\|\~]
> print(exp(5));
exp(5)
> print(~ exp(5));
148.41315910257660342111558004055227962348766759388
\end{Verbatim}
\end{minipage}\end{center}
\noindent Example 2: 
\begin{center}\begin{minipage}{15cm}\begin{Verbatim}[frame=single,commandchars=\\\|\~]
> autosimplify = off!;
\end{Verbatim}
\end{minipage}\end{center}
\noindent Example 3: 
\begin{center}\begin{minipage}{15cm}\begin{Verbatim}[frame=single,commandchars=\\\|\~]
> print(~sin(5 * pi));
0
\end{Verbatim}
\end{minipage}\end{center}
\noindent Example 4: 
\begin{center}\begin{minipage}{15cm}\begin{Verbatim}[frame=single,commandchars=\\\|\~]
> print(~exp(x));
exp(x)
> print(~ "Hello");
Hello
\end{Verbatim}
\end{minipage}\end{center}
\noindent Example 5: 
\begin{center}\begin{minipage}{15cm}\begin{Verbatim}[frame=single,commandchars=\\\|\~]
> print(~exp(x*5*Pi));
exp((pi) * 5 * x)
> print(exp(x* ~(5*Pi)));
exp(x * 15.7079632679489661923132169163975144209858469968757)
\end{Verbatim}
\end{minipage}\end{center}
\noindent Example 6: 
\begin{center}\begin{minipage}{15cm}\begin{Verbatim}[frame=single,commandchars=\\\|\~]
> print(~exp(5)*x);
148.41315910257660342111558004055227962348766759388 * x
> print( (~exp(5))*x);
148.41315910257660342111558004055227962348766759388 * x
> print(~(exp(5)*x));
exp(5) * x
\end{Verbatim}
\end{minipage}\end{center}
See also: \textbf{evaluate} (\ref{labevaluate}), \textbf{prec} (\ref{labprec}), \textbf{error} (\ref{laberror})

\subsection{asciiplot}
\label{labasciiplot}
\noindent Name: \textbf{asciiplot}\\
plots a function in a range using ASCII characters\\
\noindent Usage: 
\begin{center}
\textbf{asciiplot}(\emph{function}, \emph{range}) : (\textsf{function}, \textsf{range}) $\rightarrow$ \textsf{void}
\end{center}
Parameters: 
\begin{itemize}
\item \emph{function} represents a function to be plotted
\item \emph{range} represents a range the function is to be plotted in 
\end{itemize}
\noindent Description: \begin{itemize}

\item \textbf{asciiplot} plots the function \emph{function} in range \emph{range} using ASCII
   characters.  On systems that provide the necessary 
   \texttt{TIOCGWINSZ ioctl}, \sollya determines the size of the
   terminal for the plot size if connected to a terminal. If it is not
   connected to a terminal or if the test is not possible, the plot is of
   fixed size $77\times25$ characters.  The function is
   evaluated on a number of points equal to the number of columns
   available. Its value is rounded to the next integer in the range of
   lines available. A letter \texttt{x} is written at this place. If zero is in
   the hull of the image domain of the function, a x-axis is
   displayed. If zero is in range, an y-axis is displayed.  If the
   function is constant or if the range is reduced to one point, the
   function is evaluated to a constant and the constant is displayed
   instead of a plot.
\end{itemize}
\noindent Example 1: 
\begin{center}\begin{minipage}{15cm}\begin{Verbatim}[frame=single]
> asciiplot(exp(x),[1;2]);
                                                                           x
                                                                         xx 
                                                                      xxx   
                                                                    xx      
                                                                  xx        
                                                               xxx          
                                                             xx             
                                                          xxx               
                                                        xx                  
                                                     xxx                    
                                                  xxx                       
                                               xxx                          
                                            xxx                             
                                         xxx                                
                                      xxx                                   
                                  xxxx                                      
                              xxxx                                          
                           xxx                                              
                       xxxx                                                 
                  xxxxx                                                     
             xxxxx                                                          
         xxxx                                                               
    xxxxx                                                                   
xxxx                                                                        
\end{Verbatim}
\end{minipage}\end{center}
\noindent Example 2: 
\begin{center}\begin{minipage}{15cm}\begin{Verbatim}[frame=single]
> asciiplot(expm1(x),[-1;2]);
                         |                                                 x
                         |                                                x 
                         |                                               x  
                         |                                              x   
                         |                                             x    
                         |                                           xx     
                         |                                          x       
                         |                                         x        
                         |                                       xx         
                         |                                     xx           
                         |                                   xx             
                         |                                 xx               
                         |                                x                 
                         |                             xxx                  
                         |                           xx                     
                         |                        xxx                       
                         |                     xxx                          
                         |                 xxxx                             
                         |             xxxx                                 
                         |         xxxx                                     
                         |   xxxxxx                                         
---------------------xxxxxxxx-----------------------------------------------
         xxxxxxxxxxxx    |                                                  
xxxxxxxxx                |                                                  
\end{Verbatim}
\end{minipage}\end{center}
\noindent Example 3: 
\begin{center}\begin{minipage}{15cm}\begin{Verbatim}[frame=single]
> asciiplot(5,[-1;1]);
5
\end{Verbatim}
\end{minipage}\end{center}
\noindent Example 4: 
\begin{center}\begin{minipage}{15cm}\begin{Verbatim}[frame=single]
> asciiplot(exp(x),[1;1]);
2.71828182845904523536028747135266249775724709369998
\end{Verbatim}
\end{minipage}\end{center}
See also: \textbf{plot} (\ref{labplot})

\subsection{ asinh }
\noindent Name: \textbf{asinh}\\
the arg-hyperbolic sine function.\\

\noindent Description: \begin{itemize}

\item \textbf{asinh} is the inverse of the function \textbf{sinh}: \textbf{asinh}(y) is the unique number 
   $x \in [-\infty; +\infty]$ such that \textbf{sinh}(x)=y.

\item It is defined for every real number y.
\end{itemize}
See also: \textbf{sinh}

\subsection{ asin }
\noindent Name: \textbf{asin}\\
the arcsine function.\\

\noindent Description: \begin{itemize}

\item \textbf{asin} is the inverse of the function \textbf{sin}: \textbf{asin}(y) is the unique number 
   $x \in [-\pi/2; \pi/2]$ such that \textbf{sin}(x)=y.

\item It is defined only for $y \in [-1;1]$.
\end{itemize}
See also: \textbf{sin}

\subsection{ atanh }
\noindent Name: \textbf{atanh}\\
the hyperbolic arctangent function.\\

\noindent Description: \begin{itemize}

\item \textbf{atanh} is the inverse of the function \textbf{tanh}: \textbf{atanh}(y) is the unique number 
   $x \in [-\infty; +\infty]$ such that \textbf{tanh}(x)=y.

\item It is defined only for $y \in [-1; 1]$.
\end{itemize}
See also: \textbf{tanh}

\subsection{ atan }
\noindent Name: \textbf{atan}\\
the arctangent function.\\

\noindent Description: \begin{itemize}

\item \textbf{atan} is the inverse of the function \textbf{tan}: \textbf{atan}(y) is the unique number 
   $x \in [-\pi/2; +\pi/2]$ such that \textbf{tan}(x)=y.

\item It is defined for every real number y.
\end{itemize}
See also: \textbf{tan}

\subsection{autodiff}
\label{labautodiff}
\noindent Name: \textbf{autodiff}\\
nothing\\
\noindent Usage: 
\begin{center}
\textbf{autodiff}(\emph{f}, \emph{n}, \emph{I}) : (\textsf{function}, \textsf{integer}, \textsf{range}) $\rightarrow$ \textsf{list}\\
\end{center}
Parameters: 
\begin{itemize}
\item \emph{f} is the function to be approximated.
\item \emph{n} is the order of differentiation.
\item \emph{I} is the interval over which the function is differentiated.
\end{itemize}
\noindent Description: \begin{itemize}

\item Nothing.
\end{itemize}
\noindent Example 1: 
\begin{center}\begin{minipage}{15cm}\begin{Verbatim}[frame=single]
> L = autodiff(exp(x), 5, 0);
\end{Verbatim}
\end{minipage}\end{center}
See also: \textbf{diff} (\ref{labdiff})

\subsection{autosimplify}
\label{labautosimplify}
\noindent Name: \textbf{autosimplify}\\
activates, deactivates or inspects the value of the automatic simplification state variable\\
\noindent Usage: 
\begin{center}
\textbf{autosimplify} = \emph{activation value} : \textsf{on$|$off} $\rightarrow$ \textsf{void}\\
\textbf{autosimplify} = \emph{activation value} ! : \textsf{on$|$off} $\rightarrow$ \textsf{void}\\
\textbf{autosimplify} : \textsf{on$|$off}\\
\end{center}
Parameters: 
\begin{itemize}
\item \emph{activation value} represents \textbf{on} or \textbf{off}, i.e. activation or deactivation
\end{itemize}
\noindent Description: \begin{itemize}

\item An assignment \\textbf{autosimplify} = \\emph{activation value}, where \\emph{activation value}\n   is one of \\textbf{on} or \\textbf{off}, activates respectively deactivates the\n   automatic safe simplification of expressions of functions generated by\n   the evaluation of commands or in argument of other commands.\n    \n   \\sollya commands like \\textbf{remez}, \\textbf{taylor} or \\textbf{rationalapprox} sometimes\n   produce expressions that can be simplified. Constant subexpressions\n   can be evaluated to dyadic floating-point numbers, monomials with\n   coefficients $0$ can be eliminated. Further, expressions\n   indicated by the user perform better in many commands when simplified\n   before being passed in argument to a command. When the automatic\n   simplification of expressions is activated, \\sollya automatically\n   performs a safe (not value changing) simplification process on such\n   expressions.\n    \n   The automatic generation of subexpressions can be annoying, in\n   particular if it takes too much time for not enough benefit. Further the\n   user might want to inspect the structure of the expression tree\n   returned by a command. In this case, the automatic simplification\n   should be deactivated.\n    \n   If the assignment \\textbf{autosimplify} = \\emph{activation value} is followed by an\n   exclamation mark, no message indicating the new state is\n   displayed. Otherwise the user is informed of the new state of the\n   global mode by an indication.\n\end{itemize}
\noindent Example 1: 
\begin{center}\begin{minipage}{15cm}\begin{Verbatim}[frame=single]
\end{Verbatim}
\end{minipage}\end{center}
\noindent Example 2: 
\begin{center}\begin{minipage}{15cm}\begin{Verbatim}[frame=single]
\end{Verbatim}
\end{minipage}\end{center}
See also: \textbf{print} (\ref{labprint}), \textbf{prec} (\ref{labprec}), \textbf{points} (\ref{labpoints}), \textbf{diam} (\ref{labdiam}), \textbf{display} (\ref{labdisplay}), \textbf{verbosity} (\ref{labverbosity}), \textbf{canonical} (\ref{labcanonical}), \textbf{taylorrecursions} (\ref{labtaylorrecursions}), \textbf{timing} (\ref{labtiming}), \textbf{fullparentheses} (\ref{labfullparentheses}), \textbf{midpointmode} (\ref{labmidpointmode}), \textbf{hopitalrecursions} (\ref{labhopitalrecursions}), \textbf{remez} (\ref{labremez}), \textbf{rationalapprox} (\ref{labrationalapprox}), \textbf{taylor} (\ref{labtaylor})

\subsection{bashexecute}
\label{labbashexecute}
\noindent Name: \textbf{bashexecute}\\
\phantom{aaa}executes a shell command.\\[0.2cm]
\noindent Library name:\\
\verb|   void sollya_lib_bashexecute(sollya_obj_t)|\\[0.2cm]
\noindent Usage: 
\begin{center}
\textbf{bashexecute}(\emph{command}) : \textsf{string} $\rightarrow$ \textsf{void}\\
\end{center}
Parameters: 
\begin{itemize}
\item \emph{command} is a command to be interpreted by the shell.
\end{itemize}
\noindent Description: \begin{itemize}

\item \textbf{bashexecute}(\emph{command}) lets the shell interpret \emph{command}. It is useful to execute
   some external code within \sollya.

\item \textbf{bashexecute} does not return anything. It just executes its argument. However, if
   \emph{command} produces an output in a file, this result can be imported in \sollya
   with help of commands like \textbf{execute}, \textbf{readfile} and \textbf{parse}.
\end{itemize}
\noindent Example 1: 
\begin{center}\begin{minipage}{15cm}\begin{Verbatim}[frame=single]
> bashexecute("LANG=C date");
Wed May 11 10:16:31 CEST 2016
\end{Verbatim}
\end{minipage}\end{center}
See also: \textbf{execute} (\ref{labexecute}), \textbf{readfile} (\ref{labreadfile}), \textbf{parse} (\ref{labparse}), \textbf{bashevaluate} (\ref{labbashevaluate})

\subsection{binary}
\label{labbinary}
\noindent Name: \textbf{binary}\\
\phantom{aaa}special value for global state \textbf{display}\\[0.2cm]
\noindent Library names:\\
\verb|   sollya_obj_t sollya_lib_binary()|\\
\verb|   int sollya_lib_is_binary(sollya_obj_t)|\\[0.2cm]
\noindent Description: \begin{itemize}

\item \textbf{binary} is a special value used for the global state \textbf{display}.  If
   the global state \textbf{display} is equal to \textbf{binary}, all data will be
   output in binary notation.
    
   As any value it can be affected to a variable and stored in lists.
\end{itemize}
See also: \textbf{decimal} (\ref{labdecimal}), \textbf{dyadic} (\ref{labdyadic}), \textbf{powers} (\ref{labpowers}), \textbf{hexadecimal} (\ref{labhexadecimal}), \textbf{display} (\ref{labdisplay})

\subsection{boolean}
\label{labboolean}
\noindent Name: \textbf{boolean}\\
keyword representing a \textsf{boolean} type \\

\noindent Usage: 
\begin{center}
\textbf{boolean} : \textsf{type type}\\
\end{center}
\noindent Description: \begin{itemize}

\item \textbf{boolean} represents the \textsf{boolean} type for declarations
   of external procedures by means of \textbf{externalproc}.
   Remark that in contrast to other indicators, type indicators like
   \textbf{boolean} cannot be handled outside the \textbf{externalproc} context.  In
   particular, they cannot be assigned to variables.
\end{itemize}
See also: \textbf{externalproc} (\ref{labexternalproc}), \textbf{constant} (\ref{labconstant}), \textbf{function} (\ref{labfunction}), \textbf{integer} (\ref{labinteger}), \textbf{list of} (\ref{lablistof}), \textbf{range} (\ref{labrange}), \textbf{string} (\ref{labstring})

\subsection{ canonical }
\noindent Name: \textbf{canonical}\\
brings all polynomial subexpressions of an expression to canonical form or activates, deactivates or checks canonical form printing\\

\noindent Usage: 
\begin{center}
\textbf{canonical}(\emph{function}) : \textsf{function} $\rightarrow$ \textsf{function}\\
\textbf{canonical} = \emph{activation value} : \textsf{on|off} $\rightarrow$ \textsf{void}\\
\textbf{canonical} = \emph{activation value} ! : \textsf{on|off} $\rightarrow$ \textsf{void}\\
\textbf{canonical} = ? : \textsf{void} $\rightarrow$ \textsf{on|off}\\
\end{center}
Parameters: 
\emph{function} represents the expression to be rewritten in canonical form\\
\emph{activation value} represents \textbf{on} or \textbf{off}, i.e. activation or deactivation\\

\noindent Description: \begin{itemize}

\item The command \textbf{canonical} rewrites the expression representing the function
   \emph{function} in a way such that all polynomial subexpressions (or the
   whole expression itself, if it is a polynomial) are written in
   canonical form, i.e. as a sum of monomials in the canonical base. The
   canonical base is the base of the integer powers of the global free
   variable. The command \textbf{canonical} does not endanger the safety of
   computations even in Sollya's floating-point environment: the
   function returned is mathematically equal to the function \emph{function}.

\item An assignment \textbf{canonical} = \emph{activation value}, where \emph{activation value}
   is one of \textbf{on} or \textbf{off}, activates respectively deactivates the
   automatic printing of polynomial expressions in canonical form,
   i.e. as a sum of monomials in the canonical base. If automatic
   printing in canonical form is deactivated, automatic printing yield to
   displaying polynomial subexpressions in Horner form.
   If the assignment \textbf{canonical} = \emph{activation value} is followed by an
   exclamation mark, no message indicating the new state is
   displayed. Otherwise the user is informed of the new state of the
   global mode by an indication.

\item The expression \textbf{canonical} = ? evaluates to a variable of type
   \textsf{on|off}, indicating whether or not the automatic printing of
   subexpressions in canonical form is activated. If automatic printing
   in canonical form is deactivated, automatic printing yield to
   displaying polynomial subexpressions in Horner form.
\end{itemize}
\noindent Example 1: 
\begin{center}\begin{minipage}{14.8cm}\begin{Verbatim}[frame=single]
   > print(canonical(1 + x * (x + 3 * x^2));
   > print(canonical((x + 1)^7));
   1 + 7 * x + 21 * x^2 + 35 * x^3 + 35 * x^4 + 21 * x^5 + 7 * x^6 + x^7
\end{Verbatim}
\end{minipage}\end{center}
\noindent Example 2: 
\begin{center}\begin{minipage}{14.8cm}\begin{Verbatim}[frame=single]
   > print(canonical(exp((x + 1)^5) - log(asin(((x + 2) + x)^4 * (x + 1)) + x)));
   exp(1 + 5 * x + 10 * x^2 + 10 * x^3 + 5 * x^4 + x^5) - log(asin(16 + 80 * x + 160 * x^2 + 160 * x^3 + 80 * x^4 + 16 * x^5) + x)
\end{Verbatim}
\end{minipage}\end{center}
\noindent Example 3: 
\begin{center}\begin{minipage}{14.8cm}\begin{Verbatim}[frame=single]
   > canonical = ?;
   off
   > (x + 2)^9;
   512 + x * (2304 + x * (4608 + x * (5376 + x * (4032 + x * (2016 + x * (672 + x * (144 + x * (18 + x))))))))
   > canonical = on;
   Canonical automatic printing output has been activated.
   > (x + 2)^9;
   512 + 2304 * x + 4608 * x^2 + 5376 * x^3 + 4032 * x^4 + 2016 * x^5 + 672 * x^6 + 144 * x^7 + 18 * x^8 + x^9
   > canonical = ?;
   on
   > canonical = off!;
   > (x + 2)^9;
   512 + x * (2304 + x * (4608 + x * (5376 + x * (4032 + x * (2016 + x * (672 + x * (144 + x * (18 + x))))))))
\end{Verbatim}
\end{minipage}\end{center}
See also: \textbf{horner}, \textbf{print}

\subsection{ceil}
\label{labceil}
\noindent Name: \textbf{ceil}\\
the usual function ceil.\\

\noindent Description: \begin{itemize}

\item \textbf{ceil} is defined as usual: \textbf{ceil}(x) is the smallest integer y such that $y \ge x$.

\item It is defined for every real number x.
\end{itemize}
See also: \textbf{floor} (\ref{labfloor})

\subsection{checkinfnorm}
\label{labcheckinfnorm}
\noindent Name: \textbf{checkinfnorm}\\
checks whether the infinity norm of a function is bounded by a value\\
\noindent Usage: 
\begin{center}
\textbf{checkinfnorm}(\emph{function},\emph{range},\emph{constant}) : (\textsf{function}, \textsf{range}, \textsf{constant}) $\rightarrow$ \textsf{boolean}\\
\end{center}
Parameters: 
\begin{itemize}
\item \emph{function} represents the function whose infinity norm is to be checked
\item \emph{range} represents the infinity norm is to be considered on
\item \emph{constant} represents the upper bound the infinity norm is to be checked to
\end{itemize}
\noindent Description: \begin{itemize}

\item The command \\textbf{checkinfnorm} checks whether the infinity norm of the given\n   function \\emph{function} in the range \\emph{range} can be proven (by \\sollya) to\n   be less than the given bound \\emph{bound}. This means, if \\textbf{checkinfnorm}\n   evaluates to \\textbf{true}, the infinity norm has been proven (by \\sollya's\n   interval arithmetic) to be less than the bound. If \\textbf{checkinfnorm} evaluates\n   to \\textbf{false}, there are two possibilities: either the bound is less than\n   or equal to the infinity norm of the function or the bound is greater\n   than the infinity norm but \\sollya could not conclude using its\n   internal interval arithmetic.\n    \n   \\textbf{checkinfnorm} is sensitive to the global variable \\textbf{diam}. The smaller \\textbf{diam},\n   the more time \\sollya will spend on the evaluation of \\textbf{checkinfnorm} in\n   order to prove the bound before returning \\textbf{false} although the infinity\n   norm is bounded by the bound. If \\textbf{diam} is equal to $0$, \\sollya will\n   eventually spend infinite time on instances where the given bound\n   \\emph{bound} is less or equal to the infinity norm of the function\n   \\emph{function} in range \\emph{range}. In contrast, with \\textbf{diam} being zero,\n   \\textbf{checkinfnorm} evaluates to \\textbf{true} iff the infinity norm of the function in\n   the range is bounded by the given bound.\n\end{itemize}
\noindent Example 1: 
\begin{center}\begin{minipage}{15cm}\begin{Verbatim}[frame=single]
\end{Verbatim}
\end{minipage}\end{center}
\noindent Example 2: 
\begin{center}\begin{minipage}{15cm}\begin{Verbatim}[frame=single]
\end{Verbatim}
\end{minipage}\end{center}
See also: \textbf{infnorm} (\ref{labinfnorm}), \textbf{dirtyinfnorm} (\ref{labdirtyinfnorm}), \textbf{accurateinfnorm} (\ref{labaccurateinfnorm}), \textbf{remez} (\ref{labremez}), \textbf{diam} (\ref{labdiam})

\subsection{coeff}
\label{labcoeff}
\noindent Name: \textbf{coeff}\\
gives the coefficient of degree $n$ of a polynomial\\
\noindent Usage: 
\begin{center}
\textbf{coeff}(\emph{f},\emph{n}) : (\textsf{function}, \textsf{integer}) $\rightarrow$ \textsf{constant}\\
\end{center}
Parameters: 
\begin{itemize}
\item \emph{f} is a function (usually a polynomial).
\item \emph{n} is an integer
\end{itemize}
\noindent Description: \begin{itemize}

\item If \\emph{f} is a polynomial, \\textbf{coeff}(\\emph{f}, \\emph{n}) returns the coefficient of\n   degree \\emph{n} in \\emph{f}.\n
\item If \\emph{f} is a function that is not a polynomial, \\textbf{coeff}(\\emph{f}, \\emph{n}) returns 0.\n\end{itemize}
\noindent Example 1: 
\begin{center}\begin{minipage}{15cm}\begin{Verbatim}[frame=single]
\end{Verbatim}
\end{minipage}\end{center}
\noindent Example 2: 
\begin{center}\begin{minipage}{15cm}\begin{Verbatim}[frame=single]
\end{Verbatim}
\end{minipage}\end{center}
See also: \textbf{degree} (\ref{labdegree})

\subsection{concat}
\label{labconcat}
\noindent Name: \textbf{@}\\
concatenates two lists or strings or applies a list as arguments to a procedure\\
\noindent Usage: 
\begin{center}
\emph{L1}\textbf{@}\emph{L2} : (\textsf{list}, \textsf{list}) $\rightarrow$ \textsf{list}
\\ 
\emph{string1}\textbf{@}\emph{string2} : (\textsf{string}, \textsf{string}) $\rightarrow$ \textsf{string}
\\ 
\emph{proc}\textbf{@}\emph{L1} : (\textsf{string}, \textsf{string}) $\rightarrow$ \textsf{string}
\\ 
\end{center}
Parameters: 
\begin{itemize}
\item \emph{L1} and \emph{L2} are two lists.
\item \emph{string1} and \emph{string2} are two strings.
\item \emph{proc} is a procedure.
\end{itemize}
\noindent Description: \begin{itemize}

\item In its first usage form, \textbf{@} concatenates two lists or strings.

\item In its second usage form, \textbf{@} applies the elements of a list as
   arguments to a procedure.
\end{itemize}
\noindent Example 1: 
\begin{center}\begin{minipage}{15cm}\begin{Verbatim}[frame=single]
> [|1,...,3|]@[|7,8,9|];
[|1, 2, 3, 7, 8, 9|]
\end{Verbatim}
\end{minipage}\end{center}
\noindent Example 2: 
\begin{center}\begin{minipage}{15cm}\begin{Verbatim}[frame=single]
> "Hello "@"World!";
Hello World!
\end{Verbatim}
\end{minipage}\end{center}
\noindent Example 3: 
\begin{center}\begin{minipage}{15cm}\begin{Verbatim}[frame=single]
> procedure cool(a,b,c) { 
> write(a,", ", b," and ",c," are cool guys.
> ");
> };
> cool @ [| "Christoph", "Mioara", "Sylvain" |];
Christoph, Mioara and Sylvain are cool guys.
\end{Verbatim}
\end{minipage}\end{center}
See also: \textbf{.:} (\ref{labprepend}), \textbf{:.} (\ref{labappend}), \textbf{procedure} (\ref{labprocedure})

\subsection{constant}
\label{labconstant}
\noindent Name: \textbf{constant}\\
\phantom{aaa}keyword representing a \textsf{constant} type \\[0.2cm]
\noindent Library name:\\
\verb|   SOLLYA_EXTERNALPROC_TYPE_CONSTANT|\\[0.2cm]
\noindent Usage: 
\begin{center}
\textbf{constant} : \textsf{type type}\\
\end{center}
\noindent Description: \begin{itemize}

\item \textbf{constant} represents the \textsf{constant} type for declarations
   of external procedures \textbf{externalproc}.
    
   Remark that in contrast to other indicators, type indicators like
   \textbf{constant} cannot be handled outside the \textbf{externalproc} context.  In
   particular, they cannot be assigned to variables.
\end{itemize}
See also: \textbf{externalproc} (\ref{labexternalproc}), \textbf{boolean} (\ref{labboolean}), \textbf{function} (\ref{labfunction}), \textbf{integer} (\ref{labinteger}), \textbf{list of} (\ref{lablistof}), \textbf{range} (\ref{labrange}), \textbf{string} (\ref{labstring}), \textbf{object} (\ref{labobject})

\subsection{cosh}
\label{labcosh}
\noindent Name: \textbf{cosh}\\
\phantom{aaa}the hyperbolic cosine function.\\[0.2cm]
\noindent Library names:\\
\verb|   sollya_obj_t sollya_lib_cosh(sollya_obj_t)|\\
\verb|   sollya_obj_t sollya_lib_build_function_cosh(sollya_obj_t)|\\
\verb|   #define SOLLYA_COSH(x) sollya_lib_build_function_cosh(x)|\\[0.2cm]
\noindent Description: \begin{itemize}

\item \textbf{cosh} is the usual hyperbolic function: $\cosh(x) = \frac{e^x + e^{-x}}{2}$.

\item It is defined for every real number $x$.
\end{itemize}
See also: \textbf{acosh} (\ref{labacosh}), \textbf{sinh} (\ref{labsinh}), \textbf{tanh} (\ref{labtanh}), \textbf{exp} (\ref{labexp})

\subsection{ cos }
\noindent Name: \textbf{cos}\\
the cosine function.\\

\noindent Description: \begin{itemize}

\item \textbf{cos} is the usual cosine function.

\item It is defined for every real number x.
\end{itemize}
See also: \textbf{acos}, \textbf{sin}, \textbf{tan}

\subsection{decimal}
\label{labdecimal}
\noindent Name: \textbf{decimal}\\
special value for global state \textbf{display}\\

\noindent Description: \begin{itemize}

\item \textbf{decimal} is a special value used for the global state \textbf{display}.
   If the global state \textbf{display} is equal to \textbf{decimal}, all data will
   be output in decimal notation.
   As any value it can be affected to a variable and stored in lists.
\end{itemize}
See also: \textbf{dyadic} (\ref{labdyadic}), \textbf{powers} (\ref{labpowers}), \textbf{hexadecimal} (\ref{labhexadecimal}), \textbf{binary} (\ref{labbinary})

\subsection{default}
\label{labdefault}
\noindent Name: \textbf{default}\\
default value for some commands.\\

\noindent Description: \begin{itemize}

\item \textbf{default} is a special value and is replaced by something depending on the 
   context where it is used. It can often be used as a joker, when you want to 
   specify one of the optional parameters of a command and not the others: set 
   the value of uninteresting parameters to \textbf{default}.

\item Global variables can be reset by affecting them the special value \textbf{default}.
\end{itemize}
\noindent Example 1: 
\begin{center}\begin{minipage}{15cm}\begin{Verbatim}[frame=single]
> p = remez(exp(x),5,[0;1],default,1e-5);
> q = remez(exp(x),5,[0;1],1,1e-5);
> p==q;
true
\end{Verbatim}
\end{minipage}\end{center}
\noindent Example 2: 
\begin{center}\begin{minipage}{15cm}\begin{Verbatim}[frame=single]
> prec;
165
> prec=200;
The precision has been set to 200 bits.
> prec=default;
The precision has been set to 165 bits.
\end{Verbatim}
\end{minipage}\end{center}

\subsection{degree}
\label{labdegree}
\noindent Name: \textbf{degree}\\
gives the degree of a polynomial.\\
\noindent Usage: 
\begin{center}
\textbf{degree}(\emph{f}) : \textsf{function} $\rightarrow$ \textsf{integer}\\
\end{center}
Parameters: 
\begin{itemize}
\item \emph{f} is a function (usually a polynomial).
\end{itemize}
\noindent Description: \begin{itemize}

\item If \emph{f} is a polynomial, \textbf{degree}(\emph{f}) returns the degree of \emph{f}.

\item Contrary to the usage, \sollya considers that the degree of the null polynomial
   is 0.

\item If \emph{f} is a function that is not a polynomial, \textbf{degree}(\emph{f}) returns -1.
\end{itemize}
\noindent Example 1: 
\begin{center}\begin{minipage}{15cm}\begin{Verbatim}[frame=single]
> degree((1+x)*(2+5*x^2));
3
> degree(0);
0
\end{Verbatim}
\end{minipage}\end{center}
\noindent Example 2: 
\begin{center}\begin{minipage}{15cm}\begin{Verbatim}[frame=single]
> degree(sin(x));
-1
\end{Verbatim}
\end{minipage}\end{center}
See also: \textbf{coeff} (\ref{labcoeff})

\subsection{ denominator }
\noindent Name: \textbf{denominator}\\
gives the denominator of an expression\\

\noindent Usage: 
\begin{center}
\textbf{denominator}(\emph{expr}) : \textsf{function} $\rightarrow$ \textsf{function}\\
\end{center}
Parameters: 
\begin{itemize}
\item \emph{expr} represents an expression
\end{itemize}
\noindent Description: \begin{itemize}

\item If \emph{expr} represents a fraction \emph{expr1}/\emph{expr2}, \textbf{denominator}(\emph{expr})
   returns the denominator of this fraction, i.e. \emph{expr2}.
   If \emph{expr} represents something else, \textbf{denominator}(\emph{expr}) 
   returns 1.
   Note that for all expressions \emph{expr}, \textbf{numerator}(\emph{expr}) \textbf{/} \textbf{denominator}(\emph{expr})
   is equal to \emph{expr}.
\end{itemize}
\noindent Example 1: 
\begin{center}\begin{minipage}{15cm}\begin{Verbatim}[frame=single]
> denominator(5/3);
3
\end{Verbatim}
\end{minipage}\end{center}
\noindent Example 2: 
\begin{center}\begin{minipage}{15cm}\begin{Verbatim}[frame=single]
> denominator(exp(x));
1
\end{Verbatim}
\end{minipage}\end{center}
\noindent Example 3: 
\begin{center}\begin{minipage}{15cm}\begin{Verbatim}[frame=single]
> a = 5/3;
> b = numerator(a)/denominator(a);
> print(a);
5 / 3
> print(b);
5 / 3
\end{Verbatim}
\end{minipage}\end{center}
\noindent Example 4: 
\begin{center}\begin{minipage}{15cm}\begin{Verbatim}[frame=single]
> a = exp(x/3);
> b = numerator(a)/denominator(a);
> print(a);
exp(x / 3)
> print(b);
exp(x / 3)
\end{Verbatim}
\end{minipage}\end{center}
See also: \textbf{numerator}

\subsection{diam}
\label{labdiam}
\noindent Name: \textbf{diam}\\
parameter used in safe algorithms of \sollya and controlling the maximal length of the involved intervals.\\
\noindent Usage: 
\begin{center}
\textbf{diam} = \emph{width} : \textsf{constant} $\rightarrow$ \textsf{void}\\
\textbf{diam} = \emph{width} ! : \textsf{constant} $\rightarrow$ \textsf{void}\\
\textbf{diam} : \textsf{constant}\\
\end{center}
Parameters: 
\begin{itemize}
\item \emph{width} represents the maximal relative width of the intervals used
\end{itemize}
\noindent Description: \begin{itemize}

\item \\textbf{diam} is a global variable. Its value represents the maximal width allowed\n   for intervals involved in safe algorithms of \\sollya (namely \\textbf{infnorm},\n   \\textbf{checkinfnorm}, \\textbf{accurateinfnorm}, \\textbf{integral}, \\textbf{findzeros}).\n
\item More precisely, \\textbf{diam} is relative to the width of the input interval of\n   the command. For instance, suppose that \\textbf{diam}=1e-5: if \\textbf{infnorm} is called\n   on interval $[0,\\,1]$, the maximal width of an interval will be 1e-5. But if it\n   is called on interval $[0,\\,1\\mathrm{e}{-3}]$, the maximal width will be 1e-8.\n\end{itemize}
See also: \textbf{infnorm} (\ref{labinfnorm}), \textbf{checkinfnorm} (\ref{labcheckinfnorm}), \textbf{accurateinfnorm} (\ref{labaccurateinfnorm}), \textbf{integral} (\ref{labintegral}), \textbf{findzeros} (\ref{labfindzeros})

\subsection{dieonerrormode}
\label{labdieonerrormode}
\noindent Name: \textbf{dieonerrormode}\\
global variable controlling if \sollya is exited on an error or not.\\
\noindent Usage: 
\begin{center}
\textbf{dieonerrormode} = \emph{activation value} : \textsf{on$|$off} $\rightarrow$ \textsf{void}\\
\textbf{dieonerrormode} = \emph{activation value} ! : \textsf{on$|$off} $\rightarrow$ \textsf{void}\\
\textbf{dieonerrormode} : \textsf{on$|$off}\\
\end{center}
Parameters: 
\begin{itemize}
\item \emph{activation value} controls if \sollya is exited on an error or not.
\end{itemize}
\noindent Description: \begin{itemize}

\item \textbf{dieonerrormode} is a global variable. When its value is \textbf{off}, which is the default,
   \sollya will not exit on any syntax, typing, side-effect errors. These
   errors will be caught by the tool, even if a memory might be leaked at 
   that point. On evaluation, the \textbf{error} special value will be produced.

\item When the value of the \textbf{dieonerrormode} variable is \textbf{on}, \sollya will exit
   on any syntax, typing, side-effect errors. A warning message will
   be printed in these cases at appropriate \textbf{verbosity} levels. 
\end{itemize}
\noindent Example 1: 
\begin{center}\begin{minipage}{15cm}\begin{Verbatim}[frame=single]
> verbosity = 1!;
> dieonerrormode = off;
Die-on-error mode has been deactivated.
> for i from true to false do i + "Salut";
Warning: one of the arguments of the for loop does not evaluate to a constant.
The for loop will not be executed.
> exp(17);
Warning: rounding has happened. The value displayed is a faithful rounding of th
e true result.
2.41549527535752982147754351803858238798675673527224e7
\end{Verbatim}
\end{minipage}\end{center}
\noindent Example 2: 
\begin{center}\begin{minipage}{15cm}\begin{Verbatim}[frame=single]
> verbosity = 1!;
> dieonerrormode = off!;
> 5 */  4;
Warning: syntax error, unexpected DIVTOKEN.
The last symbol read has been "/".
Will skip input until next semicolon after the unexpected token. May leak memory
.
  exp(17);
Warning: rounding has happened. The value displayed is a faithful rounding of th
e true result.
2.41549527535752982147754351803858238798675673527224e7
\end{Verbatim}
\end{minipage}\end{center}
\noindent Example 3: 
\begin{center}\begin{minipage}{15cm}\begin{Verbatim}[frame=single]
> verbosity = 1!;
> dieonerrormode;
off
> dieonerrormode = on!;
> dieonerrormode;
on
> for i from true to false do i + "Salut";
Warning: one of the arguments of the for loop does not evaluate to a constant.
The for loop will not be executed.
Warning: some syntax, typing or side-effect error has occurred.
As the die-on-error mode is activated, the tool will be exited.
\end{Verbatim}
\end{minipage}\end{center}
\noindent Example 4: 
\begin{center}\begin{minipage}{15cm}\begin{Verbatim}[frame=single]
> verbosity = 1!;
> dieonerrormode = on!;
> 5 */  4;
Warning: syntax error, unexpected DIVTOKEN.
The last symbol read has been "/".
Will skip input until next semicolon after the unexpected token. May leak memory
.
Warning: some syntax, typing or side-effect error has occurred.
As the die-on-error mode is activated, the tool will be exited.
\end{Verbatim}
\end{minipage}\end{center}
\noindent Example 5: 
\begin{center}\begin{minipage}{15cm}\begin{Verbatim}[frame=single]
> verbosity = 0!;
> dieonerrormode = on!;
> 5 */  4;
\end{Verbatim}
\end{minipage}\end{center}
See also: \textbf{on} (\ref{labon}), \textbf{off} (\ref{laboff}), \textbf{verbosity} (\ref{labverbosity}), \textbf{error} (\ref{laberror})

\subsection{diff}
\label{labdiff}
\noindent Name: \textbf{diff}\\
\phantom{aaa}differentiation operator\\[0.2cm]
\noindent Library name:\\
\verb|   sollya_obj_t sollya_lib_diff(sollya_obj_t)|\\[0.2cm]
\noindent Usage: 
\begin{center}
\textbf{diff}(\emph{function}) : \textsf{function} $\rightarrow$ \textsf{function}\\
\end{center}
Parameters: 
\begin{itemize}
\item \emph{function} represents a function
\end{itemize}
\noindent Description: \begin{itemize}

\item \textbf{diff}(\emph{function}) returns the symbolic derivative of the function
   \emph{function} by the global free variable.
    
   If \emph{function} represents a function symbol that is externally bound
   to some code by \textbf{library}, the derivative is performed as a symbolic
   annotation to the returned expression tree.
\end{itemize}
\noindent Example 1: 
\begin{center}\begin{minipage}{15cm}\begin{Verbatim}[frame=single,commandchars=\\\|\~]
> diff(sin(x));
cos(x)
\end{Verbatim}
\end{minipage}\end{center}
\noindent Example 2: 
\begin{center}\begin{minipage}{15cm}\begin{Verbatim}[frame=single,commandchars=\\\|\~]
> diff(x);
1
\end{Verbatim}
\end{minipage}\end{center}
\noindent Example 3: 
\begin{center}\begin{minipage}{15cm}\begin{Verbatim}[frame=single,commandchars=\\\|\~]
> diff(x^x);
x^x * (1 + log(x))
\end{Verbatim}
\end{minipage}\end{center}
See also: \textbf{library} (\ref{lablibrary}), \textbf{autodiff} (\ref{labautodiff}), \textbf{taylor} (\ref{labtaylor}), \textbf{taylorform} (\ref{labtaylorform})

\subsection{ dirtyfindzeros }
\noindent Name: \textbf{dirtyfindzeros}\\
gives a list of numerical values listing the zeros of a function on an interval.\\

\noindent Usage: 
\begin{center}
\textbf{dirtyfindzeros}(\emph{f},\emph{I}) : (\textsf{function}, \textsf{range}) $\rightarrow$ \textsf{list}\\
\end{center}
Parameters: 
\emph{f} is a function.\\
\emph{I} is an interval.\\

\noindent Description: \begin{itemize}

\item \textbf{dirtyfindzeros}(\emph{f},\emph{I}) returns a list containing some zeros of \emph{f} in the
   interval \emph{I}. The values in the list are numerical approximation of the exact
   zeros. The precision of these approximations is approximately the precision
   stored in \textbf{prec}. If \emph{f} does not have two zeros very close to each other, it 
   can be expected that all zeros are listed. However, some zeros may be
   forgotten. This command should be considered as a numerical algorithm and
   should not be used if safety is critical.

\item More precisely, the algorithm relies on global variables \textbf{prec} and \textbf{points} and
   is the following: let $n$ be the value of variable \textbf{points} and $t$ be the value
   of variable \textbf{prec}.
   \begin{itemize}
   \item  Evaluate $|f|$ at $n$ evenly distributed points in the interval $I$.
     the precision used is automatically chosen in order to ensure that the sign
     is correct.
   \item  Whenever $f$ changes its sign for two consecutive points,
     find an approximation $x$ of its zero with precision $t$ using
     Newton's algorithm. The number of steps in Newton's iteration depends on $t$:
     the precision of the approximation is supposed to be doubled at each step.
   \item  Add this value to the list.
   \end{itemize}
\end{itemize}
\noindent Example 1: 
\begin{center}\begin{minipage}{14.8cm}\begin{Verbatim}[frame=single]
   > dirtyfindzeros(sin(x),[-5;5]);
   [|-0.31415926535897932384626433832795028841971693993750801e1, 0, 0.31415926535897932384626433832795028841971693993750801e1|]
\end{Verbatim}
\end{minipage}\end{center}
\noindent Example 2: 
\begin{center}\begin{minipage}{14.8cm}\begin{Verbatim}[frame=single]
   > L1=dirtyfindzeros(x^2*sin(1/x),[0;1]);
   > points=1000!;
   > L2=dirtyfindzeros(x^2*sin(1/x),[0;1]);
   > length(L1); length(L2);
   18
   25
\end{Verbatim}
\end{minipage}\end{center}
See also: \textbf{prec}, \textbf{points}, \textbf{findzeros}

\subsection{dirtyinfnorm}
\label{labdirtyinfnorm}
\noindent Name: \textbf{dirtyinfnorm}\\
computes a numerical approximation of the infinity norm of a function on an interval.\\
\noindent Usage: 
\begin{center}
\textbf{dirtyinfnorm}(\emph{f},\emph{I}) : (\textsf{function}, \textsf{range}) $\rightarrow$ \textsf{constant}\\
\end{center}
Parameters: 
\begin{itemize}
\item \emph{f} is a function.
\item \emph{I} is an interval.
\end{itemize}
\noindent Description: \begin{itemize}

\item \textbf{dirtyinfnorm}(\emph{f},\emph{I}) computes an approximation of the infinity norm of the 
   given function $f$ on the interval $I$, e.g. $\max_{x \in I} \{|f(x)|\}$.

\item The interval must be bound. If the interval contains one of -Inf or +Inf, the 
   result of \textbf{dirtyinfnorm} is NaN.

\item The result of this command depends on the global variables \textbf{prec} and \textbf{points}.
   Therefore, the returned result is generally a good approximation of the exact
   infinity norm, with precision \textbf{prec}. However, the result is generally 
   underestimated and should not be used when safety is critical.
   Use \textbf{infnorm} instead.

\item The following algorithm is used: let $n$ be the value of variable \textbf{points}
   and $t$ be the value of variable \textbf{prec}.
   \begin{itemize}
   \item Evaluate $|f|$ at $n$ evenly distributed points in the
     interval $I$. The evaluation are faithful roundings of the exact
     results at precision $t$.
   \item Whenever the derivative of $f$ changes its sign for two consecutive 
     points, find an approximation $x$ of its zero with precision $t$.
     Then compute a faithful rounding of $|f(x)|$ at precision $t$.
   \item Return the maximum of all computed values.
   \end{itemize}
\end{itemize}
\noindent Example 1: 
\begin{center}\begin{minipage}{15cm}\begin{Verbatim}[frame=single]
> dirtyinfnorm(sin(x),[-10;10]);
1
\end{Verbatim}
\end{minipage}\end{center}
\noindent Example 2: 
\begin{center}\begin{minipage}{15cm}\begin{Verbatim}[frame=single]
> prec=15!;
> dirtyinfnorm(exp(cos(x))*sin(x),[0;5]);
1.45856
> prec=40!;
> dirtyinfnorm(exp(cos(x))*sin(x),[0;5]);
1.458528537135
> prec=100!;
> dirtyinfnorm(exp(cos(x))*sin(x),[0;5]);
1.458528537136237644438147455024
> prec=200!;
> dirtyinfnorm(exp(cos(x))*sin(x),[0;5]);
1.458528537136237644438147455023841718299214087993682374094153
\end{Verbatim}
\end{minipage}\end{center}
\noindent Example 3: 
\begin{center}\begin{minipage}{15cm}\begin{Verbatim}[frame=single]
> dirtyinfnorm(x^2, [log(0);log(1)]);
@NaN@
\end{Verbatim}
\end{minipage}\end{center}
See also: \textbf{prec} (\ref{labprec}), \textbf{points} (\ref{labpoints}), \textbf{infnorm} (\ref{labinfnorm}), \textbf{checkinfnorm} (\ref{labcheckinfnorm})

\subsection{dirtyintegral}
\label{labdirtyintegral}
\noindent Name: \textbf{dirtyintegral}\\
computes a numerical approximation of the integral of a function on an interval.\\

\noindent Usage: 
\begin{center}
\textbf{dirtyintegral}(\emph{f},\emph{I}) : (\textsf{function}, \textsf{range}) $\rightarrow$ \textsf{constant}\\
\end{center}
Parameters: 
\begin{itemize}
\item \emph{f} is a function.
\item \emph{I} is an interval.
\end{itemize}
\noindent Description: \begin{itemize}

\item \textbf{dirtyintegral}(\emph{f},\emph{I}) computes an approximation of the integral of \emph{f} on \emph{I}.

\item The interval must be bound. If the interval contains one of -Inf or +Inf, the 
   result of \textbf{dirtyintegral} is NaN, even if the integral has a meaning.

\item The result of this command depends on the global variables \textbf{prec} and \textbf{points}.
   The method used is the trapezium rule applied at $n$ evenly distributed
   points in the interval, where $n$ is the value of global variable \textbf{points}.

\item This command computes a numerical approximation of the exact value of the 
   integral. It should not be used if safety is critical. In this case, use
   command \textbf{integral} instead.

\item Warning: this command is known to be currently unsatisfactory. If you really
   need to compute integrals, think of using an other tool or report a feature
   request to sylvain.chevillard@ens-lyon.fr.
\end{itemize}
\noindent Example 1: 
\begin{center}\begin{minipage}{15cm}\begin{Verbatim}[frame=single]
> sin(10);
-0.54402111088936981340474766185137728168364301291624
> dirtyintegral(cos(x),[0;10]);
-0.544003049051526298224480588824753820365362983562797
> points=2000!;
> dirtyintegral(cos(x),[0;10]);
-0.54401997751158321972222697312583199035995837926892
\end{Verbatim}
\end{minipage}\end{center}
See also: \textbf{prec} (\ref{labprec}), \textbf{points} (\ref{labpoints}), \textbf{integral} (\ref{labintegral})

\subsection{display}
\label{labdisplay}
\noindent Name: \textbf{display}\\
\phantom{aaa}sets or inspects the global variable specifying number notation\\[0.2cm]
\noindent Library names:\\
\verb|   void sollya_lib_set_display_and_print(sollya_obj_t)|\\
\verb|   void sollya_lib_set_display(sollya_obj_t)|\\
\verb|   sollya_obj_t sollya_lib_get_display()|\\[0.2cm]
\noindent Usage: 
\begin{center}
\textbf{display} = \emph{notation value} : \textsf{decimal$|$binary$|$dyadic$|$powers$|$hexadecimal} $\rightarrow$ \textsf{void}\\
\textbf{display} = \emph{notation value} ! : \textsf{decimal$|$binary$|$dyadic$|$powers$|$hexadecimal} $\rightarrow$ \textsf{void}\\
\textbf{display} : \textsf{decimal$|$binary$|$dyadic$|$powers$|$hexadecimal}\\
\end{center}
Parameters: 
\begin{itemize}
\item \emph{notation value} represents a variable of type \textsf{decimal$|$binary$|$dyadic$|$powers$|$hexadecimal}
\end{itemize}
\noindent Description: \begin{itemize}

\item An assignment \textbf{display} = \emph{notation value}, where \emph{notation value} is
   one of \textbf{decimal}, \textbf{dyadic}, \textbf{powers}, \textbf{binary} or \textbf{hexadecimal}, activates
   the corresponding notation for output of values in \textbf{print}, \textbf{write} or
   at the \sollya prompt.
    
   If the global notation variable \textbf{display} is \textbf{decimal}, all numbers will
   be output in scientific decimal notation.  If the global notation
   variable \textbf{display} is \textbf{dyadic}, all numbers will be output as dyadic
   numbers with Gappa notation.  If the global notation variable \textbf{display}
   is \textbf{powers}, all numbers will be output as dyadic numbers with a
   notation compatible with Maple and PARI/GP.  If the global notation
   variable \textbf{display} is \textbf{binary}, all numbers will be output in binary
   notation.  If the global notation variable \textbf{display} is \textbf{hexadecimal},
   all numbers will be output in C99/ IEEE754-2008 notation.  All output
   notations can be parsed back by \sollya, inducing no error if the input
   and output precisions are the same (see \textbf{prec}).
    
   If the assignment \textbf{display} = \emph{notation value} is followed by an
   exclamation mark, no message indicating the new state is
   displayed. Otherwise the user is informed of the new state of the
   global mode by an indication.
\end{itemize}
\noindent Example 1: 
\begin{center}\begin{minipage}{15cm}\begin{Verbatim}[frame=single]
> display = decimal;
Display mode is decimal numbers.
> a = evaluate(sin(pi * x), 0.25);
> a;
0.70710678118654752440084436210484903928483593768847
> display = binary;
Display mode is binary numbers.
> a;
1.011010100000100111100110011001111111001110111100110010010000100010110010111110
11000100110110011011101010100101010111110100111110001110101101111011000001011101
010001_2 * 2^(-1)
> display = hexadecimal;
Display mode is hexadecimal numbers.
> a;
0xb.504f333f9de6484597d89b3754abe9f1d6f60ba88p-4
> display = dyadic;
Display mode is dyadic numbers.
> a;
33070006991101558613323983488220944360067107133265b-165
> display = powers;
Display mode is dyadic numbers in integer-power-of-2 notation.
> a;
33070006991101558613323983488220944360067107133265 * 2^(-165)
\end{Verbatim}
\end{minipage}\end{center}
See also: \textbf{print} (\ref{labprint}), \textbf{write} (\ref{labwrite}), \textbf{decimal} (\ref{labdecimal}), \textbf{dyadic} (\ref{labdyadic}), \textbf{powers} (\ref{labpowers}), \textbf{binary} (\ref{labbinary}), \textbf{hexadecimal} (\ref{labhexadecimal}), \textbf{prec} (\ref{labprec})

\subsection{ divide }
\noindent Name: \textbf{/}\\
division function\\

\noindent Usage: 
\begin{center}
\emph{function1} \textbf{/} \emph{function2} : (\textsf{function}, \textsf{function}) $\rightarrow$ \textsf{function}\\
\end{center}
Parameters: 
\emph{function1} and \emph{function2} represent functions\\

\noindent Description: \begin{itemize}

\item \textbf{/} represents the division (function) on reals. 
   The expression \emph{function1} \textbf{/} \emph{function2} stands for
   the function composed of the division function and the two
   functions \emph{function1} and \emph{function2}, where \emph{function1} is
   the numerator and \emph{function2} the denominator.
\end{itemize}
\noindent Example 1: 
\begin{center}\begin{minipage}{14.8cm}\begin{Verbatim}[frame=single]
   > 5 / 2;
   0.25e1
\end{Verbatim}
\end{minipage}\end{center}
\noindent Example 2: 
\begin{center}\begin{minipage}{14.8cm}\begin{Verbatim}[frame=single]
   > x / 2;
   x * 0.5
\end{Verbatim}
\end{minipage}\end{center}
\noindent Example 3: 
\begin{center}\begin{minipage}{14.8cm}\begin{Verbatim}[frame=single]
   > x / x;
   1
\end{Verbatim}
\end{minipage}\end{center}
\noindent Example 4: 
\begin{center}\begin{minipage}{14.8cm}\begin{Verbatim}[frame=single]
   > 3 / 0;
   @Inf@
\end{Verbatim}
\end{minipage}\end{center}
\noindent Example 5: 
\begin{center}\begin{minipage}{14.8cm}\begin{Verbatim}[frame=single]
   > diff(sin(x) / exp(x));
   (exp(x) * cos(x) - sin(x) * exp(x)) / exp(x)^2
\end{Verbatim}
\end{minipage}\end{center}
See also: \textbf{$+$}, \textbf{$-$}, \textbf{$*$}, \textbf{\^}

\subsection{doubledouble}
\label{labdoubledouble}
\noindent Names: \textbf{doubledouble}, \textbf{DD}\\
represents a number as the sum of two IEEE doubles.\\
\noindent Description: \begin{itemize}

\item \\textbf{doubledouble} is both a function and a constant.\n
\item As a function, it rounds its argument to the nearest number that can be written\n   as the sum of two double precision numbers.\n
\item The algorithm used to compute \\textbf{doubledouble}($x$) is the following: let $x_h$ = \\textbf{double}($x$)\n   and let $x_l$ = \\textbf{double}($x-x_h$). Return the number $x_h+x_l$. Note that if the current \n   precision is not sufficient to exactly represent $x_h + x_l$, a rounding will occur\n   and the result of \\textbf{doubledouble}($x$) will be useless.\n
\item As a constant, it symbolizes the double-double precision format. It is used in \n   contexts when a precision format is necessary, e.g. in the commands \n   \\textbf{round}, \\textbf{roundcoefficients} and \\textbf{implementpoly}.\n   See the corresponding help pages for examples.\n\end{itemize}
\noindent Example 1: 
\begin{center}\begin{minipage}{15cm}\begin{Verbatim}[frame=single]
\end{Verbatim}
\end{minipage}\end{center}
See also: \textbf{single} (\ref{labsingle}), \textbf{double} (\ref{labdouble}), \textbf{doubleextended} (\ref{labdoubleextended}), \textbf{tripledouble} (\ref{labtripledouble}), \textbf{roundcoefficients} (\ref{labroundcoefficients}), \textbf{implementpoly} (\ref{labimplementpoly}), \textbf{round} (\ref{labround})

\subsection{doubleextended}
\label{labdoubleextended}
\noindent Names: \textbf{doubleextended}, \textbf{DE}\\
\phantom{aaa}computes the nearest number with 64 bits of mantissa.\\[0.2cm]
\noindent Library names:\\
\verb|   sollya_obj_t sollya_lib_doubleextended(sollya_obj_t)|\\
\verb|   sollya_obj_t sollya_lib_doubleextended_obj()|\\
\verb|   int sollya_lib_is_doubleextended_obj(sollya_obj_t)|\\
\verb|   sollya_obj_t sollya_lib_build_function_doubleextended(sollya_obj_t)|\\
\verb|   #define SOLLYA_DE(x) sollya_lib_build_function_doubleextended(x)|\\[0.2cm]
\noindent Description: \begin{itemize}

\item \textbf{doubleextended} is a function that computes the nearest floating-point number with
   64 bits of mantissa to a given number. Since it is a function, it can be
   composed with other \sollya functions such as \textbf{exp}, \textbf{sin}, etc.

\item \textbf{doubleextended} now does handle subnormal numbers for a presumed exponent width
   of the double-extended format of 15 bits. This means, with respect to 
   rounding, \textbf{doubleextended} behaves as a IEEE 754-2008 binary79 with a 64 bit 
   significand (with a hidden bit normal range), one sign bit and a 
   15 bit exponent field would behave. This behavior may be different
   from the one observed on Intel-based IA32/Intel64 processors (or compatible
   versions from other vendors). However it is the one seen on HP/Intel 
   Itanium when the precision specifier is double-extended and pseudo-denormals
   are activated.

\item Since it is a function and not a command, its behavior is a bit different from 
   the behavior of \textbf{round}(x,64,RN) even if the result is exactly the same.
   \textbf{round}(x,64,RN) is immediately evaluated whereas \textbf{doubleextended}(x) can be composed 
   with other functions (and thus be plotted and so on).
\end{itemize}
\noindent Example 1: 
\begin{center}\begin{minipage}{15cm}\begin{Verbatim}[frame=single]
> display=binary!;
> DE(0.1);
1.100110011001100110011001100110011001100110011001100110011001101_2 * 2^(-4)
> round(0.1,64,RN);
1.100110011001100110011001100110011001100110011001100110011001101_2 * 2^(-4)
\end{Verbatim}
\end{minipage}\end{center}
\noindent Example 2: 
\begin{center}\begin{minipage}{15cm}\begin{Verbatim}[frame=single]
> D(2^(-2000));
0
> DE(2^(-20000));
0
\end{Verbatim}
\end{minipage}\end{center}
\noindent Example 3: 
\begin{center}\begin{minipage}{15cm}\begin{Verbatim}[frame=single]
> verbosity=1!;
> f = sin(DE(x));
> f(pi);
Warning: rounding has happened. The value displayed is a faithful rounding of th
e true result.
-5.0165576126683320235573270803307570138315616702549e-20
> g = sin(round(x,64,RN));
Warning: at least one of the given expressions or a subexpression is not correct
ly typed
or its evaluation has failed because of some error on a side-effect.
\end{Verbatim}
\end{minipage}\end{center}
See also: \textbf{roundcoefficients} (\ref{labroundcoefficients}), \textbf{halfprecision} (\ref{labhalfprecision}), \textbf{single} (\ref{labsingle}), \textbf{double} (\ref{labdouble}), \textbf{doubledouble} (\ref{labdoubledouble}), \textbf{quad} (\ref{labquad}), \textbf{tripledouble} (\ref{labtripledouble}), \textbf{round} (\ref{labround})

\subsection{double}
\label{labdouble}
\noindent Names: \textbf{double}, \textbf{D}\\
rounding to the nearest IEEE 754 double (binary64).\\
\noindent Description: \begin{itemize}

\item \textbf{double} is both a function and a constant.

\item As a function, it rounds its argument to the nearest IEEE 754 double precision (i.e. IEEE754-2008 binary64) number.
   Subnormal numbers are supported as well as standard numbers: it is the real
   rounding described in the standard.

\item As a constant, it symbolizes the double precision format. It is used in 
   contexts when a precision format is necessary, e.g. in the commands 
   \textbf{round}, \textbf{roundcoefficients} and \textbf{implementpoly}.
   See the corresponding help pages for examples.
\end{itemize}
\noindent Example 1: 
\begin{center}\begin{minipage}{15cm}\begin{Verbatim}[frame=single]
> display=binary!;
> D(0.1);
1.100110011001100110011001100110011001100110011001101_2 * 2^(-4)
> D(4.17);
1.000010101110000101000111101011100001010001111010111_2 * 2^(2)
> D(1.011_2 * 2^(-1073));
1.1_2 * 2^(-1073)
\end{Verbatim}
\end{minipage}\end{center}
See also: \textbf{single} (\ref{labsingle}), \textbf{printdouble} (\ref{labprinthexa}), \textbf{doubleextended} (\ref{labdoubleextended}), \textbf{doubledouble} (\ref{labdoubledouble}), \textbf{tripledouble} (\ref{labtripledouble}), \textbf{roundcoefficients} (\ref{labroundcoefficients}), \textbf{implementpoly} (\ref{labimplementpoly}), \textbf{round} (\ref{labround})

\subsection{dyadic}
\label{labdyadic}
\noindent Name: \textbf{dyadic}\\
special value for global state \textbf{display}\\
\noindent Description: \begin{itemize}

\item \textbf{dyadic} is a special value used for the global state \textbf{display}.
   If the global state \textbf{display} is equal to \textbf{dyadic}, all data will
   be output in dyadic notation with numbers displayed in Gappa format.
    
   As any value it can be affected to a variable and stored in lists.
\end{itemize}
See also: \textbf{decimal} (\ref{labdecimal}), \textbf{powers} (\ref{labpowers}), \textbf{hexadecimal} (\ref{labhexadecimal}), \textbf{binary} (\ref{labbinary})

\subsection{$==$}
\label{labequal}
\noindent Name: \textbf{$==$}\\
equality test operator\\
\noindent Usage: 
\begin{center}
\emph{expr1} \textbf{$==$} \emph{expr2} : (\textsf{any type}, \textsf{any type}) $\rightarrow$ \textsf{boolean}\\
\end{center}
Parameters: 
\begin{itemize}
\item \emph{expr1} and \emph{expr2} represent expressions
\end{itemize}
\noindent Description: \begin{itemize}

\item The operator \textbf{$==$} evaluates to true iff its operands \emph{expr1} and
   \emph{expr2} are syntactically equal and different from \textbf{error} or constant
   expressions that are not constants and that evaluate to the same
   floating-point number with the global precision \textbf{prec}. The user should
   be aware of the fact that because of floating-point evaluation, the
   operator \textbf{$==$} is not exactly the same as the mathematical
   equality. Further remark that according to IEEE 754-2008 floating-point
   rules, which \sollya emulates, floating-point data which are NaN do not
   compare equal to any other floating-point datum, including NaN. 
\end{itemize}
\noindent Example 1: 
\begin{center}\begin{minipage}{15cm}\begin{Verbatim}[frame=single]
> "Hello" == "Hello";
true
> "Hello" == "Salut";
false
> "Hello" == 5;
false
> 5 + x == 5 + x;
true
\end{Verbatim}
\end{minipage}\end{center}
\noindent Example 2: 
\begin{center}\begin{minipage}{15cm}\begin{Verbatim}[frame=single]
> 1 == exp(0);
true
> asin(1) * 2 == pi;
true
> exp(5) == log(4);
false
\end{Verbatim}
\end{minipage}\end{center}
\noindent Example 3: 
\begin{center}\begin{minipage}{15cm}\begin{Verbatim}[frame=single]
> sin(pi/6) == 1/2 * sqrt(3);
false
\end{Verbatim}
\end{minipage}\end{center}
\noindent Example 4: 
\begin{center}\begin{minipage}{15cm}\begin{Verbatim}[frame=single]
> prec = 12;
The precision has been set to 12 bits.
> 16384.1 == 16385.1;
true
\end{Verbatim}
\end{minipage}\end{center}
\noindent Example 5: 
\begin{center}\begin{minipage}{15cm}\begin{Verbatim}[frame=single]
> error == error;
false
\end{Verbatim}
\end{minipage}\end{center}
\noindent Example 6: 
\begin{center}\begin{minipage}{15cm}\begin{Verbatim}[frame=single]
> a = "Biba";
> b = NaN;
> a == a;
true
> b == b;
false
\end{Verbatim}
\end{minipage}\end{center}
See also: \textbf{!$=$} (\ref{labneq}), \textbf{$>$} (\ref{labgt}), \textbf{$>=$} (\ref{labge}), \textbf{$<=$} (\ref{lable}), \textbf{$<$} (\ref{lablt}), \textbf{!} (\ref{labnot}), \textbf{$\&\&$} (\ref{laband}), \textbf{$||$} (\ref{labor}), \textbf{error} (\ref{laberror}), \textbf{prec} (\ref{labprec})

\subsection{erfc}
\label{laberfc}
\noindent Name: \textbf{erfc}\\
the complementary error function.\\
\noindent Description: \begin{itemize}

\item \textbf{erfc} is the complementary error function defined by $\mathrm{erfc}(x) = 1 - \mathrm{erf}(x)$.

\item It is defined for every real number $x$.
\end{itemize}
See also: \textbf{erf} (\ref{laberf})

\subsection{erf}
\label{laberf}
\noindent Name: \textbf{erf}\\
the error function.\\
\noindent Description: \begin{itemize}

\item \textbf{erf} is the error function defined by:
   $$\mathrm{erf}(x) = \frac{2}{\sqrt{\pi}} \int_0^x e^{-t^2} {\rm d}t.$$

\item It is defined for every real number x.
\end{itemize}
See also: \textbf{erfc} (\ref{laberfc}), \textbf{exp} (\ref{labexp})

\subsection{error}
\label{laberror}
\noindent Name: \textbf{error}\\
expression representing an input that is wrongly typed or that cannot be executed\\
\noindent Usage: 
\begin{center}
\textbf{error} : \textsf{error}\\
\end{center}
\noindent Description: \begin{itemize}

\item The variable \textbf{error} represents an input during the evaluation of
   which a type or execution error has been detected or is to be
   detected. Inputs that are syntactically correct but wrongly typed
   evaluate to \textbf{error} at some stage.  Inputs that are correctly typed
   but containing commands that depend on side-effects that cannot be
   performed or inputs that are wrongly typed at meta-level (cf. \textbf{parse}),
   evaluate to \textbf{error}.
    
   Remark that in contrast to all other elements of the \sollya language,
   \textbf{error} compares neither equal nor unequal to itself. This provides a
   means of detecting syntax errors inside the \sollya language itself
   without introducing issues of two different wrongly typed inputs being
   equal.
\end{itemize}
\noindent Example 1: 
\begin{center}\begin{minipage}{15cm}\begin{Verbatim}[frame=single]
> print(5 + "foo");
error
\end{Verbatim}
\end{minipage}\end{center}
\noindent Example 2: 
\begin{center}\begin{minipage}{15cm}\begin{Verbatim}[frame=single]
> error;
error
\end{Verbatim}
\end{minipage}\end{center}
\noindent Example 3: 
\begin{center}\begin{minipage}{15cm}\begin{Verbatim}[frame=single]
> error == error;
false
> error != error;
false
\end{Verbatim}
\end{minipage}\end{center}
\noindent Example 4: 
\begin{center}\begin{minipage}{15cm}\begin{Verbatim}[frame=single]
> correct = 5 + 6;
> incorrect = 5 + "foo";
> correct == correct;
true
> incorrect == incorrect;
false
> errorhappened = !(incorrect == incorrect);
> errorhappened;
true
\end{Verbatim}
\end{minipage}\end{center}
See also: \textbf{void} (\ref{labvoid}), \textbf{parse} (\ref{labparse}), \textbf{$==$} (\ref{labequal}), \textbf{!$=$} (\ref{labneq})

\subsection{evaluate}
\label{labevaluate}
\noindent Name: \textbf{evaluate}\\
\phantom{aaa}evaluates a function at a constant point or in a range\\[0.2cm]
\noindent Library name:\\
\verb|   sollya_obj_t sollya_lib_evaluate(sollya_obj_t, sollya_obj_t)|\\[0.2cm]
\noindent Usage: 
\begin{center}
\textbf{evaluate}(\emph{function}, \emph{constant}) : (\textsf{function}, \textsf{constant}) $\rightarrow$ \textsf{constant} $|$ \textsf{range}\\
\textbf{evaluate}(\emph{function}, \emph{range}) : (\textsf{function}, \textsf{range}) $\rightarrow$ \textsf{range}\\
\textbf{evaluate}(\emph{function}, \emph{function2}) : (\textsf{function}, \textsf{function}) $\rightarrow$ \textsf{function}\\
\end{center}
Parameters: 
\begin{itemize}
\item \emph{function} represents a function
\item \emph{constant} represents a constant point
\item \emph{range} represents a range
\item \emph{function2} represents a function that is not constant
\end{itemize}
\noindent Description: \begin{itemize}

\item If its second argument is a constant \emph{constant}, \textbf{evaluate} evaluates
   its first argument \emph{function} at the point indicated by
   \emph{constant}. This evaluation is performed in a way that the result is a
   faithful rounding of the real value of the \emph{function} at \emph{constant} to
   the current global precision. If such a faithful rounding is not
   possible, \textbf{evaluate} returns a range surely encompassing the real value
   of the function \emph{function} at \emph{constant}. If even interval evaluation
   is not possible because the expression is undefined or numerically
   unstable, NaN will be produced.

\item If its second argument is a range \emph{range}, \textbf{evaluate} evaluates its
   first argument \emph{function} by interval evaluation on this range
   \emph{range}. This ensures that the image domain of the function \emph{function}
   on the preimage domain \emph{range} is surely enclosed in the returned
   range.

\item In the case when the second argument is a range that is reduced to a
   single point (such that $[1;\,1]$ for instance), the evaluation
   is performed in the same way as when the second argument is a constant but
   it produces a range as a result: \textbf{evaluate} automatically adjusts the precision
   of the intern computations and returns a range that contains at most three floating-point
   consecutive numbers in precision \textbf{prec}. This correponds to the same accuracy
   as a faithful rounding of the actual result. If such a faithful rounding
   is not possible, \textbf{evaluate} has the same behavior as in the case when the
   second argument is a constant.

\item If its second argument is a function \emph{function2} that is not a
   constant, \textbf{evaluate} replaces all occurrences of the free variable in
   function \emph{function} by function \emph{function2}.
\end{itemize}
\noindent Example 1: 
\begin{center}\begin{minipage}{15cm}\begin{Verbatim}[frame=single,commandchars=\\\|\~]
> midpointmode=on!;
> print(evaluate(sin(pi * x), 2.25));
0.70710678118654752440084436210484903928483593768847
> print(evaluate(sin(pi * x), [2.25; 2.25]));
0.707106781186547524400844362104849039284835937688~4/5~
\end{Verbatim}
\end{minipage}\end{center}
\noindent Example 2: 
\begin{center}\begin{minipage}{15cm}\begin{Verbatim}[frame=single,commandchars=\\\|\~]
> print(evaluate(sin(pi * x), 2));
[-1.7298645251438126951650861503109812954283676799168e-12715;7.59411982011879631
45069564314525661706039084390068e-12716]
\end{Verbatim}
\end{minipage}\end{center}
\noindent Example 3: 
\begin{center}\begin{minipage}{15cm}\begin{Verbatim}[frame=single,commandchars=\\\|\~]
> print(evaluate(sin(pi * x), [2, 2.25]));
[-5.143390272677254630046998919961912407349224165421e-50;0.707106781186547524400
84436210484903928483593768866]
\end{Verbatim}
\end{minipage}\end{center}
\noindent Example 4: 
\begin{center}\begin{minipage}{15cm}\begin{Verbatim}[frame=single,commandchars=\\\|\~]
> print(evaluate(sin(pi * x), 2 + 0.25 * x));
sin((pi) * 2 + x * (pi) * 0.25)
\end{Verbatim}
\end{minipage}\end{center}
\noindent Example 5: 
\begin{center}\begin{minipage}{15cm}\begin{Verbatim}[frame=single,commandchars=\\\|\~]
> print(evaluate(sin(pi * 1/x), 0));
[-1;1]
\end{Verbatim}
\end{minipage}\end{center}
See also: \textbf{isevaluable} (\ref{labisevaluable})

\subsection{execute}
\label{labexecute}
\noindent Name: \textbf{execute}\\
executes the content of a file\\

\noindent Usage: 
\begin{center}
\textbf{execute}(\emph{filename}) : \textsf{string} $\rightarrow$ \textsf{void}\\
\end{center}
Parameters: 
\begin{itemize}
\item \emph{filename} is a string representing a file name
\end{itemize}
\noindent Description: \begin{itemize}

\item \textbf{execute} opens the file indicated by \emph{filename}, and executes the sequence of 
   commands it contains. This command is evaluated at execution time: this way you
   can modify the file \emph{filename} (for instance using \textbf{bashexecute}) and execute it
   just after.

\item If \emph{filename} contains a command \textbf{execute}, it will be executed recursively.

\item If \emph{filename} contains a call to \textbf{restart}, it will be neglected.

\item If \emph{filename} contains a call to \textbf{quit}, the commands following \textbf{quit}
   in \emph{filename} will be neglected.
\end{itemize}
\noindent Example 1: 
\begin{center}\begin{minipage}{15cm}\begin{Verbatim}[frame=single]
> a=2;
> a;
2
> print("a=1;") > "example.sollya";
> execute("example.sollya"); 
> a;
1
\end{Verbatim}
\end{minipage}\end{center}
\noindent Example 2: 
\begin{center}\begin{minipage}{15cm}\begin{Verbatim}[frame=single]
> verbosity=1!;
> print("a=1; restart; a=2;") > "example.sollya";
> execute("example.sollya"); 
Warning: a restart command has been used in a file read into another.
This restart command will be neglected.
> a;
2
\end{Verbatim}
\end{minipage}\end{center}
\noindent Example 3: 
\begin{center}\begin{minipage}{15cm}\begin{Verbatim}[frame=single]
> verbosity=1!;
> print("a=1; quit; a=2;") > "example.sollya";
> execute("example.sollya"); 
Warning: the execution of a file read by execute demanded stopping the interpret
ation but it is not stopped.
> a;
1
\end{Verbatim}
\end{minipage}\end{center}
See also: \textbf{parse} (\ref{labparse}), \textbf{readfile} (\ref{labreadfile}), \textbf{write} (\ref{labwrite}), \textbf{print} (\ref{labprint}), \textbf{bashexecute} (\ref{labbashexecute})

\subsection{expand}
\label{labexpand}
\noindent Name: \textbf{expand}\\
expands polynomial subexpressions\\

\noindent Usage: 
\begin{center}
\textbf{expand}(\emph{function}) : \textsf{function} $\rightarrow$ \textsf{function}\\
\end{center}
Parameters: 
\begin{itemize}
\item \emph{function} represents a function
\end{itemize}
\noindent Description: \begin{itemize}

\item \textbf{expand}(\emph{function}) expands all polynomial subexpressions in function
   \emph{function} as far as possible. Factors of sums are multiplied out,
   power operators with constant positive integer exponents are replaced
   by multiplications and divisions are multiplied out, i.e. denomiators
   are brought at the most interior point of expressions.
\end{itemize}
\noindent Example 1: 
\begin{center}\begin{minipage}{15cm}\begin{Verbatim}[frame=single]
> print(expand(x^3));
x * x * x
\end{Verbatim}
\end{minipage}\end{center}
\noindent Example 2: 
\begin{center}\begin{minipage}{15cm}\begin{Verbatim}[frame=single]
> print(expand((x + 2)^3 + 2 * x));
8 + 12 * x + 6 * x * x + x * x * x + 2 * x
\end{Verbatim}
\end{minipage}\end{center}
\noindent Example 3: 
\begin{center}\begin{minipage}{15cm}\begin{Verbatim}[frame=single]
> print(expand(exp((x + (x + 3))^5)));
exp(243 + 405 * x + 270 * x * x + 90 * x * x * x + 15 * x * x * x * x + x * x * 
x * x * x + x * 405 + 108 * x * 5 * x + 54 * x * x * 5 * x + 12 * x * x * x * 5 
* x + x * x * x * x * 5 * x + x * x * 270 + 27 * x * x * x * 10 + 9 * x * x * x 
* x * 10 + x * x * x * x * x * 10 + x * x * x * 90 + 6 * x * x * x * x * 10 + x 
* x * x * x * x * 10 + x * x * x * x * 5 * x + 15 * x * x * x * x + x * x * x * 
x * x)
\end{Verbatim}
\end{minipage}\end{center}
See also: \textbf{simplify} (\ref{labsimplify}), \textbf{simplifysafe} (\ref{labsimplifysafe}), \textbf{horner} (\ref{labhorner})

\subsection{ expm1 }
\noindent Name: \textbf{expm1}\\
translated exponential function.\\

\noindent Description: \begin{itemize}

\item \textbf{expm1} is defined by ${\rm expm1}(x) = \exp(x)-1$.

\item It is defined for every real number x.
\end{itemize}
See also: \textbf{exp}

\subsection{exponent}
\label{labexponent}
\noindent Name: \textbf{exponent}\\
returns the scaled binary exponent of a number.\\

\noindent Usage: 
\begin{center}
\textbf{exponent}(\emph{x}) : \textsf{constant} $\rightarrow$ \textsf{integer}\\
\end{center}
Parameters: 
\begin{itemize}
\item \emph{x} is a dyadic number.
\end{itemize}
\noindent Description: \begin{itemize}

\item \textbf{exponent}(x) is by definition 0 if x equals 0, NaN, or Inf.

\item If \emph{x} is not zero, it can be uniquely written as $x = m \cdot 2^e$ where
   $m$ is an odd integer and $e$ is an integer. \textbf{exponent}(x) returns $e$. 
\end{itemize}
\noindent Example 1: 
\begin{center}\begin{minipage}{15cm}\begin{Verbatim}[frame=single]
> a=round(Pi,20,RN);
> e=exponent(a);
> e;
-17
> m=mantissa(a);
> a-m*2^e;
0
\end{Verbatim}
\end{minipage}\end{center}
See also: \textbf{mantissa} (\ref{labmantissa}), \textbf{precision} (\ref{labprecision})

\subsection{exp}
\label{labexp}
\noindent Name: \textbf{exp}\\
the exponential function.\\
\noindent Description: \begin{itemize}

\item \textbf{exp} is the usual exponential function defined as the solution of the
   ordinary differential equation $y'=y$ with $y(0)=1$.

\item \textbf{exp}(x) is defined for every real number $x$.
\end{itemize}
See also: \textbf{exp} (\ref{labexp}), \textbf{log} (\ref{lablog})

\subsection{ externalplot }
\noindent Name: \textbf{externalplot}\\
plots the error of an external code with regard to a function\\

\noindent Usage: 
\begin{center}
\textbf{externalplot}(\emph{filename}, \emph{mode}, \emph{function}, \emph{range}, \emph{precision}) : (\textsf{string}, \textsf{absolute$|$relative}, \textsf{function}, \textsf{range}, \textsf{integer}) $\rightarrow$ \textsf{void}\\
\textbf{externalplot}(\emph{filename}, \emph{mode}, \emph{function}, \emph{range}, \emph{precision}, \emph{perturb}) : (\textsf{string}, \textsf{absolute$|$relative}, \textsf{function}, \textsf{range}, \textsf{integer}, \textsf{perturb}) $\rightarrow$ \textsf{void}\\
\textbf{externalplot}(\emph{filename}, \emph{mode}, \emph{function}, \emph{range}, \emph{precision}, \emph{plot mode}, \emph{result filename}) : (\textsf{string}, \textsf{absolute$|$relative}, \textsf{function}, \textsf{range}, \textsf{integer}, \textsf{file$|$postscript$|$postscriptfile}, \textsf{string}) $\rightarrow$ \textsf{void}\\
\textbf{externalplot}(\emph{filename}, \emph{mode}, \emph{function}, \emph{range}, \emph{precision}, \emph{perturb}, \emph{plot mode}, \emph{result filename}) : (\textsf{string}, \textsf{absolute$|$relative}, \textsf{function}, \textsf{range}, \textsf{integer}, \textsf{perturb}, \textsf{file$|$postscript$|$postscriptfile}, \textsf{string}) $\rightarrow$ \textsf{void}\\
\end{center}
\noindent Description: \begin{itemize}

\item The command \textbf{externalplot} plots the error of an external function
   evaluation code sequence implemented in the object file named
   \emph{filename} with regard to the function \emph{function}.  If \emph{mode}
   evaluates to \emph{absolute}, the difference of both functions is
   considered as an error function; if \emph{mode} evaluates to \emph{relative},
   the difference is quotiented by the function \emph{function}. The resulting
   error function is plotted on all floating-point numbers with
   \emph{precision} significant mantissa bits in the range \emph{range}. 
   If the sixth argument of the command \textbf{externalplot} is given an evaluates to
   \textbf{perturb}, each of these floating-point numbers is perturbed by a
   random value that is uniformly distributed in $\pm1$ ulp
   around the original \emph{precision} bit floating-point variable.
   If a sixth and seventh argument, respectively a seventh and eighth
   argument in the presence of \textbf{perturb} as a sixth argument, are given
   that evaluate to a variable of type \textsf{file|postscript|postscriptfile} respectively to a
   character sequence of type \textsf{string}, \textbf{externalplot} will plot
   (additionally) to a file in the same way as the command \textbf{plot}
   does. See \textbf{plot} for details.
   The external function evaluation code given in the object file name
   \emph{filename} is supposed to define a function name \texttt{f} as
   follows (here in C syntax): \texttt{void f(mpfr\_t rop, mpfr\_ op)}. 
   This function is supposed to evaluate \texttt{op} with an accuracy corresponding
   to the precision of \texttt{rop} and assign this value to
   \texttt{rop}.
\end{itemize}
\noindent Example 1: 
\begin{center}\begin{minipage}{15cm}\begin{Verbatim}[frame=single]
> bashexecute("gcc -fPIC -c externalplotexample.c");
> bashexecute("gcc -shared -o externalplotexample externalplotexample.o -lgmp -l
mpfr");
> externalplot("./externalplotexample",relative,exp(x),[-1/2;1/2],12,perturb);
\end{Verbatim}
\end{minipage}\end{center}
See also: \textbf{plot}, \textbf{asciiplot}, \textbf{perturb}, \textbf{absolute}, \textbf{relative}, \textbf{file}, \textbf{postscript}, \textbf{postscriptfile}, \textbf{bashexecute}, \textbf{externalproc}, \textbf{library}

\subsection{externalproc}
\label{labexternalproc}
\noindent Name: \textbf{externalproc}\\
binds an external code to a \sollya procedure\\
\noindent Usage: 
\begin{center}
\textbf{externalproc}(\emph{identifier}, \emph{filename}, \emph{argumenttype} $->$ \emph{resulttype}) : (\textsf{identifier type}, \textsf{string}, \textsf{type type}, \textsf{type type}) $\rightarrow$ \textsf{void}\\
\end{center}
Parameters: 
\begin{itemize}
\item \emph{identifier} represents the identifier the code is to be bound to
\item \emph{filename} of type \textsf{string} represents the name of the object file where the code of procedure can be found
\item \emph{argumenttype} represents a definition of the types of the arguments of the \sollya procedure and the external code
\item \emph{resulttype} represents a definition of the result type of the external code
\end{itemize}
\noindent Description: \begin{itemize}

\item \textbf{externalproc} allows for binding the \sollya identifier
   \emph{identifier} to an external code.  After this binding, when \sollya
   encounters \emph{identifier} applied to a list of actual parameters, it
   will evaluate these parameters and call the external code with these
   parameters. If the external code indicated success, it will receive
   the result produced by the external code, transform it to \sollya's
   internal representation and return it.
    
   In order to allow correct evaluation and typing of the data in
   parameter and in result to be passed to and received from the external
   code, \textbf{externalproc} has a third parameter \emph{argumenttype} $->$ \emph{resulttype}.
   Both \emph{argumenttype} and \emph{resulttype} are one of \textbf{void}, \textbf{constant},
   \textbf{function}, \textbf{range}, \textbf{integer}, \textbf{string}, \textbf{boolean}, \textbf{list of} \textbf{constant}, \textbf{list of} \textbf{function}, 
   \textbf{list of} \textbf{range}, \textbf{list of} \textbf{integer}, \textbf{list of} \textbf{string}, \textbf{list of} \textbf{boolean}.
    
   If upon a usage of a procedure bound to an external procedure the type
   of the actual parameters given or its number is not correct, \sollya
   produces a type error. An external function not applied to arguments
   represents itself and prints out with its argument and result types.
    
   The external function is supposed to return an integer indicating
   success.  It returns its result depending on its \sollya result type
   as follows. Here, the external procedure is assumed to be implemented
   as a C function.
    
   If the \sollya result type is void, the C function has no pointer
   argument for the result.  If the \sollya result type is \textbf{constant}, the
   first argument of the C function is of C type \texttt{mpfr\_t *}, the result is
   returned by affecting the MPFR variable.  If the \sollya result type
   is \textbf{function}, the first argument of the C function is of C type \texttt{node **},
   the result is returned by the \texttt{node *} pointed with a new \texttt{node *}.
   If the \sollya result type is \textbf{range}, the first argument of the C
   function is of C type \texttt{mpfi\_t *}, the result is returned by affecting
   the MPFI variable.  If the \sollya result type is \textbf{integer}, the first
   argument of the C function is of C type \texttt{int *}, the result is returned
   by affecting the int variable.  If the \sollya result type is \textbf{string},
   the first argument of the C function is of C type \texttt{char **}, the result
   is returned by the \texttt{char *} pointed with a new \texttt{char *}.  If the \sollya
   result type is \textbf{boolean}, the first argument of the C function is of C
   type \texttt{int *}, the result is returned by affecting the int variable with
   a boolean value.  If the \sollya result type is \textbf{list of} type, the
   first argument of the C function is of C type \texttt{chain **}, the result is
   returned by the \texttt{chain *} pointed with a new \texttt{chain *}.  This chain
   contains for \sollya type \textbf{constant} pointers \texttt{mpfr\_t *} to new MPFR
   variables, for \sollya type \textbf{function} pointers \texttt{node *} to new nodes, for
   \sollya type \textbf{range} pointers \texttt{mpfi\_t *}  to new MPFI variables, for
   \sollya type \textbf{integer} pointers \texttt{int *} to new int variables for \sollya
   type \textbf{string} pointers \texttt{char *} to new \texttt{char *} variables and for \sollya
   type \textbf{boolean} pointers \texttt{int *} to new int variables representing boolean
   values.
    	       
   The external procedure affects its possible pointer argument if and
   only if it succeeds.  This means, if the function returns an integer
   indicating failure, it does not leak any memory to the encompassing
   environment.
    
   The external procedure receives its arguments as follows: If the
   \sollya argument type is \textbf{void}, no argument array is given.  Otherwise
   the C function receives a C \texttt{void **} argument representing an array of
   size equal to the arity of the function where each entry (of C type
   \texttt{void *}) represents a value with a C type depending on the
   corresponding \sollya type. If the \sollya type is \textbf{constant}, the C
   type the \texttt{void *} is to be casted to is \texttt{mpfr\_t *}.  If the \sollya type
   is \textbf{function}, the C type the \texttt{void *} is to be casted to is \texttt{node *}.  If
   the \sollya type is \textbf{range}, the C type the \texttt{void *} is to be casted to is
   \texttt{mpfi\_t *}.  If the \sollya type is \textbf{integer}, the C type the \texttt{void *} is to
   be casted to is \texttt{int *}.  If the \sollya type is \textbf{string}, the C type the
   \texttt{void *} is to be casted to is \texttt{char *}.  If the \sollya type is \textbf{boolean},
   the C type the \texttt{void *} is to be casted to is \texttt{int *}.  If the \sollya
   type is \textbf{list of} type, the C type the \texttt{void *} is to be casted to is
   \texttt{chain *}.  Here depending on type, the values in the chain are to be
   casted to \texttt{mpfr\_t *}  for \sollya type \textbf{constant}, \texttt{node *} for \sollya type
   \textbf{function}, \texttt{mpfi\_t *} for \sollya type \textbf{range}, \texttt{int *} for \sollya type
   \textbf{integer}, \texttt{char *} for \sollya type \textbf{string} and \texttt{int *} for \sollya type
   \textbf{boolean}.
    
   The external procedure is not supposed to alter the memory pointed by
   its array argument \texttt{void **}.
    
   In both directions (argument and result values), empty lists are
   represented by \texttt{chain * NULL} pointers.
    
   In contrast to internal procedures, externally bounded procedures can
   be considered to be objects inside \sollya that can be assigned to other
   variables, stored in list etc.
\end{itemize}
\noindent Example 1: 
\begin{center}\begin{minipage}{15cm}\begin{Verbatim}[frame=single]
> bashexecute("gcc -fPIC -Wall -c externalprocexample.c");
> bashexecute("gcc -fPIC -shared -o externalprocexample externalprocexample.o");

> externalproc(foo, "./externalprocexample", (integer, integer) -> integer);
> foo;
foo(integer, integer) -> integer
> foo(5, 6);
11
> verbosity = 1!;
> foo();
Warning: at least one of the given expressions or a subexpression is not correct
ly typed
or its evaluation has failed because of some error on a side-effect.
error
> a = foo;
> a(5,6);
11
\end{Verbatim}
\end{minipage}\end{center}
See also: \textbf{library} (\ref{lablibrary}), \textbf{externalplot} (\ref{labexternalplot}), \textbf{bashexecute} (\ref{labbashexecute}), \textbf{void} (\ref{labvoid}), \textbf{constant} (\ref{labconstant}), \textbf{function} (\ref{labfunction}), \textbf{range} (\ref{labrange}), \textbf{integer} (\ref{labinteger}), \textbf{string} (\ref{labstring}), \textbf{boolean} (\ref{labboolean}), \textbf{list of} (\ref{lablistof})

\subsection{false}
\label{labfalse}
\noindent Name: \textbf{false}\\
\phantom{aaa}the boolean value representing the false.\\[0.2cm]
\noindent Library names:\\
\verb|   sollya_obj_t sollya_lib_false()|\\
\verb|   int sollya_lib_is_false(sollya_obj_t)|\\[0.2cm]
\noindent Description: \begin{itemize}

\item \textbf{false} is the usual boolean value.
\end{itemize}
\noindent Example 1: 
\begin{center}\begin{minipage}{15cm}\begin{Verbatim}[frame=single,commandchars=\\\|\~]
> true && false;
false
> 2<1;
false
\end{Verbatim}
\end{minipage}\end{center}
See also: \textbf{true} (\ref{labtrue}), \textbf{$\&\&$} (\ref{laband}), \textbf{$||$} (\ref{labor})

\subsection{ file }
\noindent Name: \textbf{file}\\
special value for commands \textbf{plot} and \textbf{externalplot}\\

\noindent Description: \begin{itemize}

\item \textbf{file} is a special value used in commands \textbf{plot} and \textbf{externalplot} to save
   the result of the command in a data file.

\item As any value it can be affected to a variable and stored in lists.
\end{itemize}
\noindent Example 1: 
\begin{center}\begin{minipage}{15cm}\begin{Verbatim}[frame=single]
> savemode=file;
> name="plotSinCos";
> plot(sin(x),0,cos(x),[-Pi,Pi],savemode, name);
\end{Verbatim}
\end{minipage}\end{center}
See also: \textbf{externalplot}, \textbf{plot}, \textbf{postscript}, \textbf{postscriptfile}

\subsection{findzeros}
\label{labfindzeros}
\noindent Name: \textbf{findzeros}\\
gives a list of intervals containing all zeros of a function on an interval.\\
\noindent Usage: 
\begin{center}
\textbf{findzeros}(\emph{f},\emph{I}) : (\textsf{function}, \textsf{range}) $\rightarrow$ \textsf{list}\\
\end{center}
Parameters: 
\begin{itemize}
\item \emph{f} is a function.
\item \emph{I} is an interval.
\end{itemize}
\noindent Description: \begin{itemize}

\item \\textbf{findzeros}(\\emph{f},\\emph{I}) returns a list of intervals $I_1, \\dots, I_n$ such that, for \n   every zero $z$ of $f$, there exists some $k$ such that $z \\in I_k$.\n
\item The list may contain intervals $I_k$ that do not contain any zero of \\emph{f}.\n   An interval \\emph{Ik} may contain many zeros of \\emph{f}.\n
\item This command is meant for cases when safety is critical. If you want to be sure\n   not to forget any zero, use \\textbf{findzeros}. However, if you just want to know \n   numerical values for the zeros of \\emph{f}, \\textbf{dirtyfindzeros} should be quite \n   satisfactory and a lot faster.\n
\item If $\\delta$ denotes the value of global variable \\textbf{diam}, the algorithm ensures\n   that for each $k$, $|I_k| \\le \\delta \\cdot |I|$.\n
\item The algorithm used is basically a bisection algorithm. It is the same algorithm\n   that the one used for \\textbf{infnorm}. See the help page of this command for more \n   details. In short, the behavior of the algorithm depends on global variables\n   \\textbf{prec}, \\textbf{diam}, \\textbf{taylorrecursions} and \\textbf{hopitalrecursions}.\n\end{itemize}
\noindent Example 1: 
\begin{center}\begin{minipage}{15cm}\begin{Verbatim}[frame=single]
\end{Verbatim}
\end{minipage}\end{center}
See also: \textbf{dirtyfindzeros} (\ref{labdirtyfindzeros}), \textbf{infnorm} (\ref{labinfnorm}), \textbf{prec} (\ref{labprec}), \textbf{diam} (\ref{labdiam}), \textbf{taylorrecursions} (\ref{labtaylorrecursions}), \textbf{hopitalrecursions} (\ref{labhopitalrecursions})

\subsection{fixed}
\label{labfixed}
\noindent Name: \textbf{fixed}\\
indicates that fixed-point formats should be used for \textbf{fpminimax}\\
\noindent Usage: 
\begin{center}
\textbf{fixed} : \textsf{fixed$|$floating}
\\ 
\end{center}
\noindent Description: \begin{itemize}

\item The use of \textbf{fixed} in the command \textbf{fpminimax} indicates that the list of
   formats given as argument is to be considered to be a list of fixed-point
   formats.
   See \textbf{fpminimax} for details.
\end{itemize}
\noindent Example 1: 
\begin{center}\begin{minipage}{15cm}\begin{Verbatim}[frame=single]
> fpminimax(cos(x),6,[|32,32,32,32,32,32,32|],[-1;1],fixed);
0.9999997480772435665130615234375 + x^2 * (-0.4999928693287074565887451171875 + 
x^2 * (4.163351492024958133697509765625e-2 + x^2 * (-1.3382239267230033874511718
75e-3)))
\end{Verbatim}
\end{minipage}\end{center}
See also: \textbf{fpminimax} (\ref{labfpminimax}), \textbf{floating} (\ref{labfloating})

\subsection{floating}
\label{labfloating}
\noindent Name: \textbf{floating}\\
\phantom{aaa}indicates that floating-point formats should be used for \textbf{fpminimax}\\[0.2cm]
\noindent Library names:\\
\verb|   sollya_obj_t sollya_lib_floating()|\\
\verb|   int sollya_lib_is_floating(sollya_obj_t)|\\[0.2cm]
\noindent Usage: 
\begin{center}
\textbf{floating} : \textsf{fixed$|$floating}\\
\end{center}
\noindent Description: \begin{itemize}

\item The use of \textbf{floating} in the command \textbf{fpminimax} indicates that the list of
   formats given as argument is to be considered to be a list of floating-point
   formats.
   See \textbf{fpminimax} for details.
\end{itemize}
\noindent Example 1: 
\begin{center}\begin{minipage}{15cm}\begin{Verbatim}[frame=single]
> fpminimax(cos(x),6,[|D...|],[-1;1],floating);
0.99999974816012215939053930924274027347564697265625 + x * (-2.79593179695850233
4440230695107655659202089892465e-15 + x * (-0.4999928698020140171998093592264922
3357439041137695 + x * (4.0484539189054105169841244454207387920433372507922e-14 
+ x * (4.1633515528919168291466235132247675210237503051758e-2 + x * (-4.01585881
8743733758578949218474363725507386355118e-14 + x * (-1.3382240885483781024645200
119493892998434603214264e-3))))))
\end{Verbatim}
\end{minipage}\end{center}
See also: \textbf{fpminimax} (\ref{labfpminimax}), \textbf{fixed} (\ref{labfixed})

\subsection{floor}
\label{labfloor}
\noindent Name: \textbf{floor}\\
the usual function floor.\\
\noindent Description: \begin{itemize}

\item \\textbf{floor} is defined as usual: \\textbf{floor}($x$) is the greatest integer y such that $y \\le x$.\n
\item It is defined for every real number $x$.\n\end{itemize}
See also: \textbf{ceil} (\ref{labceil}), \textbf{nearestint} (\ref{labnearestint})

\subsection{fpminimax}
\label{labfpminimax}
\noindent Name: \textbf{fpminimax}\\
computes a good polynomial approximation with fixed-point or floating-point coefficients\\
\noindent Usage: 
\begin{center}
\textbf{fpminimax}(\emph{f}, \emph{n}, \emph{formats}, \emph{range}, \emph{indic1}, \emph{indic2}, \emph{indic3}, \emph{P}) : (\textsf{function}, \textsf{integer}, \textsf{list}, \textsf{range}, \textsf{absolute$|$relative} $|$ \textsf{fixed$|$floating} $|$ \textsf{function}, \textsf{absolute$|$relative} $|$ \textsf{fixed$|$floating} $|$ \textsf{function}, \textsf{absolute$|$relative} $|$ \textsf{fixed$|$floating} $|$ \textsf{function}, \textsf{function}) $\rightarrow$ \textsf{function}\\
\textbf{fpminimax}(\emph{f}, \emph{monomials}, \emph{formats}, \emph{range}, \emph{indic1}, \emph{indic2}, \emph{indic3}, \emph{P}) : (\textsf{function}, \textsf{list}, \textsf{list}, \textsf{range},  \textsf{absolute$|$relative} $|$ \textsf{fixed$|$floating} $|$ \textsf{function}, \textsf{absolute$|$relative} $|$ \textsf{fixed$|$floating} $|$ \textsf{function}, \textsf{absolute$|$relative} $|$ \textsf{fixed$|$floating} $|$ \textsf{function}, \textsf{function}) $\rightarrow$ \textsf{function}\\
\textbf{fpminimax}(\emph{f}, \emph{n}, \emph{formats}, \emph{L}, \emph{indic1}, \emph{indic2}, \emph{indic3}, \emph{P}) : (\textsf{function}, \textsf{integer}, \textsf{list}, \textsf{list},  \textsf{absolute$|$relative} $|$ \textsf{fixed$|$floating} $|$ \textsf{function}, \textsf{absolute$|$relative} $|$ \textsf{fixed$|$floating} $|$ \textsf{function}, \textsf{absolute$|$relative} $|$ \textsf{fixed$|$floating} $|$ \textsf{function}, \textsf{function}) $\rightarrow$ \textsf{function}\\
\textbf{fpminimax}(\emph{f}, \emph{monomials}, \emph{formats}, \emph{L}, \emph{indic1}, \emph{indic2}, \emph{indic3}, \emph{P}) : (\textsf{function}, \textsf{list}, \textsf{list}, \textsf{list},  \textsf{absolute$|$relative} $|$ \textsf{fixed$|$floating} $|$ \textsf{function}, \textsf{absolute$|$relative} $|$ \textsf{fixed$|$floating} $|$ \textsf{function}, \textsf{absolute$|$relative} $|$ \textsf{fixed$|$floating} $|$ \textsf{function}, \textsf{function}) $\rightarrow$ \textsf{function}\\
\end{center}
Parameters: 
\begin{itemize}
\item \emph{f} is the function to be approximated
\item \emph{n} is the degree of the polynomial that must approximate \emph{f}
\item \emph{monomials} is the list of monomials that must be used to represent the polynomial that approximates~\emph{f}
\item \emph{formats} is a list indicating the formats that the coefficients of the polynomial must have
\item \emph{range} is the interval where the function must be approximated
\item \emph{L} is a list of interpolation points used by the method
\item \emph{indic1} (optional) is one of the optional indication parameters. See the detailed description below.
\item \emph{indic2} (optional) is one of the optional indication parameters. See the detailed description below.
\item \emph{indic3} (optional) is one of the optional indication parameters. See the detailed description below.
\item \emph{P} (optional) is the minimax polynomial to be considered for solving the problem.
\end{itemize}
\noindent Description: \begin{itemize}

\item \textbf{fpminimax} uses a heuristic (but practically efficient) method to find a good
   polynomial approximation of a function \emph{f} on an interval \emph{range}. It 
   implements the method published in the article:\\
   Efficient polynomial $L^\infty$-approximations\\ 
   Nicolas Brisebarre and Sylvain Chevillard\\
   Proceedings of the 18th IEEE Symposium on Computer Arithmetic (ARITH 18)\\
   pp. 169-176

\item The basic usage of this command is \textbf{fpminimax}(\emph{f}, \emph{n}, \emph{formats}, \emph{range}).
   It computes a polynomial approximation of $f$ with degree at most $n$
   on the interval \emph{range}. \emph{formats} is a list of integers or format types 
   (such as \textbf{double}, \textbf{doubledouble}, etc.). The polynomial returned by the
   command has its coefficients that fit the formats indications. For 
   instance, if formats[0] is 35, the coefficient of degree 0 of the 
   polynomial will fit a floating-point format of 35 bits. If formats[1] 
   is D, the coefficient of degree 1 will be representable by a floating-point
   number with a precision of 53 bits (which is not necessarily an IEEE 754 double
   precision number. See the remark below), etc.

\item The second argument may be either an integer or a list of integers
   interpreted as the list of desired monomials. For instance, the list
   $[|0,\,2,\,4,\,6|]$ indicates that the polynomial must be even and of
   degree at most 6. Giving an integer $n$ as second argument is equivalent
   as giving $[|0,\,\dots,\,n|]$.\\
   The list of formats is interpreted with respect to the list of monomials. For
   instance, if the list of monomials is $[|0,\,2,\,4,\,6|]$ and the list
   of formats is $[|161,\,107,\,53,\,24|]$, the coefficients of degree 0 is 
   searched as a floating-point number with precision 161, the coefficient of 
   degree 2 is searched as a number of precision 107, and so on.

\item The list of formats may contain either integers or format types (\textbf{double},
   \textbf{doubledouble}, \textbf{tripledouble} and \textbf{doubleextended}). The list may be too large
   or even infinite. Only the first indications will be considered. For 
   instance, for a degree $n$ polynomial, $\mathrm{formats}[n+1]$ and above will
   be discarded. This lets one use elliptical indications for the last
   coefficients.

\item The floating-point coefficients considered by \textbf{fpminimax} do not have an
   exponent range. In particular, in the format list, \textbf{double} or 53 does not
   imply that the corresponding coefficient is an IEEE-754 double.

\item By default, the list of formats is interpreted as a list of floating-point
   formats. This may be changed by passing \textbf{fixed} as an optional argument (see
   below). Let us take an example: \textbf{fpminimax}(f, 2, [107, DD, 53], [0;1]).
   Here the optional argument is missing (we could have set it to \textbf{floating}).
   Thus, \textbf{fpminimax} will search for a polynomial of degree 2 with a constant 
   coefficient that is a 107 bits floating-point number, etc.\\
   Currently, \textbf{doubledouble} is just a synonym for 107 and \textbf{tripledouble} a 
   synonym for 161. This behavior may change in the future (taking into
   account the fact that some double-doubles are not representable with
   107 bits).\\
   Second example: \textbf{fpminimax}(f, 2, [25, 18, 30], [0;1], \textbf{fixed}).
   In this case, \textbf{fpminimax} will search for a polynomial of degree 2 with a
   constant coefficient of the form $m/2^{25}$ where $m$ is an
   integer. In other words, it is a fixed-point number with 25 bits after
   the point. Note that even with argument \textbf{fixed}, the formats list is 
   allowed to contain \textbf{double}, \textbf{doubledouble} or \textbf{tripledouble}. In this this
   case, it is just a synonym for 53, 107 or 161. This is deprecated and may
   change in the future.

\item The fourth argument may be a range or a list. Lists are for advanced users
   that know what they are doing. The core of the  method is a kind of
   approximated interpolation. The list given here is a list of points that
   must be considered for the interpolation. It must contain at least as 
   many points as unknown coefficients. If you give a list, it is also 
   recommended that you provide the minimax polynomial as last argument.
   If you give a range, the list of points will be automatically computed.

\item The fifth, sixth and seventh arguments are optional. By default, \textbf{fpminimax}
   will approximate $f$ while optimizing the relative error, and interpreting
   the list of formats as a list of floating-point formats.\\
   This default behavior may be changed with these optional arguments. You
   may provide zero, one, two or three of the arguments in any order.
   This lets the user indicate only the non-default arguments.\\
   The three possible arguments are: \begin{itemize}
   \item \textbf{relative} or \textbf{absolute}: the error to be optimized;
   \item \textbf{floating} or \textbf{fixed}: formats of the coefficients;
   \item a constrained part $q$.
   \end{itemize}
   The constrained part lets the user assign in advance some of the
   coefficients. For instance, for approximating $\exp(x)$, it may
   be interesting to search for a polynomial $p$ of the form
                   $$p = 1 + x + \frac{x^2}{2} + a_3 x^3 + a_4 x^4.$$
   Thus, there is a constrained part $q = 1 + x + x^2/2$ and the unknown
   polynomial should be considered in the monomial basis $[|3, 4|]$.
   Calling \textbf{fpminimax} with monomial basis $[|3,\,4|]$ and constrained
   part $q$, will return a polynomial with the right form.

\item The last argument is for advanced users. It is the minimax polynomial that
   approximates the function $f$ in the monomial basis. If it is not given
   this polynomial will be automatically computed by \textbf{fpminimax}.
   \\
   This minimax polynomial is used to compute the list of interpolation
   points required by the method. In general, you do not have to provide this
   argument. But if you want to obtain several polynomials of the same degree
   that approximate the same function on the same range, just changing the
   formats, you should probably consider computing only once the minimax
   polynomial and the list of points instead of letting \textbf{fpminimax} recompute
   them each time.
   \\
   Note that in the case when a constrained part is given, the minimax 
   polynomial must take that into account. For instance, in the previous
   example, the minimax would be obtained by the following command:
          \begin{center}\verb~P = remez(1-(1+x+x^2/2)/exp(x), [|3,4|], range, 1/exp(x));~\end{center}
   Note that the constrained part is not to be added to $P$.

\item Note that \textbf{fpminimax} internally computes a minimax polynomial (using
   the same algorithm as \textbf{remez} command). Thus \textbf{fpminimax} may encounter
   the same problems as \textbf{remez}. In particular, it may be very slow 
   when Haar condition is not fulfilled. Another consequence is that
   currently \textbf{fpminimax} has to be run with a sufficiently high working precision.
\end{itemize}
\noindent Example 1: 
\begin{center}\begin{minipage}{15cm}\begin{Verbatim}[frame=single]
> P = fpminimax(cos(x),6,[|DD, DD, D...|],[-1b-5;1b-5]);
> printexpansion(P);
(0x3ff0000000000000 + 0xbc09fda20235c100) + x * ((0x3b29ecd485d34781 + 0xb7c1cbc
97120359a) + x * (0xbfdfffffffffff98 + x * (0xbbfa6e0b3183cb0d + x * (0x3fa55555
55145337 + x * (0x3ca3540480618939 + x * 0xbf56c138142d8c3b)))))
\end{Verbatim}
\end{minipage}\end{center}
\noindent Example 2: 
\begin{center}\begin{minipage}{15cm}\begin{Verbatim}[frame=single]
> P = fpminimax(sin(x),6,[|32...|],[-1b-5;1b-5], fixed, absolute);
> display = powers!;
> P;
x * (1 + x^2 * (-357913941 * 2^(-31) + x^2 * (35789873 * 2^(-32))))
\end{Verbatim}
\end{minipage}\end{center}
\noindent Example 3: 
\begin{center}\begin{minipage}{15cm}\begin{Verbatim}[frame=single]
> P = fpminimax(exp(x), [|3,4|], [|D,24|], [-1/256; 1/246], 1+x+x^2/2);
> display = powers!;
> P;
1 + x * (1 + x * (1 * 2^(-1) + x * (375300225001191 * 2^(-51) + x * (5592621 * 2
^(-27)))))
\end{Verbatim}
\end{minipage}\end{center}
\noindent Example 4: 
\begin{center}\begin{minipage}{15cm}\begin{Verbatim}[frame=single]
> f = cos(exp(x));
> pstar = remez(f, 5, [-1b-7;1b-7]);
> listpoints = dirtyfindzeros(f-pstar, [-1b-7; 1b-7]);
> P1 = fpminimax(f, 5, [|DD...|], listpoints, absolute, default, default, pstar)
;
> P2 = fpminimax(f, 5, [|D...|], listpoints, absolute, default, default, pstar);

> P3 = fpminimax(f, 5, [|D, D, D, 24...|], listpoints, absolute, default, defaul
t, pstar);
> print("Error of pstar: ", dirtyinfnorm(f-pstar, [-1b-7; 1b-7]));
Error of pstar:  7.9048441305459735102879831325718745399379329453102e-16
> print("Error of P1:    ", dirtyinfnorm(f-P1, [-1b-7; 1b-7]));
Error of P1:     7.9048441305459735159848647089192667442047469404883e-16
> print("Error of P2:    ", dirtyinfnorm(f-P2, [-1b-7; 1b-7]));
Error of P2:     8.2477144579950871061147021597406077993657714575238e-16
> print("Error of P3:    ", dirtyinfnorm(f-P3, [-1b-7; 1b-7]));
Error of P3:     1.08454277156993282593701156841863009789063333951055e-15
\end{Verbatim}
\end{minipage}\end{center}
See also: \textbf{remez} (\ref{labremez}), \textbf{dirtyfindzeros} (\ref{labdirtyfindzeros}), \textbf{absolute} (\ref{lababsolute}), \textbf{relative} (\ref{labrelative}), \textbf{fixed} (\ref{labfixed}), \textbf{floating} (\ref{labfloating}), \textbf{default} (\ref{labdefault}), \textbf{single} (\ref{labsingle}), \textbf{double} (\ref{labdouble}), \textbf{doubledouble} (\ref{labdoubledouble}), \textbf{tripledouble} (\ref{labtripledouble}), \textbf{doubleextended} (\ref{labdoubleextended}), \textbf{implementpoly} (\ref{labimplementpoly}), \textbf{coeff} (\ref{labcoeff}), \textbf{degree} (\ref{labdegree}), \textbf{roundcoefficients} (\ref{labroundcoefficients}), \textbf{guessdegree} (\ref{labguessdegree})

\subsection{fullparentheses}
\label{labfullparentheses}
\noindent Name: \textbf{fullparentheses}\\
activates, deactivates or inspects the state variable controlling output with full parenthesizing\\

\noindent Usage: 
\begin{center}
\textbf{fullparentheses} = \emph{activation value} : \textsf{on$|$off} $\rightarrow$ \textsf{void}\\
\textbf{fullparentheses} = \emph{activation value} ! : \textsf{on$|$off} $\rightarrow$ \textsf{void}\\
\textbf{fullparentheses} = ? : \textsf{void} $\rightarrow$ \textsf{on$|$off}\\
\end{center}
Parameters: 
\begin{itemize}
\item \emph{activation value} represents \textbf{on} or \textbf{off}, i.e. activation or deactivation
\end{itemize}
\noindent Description: \begin{itemize}

\item An assignment \textbf{fullparentheses} = \emph{activation value}, where \emph{activation value}
   is one of \textbf{on} or \textbf{off}, activates respectively deactivates the output
   of expressions with full parenthezing. In full parentheszing mode,
   Sollya commands like \textbf{print}, \textbf{write} and the implicit command when an
   expression is given at the prompt will output expressions with
   parentheses at all places where it is necessary for expressions
   containing infix operators to be reparsed with the same
   result. Otherwise parentheses around associative operators are
   omitted.
    
   If the assignment \textbf{fullparentheses} = \emph{activation value} is followed by an
   exclamation mark, no message indicating the new state is
   displayed. Otherwise the user is informed of the new state of the
   global mode by an indication.

\item The expression \textbf{fullparentheses} = ? evaluates to a variable of type
   \textsf{on$|$off}, indicating whether or not the full parenthesized output
   of expressions is activated or not.
\end{itemize}
\noindent Example 1: 
\begin{center}\begin{minipage}{15cm}\begin{Verbatim}[frame=single]
> autosimplify = off!;
> fullparentheses = off;
Full parentheses mode has been deactivated.
> print(1 + 2 + 3);
1 + 2 + 3
> fullparentheses = on;
Full parentheses mode has been activated.
> print(1 + 2 + 3);
(1 + 2) + 3
\end{Verbatim}
\end{minipage}\end{center}
See also: \textbf{print} (\ref{labprint}), \textbf{write} (\ref{labwrite}), \textbf{autosimplify} (\ref{labautosimplify})

\subsection{function}
\label{labfunction}
\noindent Name: \textbf{function}\\
\phantom{aaa}keyword for declaring a procedure-based function or a keyword representing a \textsf{function} type \\[0.2cm]
\noindent Usage: 
\begin{center}
\textbf{function}(\emph{procedure})  : \textsf{procedure} $\rightarrow$ \textsf{function}\\
\textbf{function} : \textsf{type type}\\
\end{center}
Parameters: 
\begin{itemize}
\item \emph{procedure} is a procedure of type (\textsf{range}, \textsf{integer}, \textsf{integer}) $\rightarrow$ \textsf{range}
\end{itemize}
\noindent Description: \begin{itemize}

\item For the sake of safety and mathematical consistency, \sollya
   distinguishes clearly between functions, seen in the mathematical
   sense of the term, i.e. mappings, and procedures, seen in the sense
   Computer Science gives to functions, i.e. pieces of code that compute
   results for arguments following an algorithm. In some cases however,
   it is interesting to use such Computer Science procedures as
   realisations of mathematical functions, e.g. in order to plot them or
   even to perform polynomial approximation on them. The \textbf{function} keyword
   allows for such a transformation of a \sollya procedure into a \sollya
   function. 

\item The procedure to be used as a function through \textbf{function}(\emph{procedure})
   must be of type (\textsf{range}, \textsf{integer}, \textsf{integer})
   $\rightarrow$ \textsf{range}. This means it must take in argument
   an interval $X$, a degree of differentiation $n$ and a
   working precision $p$. It must return in result an interval
   $Y$ encompassing the image $f^{(n)}(X)$ of the
   $n$-th derivative of the implemented function $f$,
   i.e. $f^{(n)}(X) \subseteq Y$. In order to allow
   \sollya's algorithms to work properly, the procedure must ensure that,
   whenever $(p, \textrm{diam}(X))$ tends to $(+\infty,\,0)$,
   the computed over-estimated bounding $Y$ tends to the actual image $f^{(n)}(X)$.

\item The user must be aware that they are responsible of the correctness
   of the procedure. If, for some $n$ and $X$, \emph{procedure} returns an interval $Y$
   such that $f^{(n)}(X) \not\subseteq Y$, \textbf{function} will successfully
   return a function without any complain, but this function might behave
   inconsistently in further computations.

\item For cases when the procedure does not have the correct signature or
   does not return a finite interval as a result \textbf{function}(\emph{procedure})
   evaluates to Not-A-Number (resp. to an interval of Not-A-Numbers for
   interval evaluation).

\item \textbf{function} also represents the \textsf{function} type for declarations
   of external procedures by means of \textbf{externalproc}.
    
   Remark that in contrast to other indicators, type indicators like
   \textbf{function} cannot be handled outside the \textbf{externalproc} context.  In
   particular, they cannot be assigned to variables.
\end{itemize}
\noindent Example 1: 
\begin{center}\begin{minipage}{15cm}\begin{Verbatim}[frame=single]
> procedure EXP(X,n,p) {
        var res, oldPrec;
        oldPrec = prec;
        prec = p!;
        
        res = exp(X);
        
        prec = oldPrec!;
        return res;
  };
> f = function(EXP);
> f(1);
2.71828182845904523536028747135266249775724709369998
> exp(1);
2.71828182845904523536028747135266249775724709369998
> f(x + 3);
(function(proc(X, n, p)
{
var res, oldPrec;
oldPrec = prec;
prec = p!;
res = exp(X);
prec = oldPrec!;
return res;
}))(3 + x)
> diff(f);
diff(function(proc(X, n, p)
{
var res, oldPrec;
oldPrec = prec;
prec = p!;
res = exp(X);
prec = oldPrec!;
return res;
}))
> (diff(f))(0);
1
> g = f(sin(x));
> g(17);
0.382358169993866834026905546416556413595734583420876
> diff(g);
(diff(function(proc(X, n, p)
{
var res, oldPrec;
oldPrec = prec;
prec = p!;
res = exp(X);
prec = oldPrec!;
return res;
})))(sin(x)) * cos(x)
> (diff(g))(1);
1.25338076749344683697237458088447611474812675164344
> p = remez(f,3,[-1/2;1/2]);
> p;
0.99967120901420646830315493949039176881764871951832 + x * (0.999737029835711401
34762682913614052309208076875596 + x * (0.51049729360282624921622721654643510358
3073053437 + x * 0.169814324607133287588897694747370380479108785868016))
\end{Verbatim}
\end{minipage}\end{center}
See also: \textbf{proc} (\ref{labproc}), \textbf{library} (\ref{lablibrary}), \textbf{procedure} (\ref{labprocedure}), \textbf{externalproc} (\ref{labexternalproc}), \textbf{boolean} (\ref{labboolean}), \textbf{constant} (\ref{labconstant}), \textbf{integer} (\ref{labinteger}), \textbf{list of} (\ref{lablistof}), \textbf{range} (\ref{labrange}), \textbf{string} (\ref{labstring})

\subsection{ ge }
\noindent Name: \textbf{$>=$}\\
greater-than-or-equal-to operator\\

\noindent Usage: 
\begin{center}
\emph{expr1} \textbf{$>=$} \emph{expr2} : (\textsf{constant}, \textsf{constant}) $\rightarrow$ \textsf{boolean}\\
\end{center}
Parameters: 
\begin{itemize}
\item \emph{expr1} and \emph{expr2} represent constant expressions
\end{itemize}
\noindent Description: \begin{itemize}

\item The operator \textbf{$>=$} evaluates to true iff its operands \emph{expr1} and
   \emph{expr2} evaluate to two floating-point numbers $a_1$
   respectively $a_2$ with the global precision \textbf{prec} and
   $a_1$ is greater than or equal to $a_2$. The user should
   be aware of the fact that because of floating-point evaluation, the
   operator \textbf{$>=$} is not exactly the same as the mathematical
   operation \emph{greater-than-or-equal-to}.
\end{itemize}
\noindent Example 1: 
\begin{center}\begin{minipage}{15cm}\begin{Verbatim}[frame=single]
> 5 >= 4;
true
> 5 >= 5;
true
> 5 >= 6;
false
> exp(2) >= exp(1);
true
> log(1) >= exp(2);
false
\end{Verbatim}
\end{minipage}\end{center}
\noindent Example 2: 
\begin{center}\begin{minipage}{15cm}\begin{Verbatim}[frame=single]
> prec = 12;
The precision has been set to 12 bits.
> 16384 >= 16385;
true
\end{Verbatim}
\end{minipage}\end{center}
See also: \textbf{$==$}, \textbf{!$=$}, \textbf{$>$}, \textbf{$<=$}, \textbf{$<$}, \textbf{!}, \textbf{$\&\&$}, \textbf{$||$}, \textbf{prec}

\subsection{ gt }
\noindent Name: \textbf{$>$}\\
greater-than operator\\

\noindent Usage: 
\begin{center}
\emph{expr1} \textbf{$>$} \emph{expr2} : (\textsf{constant}, \textsf{constant}) $\rightarrow$ \textsf{boolean}\\
\end{center}
Parameters: 
\emph{expr1} and \emph{expr2} represent constant expressions\\

\noindent Description: \begin{itemize}

\item The operator \textbf{$>$} evaluates to true iff its operands \emph{expr1} and
   \emph{expr2} evaluate to two floating-point numbers $a_1$
   respectively $a_2$ with the global precision \textbf{prec} and
   $a_1$ is greater than $a_2$. The user should
   be aware of the fact that because of floating-point evaluation, the
   operator \textbf{$>$} is not exactly the same as the mathematical
   operation \emph{greater-than}.
\end{itemize}
\noindent Example 1: 
\begin{center}\begin{minipage}{14.8cm}\begin{Verbatim}[frame=single]
   > 5 > 4;
   true
   > 5 > 5;
   false
   > 5 > 6;
   false
   > exp(2) > exp(1);
   true
   > log(1) > exp(2);
   false
\end{Verbatim}
\end{minipage}\end{center}
\noindent Example 2: 
\begin{center}\begin{minipage}{14.8cm}\begin{Verbatim}[frame=single]
   > prec = 12;
   The precision has been set to 12 bits.
   > 16384 > 16385;
   false
\end{Verbatim}
\end{minipage}\end{center}
See also: \textbf{$==$}, \textbf{!$=$}, \textbf{$>=$}, \textbf{$<=$}, \textbf{$<$}, \textbf{!}, \textbf{$\&\&$}, \textbf{$||$}, \textbf{prec}

\subsection{guessdegree}
\label{labguessdegree}
\noindent Name: \textbf{guessdegree}\\
returns the minimal degree needed for a polynomial to approximate a function with a certain error on an interval.\\
\noindent Usage: 
\begin{center}
\textbf{guessdegree}(\emph{f},\emph{I},\emph{eps},\emph{w}) : (\textsf{function}, \textsf{range}, \textsf{constant}, \textsf{function}) $\rightarrow$ \textsf{range}\\
\end{center}
Parameters: 
\begin{itemize}
\item \emph{f} is the function to be approximated.
\item \emph{I} is the interval where the function must be approximated.
\item \emph{eps} is the maximal acceptable error.
\item \emph{w} (optional) is a weight function. Default is 1.
\end{itemize}
\noindent Description: \begin{itemize}

\item \textbf{guessdegree} tries to find the minimal degree needed to approximate \emph{f}
   on \emph{I} by a polynomial with an infinity norm not greater than \emph{eps}.
   More precisely, it finds $n$ minimal such that there exists a
   polynomial $p$ of degree $n$ such that $\|pw-f\|_{\infty} < \mathrm{eps}$.

\item \textbf{guessdegree} returns an interval: for common cases, this interval is reduced to a 
   single number (e.g. the minimal degree). But in certain cases, \textbf{guessdegree} does
   not succeed in finding the minimal degree. In such cases the returned interval
   is of the form $[n,\,p]$ such that:
   \begin{itemize}
   \item no polynomial of degree $n-1$ gives an error less than \emph{eps}.
   \item there exists a polynomial of degree $p$ giving an error less than \emph{eps}. 
   \end{itemize}
\end{itemize}
\noindent Example 1: 
\begin{center}\begin{minipage}{15cm}\begin{Verbatim}[frame=single]
> guessdegree(exp(x),[-1;1],1e-10);
[10;10]
\end{Verbatim}
\end{minipage}\end{center}
\noindent Example 2: 
\begin{center}\begin{minipage}{15cm}\begin{Verbatim}[frame=single]
> guessdegree(1, [-1;1], 1e-8, 1/exp(x));
[8;9]
\end{Verbatim}
\end{minipage}\end{center}
See also: \textbf{dirtyinfnorm} (\ref{labdirtyinfnorm}), \textbf{remez} (\ref{labremez})

\subsection{head}
\label{labhead}
\noindent Name: \textbf{head}\\
gives the first element of a list.\\

\noindent Usage: 
\begin{center}
\textbf{head}(\emph{L}) : \textsf{list} $\rightarrow$ \textsf{any type}\\
\end{center}
Parameters: 
\begin{itemize}
\item \emph{L} is a list.
\end{itemize}
\noindent Description: \begin{itemize}

\item \textbf{head}(\emph{L}) returns the first element of the list \emph{L}. It is equivalent
   to L[0].

\item If \emph{L} is empty, the command will fail with an error.
\end{itemize}
\noindent Example 1: 
\begin{center}\begin{minipage}{15cm}\begin{Verbatim}[frame=single]
> head([|1,2,3|]);
1
> head([|1,2...|]);
1
\end{Verbatim}
\end{minipage}\end{center}
See also: \textbf{tail} (\ref{labtail})

\subsection{hexadecimal}
\label{labhexadecimal}
\noindent Name: \textbf{hexadecimal}\\
special value for global state \textbf{display}\\

\noindent Description: \begin{itemize}

\item \textbf{hexadecimal} is a special value used for the global state \textbf{display}.  If
   the global state \textbf{display} is equal to \textbf{hexadecimal}, all data will be
   output in hexadecimal C99/ IEEE 754R notation.
    
   As any value it can be affected to a variable and stored in lists.
\end{itemize}
See also: \textbf{decimal} (\ref{labdecimal}), \textbf{dyadic} (\ref{labdyadic}), \textbf{powers} (\ref{labpowers}), \textbf{binary} (\ref{labbinary})

\subsection{honorcoeffprec}
\label{labhonorcoeffprec}
\noindent Name: \textbf{honorcoeffprec}\\
indicates the (forced) honoring the precision of the coefficients in 	extbf{implementpoly}\\
\noindent Usage: 
\begin{center}
\textbf{honorcoeffprec} : \textsf{honorcoeffprec}
\end{center}
\noindent Description: \begin{itemize}

\item Used with command \textbf{implementpoly}, \textbf{honorcoeffprec} makes \textbf{implementpoly} honor
   the precision of the given polynomial. This means if a coefficient
   needs a double-double or a triple-double to be exactly stored,
   \textbf{implementpoly} will allocate appropriate space and use a double-double
   or triple-double operation even if the automatic (heuristic)
   determination implemented in command \textbf{implementpoly} indicates that the
   coefficient could be stored on less precision or, respectively, the
   operation could be performed with less precision. See \textbf{implementpoly}
   for details.
\end{itemize}
\noindent Example 1: 
\begin{center}\begin{minipage}{15cm}\begin{Verbatim}[frame=single]
> verbosity = 1!;
> q = implementpoly(1 - simplify(TD(1/6)) * x^2,[-1b-10;1b-10],1b-60,DD,"p","imp
lementation.c");
Warning: at least one of the coefficients of the given polynomial has been round
ed in a way
that the target precision can be achieved at lower cost. Nevertheless, the imple
mented polynomial
is different from the given one.
> printexpansion(q);
0x3ff0000000000000 + x^2 * 0xbfc5555555555555
> r = implementpoly(1 - simplify(TD(1/6)) * x^2,[-1b-10;1b-10],1b-60,DD,"p","imp
lementation.c",honorcoeffprec);
Warning: the infered precision of the 2th coefficient of the polynomial is great
er than
the necessary precision computed for this step. This may make the automatic dete
rmination
of precisions useless.
> printexpansion(r);
0x3ff0000000000000 + x^2 * (0xbfc5555555555555 + 0xbc65555555555555 + 0xb9055555
55555555)
\end{Verbatim}
\end{minipage}\end{center}
See also: \textbf{implementpoly} (\ref{labimplementpoly}), \textbf{printexpansion} (\ref{labprintexpansion})

\subsection{hopitalrecursions}
\label{labhopitalrecursions}
\noindent Name: \textbf{hopitalrecursions}\\
controls the number of recursion steps when applying L'Hopital's rule.\\
\noindent Description: \begin{itemize}

\item \textbf{hopitalrecursions} is a global variable. Its value represents the number of steps of
   recursion that are tried when applying L'Hopital's rule. This rule is applied
   by the interval evaluator present in the core of \sollya (and particularly
   visible in commands like \textbf{infnorm}).

\item If an expression of the form $f/g$ has to be evaluated by interval 
   arithmetic on an interval $I$ and if $f$ and $g$ have a common zero
   in $I$, a direct evaluation leads to NaN.
   \sollya implements a safe heuristic to avoid this, based on L'Hopital's rule: in 
   such a case, it can be shown that $(f/g)(I) \subseteq (f'/g')(I)$. Since
   the same problem may hold for $f'/g'$, the rule is applied recursively.
   The number of step in this recursion process is controlled by \textbf{hopitalrecursions}.

\item Setting \textbf{hopitalrecursions} to 0 makes \sollya use this rule only one time ;
   setting it to 1 makes \sollya use the rule two times, and so on.
   In particular: the rule is always applied at least once, if necessary.
\end{itemize}
\noindent Example 1: 
\begin{center}\begin{minipage}{15cm}\begin{Verbatim}[frame=single]
> hopitalrecursions=0;
The number of recursions for Hopital's rule has been set to 0.
> evaluate(log(1+x)^2/x^2,[-1/2; 1]);
[-@Inf@;@Inf@]
> hopitalrecursions=1;
The number of recursions for Hopital's rule has been set to 1.
> evaluate(log(1+x)^2/x^2,[-1/2; 1]);
[-2.52258872223978123766892848583270627230200053744108;6.77258872223978123766892
84858327062723020005374411]
\end{Verbatim}
\end{minipage}\end{center}

\subsection{horner}
\label{labhorner}
\noindent Name: \textbf{horner}\\
brings all polynomial subexpressions of an expression to Horner form\\

\noindent Usage: 
\begin{center}
\textbf{horner}(\emph{function}) : \textsf{function} $\rightarrow$ \textsf{function}\\
\end{center}
Parameters: 
\begin{itemize}
\item \emph{function} represents the expression to be rewritten in Horner form
\end{itemize}
\noindent Description: \begin{itemize}

\item The command \textbf{horner} rewrites the expression representing the function
   \emph{function} in a way such that all polynomial subexpressions (or the
   whole expression itself, if it is a polynomial) are written in Horner
   form.  The command \textbf{horner} does not endanger the safety of
   computations even in Sollya's floating-point environment: the
   function returned is mathematically equal to the function \emph{function}.
\end{itemize}
\noindent Example 1: 
\begin{center}\begin{minipage}{15cm}\begin{Verbatim}[frame=single]
> print(horner(1 + 2 * x + 3 * x^2));
1 + x * (2 + x * 3)
> print(horner((x + 1)^7));
1 + x * (7 + x * (21 + x * (35 + x * (35 + x * (21 + x * (7 + x))))))
\end{Verbatim}
\end{minipage}\end{center}
\noindent Example 2: 
\begin{center}\begin{minipage}{15cm}\begin{Verbatim}[frame=single]
> print(horner(exp((x + 1)^5) - log(asin(x + x^3) + x)));
exp(1 + x * (5 + x * (10 + x * (10 + x * (5 + x))))) - log(asin(x * (1 + x^2)) +
 x)
\end{Verbatim}
\end{minipage}\end{center}
See also: \textbf{canonical} (\ref{labcanonical}), \textbf{print} (\ref{labprint})

\subsection{implementconstant}
\label{labimplementconstant}
\noindent Name: \textbf{implementconstant}\\
implements a constant in arbitrary precision\\
\noindent Usage: 
\begin{center}
\textbf{implementconstant}(\emph{expr}) : \textsf{constant} $\rightarrow$ \textsf{void}\\
\end{center}
\noindent Description: \begin{itemize}

\item The command \textbf{implementconstant} implements the constant expresion \emph{expr} in 
   arbitrary precision. More precisely, it generates the source code (written
   in C, and using MPFR) of a function \texttt{mpfr\_const\_something} with the following
   signature:
   \begin{center}
   \texttt{void mpfr\_const\_something (mpfr\_ptr y, mp\_prec\_t prec)}
   \end{center}
   Let us denote by $c$ the exact mathematical value of the constant defined by
   the expression \emph{expr}. When called with arguments $y$ and prec (where the
   variable $y$ is supposed to be already initialized), the function
   \texttt{mpfr\_const\_something} sets the precision of $y$ to a suitable precision and
   stores in it an approximate value of c such that
   $$|y-c| \le c\,2^{1-\mathrm{prec}}.$$

\item If \emph{expr} refers to a constant defined with \textbf{libraryconstant}, the produced
   code uses the external code implementing this constant.

\item Currently, \textbf{implementconstant} makes the assumption that none of the
   non-trivial subexpressions of \emph{expr} evaluates to $0$. If \emph{expr} contains
   such a subexpression, the behavior of \textbf{implementconstant} is undefined.
\end{itemize}
\noindent Example 1: 
\begin{center}\begin{minipage}{15cm}\begin{Verbatim}[frame=single]
> implementconstant(exp(1)+log(2)/sqrt(1/10));
void
mpfr_const_something (mpfr_ptr y, mp_prec_t prec)
{
  /* Declarations */
  mpfr_t tmp1;
  mpfr_t tmp2;
  mpfr_t tmp3;
  mpfr_t tmp4;
  mpfr_t tmp5;
  mpfr_t tmp6;
  mpfr_t tmp7;

  /* Initializations */
  mpfr_init2 (tmp2, prec+5);
  mpfr_init2 (tmp1, prec+3);
  mpfr_init2 (tmp4, prec+8);
  mpfr_init2 (tmp3, prec+7);
  mpfr_init2 (tmp6, prec+11);
  mpfr_init2 (tmp7, prec+11);
  mpfr_init2 (tmp5, prec+11);
  mpfr_init2 (y, prec+3);

  /* Core */
  mpfr_set_prec (tmp2, prec+4);
  mpfr_set_ui (tmp2, 1, MPFR_RNDN);
  mpfr_set_prec (tmp1, prec+3);
  mpfr_exp (tmp1, tmp2, MPFR_RNDN);
  mpfr_set_prec (tmp4, prec+8);
  mpfr_set_ui (tmp4, 2, MPFR_RNDN);
  mpfr_set_prec (tmp3, prec+7);
  mpfr_log (tmp3, tmp4, MPFR_RNDN);
  mpfr_set_prec (tmp6, prec+11);
  mpfr_set_ui (tmp6, 1, MPFR_RNDN);
  mpfr_set_prec (tmp7, prec+11);
  mpfr_set_ui (tmp7, 10, MPFR_RNDN);
  mpfr_set_prec (tmp5, prec+11);
  mpfr_div (tmp5, tmp6, tmp7, MPFR_RNDN);
  mpfr_set_prec (tmp4, prec+7);
  mpfr_sqrt (tmp4, tmp5, MPFR_RNDN);
  mpfr_set_prec (tmp2, prec+5);
  mpfr_div (tmp2, tmp3, tmp4, MPFR_RNDN);
  mpfr_set_prec (y, prec+3);
  mpfr_add (y, tmp1, tmp2, MPFR_RNDN);

  /* Cleaning stuff */
  mpfr_clear(tmp1);
  mpfr_clear(tmp2);
  mpfr_clear(tmp3);
  mpfr_clear(tmp4);
  mpfr_clear(tmp5);
  mpfr_clear(tmp6);
  mpfr_clear(tmp7);
}
\end{Verbatim}
\end{minipage}\end{center}
\noindent Example 2: 
\begin{center}\begin{minipage}{15cm}\begin{Verbatim}[frame=single]
> bashexecute("gcc -fPIC -Wall -c libraryconstantexample.c -I$HOME/.local/includ
e");
> bashexecute("gcc -shared -o libraryconstantexample libraryconstantexample.o -l
gmp -lmpfr");
> euler_gamma = libraryconstant("./libraryconstantexample");
> implementconstant(euler_gamma^(1/3));
void
mpfr_const_something (mpfr_ptr y, mp_prec_t prec)
{
  /* Declarations */
  mpfr_t tmp1;

  /* Initializations */
  mpfr_init2 (tmp1, prec+1);
  mpfr_init2 (y, prec+2);

  /* Core */
  euler_gamma (tmp1, prec+1);
  mpfr_set_prec (y, prec+2);
  mpfr_root (y, tmp1, 3, MPFR_RNDN);

  /* Cleaning stuff */
  mpfr_clear(tmp1);
}
\end{Verbatim}
\end{minipage}\end{center}
See also: \textbf{implementpoly} (\ref{labimplementpoly}), \textbf{libraryconstant} (\ref{lablibraryconstant}), \textbf{library} (\ref{lablibrary})

\subsection{implementpoly}
\label{labimplementpoly}
\noindent Name: \textbf{implementpoly}\\
implements a polynomial using double, double-double and triple-double arithmetic and generates a Gappa proof\\
\noindent Usage: 
\begin{center}
\textbf{implementpoly}(\emph{polynomial}, \emph{range}, \emph{error bound}, \emph{format}, \emph{functionname}, \emph{filename}) : (\textsf{function}, \textsf{range}, \textsf{constant}, \textsf{D$|$double$|$DD$|$doubledouble$|$TD$|$tripledouble}, \textsf{string}, \textsf{string}) $\rightarrow$ \textsf{function}\\
\textbf{implementpoly}(\emph{polynomial}, \emph{range}, \emph{error bound}, \emph{format}, \emph{functionname}, \emph{filename}, \emph{honor coefficient precisions}) : (\textsf{function}, \textsf{range}, \textsf{constant}, \textsf{D$|$double$|$DD$|$doubledouble$|$TD$|$tripledouble}, \textsf{string}, \textsf{string}, \textsf{honorcoeffprec}) $\rightarrow$ \textsf{function}\\
\textbf{implementpoly}(\emph{polynomial}, \emph{range}, \emph{error bound}, \emph{format}, \emph{functionname}, \emph{filename}, \emph{proof filename}) : (\textsf{function}, \textsf{range}, \textsf{constant}, \textsf{D$|$double$|$DD$|$doubledouble$|$TD$|$tripledouble}, \textsf{string}, \textsf{string}, \textsf{string}) $\rightarrow$ \textsf{function}\\
\textbf{implementpoly}(\emph{polynomial}, \emph{range}, \emph{error bound}, \emph{format}, \emph{functionname}, \emph{filename}, \emph{honor coefficient precisions}, \emph{proof filename}) : (\textsf{function}, \textsf{range}, \textsf{constant}, \textsf{D$|$double$|$DD$|$doubledouble$|$TD$|$tripledouble}, \textsf{string}, \textsf{string}, \textsf{honorcoeffprec}, \textsf{string}) $\rightarrow$ \textsf{function}\\
\end{center}
\noindent Description: \begin{itemize}

\item The command \textbf{implementpoly} implements the polynomial \emph{polynomial} in range
   \emph{range} as a function called \emph{functionname} in \texttt{C} code
   using double, double-double and triple-double arithmetic in a way that
   the rounding error (estimated at its first order) is bounded by \emph{error bound}. 
   The produced code is output in a file named \emph{filename}. The
   argument \emph{format} indicates the double, double-double or triple-double
   format of the variable in which the polynomial varies, influencing
   also in the signature of the \texttt{C} function.
    
   If a seventh or eighth argument \emph{proof filename} is given and if this
   argument evaluates to a variable of type \textsf{string}, the command
   \textbf{implementpoly} will produce a \texttt{Gappa} proof that the
   rounding error is less than the given bound. This proof will be output
   in \texttt{Gappa} syntax in a file name \emph{proof filename}.
    
   The command \textbf{implementpoly} returns the polynomial that has been
   implemented. As the command \textbf{implementpoly} tries to adapt the precision
   needed in each evaluation step to its strict minimum and as it applies
   renormalization to double-double and triple-double precision
   coefficients to bring them to a round-to-nearest expansion form, the
   returned polynomial may differ from the polynomial
   \emph{polynomial}. Nevertheless the difference will be small enough that
   the rounding error bound with regard to the polynomial \emph{polynomial}
   (estimated at its first order) will be less than the given error
   bound.
    
   If a seventh argument \emph{honor coefficient precisions} is given and
   evaluates to a variable \textbf{honorcoeffprec} of type \textsf{honorcoeffprec},
   \textbf{implementpoly} will honor the precision of the given polynomial
   \emph{polynomials}. This means if a coefficient needs a double-double or a
   triple-double to be exactly stored, \textbf{implementpoly} will allocate appropriate
   space and use a double-double or triple-double operation even if the
   automatic (heuristic) determination implemented in command \textbf{implementpoly}
   indicates that the coefficient could be stored on less precision or,
   respectively, the operation could be performed with less
   precision. The use of \textbf{honorcoeffprec} has advantages and
   disadvantages. If the polynomial \emph{polynomial} given has not been
   determined by a process considering directly polynomials with
   floating-point coefficients, \textbf{honorcoeffprec} should not be
   indicated. The \textbf{implementpoly} command can then determine the needed
   precision using the same error estimation as used for the
   determination of the precisions of the operations. Generally, the
   coefficients will get rounded to double, double-double and
   triple-double precision in a way that minimizes their number and
   respects the rounding error bound \emph{error bound}.  Indicating
   \textbf{honorcoeffprec} may in this case short-circuit most precision
   estimations leading to sub-optimal code. On the other hand, if the
   polynomial \emph{polynomial} has been determined with floating-point
   precisions in mind, \textbf{honorcoeffprec} should be indicated because such
   polynomials often are very sensitive in terms of error propagation with
   regard to their coefficients' values. Indicating \textbf{honorcoeffprec}
   prevents the \textbf{implementpoly} command from rounding the coefficients and
   altering by many orders of magnitude the approximation error of the
   polynomial with regard to the function it approximates.
    
   The implementer behind the \textbf{implementpoly} command makes some assumptions on
   its input and verifies them. If some assumption cannot be verified,
   the implementation will not succeed and \textbf{implementpoly} will evaluate to a
   variable \textbf{error} of type \textsf{error}. The same behaviour is observed if
   some file is not writable or some other side-effect fails, e.g. if
   the implementer runs out of memory.
    
   As error estimation is performed only on the first order, the code
   produced by the \textbf{implementpoly} command should be considered valid iff a
   \texttt{Gappa} proof has been produced and successfully run
   in \texttt{Gappa}.
\end{itemize}
\noindent Example 1: 
\begin{center}\begin{minipage}{15cm}\begin{Verbatim}[frame=single]
> implementpoly(1 - 1/6 * x^2 + 1/120 * x^4, [-1b-10;1b-10], 1b-30, D, "p","impl
ementation.c");
1 + x^2 * (-0.166666666666666657414808128123695496469736099243164 + x^2 * 8.3333
333333333332176851016015461937058717012405395e-3)
> readfile("implementation.c");
#define p_coeff_0h 1.00000000000000000000000000000000000000000000000000000000000
000000000000000000000e+00
#define p_coeff_2h -1.6666666666666665741480812812369549646973609924316406250000
0000000000000000000000e-01
#define p_coeff_4h 8.33333333333333321768510160154619370587170124053955078125000
000000000000000000000e-03


void p(double *p_resh, double x) {
double p_x_0_pow2h;


p_x_0_pow2h = x * x;


double p_t_1_0h;
double p_t_2_0h;
double p_t_3_0h;
double p_t_4_0h;
double p_t_5_0h;
 


p_t_1_0h = p_coeff_4h;
p_t_2_0h = p_t_1_0h * p_x_0_pow2h;
p_t_3_0h = p_coeff_2h + p_t_2_0h;
p_t_4_0h = p_t_3_0h * p_x_0_pow2h;
p_t_5_0h = p_coeff_0h + p_t_4_0h;
*p_resh = p_t_5_0h;


}

\end{Verbatim}
\end{minipage}\end{center}
\noindent Example 2: 
\begin{center}\begin{minipage}{15cm}\begin{Verbatim}[frame=single]
> implementpoly(1 - 1/6 * x^2 + 1/120 * x^4, [-1b-10;1b-10], 1b-30, D, "p","impl
ementation.c","implementation.gappa");
1 + x^2 * (-0.166666666666666657414808128123695496469736099243164 + x^2 * 8.3333
333333333332176851016015461937058717012405395e-3)
\end{Verbatim}
\end{minipage}\end{center}
\noindent Example 3: 
\begin{center}\begin{minipage}{15cm}\begin{Verbatim}[frame=single]
> verbosity = 1!;
> q = implementpoly(1 - simplify(TD(1/6)) * x^2,[-1b-10;1b-10],1b-60,DD,"p","imp
lementation.c");
Warning: at least one of the coefficients of the given polynomial has been round
ed in a way
that the target precision can be achieved at lower cost. Nevertheless, the imple
mented polynomial
is different from the given one.
> printexpansion(q);
0x3ff0000000000000 + x^2 * 0xbfc5555555555555
> r = implementpoly(1 - simplify(TD(1/6)) * x^2,[-1b-10;1b-10],1b-60,DD,"p","imp
lementation.c",honorcoeffprec);
Warning: the infered precision of the 2th coefficient of the polynomial is great
er than
the necessary precision computed for this step. This may make the automatic dete
rmination
of precisions useless.
> printexpansion(r);
0x3ff0000000000000 + x^2 * (0xbfc5555555555555 + 0xbc65555555555555 + 0xb9055555
55555555)
\end{Verbatim}
\end{minipage}\end{center}
\noindent Example 4: 
\begin{center}\begin{minipage}{15cm}\begin{Verbatim}[frame=single]
> p = 0x3ff0000000000000 + x * (0x3ff0000000000000 + x * (0x3fe0000000000000 + x
 * (0x3fc5555555555559 + x * (0x3fa55555555555bd + x * (0x3f811111111106e2 + x
 * (0x3f56c16c16bf5eb7 + x * (0x3f2a01a01a292dcd + x * (0x3efa01a0218a016a + x
 * (0x3ec71de360331aad + x * (0x3e927e42e3823bf3 + x * (0x3e5ae6b2710c2c9a + x
 * (0x3e2203730c0a7c1d + x * 0x3de5da557e0781df))))))))))));
> q = implementpoly(p,[-1/2;1/2],1b-60,D,"p","implementation.c",honorcoeffprec,"
implementation.gappa");
> if (q != p) then print("During implementation, rounding has happened.") else p
rint("Polynomial implemented as given.");    
Polynomial implemented as given.
\end{Verbatim}
\end{minipage}\end{center}
See also: \textbf{honorcoeffprec} (\ref{labhonorcoeffprec}), \textbf{roundcoefficients} (\ref{labroundcoefficients}), \textbf{double} (\ref{labdouble}), \textbf{doubledouble} (\ref{labdoubledouble}), \textbf{tripledouble} (\ref{labtripledouble}), \textbf{readfile} (\ref{labreadfile}), \textbf{printexpansion} (\ref{labprintexpansion}), \textbf{error} (\ref{laberror}), \textbf{remez} (\ref{labremez}), \textbf{fpminimax} (\ref{labfpminimax}), \textbf{taylor} (\ref{labtaylor}), \textbf{implementconstant} (\ref{labimplementconstant})

\subsection{infnorm}
\label{labinfnorm}
\noindent Name: \textbf{infnorm}\\
\phantom{aaa}computes an interval bounding the infinity norm of a function on an interval.\\[0.2cm]
\noindent Library names:\\
\verb|   sollya_obj_t sollya_lib_infnorm(sollya_obj_t, sollya_obj_t, ...)|\\
\verb|   sollya_obj_t sollya_lib_v_infnorm(sollya_obj_t, sollya_obj_t, va_list)|\\[0.2cm]
\noindent Usage: 
\begin{center}
\textbf{infnorm}(\emph{f},\emph{I},\emph{filename},\emph{Ilist}) : (\textsf{function}, \textsf{range}, \textsf{string}, \textsf{list}) $\rightarrow$ \textsf{range}\\
\end{center}
Parameters: 
\begin{itemize}
\item \emph{f} is a function.
\item \emph{I} is an interval.
\item \emph{filename} (optional) is the name of the file into a proof will be saved.
\item \emph{IList} (optional) is a list of intervals to be excluded.
\end{itemize}
\noindent Description: \begin{itemize}

\item \textbf{infnorm}(\emph{f},\emph{range}) computes an interval bounding the infinity norm of the 
   given function $f$ on the interval $I$, e.g. computes an interval $J$
   such that $\max_{x \in I} \{|f(x)|\} \subseteq J$.

\item If \emph{filename} is given, a proof in English will be produced (and stored in file
   called \emph{filename}) proving that  $\max_{x \in I} \{|f(x)|\} \subseteq J$.

\item If a list \emph{IList} of intervals $I_1, \dots, I_n$ is given, the infinity norm will
   be computed on $I \backslash (I_1 \cup \dots \cup I_n)$.

\item The function \emph{f} is assumed to be at least twice continuous on \emph{I}. More 
   generally, if \emph{f} is $\mathcal{C}^k$, global variables \textbf{hopitalrecursions} and
   \textbf{taylorrecursions} must have values not greater than $k$.  

\item If the interval is reduced to a single point, the result of \textbf{infnorm} is an 
   interval containing the exact absolute value of \emph{f} at this point.

\item If the interval is not bound, the result will be $[0,\,+\infty]$ 
   which is correct but perfectly useless. \textbf{infnorm} is not meant to be used with 
   infinite intervals.

\item The result of this command depends on the global variables \textbf{prec}, \textbf{diam},
   \textbf{taylorrecursions} and \textbf{hopitalrecursions}. The contribution of each variable is 
   not easy even to analyse.
   \begin{itemize}
   \item The algorithm uses interval arithmetic with precision \textbf{prec}. The
     precision should thus be set high enough to ensure that no critical
     cancellation will occur.
   \item When an evaluation is performed on an interval $[a,\,b]$, if the result
     is considered being too large, the interval is split into $[a,\,\frac{a+b}{2}]$
     and $[\frac{a+b}{2},\,b]$ and so on recursively. This recursion step
     is  not performed if the $(b-a) < \delta \cdot |I|$ where $\delta$ is the value
     of variable \textbf{diam}. In other words, \textbf{diam} controls the minimum length of an
     interval during the algorithm.
   \item To perform the evaluation of a function on an interval, Taylor's rule is
     applied, e.g. $f([a,b]) \subseteq f(m) + [a-m,\,b-m] \cdot f'([a,\,b])$
     where $m=\frac{a+b}{2}$. This rule is recursively applied $n$ times
     where $n$ is the value of variable \textbf{taylorrecursions}. Roughly speaking,
     the evaluations will avoid decorrelation up to order $n$.
   \item When a function of the form $\frac{g}{h}$ has to be evaluated on an
     interval $[a,\,b]$ and when $g$ and $h$ vanish at a same point
     $z$ of the interval, the ratio may be defined even if the expression
     $\frac{g(z)}{h(z)}=\frac{0}{0}$ does not make any sense. In this case, L'Hopital's rule
     may be used and $\left(\frac{g}{h}\right)([a,\,b]) \subseteq \left(\frac{g'}{h'}\right)([a,\,b])$.
     Since the same can occur with the ratio $\frac{g'}{h'}$, the rule is applied
     recursively. The variable \textbf{hopitalrecursions} controls the number of 
     recursion steps.
   \end{itemize}

\item The algorithm used for this command is quite complex to be explained here. 
   Please find a complete description in the following article:\\
        S. Chevillard and C. Lauter\\
        A certified infinity norm for the implementation of elementary functions\\
        LIP Research Report number RR2007-26\\
        http://prunel.ccsd.cnrs.fr/ensl-00119810\\

\item Users should be aware about the fact that the algorithm behind
   \textbf{infnorm} is inefficient in most cases and that other, better suited
   algorithms, such as \textbf{supnorm}, are available inside \sollya. As a
   matter of fact, while \textbf{infnorm} is maintained for compatibility reasons
   with legacy \sollya codes, users are advised to avoid using \textbf{infnorm}
   in new \sollya scripts and to replace it, where possible, by the
   \textbf{supnorm} command.
\end{itemize}
\noindent Example 1: 
\begin{center}\begin{minipage}{15cm}\begin{Verbatim}[frame=single,commandchars=\\\|\~]
> infnorm(exp(x),[-2;3]);
[20.085536923187667740928529654581717896987907838554;20.085536923187667740928529
6545817178969879078385544]
\end{Verbatim}
\end{minipage}\end{center}
\noindent Example 2: 
\begin{center}\begin{minipage}{15cm}\begin{Verbatim}[frame=single,commandchars=\\\|\~]
> infnorm(exp(x),[-2;3],"proof.txt");
[20.085536923187667740928529654581717896987907838554;20.085536923187667740928529
6545817178969879078385544]
\end{Verbatim}
\end{minipage}\end{center}
\noindent Example 3: 
\begin{center}\begin{minipage}{15cm}\begin{Verbatim}[frame=single,commandchars=\\\|\~]
> infnorm(exp(x),[-2;3],[| [0;1], [2;2.5] |]);
[20.085536923187667740928529654581717896987907838554;20.085536923187667740928529
6545817178969879078385544]
\end{Verbatim}
\end{minipage}\end{center}
\noindent Example 4: 
\begin{center}\begin{minipage}{15cm}\begin{Verbatim}[frame=single,commandchars=\\\|\~]
> infnorm(exp(x),[-2;3],"proof.txt", [| [0;1], [2;2.5] |]);
[20.085536923187667740928529654581717896987907838554;20.085536923187667740928529
6545817178969879078385544]
\end{Verbatim}
\end{minipage}\end{center}
\noindent Example 5: 
\begin{center}\begin{minipage}{15cm}\begin{Verbatim}[frame=single,commandchars=\\\|\~]
> infnorm(exp(x),[1;1]);
[2.7182818284590452353602874713526624977572470936999;2.7182818284590452353602874
713526624977572470937]
\end{Verbatim}
\end{minipage}\end{center}
\noindent Example 6: 
\begin{center}\begin{minipage}{15cm}\begin{Verbatim}[frame=single,commandchars=\\\|\~]
> infnorm(exp(x), [log(0);log(1)]);
[0;infty]
\end{Verbatim}
\end{minipage}\end{center}
See also: \textbf{prec} (\ref{labprec}), \textbf{diam} (\ref{labdiam}), \textbf{hopitalrecursions} (\ref{labhopitalrecursions}), \textbf{taylorrecursions} (\ref{labtaylorrecursions}), \textbf{dirtyinfnorm} (\ref{labdirtyinfnorm}), \textbf{checkinfnorm} (\ref{labcheckinfnorm}), \textbf{supnorm} (\ref{labsupnorm}), \textbf{findzeros} (\ref{labfindzeros}), \textbf{diff} (\ref{labdiff}), \textbf{taylorrecursions} (\ref{labtaylorrecursions}), \textbf{autodiff} (\ref{labautodiff}), \textbf{numberroots} (\ref{labnumberroots}), \textbf{taylorform} (\ref{labtaylorform})

\subsection{inf}
\label{labinf}
\noindent Name: \textbf{inf}\\
gives the lower bound of an interval.\\
\noindent Usage: 
\begin{center}
\textbf{inf}(\emph{I}) : \textsf{range} $\rightarrow$ \textsf{constant}
\textbf{inf}(\emph{x}) : \textsf{constant} $\rightarrow$ \textsf{constant}
\end{center}
Parameters: 
\begin{itemize}
\item \emph{I} is an interval.
\item \emph{x} is a real number.
\end{itemize}
\noindent Description: \begin{itemize}

\item Returns the lower bound of the interval \emph{I}. Each bound of an interval has its 
   own precision, so this command is exact, even if the current precision is too 
   small to represent the bound.

\item When called on a real number \emph{x}, \textbf{inf} considers it as an interval formed
   of a single point: [x, x]. In other words, \textbf{inf} behaves like the identity.
\end{itemize}
\noindent Example 1: 
\begin{center}\begin{minipage}{15cm}\begin{Verbatim}[frame=single]
> inf([1;3]);
1
> inf(0);
0
\end{Verbatim}
\end{minipage}\end{center}
\noindent Example 2: 
\begin{center}\begin{minipage}{15cm}\begin{Verbatim}[frame=single]
> display=binary!;
> I=[0.111110000011111_2; 1];
> inf(I);
1.11110000011111_2 * 2^(-1)
> prec=12!;
> inf(I);
1.11110000011111_2 * 2^(-1)
\end{Verbatim}
\end{minipage}\end{center}
See also: \textbf{mid} (\ref{labmid}), \textbf{sup} (\ref{labsup})

\subsection{integer}
\label{labinteger}
\noindent Name: \textbf{integer}\\
keyword representing a machine integer type \\
\noindent Usage: 
\begin{center}
\textbf{integer} : \textsf{type type}
\\ 
\end{center}
\noindent Description: \begin{itemize}

\item \textbf{integer} represents the machine integer type for declarations
   of external procedures \textbf{externalproc}.
    
   Remark that in contrast to other indicators, type indicators like
   \textbf{integer} cannot be handled outside the \textbf{externalproc} context.  In
   particular, they cannot be assigned to variables.
\end{itemize}
See also: \textbf{externalproc} (\ref{labexternalproc}), \textbf{boolean} (\ref{labboolean}), \textbf{constant} (\ref{labconstant}), \textbf{function} (\ref{labfunction}), \textbf{list of} (\ref{lablistof}), \textbf{range} (\ref{labrange}), \textbf{string} (\ref{labstring})

\subsection{integral}
\label{labintegral}
\noindent Name: \textbf{integral}\\
\phantom{aaa}computes an interval bounding the integral of a function on an interval.\\[0.2cm]
\noindent Library name:\\
\verb|   sollya_obj_t sollya_lib_integral(sollya_obj_t, sollya_obj_t)|\\[0.2cm]
\noindent Usage: 
\begin{center}
\textbf{integral}(\emph{f},\emph{I}) : (\textsf{function}, \textsf{range}) $\rightarrow$ \textsf{range}\\
\end{center}
Parameters: 
\begin{itemize}
\item \emph{f} is a function.
\item \emph{I} is an interval.
\end{itemize}
\noindent Description: \begin{itemize}

\item \textbf{integral}(\emph{f},\emph{I}) returns an interval $J$ such that the exact value of 
   the integral of \emph{f} on \emph{I} lies in $J$.

\item This command is safe but very inefficient. Use \textbf{dirtyintegral} if you just want
   an approximate value.

\item The result of this command depends on the global variable \textbf{diam}.
   The method used is the following: \emph{I} is cut into intervals of length not 
   greater then $\delta \cdot |I|$ where $\delta$ is the value
   of global variable \textbf{diam}.
   On each small interval \emph{J}, an evaluation of \emph{f} by interval is
   performed. The result is multiplied by the length of \emph{J}. Finally all values 
   are summed.
\end{itemize}
\noindent Example 1: 
\begin{center}\begin{minipage}{15cm}\begin{Verbatim}[frame=single]
> sin(10);
-0.54402111088936981340474766185137728168364301291622
> integral(cos(x),[0;10]);
[-0.54710197983579690224097637163525943075698599257333;-0.5409401513001318384815
0540881373370744053741191729]
> diam=1e-5!;
> integral(cos(x),[0;10]);
[-0.54432915685955427101857780295936956775293876382777;-0.5437130640124996950803
9644221927489010425803173555]
\end{Verbatim}
\end{minipage}\end{center}
See also: \textbf{diam} (\ref{labdiam}), \textbf{dirtyintegral} (\ref{labdirtyintegral}), \textbf{prec} (\ref{labprec})

\subsection{in}
\label{labin}
\noindent Name: \textbf{in}\\
\phantom{aaa}containment test operator\\[0.2cm]
\noindent Library name:\\
\verb|   sollya_obj_t sollya_lib_cmp_in(sollya_obj_t, sollya_obj_t)|\\[0.2cm]
\noindent Usage: 
\begin{center}
\emph{expr} \textbf{in} \emph{range1} : (\textsf{constant}, \textsf{range}) $\rightarrow$ \textsf{boolean}\\
\emph{range1} \textbf{in} \emph{range2} : (\textsf{range}, \textsf{range}) $\rightarrow$ \textsf{boolean}\\
\end{center}
Parameters: 
\begin{itemize}
\item \emph{expr} represents a constant expression
\item \emph{range1} and \emph{range2} represent ranges (intervals)
\end{itemize}
\noindent Description: \begin{itemize}

\item When its first operand is a constant expression \emph{expr},
   the operator \textbf{in} evaluates to true iff the constant value
   of the expression \emph{expr} is contained in the interval \emph{range1}.

\item When both its operands are ranges (intervals), 
   the operator \textbf{in} evaluates to true iff all values
   in \emph{range1} are contained in the interval \emph{range2}.

\item \textbf{in} is also used as a keyword for loops over the different
   elements of a list.
\end{itemize}
\noindent Example 1: 
\begin{center}\begin{minipage}{15cm}\begin{Verbatim}[frame=single,commandchars=\\\|\~]
> 5 in [-4;7];
true
> 4 in [-1;1];
false
> 0 in sin([-17;17]);
true
\end{Verbatim}
\end{minipage}\end{center}
\noindent Example 2: 
\begin{center}\begin{minipage}{15cm}\begin{Verbatim}[frame=single,commandchars=\\\|\~]
> [5;7] in [2;8];
true
> [2;3] in [4;5];
false
> [2;3] in [2.5;5];
false
\end{Verbatim}
\end{minipage}\end{center}
\noindent Example 3: 
\begin{center}\begin{minipage}{15cm}\begin{Verbatim}[frame=single,commandchars=\\\|\~]
> for i in [|1,...,5|] do print(i);
1
2
3
4
5
\end{Verbatim}
\end{minipage}\end{center}
See also: \textbf{$==$} (\ref{labequal}), \textbf{!$=$} (\ref{labneq}), \textbf{$>=$} (\ref{labge}), \textbf{$>$} (\ref{labgt}), \textbf{$<=$} (\ref{lable}), \textbf{$<$} (\ref{lablt}), \textbf{!} (\ref{labnot}), \textbf{$\&\&$} (\ref{laband}), \textbf{$||$} (\ref{labor}), \textbf{prec} (\ref{labprec}), \textbf{print} (\ref{labprint})

\subsection{isbound}
\label{labisbound}
\noindent Name: \textbf{isbound}\\
\phantom{aaa}indicates whether a variable is bound or not.\\[0.2cm]
\noindent Usage: 
\begin{center}
\textbf{isbound}(\emph{ident}) : \textsf{boolean}\\
\end{center}
Parameters: 
\begin{itemize}
\item \emph{ident} is a name.
\end{itemize}
\noindent Description: \begin{itemize}

\item \textbf{isbound}(\emph{ident}) returns a boolean value indicating whether the name \emph{ident}
   is used or not to represent a variable. It returns true when \emph{ident} is the 
   name used to represent the global variable or if the name is currently used
   to refer to a (possibly local) variable.

\item When a variable is defined in a block and has not been defined outside, 
   \textbf{isbound} returns true when called inside the block, and false outside.
   Note that \textbf{isbound} returns true as soon as a variable has been declared with 
   \textbf{var}, even if no value is actually stored in it.

\item If \emph{ident1} is bound to a variable and if \emph{ident2} refers to the global 
   variable, the command \textbf{rename}(\emph{ident2}, \emph{ident1}) hides the value of \emph{ident1}
   which becomes the global variable. However, if the global variable is again
   renamed, \emph{ident1} gets its value back. In this case, \textbf{isbound}(\emph{ident1}) returns
   true. If \emph{ident1} was not bound before, \textbf{isbound}(\emph{ident1}) returns false after
   that \emph{ident1} has been renamed.
\end{itemize}
\noindent Example 1: 
\begin{center}\begin{minipage}{15cm}\begin{Verbatim}[frame=single]
> isbound(x);
false
> isbound(f);
false
> isbound(g);
false
> f=sin(x);
> isbound(x);
true
> isbound(f);
true
> isbound(g);
false
\end{Verbatim}
\end{minipage}\end{center}
\noindent Example 2: 
\begin{center}\begin{minipage}{15cm}\begin{Verbatim}[frame=single]
> isbound(a);
false
> { var a; isbound(a); };
true
> isbound(a);
false
\end{Verbatim}
\end{minipage}\end{center}
\noindent Example 3: 
\begin{center}\begin{minipage}{15cm}\begin{Verbatim}[frame=single]
> f=sin(x);
> isbound(x);
true
> rename(x,y);
> isbound(x);
false
\end{Verbatim}
\end{minipage}\end{center}
\noindent Example 4: 
\begin{center}\begin{minipage}{15cm}\begin{Verbatim}[frame=single]
> x=1;
> f=sin(y);
> rename(y,x);
> f;
sin(x)
> x;
x
> isbound(x);
true
> rename(x,y);
> isbound(x);
true
> x;
1
\end{Verbatim}
\end{minipage}\end{center}
See also: \textbf{rename} (\ref{labrename})

\subsection{isevaluable}
\label{labisevaluable}
\noindent Name: \textbf{isevaluable}\\
tests whether a function can be evaluated at a point \\

\noindent Usage: 
\begin{center}
\textbf{isevaluable}(\emph{function}, \emph{constant}) : (\textsf{function}, \textsf{constant}) $\rightarrow$ \textsf{boolean}\\
\end{center}
Parameters: 
\begin{itemize}
\item \emph{function} represents a function
\item \emph{constant} represents a constant point
\end{itemize}
\noindent Description: \begin{itemize}

\item \textbf{isevaluable} applied to function \emph{function} and a constant \emph{constant} returns
   a boolean indicating whether or not a subsequent call to \textbf{evaluate} on the
   same function \emph{function} and constant \emph{constant} will produce a numerical
   result or NaN. I.e. \textbf{isevaluable} returns false iff \textbf{evaluate} will return NaN.
\end{itemize}
\noindent Example 1: 
\begin{center}\begin{minipage}{15cm}\begin{Verbatim}[frame=single]
> isevaluable(sin(pi * 1/x), 0.75);
true
> print(evaluate(sin(pi * 1/x), 0.75));
-0.86602540378443864676372317075293618347140262690518
\end{Verbatim}
\end{minipage}\end{center}
\noindent Example 2: 
\begin{center}\begin{minipage}{15cm}\begin{Verbatim}[frame=single]
> isevaluable(sin(pi * 1/x), 0.5);
true
> print(evaluate(sin(pi * 1/x), 0.5));
[-0.172986452514381269516508615031098129542836767991679e-12714;0.759411982011879
63145069564314525661706039084390067e-12715]
\end{Verbatim}
\end{minipage}\end{center}
\noindent Example 3: 
\begin{center}\begin{minipage}{15cm}\begin{Verbatim}[frame=single]
> isevaluable(sin(pi * 1/x), 0);
false
> print(evaluate(sin(pi * 1/x), 0));
[@NaN@;@NaN@]
\end{Verbatim}
\end{minipage}\end{center}
See also: \textbf{evaluate} (\ref{labevaluate})

\subsection{length}
\label{lablength}
\noindent Name: \textbf{length}\\
computes the length of a list or string.\\

\noindent Usage: 
\begin{center}
\textbf{length}(\emph{L}) : \textsf{list} $\rightarrow$ \textsf{integer}\\
\textbf{length}(\emph{s}) : \textsf{string} $\rightarrow$ \textsf{integer}\\
\end{center}
Parameters: 
\begin{itemize}
\item \emph{L} is a list.
\item \emph{s} is a string.
\end{itemize}
\noindent Description: \begin{itemize}

\item \textbf{length} returns the length of a list or a string, e.g. the number of elements
   or letters.

\item The empty list or string have length 0.
   If \emph{L} is an end-elliptic list, \textbf{length} returns +Inf.
\end{itemize}
\noindent Example 1: 
\begin{center}\begin{minipage}{15cm}\begin{Verbatim}[frame=single]
> length("Hello World!");
12
\end{Verbatim}
\end{minipage}\end{center}
\noindent Example 2: 
\begin{center}\begin{minipage}{15cm}\begin{Verbatim}[frame=single]
> length([|1,...,5|]);
5
\end{Verbatim}
\end{minipage}\end{center}
\noindent Example 3: 
\begin{center}\begin{minipage}{15cm}\begin{Verbatim}[frame=single]
> length([| |]);
1
\end{Verbatim}
\end{minipage}\end{center}
\noindent Example 4: 
\begin{center}\begin{minipage}{15cm}\begin{Verbatim}[frame=single]
> length([|1,2...|]);
@Inf@
\end{Verbatim}
\end{minipage}\end{center}

\subsection{$<=$}
\label{lable}
\noindent Name: \textbf{$<=$}\\
\phantom{aaa}less-than-or-equal-to operator\\[0.2cm]
\noindent Library name:\\
\verb|   sollya_obj_t sollya_lib_cmp_less_equal(sollya_obj_t, sollya_obj_t)|\\[0.2cm]
\noindent Usage: 
\begin{center}
\emph{expr1} \textbf{$<=$} \emph{expr2} : (\textsf{constant}, \textsf{constant}) $\rightarrow$ \textsf{boolean}\\
\end{center}
Parameters: 
\begin{itemize}
\item \emph{expr1} and \emph{expr2} represent constant expressions
\end{itemize}
\noindent Description: \begin{itemize}

\item The operator \textbf{$<=$} evaluates to true iff its operands \emph{expr1} and
   \emph{expr2} evaluate to two floating-point numbers $a_1$
   respectively $a_2$ with the global precision \textbf{prec} and
   $a_1$ is less than or equal to $a_2$. The user should
   be aware of the fact that because of floating-point evaluation, the
   operator \textbf{$<=$} is not exactly the same as the mathematical
   operation \emph{less-than-or-equal-to}.
\end{itemize}
\noindent Example 1: 
\begin{center}\begin{minipage}{15cm}\begin{Verbatim}[frame=single,commandchars=\\\|\~]
> 5 <= 4;
false
> 5 <= 5;
true
> 5 <= 6;
true
> exp(2) <= exp(1);
false
> log(1) <= exp(2);
true
\end{Verbatim}
\end{minipage}\end{center}
\noindent Example 2: 
\begin{center}\begin{minipage}{15cm}\begin{Verbatim}[frame=single,commandchars=\\\|\~]
> prec = 12;
The precision has been set to 12 bits.
> 16385.1 <= 16384.1;
true
\end{Verbatim}
\end{minipage}\end{center}
See also: \textbf{$==$} (\ref{labequal}), \textbf{!$=$} (\ref{labneq}), \textbf{$>=$} (\ref{labge}), \textbf{$>$} (\ref{labgt}), \textbf{$<$} (\ref{lablt}), \textbf{in} (\ref{labin}), \textbf{!} (\ref{labnot}), \textbf{$\&\&$} (\ref{laband}), \textbf{$||$} (\ref{labor}), \textbf{prec} (\ref{labprec}), \textbf{max} (\ref{labmax}), \textbf{min} (\ref{labmin})

\subsection{libraryconstant}
\label{lablibraryconstant}
\noindent Name: \textbf{libraryconstant}\\
\phantom{aaa}binds an external mathematical constant to a variable in \sollya\\[0.2cm]
\noindent Library names:\\
\verb|   sollya_obj_t sollya_lib_libraryconstant(char *, void (*)(mpfr_t, mp_prec_t))|\\
\verb|   sollya_obj_t sollya_lib_build_function_libraryconstant(char *,|\\
\verb|                                                          void (*)(mpfr_t,|\\
\verb|                                                                   mp_prec_t))|\\
\verb|   sollya_obj_t sollya_lib_libraryconstant_with_data(char *, void (*)(mpfr_t, mp_prec_t, void *), void *)|\\
\verb|   sollya_obj_t sollya_lib_build_function_libraryconstant_with_data(char *,|\\
\verb|                                                                    void (*)(mpfr_t,|\\
\verb|                                                                    mp_prec_t, void *), void *)|\\[0.2cm]
\noindent Usage: 
\begin{center}
\textbf{libraryconstant}(\emph{path}) : \textsf{string} $\rightarrow$ \textsf{function}\\
\end{center}
\noindent Description: \begin{itemize}

\item The command \textbf{libraryconstant} lets you extend the set of mathematical
   constants known to \sollya.
   By default, the only mathematical constant known by \sollya is \textbf{pi}.
   For particular applications, one may want to
   manipulate other constants, such as Euler's gamma constant, for instance.

\item \textbf{libraryconstant} makes it possible to let \sollya know about new constants.
   In order to let it know, you have to provide an implementation of the
   constant you are interested in. This implementation is a C file containing
   a function of the form:
   \begin{verbatim} void my_ident(mpfr_t result, mp_prec_t prec)\end{verbatim}
   The semantic of this function is the following: it is an implementation of
   the constant in arbitrary precision.
   \verb|my_ident(result, prec)| shall set the
   precision of the variable result to a suitable precision (the variable is
   assumed to be already initialized) and store in result an approximate value
   of the constant with a relative error not greater than $2^{1-\mathrm{prec}}$.
   More precisely, if $c$ is the exact value of the constant, the value stored
   in result should satisfy $$|\mathrm{result}-c| \le |c|\,2^{1-\mathrm{prec}}.$$

\item You can include sollya.h in your implementation and use library 
   functionnalities of \sollya for your implementation. However, this requires to
   have compiled \sollya with \texttt{-fPIC} in order to make the \sollya executable
   code position independent and to use a system on with programs, using \texttt{dlopen}
   to open dynamic routines can dynamically open themselves.

\item To bind your constant into \sollya, you must use the same identifier as the
   function name used in your implementation file (\verb|my_ident| in the previous
   example). Once the function code has been bound to an identifier, you can use
   a simple assignment to assign the bound identifier to yet another identifier.
   This way, you may use convenient names inside \sollya even if your
   implementation environment requires you to use a less convenient name.

\item Once your constant is bound, it is considered by \sollya as an infinitely
   accurate constant (i.e. a 0-ary function, exactly like \textbf{pi}).

\item The dynamic object file whose name is given to \textbf{libraryconstant} for binding of an
   external library constant may also define a destructor function \verb|int sollya_external_lib_close(void)|.
   If \sollya finds such a destructor function in the dynamic object file, it will call 
   that function when closing the dynamic object file again. This happens when \sollya
   is terminated or when the current \sollya session is restarted using \textbf{restart}.
   The purpose of the destructor function is to allow the dynamically bound code
   to free any memory that it might have allocated before \sollya is terminated 
   or restarted. 
   The dynamic object file is not necessarily needed to define a destructor
   function. This ensure backward compatibility with older \sollya external 
   library function object files.
   When defined, the destructor function is supposed to return an integer
   value indicating if an error has happened. Upon success, the destructor
   functions is to return a zero value, upon error a non-zero value.
\end{itemize}
\noindent Example 1: 
\begin{center}\begin{minipage}{15cm}\begin{Verbatim}[frame=single]
> bashexecute("gcc -fPIC -Wall -c libraryconstantexample.c -I$HOME/.local/includ
e");
> bashexecute("gcc -shared -o libraryconstantexample libraryconstantexample.o -l
gmp -lmpfr");
> euler_gamma = libraryconstant("./libraryconstantexample");
> prec = 20!;
> euler_gamma;
0.577215
> prec = 100!;
> euler_gamma;
0.577215664901532860606512090082
> midpointmode = on;
Midpoint mode has been activated.
> [euler_gamma];
0.57721566490153286060651209008~2/4~
\end{Verbatim}
\end{minipage}\end{center}
See also: \textbf{bashexecute} (\ref{labbashexecute}), \textbf{externalproc} (\ref{labexternalproc}), \textbf{externalplot} (\ref{labexternalplot}), \textbf{pi} (\ref{labpi}), \textbf{library} (\ref{lablibrary}), \textbf{evaluate} (\ref{labevaluate}), \textbf{implementconstant} (\ref{labimplementconstant})

\subsection{library}
\label{lablibrary}
\noindent Name: \textbf{library}\\
binds an external mathematical function to a variable in \sollya\\
\noindent Usage: 
\begin{center}
\textbf{library}(\emph{path}) : \textsf{string} $\rightarrow$ \textsf{function}
\end{center}
\noindent Description: \begin{itemize}

\item The command \textbf{library} lets you extends the set of mathematical
   functions known by \sollya.
   By default, \sollya knows the most common mathematical functions such
   as \textbf{exp}, \textbf{sin}, \textbf{erf}, etc. Within \sollya, these functions may be
   composed. This way, \sollya should satisfy the needs of a lot of
   users. However, for particular applications, one may want to
   manipulates other functions such as Bessel functions, or functions
   defined by an integral or even a particular solution of an ODE.

\item \textbf{library} makes it possible to let \sollya know about new functions. In
   order to let it know, you have to provide an implementation of the
   function you are interested with. This implementation is a C file containing
   a function of the form:
   \begin{verbatim} int my_ident(mpfi_t result, mpfi_t op, int n)\end{verbatim}
   The semantic of this function is the following: it is an implementation of
   the function and its derivatives in interval arithmetic.
   \verb|my_ident(result, I, n)| shall store in \verb|result| an enclosure 
   of the image set of the n-th derivative
   of the function f over \verb|I|: $f^{(n)}(I) \subseteq \mathrm{result}$.

\item The integer returned value has no meaning currently.

\item You must not provide a non trivial implementation for any \verb|n|. Most functions
   of \sollya needs a relevant implementation of $f$, $f'$ and $f''$. For higher 
   derivatives, its is not so critical and the implementation may just store 
   $[-\infty,\,+\infty]$ in result whenever $n>2$.

\item Note that you should respect somehow MPFI standards in your implementation:
   \verb|result| has its own precision and you should perform the 
   intermediate computations so that \verb|result| is as tighter as possible.

\item You can include sollya.h in your implementation and use library 
   functionnalities of \sollya for your implementation.

\item To bind your function into \sollya, you must use the same identifier as the
   function name used in your implementation file (\verb|my_ident| in the previous
   example).
\end{itemize}
\noindent Example 1: 
\begin{center}\begin{minipage}{15cm}\begin{Verbatim}[frame=single]
> bashexecute("gcc -fPIC -Wall -c libraryexample.c");
> bashexecute("gcc -shared -o libraryexample libraryexample.o -lgmp -lmpfr");
> myownlog = library("./libraryexample");
> evaluate(log(x), 2);
0.69314718055994530941723212145817656807550013436024
> evaluate(myownlog(x), 2);
0.69314718055994530941723212145817656807550013436024
\end{Verbatim}
\end{minipage}\end{center}
See also: \textbf{bashexecute} (\ref{labbashexecute}), \textbf{externalproc} (\ref{labexternalproc}), \textbf{externalplot} (\ref{labexternalplot})

\subsection{list of}
\label{lablistof}
\noindent Name: \textbf{list of}\\
keyword used in combination with a type keyword\\
\noindent Description: \begin{itemize}

\item \\textbf{list of} is used in combination with one of the following keywords for\n   indicating lists of the respective type in declarations of external\n   procedures using \\textbf{externalproc}: \\textbf{boolean}, \\textbf{constant}, \\textbf{function},\n   \\textbf{integer}, \\textbf{range} and \\textbf{string}.\n\end{itemize}
See also: \textbf{externalproc} (\ref{labexternalproc}), \textbf{boolean} (\ref{labboolean}), \textbf{constant} (\ref{labconstant}), \textbf{function} (\ref{labfunction}), \textbf{integer} (\ref{labinteger}), \textbf{range} (\ref{labrange}), \textbf{string} (\ref{labstring})

\subsection{log10}
\label{lablog10}
\noindent Name: \textbf{log10}\\
decimal logarithm.\\

\noindent Description: \begin{itemize}

\item \textbf{log10} is the decimal logarithm defined by: ${\rm log10}(x) = \log(x)/\log(10)$.

\item It is defined only for $x \in [0; +\infty]$.
\end{itemize}
See also: \textbf{log} (\ref{lablog}), \textbf{log2} (\ref{lablog2})

\subsection{log1p}
\label{lablog1p}
\noindent Name: \textbf{log1p}\\
translated logarithm.\\
\noindent Description: \begin{itemize}

\item \textbf{log1p} is the function defined by ${\rm log1p}(x) = \log(1+x)$.

\item It is defined only for $x \in [-1; +\infty]$.
\end{itemize}
See also: \textbf{log} (\ref{lablog})

\subsection{log2}
\label{lablog2}
\noindent Name: \textbf{log2}\\
binary logarithm.\\
\noindent Description: \begin{itemize}

\item \\textbf{log2} is the binary logarithm defined by: ${\\rm log2}(x) = \\log(x)/\\log(2)$.\n
\item It is defined only for $x \\in [0; +\\infty]$.\n\end{itemize}
See also: \textbf{log} (\ref{lablog}), \textbf{log10} (\ref{lablog10})

\subsection{log}
\label{lablog}
\noindent Name: \textbf{log}\\
natural logarithm.\\

\noindent Description: \begin{itemize}

\item \textbf{log} is the natural logarithm defined as the inverse of the exponential
   function: $\log(y)$ is the unique real number $x$ such that $\exp(x)=y$.

\item It is defined only for $y \in [0; +\infty]$.
\end{itemize}
See also: \textbf{exp} (\ref{labexp}), \textbf{log2} (\ref{lablog2}), \textbf{log10} (\ref{lablog10})

\subsection{lt}
\label{lablt}
\noindent Name: \textbf{$<$}\\
less-than operator\\
\noindent Usage: 
\begin{center}
\emph{expr1} \textbf{$<$} \emph{expr2} : (\textsf{constant}, \textsf{constant}) $\rightarrow$ \textsf{boolean}
\end{center}
Parameters: 
\begin{itemize}
\item \emph{expr1} and \emph{expr2} represent constant expressions
\end{itemize}
\noindent Description: \begin{itemize}

\item The operator \textbf{$<$} evaluates to true iff its operands \emph{expr1} and
   \emph{expr2} evaluate to two floating-point numbers $a_1$
   respectively $a_2$ with the global precision \textbf{prec} and
   $a_1$ is less than $a_2$. The user should
   be aware of the fact that because of floating-point evaluation, the
   operator \textbf{$<$} is not exactly the same as the mathematical
   operation \emph{less-than}.
\end{itemize}
\noindent Example 1: 
\begin{center}\begin{minipage}{15cm}\begin{Verbatim}[frame=single]
> 5 < 4;
false
> 5 < 5;
false
> 5 < 6;
true
> exp(2) < exp(1);
false
> log(1) < exp(2);
true
\end{Verbatim}
\end{minipage}\end{center}
\noindent Example 2: 
\begin{center}\begin{minipage}{15cm}\begin{Verbatim}[frame=single]
> prec = 12;
The precision has been set to 12 bits.
> 16384.1 < 16385.1;
false
\end{Verbatim}
\end{minipage}\end{center}
See also: \textbf{$==$} (\ref{labequal}), \textbf{!$=$} (\ref{labneq}), \textbf{$>=$} (\ref{labge}), \textbf{$>$} (\ref{labgt}), \textbf{$<=$} (\ref{lable}), \textbf{!} (\ref{labnot}), \textbf{$\&\&$} (\ref{laband}), \textbf{$||$} (\ref{labor}), \textbf{prec} (\ref{labprec})

\subsection{ mantissa }
\noindent Name: \textbf{mantissa}\\
returns the integer mantissa of a number.\\

\noindent Usage: 
\begin{center}
\textbf{mantissa}(\emph{x}) : \textsf{constant} $\rightarrow$ \textsf{integer}\\
\end{center}
Parameters: 
\emph{x} is a dyadic number.\\

\noindent Description: \begin{itemize}

\item \textbf{mantissa}(x) is by definition x if x equals 0, NaN, or Inf.

\item If \emph{x} is not zero, it can be uniquely written as $x = m \cdot 2^e$ where
   $m$ is an odd integer and $e$ is an integer. \textbf{mantissa}(x) returns $m$. 
\end{itemize}
\noindent Example 1: 
\begin{center}\begin{minipage}{14.8cm}\begin{Verbatim}[frame=single]
   > a=round(Pi,20,RN);
   > e=exponent(a);
   > m=mantissa(a);
   > m;
   411775
   > a-m*2^e;
   0
\end{Verbatim}
\end{minipage}\end{center}
See also: \textbf{exponent}, \textbf{precision}

\subsection{max}
\label{labmax}
\noindent Name: \textbf{max}\\
determines which of given constant expressions has maximum value\\
\noindent Usage: 
\begin{center}
\textbf{max}(\emph{expr1},\emph{expr2},...,\emph{exprn}) : (\textsf{constant}, \textsf{constant}, ..., \textsf{constant}) $\rightarrow$ \textsf{constant}\\
\textbf{max}(\emph{l}) : \textsf{list} $\rightarrow$ \textsf{constant}\\
\end{center}
Parameters: 
\begin{itemize}
\item \emph{expr} are constant expressions.
\item \emph{l} is a list of constant expressions.
\end{itemize}
\noindent Description: \begin{itemize}

\item \textbf{max} determines which of a given set of constant expressions
   \emph{expr} has maximum value. To do so, \textbf{max} tries to increase the
   precision used for evaluation until it can decide the ordering or some
   maximum precision is reached. In the latter case, a warning is printed
   indicating that there might actually be another expression that has a
   greater value.

\item Even though \textbf{max} determines the maximum expression by evaluation, it 
   returns the expression that is maximum as is, i.e. as an expression
   tree that might be evaluated to any accuracy afterwards.

\item \textbf{max} can be given either an arbitrary number of constant
   expressions in argument or a list of constant expressions. The list
   however must not be end-elliptic.

\item Users should be aware that the behavior of \textbf{max} follows the IEEE
   754-2008 standard with respect to NaNs. In particular, a NaN given as
   the first argument will not be promoted as a result unless the other
   argument is a NaN. This means that NaNs may seem to disappear during
   computations.
\end{itemize}
\noindent Example 1: 
\begin{center}\begin{minipage}{15cm}\begin{Verbatim}[frame=single]
> max(1,2,3,exp(5),log(0.25));
1.48413159102576603421115580040552279623487667593878e2
> max(17);
17
\end{Verbatim}
\end{minipage}\end{center}
\noindent Example 2: 
\begin{center}\begin{minipage}{15cm}\begin{Verbatim}[frame=single]
> l = [|1,2,3,exp(5),log(0.25)|];
> max(l);
1.48413159102576603421115580040552279623487667593878e2
\end{Verbatim}
\end{minipage}\end{center}
\noindent Example 3: 
\begin{center}\begin{minipage}{15cm}\begin{Verbatim}[frame=single]
> print(max(exp(17),sin(62)));
exp(17)
\end{Verbatim}
\end{minipage}\end{center}
\noindent Example 4: 
\begin{center}\begin{minipage}{15cm}\begin{Verbatim}[frame=single]
> verbosity = 1!;
> print(max(17 + log2(13)/log2(9),17 + log(13)/log(9)));
Warning: maximum computation relies on floating-point result that is faithfully 
evaluated and different faithful roundings toggle the result.
17 + log2(13) / log2(9)
\end{Verbatim}
\end{minipage}\end{center}
See also: \textbf{min} (\ref{labmin}), \textbf{$==$} (\ref{labequal}), \textbf{!$=$} (\ref{labneq}), \textbf{$>=$} (\ref{labge}), \textbf{$>$} (\ref{labgt}), \textbf{$<$} (\ref{lablt}), \textbf{$<=$} (\ref{lable}), \textbf{in} (\ref{labin}), \textbf{inf} (\ref{labinf}), \textbf{sup} (\ref{labsup})

\subsection{ midpointmode }
\noindent Name: \textbf{midpointmode}\\
global variable controlling the way intervals are displayed.\\

\noindent Description: \begin{itemize}

\item \textbf{midpointmode} is a global variable. When its value is \textbf{off}, intervals are displayed
   as usual (with the form [a;b]).
   When its value is \textbf{on}, and if a and b have the same first significant digits,
   the interval in displayed in a way that lets one immediately see the common
   digits of the two bounds.

\item This mode is supported only with \textbf{display} set to \textbf{decimal}. In other modes of 
   display, \textbf{midpointmode} value is simply ignored.
\end{itemize}
\noindent Example 1: 
\begin{center}\begin{minipage}{14.8cm}\begin{Verbatim}[frame=single]
   > a = round(Pi,30,RD);
   > b = round(Pi,30,RU);
   > d = [a,b];
   > d;
   [0.31415926516056060791015625e1;0.31415926553308963775634765625e1]
   > midpointmode=on!;
   > d;
   0.314159265~1/5~e1
\end{Verbatim}
\end{minipage}\end{center}
See also: \textbf{on}, \textbf{off}

\subsection{mid}
\label{labmid}
\noindent Name: \textbf{mid}\\
gives the middle of an interval.\\

\noindent Usage: 
\begin{center}
\textbf{mid}(\emph{I}) : \textsf{range} $\rightarrow$ \textsf{constant}\\
\textbf{mid}(\emph{x}) : \textsf{constant} $\rightarrow$ \textsf{constant}\\
\end{center}
Parameters: 
\begin{itemize}
\item \emph{I} is an interval.
\item \emph{x} is a real number.
\end{itemize}
\noindent Description: \begin{itemize}

\item Returns the middle of the interval \emph{I}. If the middle is not exactly
   representable at the current precision, the value is returned as an
   unevaluated expression.

\item When called on a real number \emph{x}, \textbf{mid} considers it as an interval formed
   of a single point: [x, x]. In other words, \textbf{mid} behaves like the identity.
\end{itemize}
\noindent Example 1: 
\begin{center}\begin{minipage}{15cm}\begin{Verbatim}[frame=single]
> mid([1;3]);
2
> mid(17);
17
\end{Verbatim}
\end{minipage}\end{center}
See also: \textbf{inf} (\ref{labinf}), \textbf{sup} (\ref{labsup})

\subsection{minus}
\label{labminus}
\noindent Name: \textbf{$-$}\\
substraction function\\

\noindent Usage: 
\begin{center}
\emph{function1} \textbf{$-$} \emph{function2} : (\textsf{function}, \textsf{function}) $\rightarrow$ \textsf{function}\\
\end{center}
Parameters: 
\begin{itemize}
\item \emph{function1} and \emph{function2} represent functions
\end{itemize}
\noindent Description: \begin{itemize}

\item \textbf{$-$} represents the substraction (function) on reals. 
   The expression \emph{function1} \textbf{$-$} \emph{function2} stands for
   the function composed of the substraction function and the two
   functions \emph{function1} and \emph{function2}, where \emph{function1} is 
   the subtrahent and \emph{function2} the substractor.
\end{itemize}
\noindent Example 1: 
\begin{center}\begin{minipage}{15cm}\begin{Verbatim}[frame=single]
> 5 - 2;
3
\end{Verbatim}
\end{minipage}\end{center}
\noindent Example 2: 
\begin{center}\begin{minipage}{15cm}\begin{Verbatim}[frame=single]
> x - 2;
-2 + x
\end{Verbatim}
\end{minipage}\end{center}
\noindent Example 3: 
\begin{center}\begin{minipage}{15cm}\begin{Verbatim}[frame=single]
> x - x;
0
\end{Verbatim}
\end{minipage}\end{center}
\noindent Example 4: 
\begin{center}\begin{minipage}{15cm}\begin{Verbatim}[frame=single]
> diff(sin(x) - exp(x));
cos(x) - exp(x)
\end{Verbatim}
\end{minipage}\end{center}
See also: \textbf{$+$} (\ref{labplus}), \textbf{$*$} (\ref{labmult}), \textbf{/} (\ref{labdivide}), \textbf{\^} (\ref{labpower})

\subsection{min}
\label{labmin}
\noindent Name: \textbf{min}\\
\phantom{aaa}determines which of given constant expressions has minimum value\\[0.2cm]
\noindent Library names:\\
\verb|   sollya_obj_t sollya_lib_min(sollya_obj_t, ...)|\\
\verb|   sollya_obj_t sollya_lib_v_min(sollya_obj_t, va_list)|\\[0.2cm]
\noindent Usage: 
\begin{center}
\textbf{min}(\emph{expr1},\emph{expr2},...,\emph{exprn}) : (\textsf{constant}, \textsf{constant}, ..., \textsf{constant}) $\rightarrow$ \textsf{constant}\\
\textbf{min}(\emph{l}) : \textsf{list} $\rightarrow$ \textsf{constant}\\
\end{center}
Parameters: 
\begin{itemize}
\item \emph{expr} are constant expressions.
\item \emph{l} is a list of constant expressions.
\end{itemize}
\noindent Description: \begin{itemize}

\item \textbf{min} determines which of a given set of constant expressions
   \emph{expr} has minimum value. To do so, \textbf{min} tries to increase the
   precision used for evaluation until it can decide the ordering or some
   maximum precision is reached. In the latter case, a warning is printed
   indicating that there might actually be another expression that has a
   lesser value.

\item Even though \textbf{min} determines the minimum expression by evaluation, it 
   returns the expression that is minimum as is, i.e. as an expression
   tree that might be evaluated to any accuracy afterwards.

\item \textbf{min} can be given either an arbitrary number of constant
   expressions in argument or a list of constant expressions. The list
   however must not be end-elliptic.

\item Users should be aware that the behavior of \textbf{min} follows the IEEE
   754-2008 standard with respect to NaNs. In particular, \textbf{min}
   evaluates to NaN if and only if all arguments of \textbf{min} are
   NaNs. This means that NaNs may disappear during computations.
\end{itemize}
\noindent Example 1: 
\begin{center}\begin{minipage}{15cm}\begin{Verbatim}[frame=single,commandchars=\\\|\~]
> min(1,2,3,exp(5),log(0.25));
-1.3862943611198906188344642429163531361510002687205
> min(17);
17
\end{Verbatim}
\end{minipage}\end{center}
\noindent Example 2: 
\begin{center}\begin{minipage}{15cm}\begin{Verbatim}[frame=single,commandchars=\\\|\~]
> l = [|1,2,3,exp(5),log(0.25)|];
> min(l);
-1.3862943611198906188344642429163531361510002687205
\end{Verbatim}
\end{minipage}\end{center}
\noindent Example 3: 
\begin{center}\begin{minipage}{15cm}\begin{Verbatim}[frame=single,commandchars=\\\|\~]
> print(min(exp(17),sin(62)));
sin(62)
\end{Verbatim}
\end{minipage}\end{center}
\noindent Example 4: 
\begin{center}\begin{minipage}{15cm}\begin{Verbatim}[frame=single,commandchars=\\\|\~]
> verbosity = 1!;
> print(min(17 + log2(13)/log2(9),17 + log(13)/log(9)));
Warning: the tool is unable to decide a minimum computation by evaluation even t
hough faithful evaluation of the terms has been possible. The terms will be cons
idered to be equal.
17 + log(13) / log(9)
\end{Verbatim}
\end{minipage}\end{center}
See also: \textbf{max} (\ref{labmax}), \textbf{$==$} (\ref{labequal}), \textbf{!$=$} (\ref{labneq}), \textbf{$>=$} (\ref{labge}), \textbf{$>$} (\ref{labgt}), \textbf{$<$} (\ref{lablt}), \textbf{$<=$} (\ref{lable}), \textbf{in} (\ref{labin}), \textbf{inf} (\ref{labinf}), \textbf{sup} (\ref{labsup})

\subsection{$*$}
\label{labmult}
\noindent Name: \textbf{$*$}\\
multiplication function\\
\noindent Usage: 
\begin{center}
\emph{function1} \textbf{$*$} \emph{function2} : (\textsf{function}, \textsf{function}) $\rightarrow$ \textsf{function}\\
\emph{interval1} \textbf{$*$} \emph{interval2} : (\textsf{range}, \textsf{range}) $\rightarrow$ \textsf{range}\\
\emph{interval1} \textbf{$*$} \emph{constant} : (\textsf{range}, \textsf{constant}) $\rightarrow$ \textsf{range}\\
\emph{interval1} \textbf{$*$} \emph{constant} : (\textsf{constant}, \textsf{range}) $\rightarrow$ \textsf{range}\\
\end{center}
Parameters: 
\begin{itemize}
\item \emph{function1} and \emph{function2} represent functions
\item \emph{interval1} and \emph{interval2} represent intervals (ranges)
\item \emph{constant} represents a constant or constant expression
\end{itemize}
\noindent Description: \begin{itemize}

\item \textbf{$*$} represents the multiplication (function) on reals. 
   The expression \emph{function1} \textbf{$*$} \emph{function2} stands for
   the function composed of the multiplication function and the two
   functions \emph{function1} and \emph{function2}.

\item \textbf{$*$} can be used for interval arithmetic on intervals
   (ranges). \textbf{$*$} will evaluate to an interval that safely
   encompasses all images of the multiplication function with arguments varying
   in the given intervals.  Any combination of intervals with intervals
   or constants (resp. constant expressions) is supported. However, it is
   not possible to represent families of functions using an interval as
   one argument and a function (varying in the free variable) as the
   other one.
\end{itemize}
\noindent Example 1: 
\begin{center}\begin{minipage}{15cm}\begin{Verbatim}[frame=single]
> 5 * 2;
10
\end{Verbatim}
\end{minipage}\end{center}
\noindent Example 2: 
\begin{center}\begin{minipage}{15cm}\begin{Verbatim}[frame=single]
> x * 2;
x * 2
\end{Verbatim}
\end{minipage}\end{center}
\noindent Example 3: 
\begin{center}\begin{minipage}{15cm}\begin{Verbatim}[frame=single]
> x * x;
x^2
\end{Verbatim}
\end{minipage}\end{center}
\noindent Example 4: 
\begin{center}\begin{minipage}{15cm}\begin{Verbatim}[frame=single]
> diff(sin(x) * exp(x));
sin(x) * exp(x) + exp(x) * cos(x)
\end{Verbatim}
\end{minipage}\end{center}
\noindent Example 5: 
\begin{center}\begin{minipage}{15cm}\begin{Verbatim}[frame=single]
> [1;2] * [3;4];
[3;8]
> [1;2] * 17;
[17;34]
> 13 * [-4;17];
[-52;221]
\end{Verbatim}
\end{minipage}\end{center}
See also: \textbf{$+$} (\ref{labplus}), \textbf{$-$} (\ref{labminus}), \textbf{/} (\ref{labdivide}), \textbf{$\mathbf{\hat{~}}$} (\ref{labpower})

\subsection{nearestint}
\label{labnearestint}
\noindent Name: \textbf{nearestint}\\
\phantom{aaa}the function mapping the reals to the integers nearest to them.\\[0.2cm]
\noindent Library names:\\
\verb|   sollya_obj_t sollya_lib_nearestint(sollya_obj_t)|\\
\verb|   sollya_obj_t sollya_lib_build_function_nearestint(sollya_obj_t)|\\
\verb|   #define SOLLYA_NEARESTINT(x) sollya_lib_build_function_nearestint(x)|\\[0.2cm]
\noindent Description: \begin{itemize}

\item \textbf{nearestint} is defined as usual: \textbf{nearestint}($x$) is the integer nearest to $x$, with the
   special rule that the even integer is chosen if there exist two integers equally near to $x$.

\item It is defined for every real number $x$.
\end{itemize}
See also: \textbf{ceil} (\ref{labceil}), \textbf{floor} (\ref{labfloor}), \textbf{round} (\ref{labround}), \textbf{RN} (\ref{labrn})

\subsection{neq}
\label{labneq}
\noindent Name: \textbf{!$=$}\\
negated equality test operator\\

\noindent Usage: 
\begin{center}
\emph{expr1} \textbf{!$=$} \emph{expr2} : (\textsf{any type}, \textsf{any type}) $\rightarrow$ \textsf{boolean}\\
\end{center}
Parameters: 
\begin{itemize}
\item \emph{expr1} and \emph{expr2} represent expressions
\end{itemize}
\noindent Description: \begin{itemize}

\item The operator \textbf{!$=$} evaluates to true iff its operands \emph{expr1} and
   \emph{expr2} are syntactically unequal and both different from \textbf{error} or
   constant expressions that evaluate to two different floating-point
   number with the global precision \textbf{prec}. The user should be aware of
   the fact that because of floating-point evaluation, the operator
   \textbf{!$=$} is not exactly the same as the negation of the mathematical
   equality. 
   Note that the expressions \textbf{!}(\emph{expr1} \textbf{!$=$} \emph{expr2}) and \emph{expr1} \textbf{$==$}
   \emph{expr2} do not evaluate to the same boolean value. See \textbf{error} for
   details.
\end{itemize}
\noindent Example 1: 
\begin{center}\begin{minipage}{15cm}\begin{Verbatim}[frame=single]
> "Hello" != "Hello";
false
> "Hello" != "Salut";
true
> "Hello" != 5;
true
> 5 + x != 5 + x;
false
\end{Verbatim}
\end{minipage}\end{center}
\noindent Example 2: 
\begin{center}\begin{minipage}{15cm}\begin{Verbatim}[frame=single]
> 1 != exp(0);
false
> asin(1) * 2 != pi;
false
> exp(5) != log(4);
true
\end{Verbatim}
\end{minipage}\end{center}
\noindent Example 3: 
\begin{center}\begin{minipage}{15cm}\begin{Verbatim}[frame=single]
> prec = 12;
The precision has been set to 12 bits.
> 16384 != 16385;
false
\end{Verbatim}
\end{minipage}\end{center}
\noindent Example 4: 
\begin{center}\begin{minipage}{15cm}\begin{Verbatim}[frame=single]
> error != error;
false
\end{Verbatim}
\end{minipage}\end{center}
See also: \textbf{$==$} (\ref{labequal}), \textbf{$>$} (\ref{labgt}), \textbf{$>=$} (\ref{labge}), \textbf{$<=$} (\ref{lable}), \textbf{$<$} (\ref{lablt}), \textbf{!} (\ref{labnot}), \textbf{$\&\&$} (\ref{laband}), \textbf{$||$} (\ref{labor}), \textbf{error} (\ref{laberror}), \textbf{prec} (\ref{labprec})

\subsection{nop}
\label{labnop}
\noindent Name: \textbf{nop}\\
no operation\\
\noindent Usage: 
\begin{center}
\textbf{nop} : \textsf{void} $\rightarrow$ \textsf{void}\\
\textbf{nop}() : \textsf{void} $\rightarrow$ \textsf{void}\\
\textbf{nop}(\emph{n}) : \textsf{integer} $\rightarrow$ \textsf{void}\\
\end{center}
\noindent Description: \begin{itemize}

\item The command \\textbf{nop} does nothing. This means it is an explicit parse\n   element in the \\sollya language that finally does not produce any\n   result or side-effect.\n
\item The command \\textbf{nop} may take an optional positive integer argument \\emph{n}. The argument controls how much (useless) integer additions \\sollya performs while doing nothing. \n   With this behaviour, \\textbf{nop} can be used for calibration of timing tests.\n
\item The keyword \\textbf{nop} is implicit in some procedure\n   definitions. Procedures without imperative body get parsed as if they\n   had an imperative body containing one \\textbf{nop} statement.\n\end{itemize}
\noindent Example 1: 
\begin{center}\begin{minipage}{15cm}\begin{Verbatim}[frame=single]
\end{Verbatim}
\end{minipage}\end{center}
\noindent Example 2: 
\begin{center}\begin{minipage}{15cm}\begin{Verbatim}[frame=single]
\end{Verbatim}
\end{minipage}\end{center}
\noindent Example 3: 
\begin{center}\begin{minipage}{15cm}\begin{Verbatim}[frame=single]
\end{Verbatim}
\end{minipage}\end{center}
See also: \textbf{proc} (\ref{labproc})

\subsection{!}
\label{labnot}
\noindent Name: \textbf{!}\\
\phantom{aaa}boolean NOT operator\\[0.2cm]
\noindent Library name:\\
\verb|   sollya_obj_t sollya_lib_negate(sollya_obj_t)|\\[0.2cm]
\noindent Usage: 
\begin{center}
\textbf{!} \emph{expr} : \textsf{boolean} $\rightarrow$ \textsf{boolean}\\
\end{center}
Parameters: 
\begin{itemize}
\item \emph{expr} represents a boolean expression
\end{itemize}
\noindent Description: \begin{itemize}

\item \textbf{!} evaluates to the boolean NOT of the boolean expression
   \emph{expr}. \textbf{!} \emph{expr} evaluates to true iff \emph{expr} does not evaluate
   to true.
\end{itemize}
\noindent Example 1: 
\begin{center}\begin{minipage}{15cm}\begin{Verbatim}[frame=single,commandchars=\\\|\~]
> ! false;
true
\end{Verbatim}
\end{minipage}\end{center}
\noindent Example 2: 
\begin{center}\begin{minipage}{15cm}\begin{Verbatim}[frame=single,commandchars=\\\|\~]
> ! (1 == exp(0));
false
\end{Verbatim}
\end{minipage}\end{center}
See also: \textbf{$\&\&$} (\ref{laband}), \textbf{$||$} (\ref{labor})

\subsection{numberroots}
\label{labnumberroots}
\noindent Name: \textbf{numberroots}\\
Computes the number of roots of a polynomial in a given range.\\
\noindent Usage: 
\begin{center}
\textbf{numberroots}(\emph{p}, \emph{I}) : (\textsf{function}, \textsf{range}) $\rightarrow$ \textsf{integer}\\
\end{center}
Parameters: 
\begin{itemize}
\item \emph{p} is a polynomial.
\item \emph{I} is an interval.
\end{itemize}
\noindent Description: \begin{itemize}

\item \textbf{numberroots} rigorously computes the number of roots of polynomial the $p$ in
   the interval $I$. The technique used is Sturm's algorithm. The value returned
   is not just a numerical estimation of the number of roots of $p$ in $I$: it is
   the exact number of roots.

\item The command \textbf{findzeros} computes safe enclosures of all the zeros of a
   function, without forgetting any, but it is not guaranteed to separate them
   all in distinct intervals. \textbf{numberroots} is more accurate since it guarantees 
   the exact number of roots. However, it does not compute them. It may be used,
   for instance, to certify that \textbf{findzeros} did not put two distinct roots in 
   the same interval.

\item Multiple roots are counted only once.

\item The interval $I$ must be bounded. The algorithm cannot handle unbounded
   intervals. Moreover, the interval is considered as a closed interval: if one
   (or both) of the endpoints of $I$ are roots of $p$, they are counted.

\item The argument $p$ can be any expression, but if \sollya fails to prove that
   it is a polynomial an error is produced. Also, please note that if the
   coefficients of $p$ or the endpoints of $I$ are not exactly representable,
   they are first numerically evaluated, before the algorithm is used. In that
   case, the counted number of roots corresponds to the rounded polynomial on
   the rounded interval \textbf{and not} to the exact parameters given by the user.
   A warning is displayed to inform the user.
\end{itemize}
\noindent Example 1: 
\begin{center}\begin{minipage}{15cm}\begin{Verbatim}[frame=single]
> numberroots(1+x-x^2, [1,2]);
1
> findzeros(1+x-x^2, [1,2]);
[|[1.617919921875;1.6180419921875]|]
\end{Verbatim}
\end{minipage}\end{center}
\noindent Example 2: 
\begin{center}\begin{minipage}{15cm}\begin{Verbatim}[frame=single]
> numberroots((1+x)*(1-x), [-1,1]);
2
> numberroots(x^2, [-1,1]);
1
\end{Verbatim}
\end{minipage}\end{center}
\noindent Example 3: 
\begin{center}\begin{minipage}{15cm}\begin{Verbatim}[frame=single]
> verbosity = 1!;
> numberroots(x-pi, [0,4]);
Warning: the 0th coefficient of the polynomial is neither a floating point
constant nor can be evaluated without rounding to a floating point constant.
Will faithfully evaluate it with the current precision (165 bits) 
1
\end{Verbatim}
\end{minipage}\end{center}
\noindent Example 4: 
\begin{center}\begin{minipage}{15cm}\begin{Verbatim}[frame=single]
> verbosity = 1!;
> numberroots(1+x-x^2, [0, @Inf@]);
1
> numberroots(exp(x), [0, 1]);
Warning: the given function must be a polynomial in this context.
Warning: at least one of the given expressions or a subexpression is not correct
ly typed
or its evaluation has failed because of some error on a side-effect.
error
\end{Verbatim}
\end{minipage}\end{center}
See also: \textbf{dirtyfindzeros} (\ref{labdirtyfindzeros}), \textbf{findzeros} (\ref{labfindzeros})

\subsection{numerator}
\label{labnumerator}
\noindent Name: \textbf{numerator}\\
gives the numerator of an expression\\
\noindent Usage: 
\begin{center}
\textbf{numerator}(\emph{expr}) : \textsf{function} $\rightarrow$ \textsf{function}\\
\end{center}
Parameters: 
\begin{itemize}
\item \emph{expr} represents an expression
\end{itemize}
\noindent Description: \begin{itemize}

\item If \emph{expr} represents a fraction \emph{expr1}/\emph{expr2}, \textbf{numerator}(\emph{expr})
   returns the numerator of this fraction, i.e. \emph{expr1}.
    
   If \emph{expr} represents something else, \textbf{numerator}(\emph{expr}) 
   returns the expression itself, i.e. \emph{expr}.
    
   Note that for all expressions \emph{expr}, \textbf{numerator}(\emph{expr}) \textbf{/} \textbf{denominator}(\emph{expr})
   is equal to \emph{expr}.
\end{itemize}
\noindent Example 1: 
\begin{center}\begin{minipage}{15cm}\begin{Verbatim}[frame=single]
> numerator(5/3);
5
\end{Verbatim}
\end{minipage}\end{center}
\noindent Example 2: 
\begin{center}\begin{minipage}{15cm}\begin{Verbatim}[frame=single]
> numerator(exp(x));
exp(x)
\end{Verbatim}
\end{minipage}\end{center}
\noindent Example 3: 
\begin{center}\begin{minipage}{15cm}\begin{Verbatim}[frame=single]
> a = 5/3;
> b = numerator(a)/denominator(a);
> print(a);
5 / 3
> print(b);
5 / 3
\end{Verbatim}
\end{minipage}\end{center}
\noindent Example 4: 
\begin{center}\begin{minipage}{15cm}\begin{Verbatim}[frame=single]
> a = exp(x/3);
> b = numerator(a)/denominator(a);
> print(a);
exp(x / 3)
> print(b);
exp(x / 3)
\end{Verbatim}
\end{minipage}\end{center}
See also: \textbf{denominator} (\ref{labdenominator})

\subsection{off}
\label{laboff}
\noindent Name: \textbf{off}\\
\phantom{aaa}special value for certain global variables.\\[0.2cm]
\noindent Library names:\\
\verb|   sollya_obj_t sollya_lib_off()|\\
\verb|   int sollya_lib_is_off(sollya_obj_t)|\\[0.2cm]
\noindent Description: \begin{itemize}

\item \textbf{off} is a special value used to deactivate certain functionnalities
   of \sollya.

\item As any value it can be affected to a variable and stored in lists.
\end{itemize}
\noindent Example 1: 
\begin{center}\begin{minipage}{15cm}\begin{Verbatim}[frame=single,commandchars=\\\|\~]
> canonical=on;
Canonical automatic printing output has been activated.
> p=1+x+x^2;
> mode=off;
> p;
1 + x + x^2
> canonical=mode;
Canonical automatic printing output has been deactivated.
> p;
1 + x * (1 + x)
\end{Verbatim}
\end{minipage}\end{center}
See also: \textbf{on} (\ref{labon}), \textbf{autosimplify} (\ref{labautosimplify}), \textbf{canonical} (\ref{labcanonical}), \textbf{timing} (\ref{labtiming}), \textbf{fullparentheses} (\ref{labfullparentheses}), \textbf{midpointmode} (\ref{labmidpointmode}), \textbf{rationalmode} (\ref{labrationalmode}), \textbf{roundingwarnings} (\ref{labroundingwarnings}), \textbf{timing} (\ref{labtiming}), \textbf{dieonerrormode} (\ref{labdieonerrormode})

\subsection{on}
\label{labon}
\noindent Name: \textbf{on}\\
special value for certain global variables.\\
\noindent Description: \begin{itemize}

\item \textbf{on} is a special value used to activate certain functionnalities 
   of \sollya.

\item As any value it can be affected to a variable and stored in lists.
\end{itemize}
\noindent Example 1: 
\begin{center}\begin{minipage}{15cm}\begin{Verbatim}[frame=single]
> p=1+x+x^2;
> mode=on;
> p;
1 + x * (1 + x)
> canonical=mode;
Canonical automatic printing output has been activated.
> p;
1 + x + x^2
\end{Verbatim}
\end{minipage}\end{center}
See also: \textbf{off} (\ref{laboff}), \textbf{autosimplify} (\ref{labautosimplify}), \textbf{canonical} (\ref{labcanonical}), \textbf{timing} (\ref{labtiming}), \textbf{fullparentheses} (\ref{labfullparentheses}), \textbf{midpointmode} (\ref{labmidpointmode}), \textbf{rationalmode} (\ref{labrationalmode}), \textbf{roundingwarnings} (\ref{labroundingwarnings}), \textbf{timing} (\ref{labtiming}), \textbf{dieonerrormode} (\ref{labdieonerrormode})

\subsection{ or }
\noindent Name: \textbf{$||$}\\
boolean OR operator\\

\noindent Usage: 
\begin{center}
\emph{expr1} \textbf{$||$} \emph{expr2} : (\textsf{boolean}, \textsf{boolean}) $\rightarrow$ \textsf{boolean}\\
\end{center}
Parameters: 
\emph{expr1} and \emph{expr2} represent boolean expressions\\

\noindent Description: \begin{itemize}

\item \textbf{$||$} evaluates to the boolean OR of the two
   boolean expressions \emph{expr1} and \emph{expr2}. \textbf{$||$} evaluates to 
   true iff at least one of \emph{expr1} or \emph{expr2} evaluate to true.
\end{itemize}
\noindent Example 1: 
\begin{center}\begin{minipage}{14.8cm}\begin{Verbatim}[frame=single]
   > false || false;
   false
\end{Verbatim}
\end{minipage}\end{center}
\noindent Example 2: 
\begin{center}\begin{minipage}{14.8cm}\begin{Verbatim}[frame=single]
   > (1 == exp(0)) || (0 == log(1));
   true
\end{Verbatim}
\end{minipage}\end{center}
See also: \textbf{$\&\&$}, \textbf{!}

\subsection{parse}
\label{labparse}
\noindent Name: \textbf{parse}\\
\phantom{aaa}parses an expression contained in a string\\[0.2cm]
\noindent Library name:\\
\verb|   sollya_obj_t sollya_lib_parse(sollya_obj_t)|\\[0.2cm]
\noindent Usage: 
\begin{center}
\textbf{parse}(\emph{string}) : \textsf{string} $\rightarrow$ \textsf{function} $|$ \textsf{error}\\
\end{center}
Parameters: 
\begin{itemize}
\item \emph{string} represents a character sequence
\end{itemize}
\noindent Description: \begin{itemize}

\item \textbf{parse}(\emph{string}) parses the character sequence \emph{string} containing
   an expression built on constants and base functions.
    
   If the character sequence does not contain a well-defined expression,
   a warning is displayed indicating a syntax error and \textbf{parse} returns
   a \textbf{error} of type \textsf{error}.

\item The character sequence to be parsed by \textbf{parse} may contain commands that 
   return expressions, including \textbf{parse} itself. Those commands get executed after the string has been parsed.
   \textbf{parse}(\emph{string}) will return the expression computed by the commands contained in the character 
   sequence \emph{string}.
\end{itemize}
\noindent Example 1: 
\begin{center}\begin{minipage}{15cm}\begin{Verbatim}[frame=single]
> parse("exp(x)");
exp(x)
\end{Verbatim}
\end{minipage}\end{center}
\noindent Example 2: 
\begin{center}\begin{minipage}{15cm}\begin{Verbatim}[frame=single]
> text = "remez(exp(x),5,[-1;1])";
> print("The string", text, "gives", parse(text));
The string remez(exp(x),5,[-1;1]) gives 8.73819103880655511401584202783309604799
60476712009e-3 * x^5 + 4.3793696387328047027125756620718349665957546423673e-2 * 
x^4 + 0.16642465607515519441592059732272738093227960290946 * x^3 + 0.49919698262
8829864921688244942403747719690128612986 * x^2 + 1.00003834652983970735244541124
504033817544233075343 * x + 1.00004475029055070643077052482053398765426158966754

\end{Verbatim}
\end{minipage}\end{center}
\noindent Example 3: 
\begin{center}\begin{minipage}{15cm}\begin{Verbatim}[frame=single]
> verbosity = 1!;
> parse("5 + * 3");
Warning: syntax error, unexpected MULTOKEN. Will try to continue parsing (expect
ing ";"). May leak memory.
Warning: the string "5 + * 3" could not be parsed by the miniparser.
Warning: at least one of the given expressions or a subexpression is not correct
ly typed
or its evaluation has failed because of some error on a side-effect.
error
\end{Verbatim}
\end{minipage}\end{center}
See also: \textbf{execute} (\ref{labexecute}), \textbf{readfile} (\ref{labreadfile}), \textbf{print} (\ref{labprint}), \textbf{error} (\ref{laberror}), \textbf{dieonerrormode} (\ref{labdieonerrormode})

\subsection{perturb}
\label{labperturb}
\noindent Name: \textbf{perturb}\\
indicates random perturbation of sampling points for \textbf{externalplot}\\

\noindent Usage: 
\begin{center}
\textbf{perturb} : \textsf{perturb}\\
\end{center}
\noindent Description: \begin{itemize}

\item The use of \textbf{perturb} in the command \textbf{externalplot} enables the addition
   of some random noise around each sampling point in \textbf{externalplot}.
    
   See \textbf{externalplot} for details.
\end{itemize}
\noindent Example 1: 
\begin{center}\begin{minipage}{15cm}\begin{Verbatim}[frame=single]
> bashexecute("gcc -fPIC -c externalplotexample.c");
> bashexecute("gcc -shared -o externalplotexample externalplotexample.o -lgmp -l
mpfr");
> externalplot("./externalplotexample",relative,exp(x),[-1/2;1/2],12,perturb);
\end{Verbatim}
\end{minipage}\end{center}
See also: \textbf{externalplot} (\ref{labexternalplot}), \textbf{absolute} (\ref{lababsolute}), \textbf{relative} (\ref{labrelative}), \textbf{bashexecute} (\ref{labbashexecute})

\subsection{pi}
\label{labpi}
\noindent Name: \textbf{pi}\\
\phantom{aaa}the constant $\pi$.\\[0.2cm]
\noindent Library names:\\
\verb|   sollya_obj_t sollya_lib_pi()|\\
\verb|   int sollya_lib_is_pi(sollya_obj_t)|\\
\verb|   sollya_obj_t sollya_lib_build_function_pi()|\\
\verb|   #define SOLLYA_PI (sollya_lib_build_function_pi())|\\[0.2cm]
\noindent Description: \begin{itemize}

\item \textbf{pi} is the constant $\pi$, defined as half the period of sine and cosine.

\item In \sollya, \textbf{pi} is considered a 0-ary function. This way, the constant 
   is not evaluated at the time of its definition but at the time of its use. For 
   instance, when you define a constant or a function relating to $\pi$, the current
   precision at the time of the definition does not matter. What is important is 
   the current precision when you evaluate the function or the constant value.

\item Remark that when you define an interval, the bounds are first evaluated and 
   then the interval is defined. In this case, \textbf{pi} will be evaluated as any 
   other constant value at the definition time of the interval, thus using the 
   current precision at this time.
\end{itemize}
\noindent Example 1: 
\begin{center}\begin{minipage}{15cm}\begin{Verbatim}[frame=single]
> verbosity=1!; prec=12!;
> a = 2*pi;
> a;
Warning: rounding has happened. The value displayed is a faithful rounding to 12
 bits of the true result.
6.283
> prec=20!;
> a;
Warning: rounding has happened. The value displayed is a faithful rounding to 20
 bits of the true result.
6.283187
\end{Verbatim}
\end{minipage}\end{center}
\noindent Example 2: 
\begin{center}\begin{minipage}{15cm}\begin{Verbatim}[frame=single]
> display=binary;
Display mode is binary numbers.
> prec=12!;
> d = [pi; 5];
> d;
[1.1001001_2 * 2^(1);1.01_2 * 2^(2)]
> prec=20!;
> d;
[1.1001001_2 * 2^(1);1.01_2 * 2^(2)]
\end{Verbatim}
\end{minipage}\end{center}
See also: \textbf{cos} (\ref{labcos}), \textbf{sin} (\ref{labsin}), \textbf{tan} (\ref{labtan}), \textbf{asin} (\ref{labasin}), \textbf{acos} (\ref{labacos}), \textbf{atan} (\ref{labatan}), \textbf{evaluate} (\ref{labevaluate}), \textbf{prec} (\ref{labprec}), \textbf{libraryconstant} (\ref{lablibraryconstant})

\subsection{plot}
\label{labplot}
\noindent Name: \textbf{plot}\\
\phantom{aaa}plots one or several functions\\[0.2cm]
\noindent Library names:\\
\verb|   void sollya_lib_plot(sollya_obj_t, sollya_obj_t, ...)|\\
\verb|   void sollya_lib_v_plot(sollya_obj_t, sollya_obj_t, va_list)|\\[0.2cm]
\noindent Usage: 
\begin{center}
\textbf{plot}(\emph{f1}, ... ,\emph{fn}, \emph{I}) : (\textsf{function}, ... ,\textsf{function}, \textsf{range}) $\rightarrow$ \textsf{void}\\
\textbf{plot}(\emph{f1}, ... ,\emph{fn}, \emph{I}, \textbf{file}, \emph{name}) : (\textsf{function}, ... ,\textsf{function}, \textsf{range}, \textbf{file}, \textsf{string}) $\rightarrow$ \textsf{void}\\
\textbf{plot}(\emph{f1}, ... ,\emph{fn}, \emph{I}, \textbf{postscript}, \emph{name}) : (\textsf{function}, ... ,\textsf{function}, \textsf{range}, \textbf{postscript}, \textsf{string}) $\rightarrow$ \textsf{void}\\
\textbf{plot}(\emph{f1}, ... ,\emph{fn}, \emph{I}, \textbf{postscriptfile}, \emph{name}) : (\textsf{function}, ... ,\textsf{function}, \textsf{range}, \textbf{postscriptfile}, \textsf{string}) $\rightarrow$ \textsf{void}\\
\textbf{plot}(\emph{L}, \emph{I}) : (\textsf{list}, \textsf{range}) $\rightarrow$ \textsf{void}\\
\textbf{plot}(\emph{L}, \emph{I}, \textbf{file}, \emph{name}) : (\textsf{list}, \textsf{range}, \textbf{file}, \textsf{string}) $\rightarrow$ \textsf{void}\\
\textbf{plot}(\emph{L}, \emph{I}, \textbf{postscript}, \emph{name}) : (\textsf{list}, \textsf{range}, \textbf{postscript}, \textsf{string}) $\rightarrow$ \textsf{void}\\
\textbf{plot}(\emph{L}, \emph{I}, \textbf{postscriptfile}, \emph{name}) : (\textsf{list}, \textsf{range}, \textbf{postscriptfile}, \textsf{string}) $\rightarrow$ \textsf{void}\\
\end{center}
Parameters: 
\begin{itemize}
\item \emph{f1}, ..., \emph{fn} are functions to be plotted.
\item \emph{L} is a list of functions to be plotted.
\item \emph{I} is the interval where the functions have to be plotted.
\item \emph{name} is a string representing the name of a file.
\end{itemize}
\noindent Description: \begin{itemize}

\item This command plots one or several functions \emph{f1}, ... ,\emph{fn} on an interval \emph{I}.
   Functions can be either given as parameters of \textbf{plot} or as a list \emph{L}
   which elements are functions.
   The functions are drawn on the same plot with different colors.

\item If \emph{L} contains an element that is not a function (or a constant), an error
   occurs.

\item \textbf{plot} relies on the value of global variable \textbf{points}. Let $n$ be the 
   value of this variable. The algorithm is the following: each function is 
   evaluated at $n$ evenly distributed points in \emph{I}. At each point, the 
   computed value is a faithful rounding of the exact value with a sufficiently
   high precision. Each point is finally plotted.
   This should avoid numerical artefacts such as critical cancellations.

\item You can save the function plot either as a data file or as a postscript file.

\item If you use argument \textbf{file} with a string \emph{name}, \sollya will save a data file
   called name.dat and a gnuplot directives file called name.p. Invoking gnuplot
   on name.p will plot the data stored in name.dat.

\item If you use argument \textbf{postscript} with a string \emph{name}, \sollya will save a 
   postscript file called name.eps representing your plot.

\item If you use argument \textbf{postscriptfile} with a string \emph{name}, \sollya will 
   produce the corresponding name.dat, name.p and name.eps.

\item This command uses gnuplot to produce the final plot.
   If your terminal is not graphic (typically if you use \sollya through 
   ssh without -X)
   gnuplot should be able to detect that and produce an ASCII-art version on the
   standard output. If it is not the case, you can either store the plot in a
   postscript file to view it locally, or use \textbf{asciiplot} command.

\item If every function is constant, \textbf{plot} will not plot them but just display
   their value.

\item If the interval is reduced to a single point, \textbf{plot} will just display the
   value of the functions at this point.
\end{itemize}
\noindent Example 1: 
\begin{center}\begin{minipage}{15cm}\begin{Verbatim}[frame=single,commandchars=\\\|\~]
> plot(sin(x),0,cos(x),[-Pi,Pi]);
\end{Verbatim}
\end{minipage}\end{center}
\noindent Example 2: 
\begin{center}\begin{minipage}{15cm}\begin{Verbatim}[frame=single,commandchars=\\\|\~]
> plot(sin(x),0,cos(x),[-Pi,Pi],postscriptfile,"plotSinCos");
\end{Verbatim}
\end{minipage}\end{center}
\noindent Example 3: 
\begin{center}\begin{minipage}{15cm}\begin{Verbatim}[frame=single,commandchars=\\\|\~]
> plot(exp(0), sin(1), [0;1]);
1
0.84147098480789650665250232163029899962256306079837
\end{Verbatim}
\end{minipage}\end{center}
\noindent Example 4: 
\begin{center}\begin{minipage}{15cm}\begin{Verbatim}[frame=single,commandchars=\\\|\~]
> plot(sin(x), cos(x), [1;1]);
0.84147098480789650665250232163029899962256306079837
0.54030230586813971740093660744297660373231042061792
\end{Verbatim}
\end{minipage}\end{center}
See also: \textbf{externalplot} (\ref{labexternalplot}), \textbf{asciiplot} (\ref{labasciiplot}), \textbf{file} (\ref{labfile}), \textbf{postscript} (\ref{labpostscript}), \textbf{postscriptfile} (\ref{labpostscriptfile}), \textbf{points} (\ref{labpoints})

\subsection{$+$}
\label{labplus}
\noindent Name: \textbf{$+$}\\
addition function\\
\noindent Usage: 
\begin{center}
\emph{function1} \textbf{$+$} \emph{function2} : (\textsf{function}, \textsf{function}) $\rightarrow$ \textsf{function}\\
\emph{interval1} \textbf{$+$} \emph{interval2} : (\textsf{range}, \textsf{range}) $\rightarrow$ \textsf{range}\\
\emph{interval1} \textbf{$+$} \emph{constant} : (\textsf{range}, \textsf{constant}) $\rightarrow$ \textsf{range}\\
\emph{interval1} \textbf{$+$} \emph{constant} : (\textsf{constant}, \textsf{range}) $\rightarrow$ \textsf{range}\\
\end{center}
Parameters: 
\begin{itemize}
\item \emph{function1} and \emph{function2} represent functions
\item \emph{interval1} and \emph{interval2} represent intervals (ranges)
\item \emph{constant} represents a constant or constant expression
\end{itemize}
\noindent Description: \begin{itemize}

\item \\textbf{$+$} represents the addition (function) on reals. \n   The expression \\emph{function1} \\textbf{$+$} \\emph{function2} stands for\n   the function composed of the addition function and the two\n   functions \\emph{function1} and \\emph{function2}.\n
\item \\textbf{$+$} can be used for interval arithmetic on intervals\n   (ranges). \\textbf{$+$} will evaluate to an interval that safely\n   encompasses all images of the addition function with arguments varying\n   in the given intervals.  Any combination of intervals with intervals\n   or constants (resp. constant expressions) is supported. However, it is\n   not possible to represent families of functions using an interval as\n   one argument and a function (varying in the free variable) as the\n   other one.\n\end{itemize}
\noindent Example 1: 
\begin{center}\begin{minipage}{15cm}\begin{Verbatim}[frame=single]
\end{Verbatim}
\end{minipage}\end{center}
\noindent Example 2: 
\begin{center}\begin{minipage}{15cm}\begin{Verbatim}[frame=single]
\end{Verbatim}
\end{minipage}\end{center}
\noindent Example 3: 
\begin{center}\begin{minipage}{15cm}\begin{Verbatim}[frame=single]
\end{Verbatim}
\end{minipage}\end{center}
\noindent Example 4: 
\begin{center}\begin{minipage}{15cm}\begin{Verbatim}[frame=single]
\end{Verbatim}
\end{minipage}\end{center}
\noindent Example 5: 
\begin{center}\begin{minipage}{15cm}\begin{Verbatim}[frame=single]
\end{Verbatim}
\end{minipage}\end{center}
See also: \textbf{$-$} (\ref{labminus}), \textbf{$*$} (\ref{labmult}), \textbf{/} (\ref{labdivide}), \textbf{$\mathbf{\hat{~}}$} (\ref{labpower})

\subsection{points}
\label{labpoints}
\noindent Name: \textbf{points}\\
\phantom{aaa}controls the number of points chosen by \sollya in certain commands.\\[0.2cm]
\noindent Library names:\\
\verb|   void sollya_lib_set_points_and_print(sollya_obj_t)|\\
\verb|   void sollya_lib_set_points(sollya_obj_t)|\\
\verb|   sollya_obj_t sollya_lib_get_points()|\\[0.2cm]
\noindent Usage: 
\begin{center}
\textbf{points} = \emph{n} : \textsf{integer} $\rightarrow$ \textsf{void}\\
\textbf{points} = \emph{n} ! : \textsf{integer} $\rightarrow$ \textsf{void}\\
\textbf{points} : \textsf{constant}\\
\end{center}
Parameters: 
\begin{itemize}
\item \emph{n} represents the number of points
\end{itemize}
\noindent Description: \begin{itemize}

\item \textbf{points} is a global variable. Its value represents the number of points
   used in numerical algorithms of \sollya (namely \textbf{dirtyinfnorm},
   \textbf{dirtyintegral}, \textbf{dirtyfindzeros}, \textbf{plot}).
\end{itemize}
\noindent Example 1: 
\begin{center}\begin{minipage}{15cm}\begin{Verbatim}[frame=single]
> f=x^2*sin(1/x);
> points=10;
The number of points has been set to 10.
> dirtyfindzeros(f, [0;1]);
[|0, 0.31830988618379067153776752674502872406891929148092|]
> points=100;
The number of points has been set to 100.
> dirtyfindzeros(f, [0;1]);
[|0, 2.4485375860291590118289809749617594159147637806224e-2, 3.97887357729738339
42220940843128590508614911435115e-2, 4.54728408833986673625382181064326748669884
70211559e-2, 5.3051647697298445256294587790838120678153215246819e-2, 6.366197723
6758134307553505349005744813783858296184e-2, 7.957747154594766788444188168625718
101722982287023e-2, 0.106103295394596890512589175581676241356306430493638, 0.159
15494309189533576888376337251436203445964574046, 0.31830988618379067153776752674
502872406891929148092|]
\end{Verbatim}
\end{minipage}\end{center}
See also: \textbf{dirtyinfnorm} (\ref{labdirtyinfnorm}), \textbf{dirtyintegral} (\ref{labdirtyintegral}), \textbf{dirtyfindzeros} (\ref{labdirtyfindzeros}), \textbf{plot} (\ref{labplot}), \textbf{diam} (\ref{labdiam}), \textbf{prec} (\ref{labprec})

\subsection{postscriptfile}
\label{labpostscriptfile}
\noindent Name: \textbf{postscriptfile}\\
special value for commands 	extbf{plot} and 	extbf{externalplot}\\
\noindent Description: \begin{itemize}

\item \textbf{postscriptfile} is a special value used in commands \textbf{plot} and \textbf{externalplot} to save
   the result of the command in a data file and a postscript file.

\item As any value it can be affected to a variable and stored in lists.
\end{itemize}
\noindent Example 1: 
\begin{center}\begin{minipage}{15cm}\begin{Verbatim}[frame=single]
> savemode=postscriptfile;
> name="plotSinCos";
> plot(sin(x),0,cos(x),[-Pi,Pi],savemode, name);
\end{Verbatim}
\end{minipage}\end{center}
See also: \textbf{externalplot} (\ref{labexternalplot}), \textbf{plot} (\ref{labplot}), \textbf{file} (\ref{labfile}), \textbf{postscript} (\ref{labpostscript})

\subsection{postscript}
\label{labpostscript}
\noindent Name: \textbf{postscript}\\
\phantom{aaa}special value for commands \textbf{plot} and \textbf{externalplot}\\[0.2cm]
\noindent Library names:\\
\verb|   sollya_obj_t sollya_lib_postscript()|\\
\verb|   int sollya_lib_is_postscript(sollya_obj_t)|\\[0.2cm]
\noindent Description: \begin{itemize}

\item \textbf{postscript} is a special value used in commands \textbf{plot} and \textbf{externalplot} to save
   the result of the command in a postscript file.

\item As any value it can be affected to a variable and stored in lists.
\end{itemize}
\noindent Example 1: 
\begin{center}\begin{minipage}{15cm}\begin{Verbatim}[frame=single,commandchars=\\\|\~]
> savemode=postscript;
> name="plotSinCos";
> plot(sin(x),0,cos(x),[-Pi,Pi],savemode, name);
\end{Verbatim}
\end{minipage}\end{center}
See also: \textbf{externalplot} (\ref{labexternalplot}), \textbf{plot} (\ref{labplot}), \textbf{file} (\ref{labfile}), \textbf{postscriptfile} (\ref{labpostscriptfile})

\subsection{powers}
\label{labpowers}
\noindent Name: \textbf{powers}\\
\phantom{aaa}special value for global state \textbf{display}\\[0.2cm]
\noindent Library names:\\
\verb|   sollya_obj_t sollya_lib_powers()|\\
\verb|   int sollya_lib_is_powers(sollya_obj_t)|\\[0.2cm]
\noindent Description: \begin{itemize}

\item \textbf{powers} is a special value used for the global state \textbf{display}.  If
   the global state \textbf{display} is equal to \textbf{powers}, all data will be
   output in dyadic notation with numbers displayed in a Maple and
   PARI/GP compatible format.
    
   As any value it can be affected to a variable and stored in lists.
\end{itemize}
See also: \textbf{decimal} (\ref{labdecimal}), \textbf{dyadic} (\ref{labdyadic}), \textbf{hexadecimal} (\ref{labhexadecimal}), \textbf{binary} (\ref{labbinary}), \textbf{display} (\ref{labdisplay})

\subsection{ power }
\noindent Name: \textbf{\^}\\
power function\\

\noindent Usage: 
\begin{center}
\emph{function1} \textbf{\^} \emph{function2} : (\textsf{function}, \textsf{function}) $\rightarrow$ \textsf{function}\\
\end{center}
Parameters: 
\emph{function1} and \emph{function2} represent functions\\

\noindent Description: \begin{itemize}

\item \textbf{\^} represents the power (function) on reals. 
   The expression \emph{function1} \textbf{\^} \emph{function2} stands for
   the function composed of the power function and the two
   functions \emph{function1} and \emph{function2}, where \emph{function1} is
   the base and \emph{function2} the exponent.
   If \emph{function2} is a constant integer, \textbf{\^} is defined
   on negative values of \emph{function1}. Otherwise \textbf{\^}
   is defined as exp(y * log(x)).
\end{itemize}
\noindent Example 1: 
\begin{center}\begin{minipage}{14.8cm}\begin{Verbatim}[frame=single]
   > 5 ^ 2;
   25
\end{Verbatim}
\end{minipage}\end{center}
\noindent Example 2: 
\begin{center}\begin{minipage}{14.8cm}\begin{Verbatim}[frame=single]
   > x ^ 2;
   x^2
\end{Verbatim}
\end{minipage}\end{center}
\noindent Example 3: 
\begin{center}\begin{minipage}{14.8cm}\begin{Verbatim}[frame=single]
   > 3 ^ (-5);
   0.411522633744855967078189300411522633744855967078186e-2
\end{Verbatim}
\end{minipage}\end{center}
\noindent Example 4: 
\begin{center}\begin{minipage}{14.8cm}\begin{Verbatim}[frame=single]
   > (-3) ^ (-2.5);
   @NaN@
\end{Verbatim}
\end{minipage}\end{center}
\noindent Example 5: 
\begin{center}\begin{minipage}{14.8cm}\begin{Verbatim}[frame=single]
   > diff(sin(x) ^ exp(x));
   sin(x)^exp(x) * ((cos(x) * exp(x)) / sin(x) + exp(x) * log(sin(x)))
\end{Verbatim}
\end{minipage}\end{center}
See also: \textbf{$+$}, \textbf{$-$}, \textbf{$*$}, \textbf{/}

\subsection{precision}
\label{labprecision}
\noindent Name: \textbf{precision}\\
returns the precision necessary to represent a number.\\

\noindent Usage: 
\begin{center}
\textbf{precision}(\emph{x}) : \textsf{constant} $\rightarrow$ \textsf{integer}\\
\end{center}
Parameters: 
\begin{itemize}
\item \emph{x} is a dyadic number.
\end{itemize}
\noindent Description: \begin{itemize}

\item \textbf{precision}(x) is by definition $\vert x \vert$ if x equals 0, NaN, or Inf.

\item If \emph{x} is not zero, it can be uniquely written as $x = m \cdot 2^e$ where
   $m$ is an odd integer and $e$ is an integer. \textbf{precision}(x) returns the number
   of bits necessary to write $m$ (e.g. $\lceil \log_2(m) \rceil$).
\end{itemize}
\noindent Example 1: 
\begin{center}\begin{minipage}{15cm}\begin{Verbatim}[frame=single]
> a=round(Pi,20,RN);
> precision(a);
19
> m=mantissa(a);
> ceil(log2(m));
19
\end{Verbatim}
\end{minipage}\end{center}
See also: \textbf{mantissa} (\ref{labmantissa}), \textbf{exponent} (\ref{labexponent})

\subsection{prec}
\label{labprec}
\noindent Name: \textbf{prec}\\
controls the precision used in numerical computations.\\
\noindent Description: \begin{itemize}

\item \textbf{prec} is a global variable. Its value represents the precision of the 
   floating-point format used in numerical computations.

\item Many commands try to adapt their working precision in order to have 
   approximately $n$ correct bits in output, where $n$ is the value of \textbf{prec}.
\end{itemize}
\noindent Example 1: 
\begin{center}\begin{minipage}{15cm}\begin{Verbatim}[frame=single]
> display=binary!;
> prec=50;
The precision has been set to 50 bits.
> dirtyinfnorm(exp(x),[1;2]);
1.110110001110011001001011100011010100110111011011_2 * 2^(2)
> prec=100;
The precision has been set to 100 bits.
> dirtyinfnorm(exp(x),[1;2]);
1.110110001110011001001011100011010100110111011010110111001100001100111010001110
11101000100000011011_2 * 2^(2)
\end{Verbatim}
\end{minipage}\end{center}
See also: \textbf{evaluate} (\ref{labevaluate}), \textbf{diam} (\ref{labdiam})

\subsection{.:}
\label{labprepend}
\noindent Name: \textbf{.:}\\
add an element at the beginning of a list.\\
\noindent Usage: 
\begin{center}
\emph{x}\textbf{.:}\emph{L} : (\textsf{any type}, \textsf{list}) $\rightarrow$ \textsf{list}\\
\end{center}
Parameters: 
\begin{itemize}
\item \emph{x} is an object of any type.
\item \emph{L} is a list (possibly empty).
\end{itemize}
\noindent Description: \begin{itemize}

\item \\textbf{.:} adds the element \\emph{x} at the beginning of the list \\emph{L}.\n
\item Note that since \\emph{x} may be of any type, it can be in particular a list.\n\end{itemize}
\noindent Example 1: 
\begin{center}\begin{minipage}{15cm}\begin{Verbatim}[frame=single]
\end{Verbatim}
\end{minipage}\end{center}
\noindent Example 2: 
\begin{center}\begin{minipage}{15cm}\begin{Verbatim}[frame=single]
\end{Verbatim}
\end{minipage}\end{center}
\noindent Example 3: 
\begin{center}\begin{minipage}{15cm}\begin{Verbatim}[frame=single]
\end{Verbatim}
\end{minipage}\end{center}
See also: \textbf{:.} (\ref{labappend}), \textbf{@} (\ref{labconcat})

\subsection{printdouble}
\label{labprintdouble}
\noindent Name: \textbf{printdouble}\\
prints a constant value as a hexadecimal double precision number\\
\noindent Usage: 
\begin{center}
\textbf{printdouble}(\emph{constant}) : \textsf{constant} $\rightarrow$ \textsf{void}\\
\end{center}
Parameters: 
\begin{itemize}
\item \emph{constant} represents a constant
\end{itemize}
\noindent Description: \begin{itemize}

\item Prints a constant value as a hexadecimal number on 16 hexadecimal
   digits. The hexadecimal number represents the integer equivalent to
   the 64 bit memory representation of the constant considered as a
   double precision number.
    
   If the constant value does not hold on a double precision number, it
   is first rounded to the nearest double precision number before
   displayed. A warning is displayed in this case.
\end{itemize}
\noindent Example 1: 
\begin{center}\begin{minipage}{15cm}\begin{Verbatim}[frame=single]
> printdouble(3);
0x4008000000000000
\end{Verbatim}
\end{minipage}\end{center}
\noindent Example 2: 
\begin{center}\begin{minipage}{15cm}\begin{Verbatim}[frame=single]
> prec=100!;
> verbosity = 1!;
> printdouble(exp(5));
Warning: the given expression is not a constant but an expression to evaluate. A
 faithful evaluation will be used.
Warning: rounding down occurred before printing a value as a double.
0x40628d389970338f
\end{Verbatim}
\end{minipage}\end{center}
See also: \textbf{printsingle} (\ref{labprintsingle}), \textbf{printexpansion} (\ref{labprintexpansion}), \textbf{double} (\ref{labdouble})

\subsection{printexpansion}
\label{labprintexpansion}
\noindent Name: \textbf{printexpansion}\\
prints a polynomial in Horner form with its coefficients written as a expansions of double precision numbers\\
\noindent Usage: 
\begin{center}
\textbf{printexpansion}(\emph{polynomial}) : \textsf{function} $\rightarrow$ \textsf{void}
\\ 
\end{center}
Parameters: 
\begin{itemize}
\item \emph{polynomial} represents the polynomial to be printed
\end{itemize}
\noindent Description: \begin{itemize}

\item The command \textbf{printexpansion} prints the polynomial \emph{polynomial} in Horner form
   writing its coefficients as expansions of double precision
   numbers. The double precision numbers themselves are displayed in
   hexadecimal memory notation (see \textbf{printhexa}). 
    
   If some of the coefficients of the polynomial \emph{polynomial} are not
   floating-point constants but constant expressions, they are evaluated
   to floating-point constants using the global precision \textbf{prec}.  If a
   rounding occurs in this evaluation, a warning is displayed.
    
   If the exponent range of double precision is not sufficient to display
   all the mantissa bits of a coefficient, the coefficient is displayed
   rounded and a warning is displayed.
    
   If the argument \emph{polynomial} does not a polynomial, nothing but a
   warning or a newline is displayed. Constants can be displayed using
   \textbf{printexpansion} since they are polynomials of degree $0$.
\end{itemize}
\noindent Example 1: 
\begin{center}\begin{minipage}{15cm}\begin{Verbatim}[frame=single]
> printexpansion(roundcoefficients(taylor(exp(x),5,0),[|DD...|]));
0x3ff0000000000000 + x * (0x3ff0000000000000 + x * (0x3fe0000000000000 + x * ((0
x3fc5555555555555 + 0x3c65555555555555) + x * ((0x3fa5555555555555 + 0x3c4555555
5555555) + x * (0x3f81111111111111 + 0x3c01111111111111)))))
\end{Verbatim}
\end{minipage}\end{center}
\noindent Example 2: 
\begin{center}\begin{minipage}{15cm}\begin{Verbatim}[frame=single]
> printexpansion(remez(exp(x),5,[-1;1]));
(0x3ff0002eec908ce9 + 0xbc7df99eb225af5b + 0xb8d55834b08b1f18) + x * ((0x3ff0002
835917719 + 0x3c6d82c073b25ebf + 0xb902cf062b54b7b6 + 0x35b0000000000000) + x * 
((0x3fdff2d7e6a9c5e9 + 0xbc7b09a95b0d520f + 0xb915b639add55731 + 0x35a8000000000
000) + x * ((0x3fc54d67338ba09f + 0x3c4867596d0631cf + 0xb8ef0756bdb4af6e) + x *
 ((0x3fa66c209b825167 + 0x3c45ec5b6655b076 + 0xb8d8c125286400bc) + x * (0x3f81e5
5425e72ab4 + 0x3c263b25a1bf597b + 0xb8c843e0401dadd0 + 0x3540000000000000)))))
\end{Verbatim}
\end{minipage}\end{center}
\noindent Example 3: 
\begin{center}\begin{minipage}{15cm}\begin{Verbatim}[frame=single]
> verbosity = 1!;
> prec = 3500!;
> printexpansion(pi);
(0x400921fb54442d18 + 0x3ca1a62633145c07 + 0xb92f1976b7ed8fbc + 0x35c4cf98e80417
7d + 0x32631d89cd9128a5 + 0x2ec0f31c6809bbdf + 0x2b5519b3cd3a431b + 0x27e8158536
f92f8a + 0x246ba7f09ab6b6a9 + 0xa0eedd0dbd2544cf + 0x1d779fb1bd1310ba + 0x1a1a63
7ed6b0bff6 + 0x96aa485fca40908e + 0x933e501295d98169 + 0x8fd160dbee83b4e0 + 0x8c
59b6d799ae131c + 0x08f6cf70801f2e28 + 0x05963bf0598da483 + 0x023871574e69a459 + 
0x8000000005702db3 + 0x8000000000000000)
Warning: the expansion is not complete because of the limited exponent range of 
double precision.
Warning: rounding occurred while printing.
\end{Verbatim}
\end{minipage}\end{center}
See also: \textbf{printhexa} (\ref{labprinthexa}), \textbf{horner} (\ref{labhorner}), \textbf{print} (\ref{labprint}), \textbf{prec} (\ref{labprec}), \textbf{remez} (\ref{labremez}), \textbf{taylor} (\ref{labtaylor}), \textbf{roundcoefficients} (\ref{labroundcoefficients})

\subsection{printsingle}
\label{labprintsingle}
\noindent Name: \textbf{printsingle}\\
\phantom{aaa}prints a constant value as a hexadecimal single precision number\\[0.2cm]
\noindent Library name:\\
\verb|   void sollya_lib_printsingle(sollya_obj_t)|\\[0.2cm]
\noindent Usage: 
\begin{center}
\textbf{printsingle}(\emph{constant}) : \textsf{constant} $\rightarrow$ \textsf{void}\\
\end{center}
Parameters: 
\begin{itemize}
\item \emph{constant} represents a constant
\end{itemize}
\noindent Description: \begin{itemize}

\item Prints a constant value as a hexadecimal number on 8 hexadecimal
   digits. The hexadecimal number represents the integer equivalent to
   the 32 bit memory representation of the constant considered as a
   single precision number.
    
   If the constant value does not hold on a single precision number, it
   is first rounded to the nearest single precision number before it is
   displayed. A warning is displayed in this case.
\end{itemize}
\noindent Example 1: 
\begin{center}\begin{minipage}{15cm}\begin{Verbatim}[frame=single]
> printsingle(3);
0x40400000
\end{Verbatim}
\end{minipage}\end{center}
\noindent Example 2: 
\begin{center}\begin{minipage}{15cm}\begin{Verbatim}[frame=single]
> prec=100!;
> verbosity = 1!;
> printsingle(exp(5));
Warning: the given expression is not a constant but an expression to evaluate. A
 faithful evaluation will be used.
Warning: rounding down occurred before printing a value as a single.
0x431469c5
\end{Verbatim}
\end{minipage}\end{center}
See also: \textbf{printdouble} (\ref{labprintdouble}), \textbf{single} (\ref{labsingle})

\subsection{printxml}
\label{labprintxml}
\noindent Name: \textbf{printxml}\\
prints an expression as an MathML-Content-Tree\\
\noindent Usage: 
\begin{center}
\textbf{printxml}(\emph{expr}) : \textsf{function} $\rightarrow$ \textsf{void}\\
\textbf{printxml}(\emph{expr}) $>$ \emph{filename} : (\textsf{function}, \textsf{string}) $\rightarrow$ \textsf{void}\\
\textbf{printxml}(\emph{expr}) $>$ $>$ \emph{filename} : (\textsf{function}, \textsf{string}) $\rightarrow$ \textsf{void}\\
\end{center}
Parameters: 
\begin{itemize}
\item \emph{expr} represents a functional expression
\item \emph{filename} represents a character sequence indicating a file name
\end{itemize}
\noindent Description: \begin{itemize}

\item \textbf{printxml}(\emph{expr}) prints the functional expression \emph{expr} as a tree of
   MathML Content Definition Markups. This XML tree can be re-read in
   external tools or by usage of the \textbf{readxml} command.
    
   If a second argument \emph{filename} is given after a single $>$, the
   MathML tree is not output on the standard output of \sollya but if in
   the file \emph{filename} that get newly created or overwritten. If a double
   $>$ $>$ is given, the output will be appended to the file \emph{filename}.
\end{itemize}
\noindent Example 1: 
\begin{center}\begin{minipage}{15cm}\begin{Verbatim}[frame=single]
> printxml(x + 2 + exp(sin(x)));

<?xml version="1.0" encoding="UTF-8"?>
<!-- generated by sollya: http://sollya.gforge.inria.fr/ -->
<!-- syntax: printxml(...);   example: printxml(x^2-2*x+5); -->
<?xml-stylesheet type="text/xsl" href="http://sollya.gforge.inria.fr/mathmlc2p-w
eb.xsl"?>
<?xml-stylesheet type="text/xsl" href="mathmlc2p-web.xsl"?>
<!-- This stylesheet allows direct web browsing of MathML-c XML files (http:// o
r file://) -->

<math xmlns="http://www.w3.org/1998/Math/MathML">
<semantics>
<annotation-xml encoding="MathML-Content">
<lambda>
<bvar><ci> x </ci></bvar>
<apply>
<apply>
<plus/>
<apply>
<plus/>
<ci> x </ci>
<cn type="integer" base="10"> 2 </cn>
</apply>
<apply>
<exp/>
<apply>
<sin/>
<ci> x </ci>
</apply>
</apply>
</apply>
</apply>
</lambda>
</annotation-xml>
<annotation encoding="sollya/text">(x + 1b1) + exp(sin(x))</annotation>
</semantics>
</math>

\end{Verbatim}
\end{minipage}\end{center}
\noindent Example 2: 
\begin{center}\begin{minipage}{15cm}\begin{Verbatim}[frame=single]
> printxml(x + 2 + exp(sin(x))) > "foo.xml";
\end{Verbatim}
\end{minipage}\end{center}
\noindent Example 3: 
\begin{center}\begin{minipage}{15cm}\begin{Verbatim}[frame=single]
> printxml(x + 2 + exp(sin(x))) >> "foo.xml";
\end{Verbatim}
\end{minipage}\end{center}
See also: \textbf{readxml} (\ref{labreadxml}), \textbf{print} (\ref{labprint}), \textbf{write} (\ref{labwrite})

\subsection{print}
\label{labprint}
\noindent Name: \textbf{print}\\
\phantom{aaa}prints an expression\\[0.2cm]
\noindent Usage: 
\begin{center}
\textbf{print}(\emph{expr1},...,\emph{exprn}) : (\textsf{any type},..., \textsf{any type}) $\rightarrow$ \textsf{void}\\
\textbf{print}(\emph{expr1},...,\emph{exprn}) $>$ \emph{filename} : (\textsf{any type},..., \textsf{any type}, \textsf{string}) $\rightarrow$ \textsf{void}\\
\textbf{print}(\emph{expr1},...,\emph{exprn}) $>>$ \emph{filename} : (\textsf{any type},...,\textsf{any type}, \textsf{string}) $\rightarrow$ \textsf{void}\\
\end{center}
Parameters: 
\begin{itemize}
\item \emph{expr} represents an expression
\item \emph{filename} represents a character sequence indicating a file name
\end{itemize}
\noindent Description: \begin{itemize}

\item \textbf{print}(\emph{expr1},...,\emph{exprn}) prints the expressions \emph{expr1} through
   \emph{exprn} separated by spaces and followed by a newline.
    
   If a second argument \emph{filename} is given after a single  "$>$", the
   displaying is not output on the standard output of \sollya but if in
   the file \emph{filename} that get newly created or overwritten. If a double
    "$>>$" is given, the output will be appended to the file \emph{filename}.
    
   The global variables \textbf{display}, \textbf{midpointmode} and \textbf{fullparentheses} have
   some influence on the formatting of the output (see \textbf{display},
   \textbf{midpointmode} and \textbf{fullparentheses}).
    
   Remark that if one of the expressions \emph{expri} given in argument is of
   type \textsf{string}, the character sequence \emph{expri} evaluates to is
   displayed. However, if \emph{expri} is of type \textsf{list} and this list
   contains a variable of type \textsf{string}, the expression for the list
   is displayed, i.e.  all character sequences get displayed surrounded
   by double quotes ("). Nevertheless, escape sequences used upon defining
   character sequences are interpreted immediately.
\end{itemize}
\noindent Example 1: 
\begin{center}\begin{minipage}{15cm}\begin{Verbatim}[frame=single]
> print(x + 2 + exp(sin(x))); 
x + 2 + exp(sin(x))
> print("Hello","world");
Hello world
> print("Hello","you", 4 + 3, "other persons.");
Hello you 7 other persons.
\end{Verbatim}
\end{minipage}\end{center}
\noindent Example 2: 
\begin{center}\begin{minipage}{15cm}\begin{Verbatim}[frame=single]
> print("Hello");
Hello
> print([|"Hello"|]);
[|"Hello"|]
> s = "Hello";
> print(s,[|s|]);
Hello [|"Hello"|]
> t = "Hello\tyou";
> print(t,[|t|]);
Hello    you [|"Hello\tyou"|]
\end{Verbatim}
\end{minipage}\end{center}
\noindent Example 3: 
\begin{center}\begin{minipage}{15cm}\begin{Verbatim}[frame=single]
> print(x + 2 + exp(sin(x))) > "foo.sol";
> readfile("foo.sol");
x + 2 + exp(sin(x))

\end{Verbatim}
\end{minipage}\end{center}
\noindent Example 4: 
\begin{center}\begin{minipage}{15cm}\begin{Verbatim}[frame=single]
> print(x + 2 + exp(sin(x))) >> "foo.sol";
\end{Verbatim}
\end{minipage}\end{center}
\noindent Example 5: 
\begin{center}\begin{minipage}{15cm}\begin{Verbatim}[frame=single]
> display = decimal;
Display mode is decimal numbers.
> a = evaluate(sin(pi * x), 0.25);
> b = evaluate(sin(pi * x), [0.25; 0.25 + 1b-50]);
> print(a);
0.70710678118654752440084436210484903928483593768847
> display = binary;
Display mode is binary numbers.
> print(a);
1.011010100000100111100110011001111111001110111100110010010000100010110010111110
11000100110110011011101010100101010111110100111110001110101101111011000001011101
010001_2 * 2^(-1)
> display = hexadecimal;
Display mode is hexadecimal numbers.
> print(a);
0xb.504f333f9de6484597d89b3754abe9f1d6f60ba88p-4
> display = dyadic;
Display mode is dyadic numbers.
> print(a);
33070006991101558613323983488220944360067107133265b-165
> display = powers;
Display mode is dyadic numbers in integer-power-of-2 notation.
> print(a);
33070006991101558613323983488220944360067107133265 * 2^(-165)
> display = decimal;
Display mode is decimal numbers.
> midpointmode = off;
Midpoint mode has been deactivated.
> print(b);
[0.70710678118654752440084436210484903928483593768844;0.707106781186549497437217
82517557347782646274417048]
> midpointmode = on;
Midpoint mode has been activated.
> print(b);
0.7071067811865~4/5~
> display = dyadic;
Display mode is dyadic numbers.
> print(b);
[2066875436943847413332748968013809022504194195829b-161;165350034955508254441962
37019385936414432675156571b-164]
> display = decimal;
Display mode is decimal numbers.
> autosimplify = off;
Automatic pure tree simplification has been deactivated.
> fullparentheses = off;
Full parentheses mode has been deactivated.
> print(x + x * ((x + 1) + 1));
x + x * (x + 1 + 1)
> fullparentheses = on;
Full parentheses mode has been activated.
> print(x + x * ((x + 1) + 1));
x + (x * ((x + 1) + 1))
\end{Verbatim}
\end{minipage}\end{center}
See also: \textbf{write} (\ref{labwrite}), \textbf{printexpansion} (\ref{labprintexpansion}), \textbf{printdouble} (\ref{labprintdouble}), \textbf{printsingle} (\ref{labprintsingle}), \textbf{printxml} (\ref{labprintxml}), \textbf{readfile} (\ref{labreadfile}), \textbf{autosimplify} (\ref{labautosimplify}), \textbf{display} (\ref{labdisplay}), \textbf{midpointmode} (\ref{labmidpointmode}), \textbf{fullparentheses} (\ref{labfullparentheses}), \textbf{evaluate} (\ref{labevaluate}), \textbf{rationalmode} (\ref{labrationalmode})

\subsection{procedure}
\label{labprocedure}
\noindent Name: \textbf{procedure}\\
\phantom{aaa}defines and assigns a \sollya procedure\\[0.2cm]
\noindent Usage: 
\begin{center}
\textbf{procedure} \emph{identifier}(\emph{formal parameter1}, \emph{formal parameter2},..., \emph{formal parameter n}) \key{$\lbrace$} \emph{procedure body} \key{$\rbrace$} : \textsf{void} $\rightarrow$ \textsf{void}\\
\textbf{procedure} \emph{identifier}(\emph{formal parameter1}, \emph{formal parameter2},..., \emph{formal parameter n}) \key{$\lbrace$} \emph{procedure body} \textbf{return} \emph{expression}; \key{$\rbrace$} : \textsf{void} $\rightarrow$ \textsf{void}\\
\textbf{procedure} \emph{identifier}(\emph{formal list parameter} = ...) \key{$\lbrace$} \emph{procedure body} \key{$\rbrace$} : \textsf{void} $\rightarrow$ \textsf{void}\\
\textbf{procedure} \emph{identifier}(\emph{formal list parameter} = ...) \key{$\lbrace$} \emph{procedure body} \textbf{return} \emph{expression}; \key{$\rbrace$} : \textsf{void} $\rightarrow$ \textsf{void}\\
\end{center}
Parameters: 
\begin{itemize}
\item \emph{identifier} represents the name of the procedure to be defined and assigned
\item \emph{formal parameter1}, \emph{formal parameter2} through \emph{formal parameter n} represent identifiers used as formal parameters
\item \emph{formal list parameter} represents an identifier used as a formal parameter for the list of an arbitrary number of parameters
\item \emph{procedure body} represents the imperative statements in the body of the procedure
\item \emph{expression} represents the expression \textbf{procedure} shall evaluate to
\end{itemize}
\noindent Description: \begin{itemize}

\item The \textbf{procedure} keyword allows for defining and assigning procedures in
   the \sollya language. It is an abbreviation to a procedure definition
   using \textbf{proc} with the same formal parameters, procedure body and
   return-expression followed by an assignment of the procedure (object)
   to the identifier \emph{identifier}. In particular, all rules concerning
   local variables declared using the \textbf{var} keyword apply for \textbf{procedure}.
\end{itemize}
\noindent Example 1: 
\begin{center}\begin{minipage}{15cm}\begin{Verbatim}[frame=single]
> procedure succ(n) { return n + 1; };
> succ(5);
6
> 3 + succ(0);
4
> succ;
proc(n)
{
nop;
return (n) + (1);
}
\end{Verbatim}
\end{minipage}\end{center}
\noindent Example 2: 
\begin{center}\begin{minipage}{15cm}\begin{Verbatim}[frame=single]
> procedure myprint(L = ...) { var i; for i in L do i; };
> myprint("Lyon","Nancy","Beaverton","Coye-la-Foret","Amberg","Nizhny Novgorod",
"Cluj-Napoca");
Lyon
Nancy
Beaverton
Coye-la-Foret
Amberg
Nizhny Novgorod
Cluj-Napoca
\end{Verbatim}
\end{minipage}\end{center}
See also: \textbf{proc} (\ref{labproc}), \textbf{var} (\ref{labvar}), \textbf{bind} (\ref{labbind}), \textbf{getbacktrace} (\ref{labgetbacktrace})

\subsection{proc}
\label{labproc}
\noindent Name: \textbf{proc}\\
\phantom{aaa}defines a \sollya procedure\\[0.2cm]
\noindent Usage: 
\begin{center}
\textbf{proc}(\emph{formal parameter1}, \emph{formal parameter2},..., \emph{formal parameter n}) \key{$\lbrace$} \emph{procedure body} \key{$\rbrace$} : \textsf{void} $\rightarrow$ \textsf{procedure}\\
\textbf{proc}(\emph{formal parameter1}, \emph{formal parameter2},..., \emph{formal parameter n}) \key{$\lbrace$} \emph{procedure body} \textbf{return} \emph{expression}; \key{$\rbrace$} : \textsf{void} $\rightarrow$ \textsf{procedure}\\
\textbf{proc}(\emph{formal list parameter} = ...) \key{$\lbrace$} \emph{procedure body} \key{$\rbrace$} : \textsf{void} $\rightarrow$ \textsf{procedure}\\
\textbf{proc}(\emph{formal list parameter} = ...) \key{$\lbrace$} \emph{procedure body} \textbf{return} \emph{expression}; \key{$\rbrace$} : \textsf{void} $\rightarrow$ \textsf{procedure}\\
\end{center}
Parameters: 
\begin{itemize}
\item \emph{formal parameter1}, \emph{formal parameter2} through \emph{formal parameter n} represent identifiers used as formal parameters
\item \emph{formal list parameter} represents an identifier used as a formal parameter for the list of an arbitrary number of parameters
\item \emph{procedure body} represents the imperative statements in the body of the procedure
\item \emph{expression} represents the expression \textbf{proc} shall evaluate to
\end{itemize}
\noindent Description: \begin{itemize}

\item The \textbf{proc} keyword allows for defining procedures in the \sollya
   language. These procedures are common \sollya objects that can be
   applied to actual parameters after definition. Upon such an
   application, the \sollya interpreter applies the actual parameters to
   the formal parameters \emph{formal parameter1} through \emph{formal parameter n}
   (resp. builds up the list of arguments and applies it to the list
   \emph{formal list parameter}) and executes the \emph{procedure body}. The
   procedure applied to actual parameters evaluates then to the
   expression \emph{expression} in the \textbf{return} statement after the \emph{procedure body} 
   or to \textbf{void}, if no return statement is given (i.e. a \textbf{return}
   \textbf{void} statement is implicitly given).

\item \sollya procedures defined by \textbf{proc} have no name. They can be bound
   to an identifier by assigning the procedure object a \textbf{proc}
   expression produces to an identifier. However, it is possible to use
   procedures without giving them any name. For instance, \sollya
   procedures, i.e. procedure objects, can be elements of lists. They can
   even be given as an argument to other internal \sollya procedures. See
   also \textbf{procedure} on this subject.

\item Upon definition of a \sollya procedure using \textbf{proc}, no type check
   is performed. More precisely, the statements in \emph{procedure body} are
   merely parsed but not interpreted upon procedure definition with
   \textbf{proc}. Type checks are performed once the procedure is applied to
   actual parameters or to \textbf{void}. At this time, if the procedure was
   defined using several different formal parameters \emph{formal parameter 1}
   through \emph{formal parameter n}, it is checked whether the number of
   actual parameters corresponds to the number of formal parameters. If
   the procedure was defined using the syntax for a procedure with an
   arbitrary number of parameters by giving a \emph{formal list parameter},
   the number of actual arguments is not checked but only a list
   \emph{formal list parameter} of appropriate length is built up. Type checks are
   further performed upon execution of each statement in \emph{procedure body}
   and upon evaluation of the expression \emph{expression} to be returned.
    
   Procedures defined by \textbf{proc} containing a \textbf{quit} or \textbf{restart} command
   cannot be executed (i.e. applied). Upon application of a procedure,
   the \sollya interpreter checks beforehand for such a statement. If one
   is found, the application of the procedure to its arguments evaluates
   to \textbf{error}. A warning is displayed. Remark that in contrast to other
   type or semantic correctness checks, this check is really performed
   before interpreting any other statement in the body of the procedure.

\item Through the \textbf{var} keyword it is possible to declare local
   variables and thus to have full support of recursive procedures. This
   means a procedure defined using \textbf{proc} may contain in its \emph{procedure body} 
   an application of itself to some actual parameters: it suffices
   to assign the procedure (object) to an identifier with an appropriate
   name.

\item \sollya procedures defined using \textbf{proc} may return other
   procedures. Further \emph{procedure body} may contain assignments of
   locally defined procedure objects to identifiers. See \textbf{var} for the
   particular behaviour of local and global variables.

\item The expression \emph{expression} returned by a procedure is evaluated with
   regard to \sollya commands, procedures and external
   procedures. Simplification may be performed.  However, an application
   of a procedure defined by \textbf{proc} to actual parameters evaluates to the
   expression \emph{expression} that may contain the free global variable or
   that may be composed.
\end{itemize}
\noindent Example 1: 
\begin{center}\begin{minipage}{15cm}\begin{Verbatim}[frame=single,commandchars=\\\|\~]
> succ = proc(n) { return n + 1; };
> succ(5);
6
> 3 + succ(0);
4
> succ;
proc(n)
{
nop;
return (n) + (1);
}
\end{Verbatim}
\end{minipage}\end{center}
\noindent Example 2: 
\begin{center}\begin{minipage}{15cm}\begin{Verbatim}[frame=single,commandchars=\\\|\~]
> add = proc(m,n) { var res; res := m + n; return res; };
> add(5,6);
11
> add;
proc(m, n)
{
var res;
res := (m) + (n);
return res;
}
> verbosity = 1!;
> add(3);
Warning: at least one of the given expressions or a subexpression is not correct
ly typed
or its evaluation has failed because of some error on a side-effect.
error
> add(true,false);
Warning: at least one of the given expressions or a subexpression is not correct
ly typed
or its evaluation has failed because of some error on a side-effect.
Warning: the given expression or command could not be handled.
error
\end{Verbatim}
\end{minipage}\end{center}
\noindent Example 3: 
\begin{center}\begin{minipage}{15cm}\begin{Verbatim}[frame=single,commandchars=\\\|\~]
> succ = proc(n) { return n + 1; };
> succ(5);
6
> succ(x);
1 + x
\end{Verbatim}
\end{minipage}\end{center}
\noindent Example 4: 
\begin{center}\begin{minipage}{15cm}\begin{Verbatim}[frame=single,commandchars=\\\|\~]
> hey = proc() { print("Hello world."); };
> hey();
Hello world.
> print(hey());
Hello world.
void
> hey;
proc()
{
print("Hello world.");
return void;
}
\end{Verbatim}
\end{minipage}\end{center}
\noindent Example 5: 
\begin{center}\begin{minipage}{15cm}\begin{Verbatim}[frame=single,commandchars=\\\|\~]
> fac = proc(n) { var res; if (n == 0) then res := 1 else res := n * fac(n - 1);
 return res; };
> fac(5);
120
> fac(11);
39916800
> fac;
proc(n)
{
var res;
if (n) == (0) then
res := 1
else
res := (n) * (fac((n) - (1)));
return res;
}
\end{Verbatim}
\end{minipage}\end{center}
\noindent Example 6: 
\begin{center}\begin{minipage}{15cm}\begin{Verbatim}[frame=single,commandchars=\\\|\~]
> myprocs = [| proc(m,n) { return m + n; }, proc(m,n) { return m - n; } |];
> (myprocs[0])(5,6);
11
> (myprocs[1])(5,6);
-1
> succ = proc(n) { return n + 1; };
> pred = proc(n) { return n - 1; };
> applier = proc(p,n) { return p(n); };
> applier(succ,5);
6
> applier(pred,5);
4
\end{Verbatim}
\end{minipage}\end{center}
\noindent Example 7: 
\begin{center}\begin{minipage}{15cm}\begin{Verbatim}[frame=single,commandchars=\\\|\~]
> verbosity = 1!;
> myquit = proc(n) { print(n); quit; };
> myquit;
proc(n)
{
print(n);
quit;
return void;
}
> myquit(5);
Warning: a quit or restart command may not be part of a procedure body.
The procedure will not be executed.
Warning: an error occurred while executing a procedure.
Warning: the given expression or command could not be handled.
error
\end{Verbatim}
\end{minipage}\end{center}
\noindent Example 8: 
\begin{center}\begin{minipage}{15cm}\begin{Verbatim}[frame=single,commandchars=\\\|\~]
> printsucc = proc(n) { var succ; succ = proc(n) { return n + 1; }; print("Succe
ssor of",n,"is",succ(n)); };
> printsucc(5);
Successor of 5 is 6
\end{Verbatim}
\end{minipage}\end{center}
\noindent Example 9: 
\begin{center}\begin{minipage}{15cm}\begin{Verbatim}[frame=single,commandchars=\\\|\~]
> makeadd = proc(n) { var add; print("n =",n); add = proc(m,n) { return n + m; }
; return add; };
> makeadd(4);
n = 4
proc(m, n)
{
nop;
return (n) + (m);
}
> (makeadd(4))(5,6);
n = 4
11
\end{Verbatim}
\end{minipage}\end{center}
\noindent Example 10: 
\begin{center}\begin{minipage}{15cm}\begin{Verbatim}[frame=single,commandchars=\\\|\~]
> sumall = proc(L = ...) { var acc, i; acc = 0; for i in L do acc = acc + i; ret
urn acc; };
> sumall;
proc(L = ...)
{
var acc, i;
acc = 0;
for i in L do
acc = (acc) + (i);
return acc;
}
> sumall();
0
> sumall(2);
2
> sumall(2,5);
7
> sumall(2,5,7,9,16);
39
> sumall @ [|1,...,10|];
55
\end{Verbatim}
\end{minipage}\end{center}
See also: \textbf{return} (\ref{labreturn}), \textbf{externalproc} (\ref{labexternalproc}), \textbf{void} (\ref{labvoid}), \textbf{quit} (\ref{labquit}), \textbf{restart} (\ref{labrestart}), \textbf{var} (\ref{labvar}), \textbf{@} (\ref{labconcat}), \textbf{bind} (\ref{labbind}), \textbf{getbacktrace} (\ref{labgetbacktrace}), \textbf{error} (\ref{laberror})

\subsection{quit}
\label{labquit}
\noindent Name: \textbf{quit}\\
\phantom{aaa}quits \sollya\\[0.2cm]
\noindent Usage: 
\begin{center}
\textbf{quit} : \textsf{void} $\rightarrow$ \textsf{void}\\
\end{center}
\noindent Description: \begin{itemize}

\item The command \textbf{quit}, when executed, stops the execution of a \sollya
   script and leaves the \sollya interpreter unless the \textbf{quit} command 
   is executed in a \sollya script read into a main \sollya script by
   \textbf{execute} or $\#$include.
    
   Upon exiting the \sollya interpreter, all state is thrown away, all
   memory is deallocated, all bound libraries are unbound and the
   temporary files produced by \textbf{plot} and \textbf{externalplot} are deleted.
    
   If the \textbf{quit} command does not lead to a halt of the \sollya
   interpreter, a warning is displayed.
\end{itemize}
\noindent Example 1: 
\begin{center}\begin{minipage}{15cm}\begin{Verbatim}[frame=single]
> quit;
\end{Verbatim}
\end{minipage}\end{center}
See also: \textbf{restart} (\ref{labrestart}), \textbf{execute} (\ref{labexecute}), \textbf{plot} (\ref{labplot}), \textbf{externalplot} (\ref{labexternalplot}), \textbf{return} (\ref{labreturn})

\subsection{range}
\label{labrange}
\noindent Name: \textbf{range}\\
keyword representing a \textsf{range} type \\
\noindent Usage: 
\begin{center}
\textbf{range} : \textsf{type type}\\
\end{center}
\noindent Description: \begin{itemize}

\item \textbf{range} represents the \textsf{range} type for declarations
   of external procedures by means of \textbf{externalproc}.
    
   Remark that in contrast to other indicators, type indicators like
   \textbf{range} cannot be handled outside the \textbf{externalproc} context.  In
   particular, they cannot be assigned to variables.
\end{itemize}
See also: \textbf{externalproc} (\ref{labexternalproc}), \textbf{boolean} (\ref{labboolean}), \textbf{constant} (\ref{labconstant}), \textbf{function} (\ref{labfunction}), \textbf{integer} (\ref{labinteger}), \textbf{list of} (\ref{lablistof}), \textbf{string} (\ref{labstring})

\subsection{rationalapprox}
\label{labrationalapprox}
\noindent Name: \textbf{rationalapprox}\\
returns a fraction close to a given number.\\
\noindent Usage: 
\begin{center}
\textbf{rationalapprox}(\emph{x},\emph{n}) : (\textsf{constant}, \textsf{integer}) $\rightarrow$ \textsf{function}
\end{center}
Parameters: 
\begin{itemize}
\item \emph{x} is a number to approximate.
\item \emph{n} is a integer (representing a format).
\end{itemize}
\noindent Description: \begin{itemize}

\item \textbf{rationalapprox}(\emph{x},\emph{n}) returns a constant function of the form $a/b$ where $a$ and $b$ are
   integers. The value $a/b$ is an approximation of \emph{x}. The quality of this 
   approximation is determined by the parameter \emph{n} that indicates the number of
   correct bits that $a/b$ should have.

\item The command is not safe in the sense that it is not ensured that the error 
   between $a/b$ and \emph{x} is less than $2^{-n}$.

\item The following algorithm is used: \emph{x} is first rounded downwards and upwards to
   a format of \emph{n} bits, thus obtaining an interval $[x_l,\,x_u]$. This interval is then
   developped into a continued fraction as far as the representation is the same
   for every elements of $[x_l,\,x_u]$. The corresponding fraction is returned.

\item Since rational numbers are not a primitive object of \sollya, the fraction is
   returned as a constant function. It can be quite amazing, because \sollya
   immediately simplifies a constant function by evaluating it when the constant
   has to be displayed.
   To avoid this, you can use \textbf{print} (that displays the expression representing
   the constant and not the constant itself) or the commands \textbf{numerator} 
   and \textbf{denominator}.
\end{itemize}
\noindent Example 1: 
\begin{center}\begin{minipage}{15cm}\begin{Verbatim}[frame=single]
> pi10 = rationalapprox(Pi,10);
> pi50 = rationalapprox(Pi,50);
> pi100 = rationalapprox(Pi,100);
> print( pi10, ": ", simplify(floor(-log2(abs(pi10-Pi)/Pi))), "bits." );
22 / 7 :  11 bits.
> print( pi50, ": ", simplify(floor(-log2(abs(pi50-Pi)/Pi))), "bits." );
90982559 / 28960648 :  50 bits.
> print( pi100, ": ", simplify(floor(-log2(abs(pi100-Pi)/Pi))), "bits." );
4850225745369133 / 1543874804974140 :  101 bits.
\end{Verbatim}
\end{minipage}\end{center}
\noindent Example 2: 
\begin{center}\begin{minipage}{15cm}\begin{Verbatim}[frame=single]
> a=0.1;
> b=rationalapprox(a,4);
> numerator(b); denominator(b);
1
10
> print(simplify(floor(-log2(abs((b-a)/a)))), "bits.");
166 bits.
\end{Verbatim}
\end{minipage}\end{center}
See also: \textbf{print} (\ref{labprint}), \textbf{numerator} (\ref{labnumerator}), \textbf{denominator} (\ref{labdenominator})

\subsection{rationalmode}
\label{labrationalmode}
\noindent Name: \textbf{rationalmode}\\
global variable controlling if rational arithmetic is used or not.\\
\noindent Usage: 
\begin{center}
\textbf{rationalmode} = \emph{activation value} : \textsf{on$|$off} $\rightarrow$ \textsf{void}\\
\textbf{rationalmode} = \emph{activation value} ! : \textsf{on$|$off} $\rightarrow$ \textsf{void}\\
\textbf{rationalmode} : \textsf{on$|$off}\\
\end{center}
Parameters: 
\begin{itemize}
\item \emph{activation value} controls if rational arithmetic should be used or not
\end{itemize}
\noindent Description: \begin{itemize}

\item \textbf{rationalmode} is a global variable. When its value is \textbf{off}, which is the default,
   \sollya will not use rational arithmetic to simplify expressions. All computations,
   including the evaluation of constant expressions given on the \sollya prompt,
   will be performed using floating-point and interval arithmetic. Constant expressions
   will be approximated by floating-point numbers, which are in most cases faithful 
   roundings of the expressions, when shown at the prompt. 

\item When the value of the global variable \textbf{rationalmode} is \textbf{on}, \sollya will use 
   rational arithmetic when simplifying expressions. Constant expressions, given 
   at the \sollya prompt, will be simplified to rational numbers and displayed 
   as such when they are in the set of the rational numbers. Otherwise, flaoting-point
   and interval arithmetic will be used to compute a floating-point approximation,
   which is in most cases a faithful rounding of the constant expression.
\end{itemize}
\noindent Example 1: 
\begin{center}\begin{minipage}{15cm}\begin{Verbatim}[frame=single]
> rationalmode=off!;
> 19/17 + 3/94;
1.1495619524405506883604505632040050062578222778473
> rationalmode=on!;
> 19/17 + 3/94;
1837 / 1598
\end{Verbatim}
\end{minipage}\end{center}
\noindent Example 2: 
\begin{center}\begin{minipage}{15cm}\begin{Verbatim}[frame=single]
> rationalmode=off!;
> exp(19/17 + 3/94);
3.15680977395514136754709208944824276340328162814418
> rationalmode=on!;
> exp(19/17 + 3/94);
3.15680977395514136754709208944824276340328162814418
\end{Verbatim}
\end{minipage}\end{center}
See also: \textbf{on} (\ref{labon}), \textbf{off} (\ref{laboff}), \textbf{numerator} (\ref{labnumerator}), \textbf{denominator} (\ref{labdenominator}), \textbf{simplifysafe} (\ref{labsimplifysafe})

\subsection{RD}
\label{labrd}
\noindent Name: \textbf{RD}\\
constant representing rounding-downwards mode.\\
\noindent Description: \begin{itemize}

\item \\textbf{RD} is used in command \\textbf{round} to specify that the value $x$ must be rounded\n   to the greatest floating-point number $y$ such that $y \\le x$.\n\end{itemize}
\noindent Example 1: 
\begin{center}\begin{minipage}{15cm}\begin{Verbatim}[frame=single]
\end{Verbatim}
\end{minipage}\end{center}
See also: \textbf{RZ} (\ref{labrz}), \textbf{RU} (\ref{labru}), \textbf{RN} (\ref{labrn}), \textbf{round} (\ref{labround})

\subsection{readfile}
\label{labreadfile}
\noindent Name: \textbf{readfile}\\
reads the content of a file into a string variable\\
\noindent Usage: 
\begin{center}
\textbf{readfile}(\emph{filename}) : \textsf{string} $\rightarrow$ \textsf{string}\\
\end{center}
Parameters: 
\begin{itemize}
\item \emph{filename} represents a character sequence indicating a file name
\end{itemize}
\noindent Description: \begin{itemize}

\item \textbf{readfile} opens the file indicated by \emph{filename}, reads it and puts its
   contents in a character sequence of type \textsf{string} that is returned.
    
   If the file indicated by \emph{filename} cannot be opened for reading, a
   warning is displayed and \textbf{readfile} evaluates to an \textbf{error} variable of
   type \textsf{error}.
\end{itemize}
\noindent Example 1: 
\begin{center}\begin{minipage}{15cm}\begin{Verbatim}[frame=single]
> print("Hello world") > "myfile.txt";
> t = readfile("myfile.txt"); 
> t;
Hello world

\end{Verbatim}
\end{minipage}\end{center}
\noindent Example 2: 
\begin{center}\begin{minipage}{15cm}\begin{Verbatim}[frame=single]
> verbosity=1!;
> readfile("afile.txt");
Warning: the file "afile.txt" could not be opened for reading.
Warning: at least one of the given expressions or a subexpression is not correct
ly typed
or its evaluation has failed because of some error on a side-effect.
error
\end{Verbatim}
\end{minipage}\end{center}
See also: \textbf{parse} (\ref{labparse}), \textbf{execute} (\ref{labexecute}), \textbf{write} (\ref{labwrite}), \textbf{print} (\ref{labprint})

\subsection{readxml}
\label{labreadxml}
\noindent Name: \textbf{readxml}\\
reads an expression written as a MathML-Content-Tree in a file\\
\noindent Usage: 
\begin{center}
\textbf{readxml}(\emph{filename}) : \textsf{string} $\rightarrow$ \textsf{function} $|$ \textsf{error}
\end{center}
Parameters: 
\begin{itemize}
\item \emph{filename} represents a character sequence indicating a file name
\end{itemize}
\noindent Description: \begin{itemize}

\item \textbf{readxml}(\emph{filename}) reads the first occurrence of a lambda
   application with one bounded variable on applications of the supported
   basic functions in file \emph{filename} and returns it as a \sollya
   functional expression.
    
   If the file \emph{filename} does not contain a valid MathML-Content tree,
   \textbf{readxml} tries to find an "annotation encoding" markup of type
   "sollya/text". If this annotation contains a character sequence
   that can be parsed by \textbf{parse}, \textbf{readxml} returns that expression.  Otherwise
   \textbf{readxml} displays a warning and returns an \textbf{error} variable of type
   \textsf{error}.
\end{itemize}
\noindent Example 1: 
\begin{center}\begin{minipage}{15cm}\begin{Verbatim}[frame=single]
> readxml("readxmlexample.xml");
2 + x + exp(sin(x))
\end{Verbatim}
\end{minipage}\end{center}
See also: \textbf{printxml} (\ref{labprintxml}), \textbf{readfile} (\ref{labreadfile}), \textbf{parse} (\ref{labparse})

\subsection{relative}
\label{labrelative}
\noindent Name: \textbf{relative}\\
indicates a relative error for \textbf{externalplot}, \textbf{fpminimax} or \textbf{supnorm}\\
\noindent Usage: 
\begin{center}
\textbf{relative} : \textsf{absolute$|$relative}\\
\end{center}
\noindent Description: \begin{itemize}

\item The use of \textbf{relative} in the command \textbf{externalplot} indicates that during
   plotting in \textbf{externalplot} a relative error is to be considered.
    
   See \textbf{externalplot} for details.

\item Used with \textbf{fpminimax}, \textbf{relative} indicates that \textbf{fpminimax} must try to minimize
   the relative error.
    
   See \textbf{fpminimax} for details.

\item When given in argument to \textbf{supnorm}, \textbf{relative} indicates that an absolute error
   is to be considered for supremum norm computation.
    
   See \textbf{supnorm} for details.
\end{itemize}
\noindent Example 1: 
\begin{center}\begin{minipage}{15cm}\begin{Verbatim}[frame=single]
> bashexecute("gcc -fPIC -c externalplotexample.c");
> bashexecute("gcc -shared -o externalplotexample externalplotexample.o -lgmp -l
mpfr");
> externalplot("./externalplotexample",absolute,exp(x),[-1/2;1/2],12,perturb);
\end{Verbatim}
\end{minipage}\end{center}
See also: \textbf{externalplot} (\ref{labexternalplot}), \textbf{fpminimax} (\ref{labfpminimax}), \textbf{absolute} (\ref{lababsolute}), \textbf{bashexecute} (\ref{labbashexecute}), \textbf{supnorm} (\ref{labsupnorm})

\subsection{remez}
\label{labremez}
\noindent Name: \textbf{remez}\\
\phantom{aaa}computes the minimax of a function on an interval.\\[0.2cm]
\noindent Library names:\\
\verb|   sollya_obj_t sollya_lib_remez(sollya_obj_t, sollya_obj_t, sollya_obj_t, ...)|\\
\verb|   sollya_obj_t sollya_lib_v_remez(sollya_obj_t, sollya_obj_t, sollya_obj_t, va_list)|\\[0.2cm]
\noindent Usage: 
\begin{center}
\textbf{remez}(\emph{f}, \emph{n}, \emph{range}, \emph{w}, \emph{quality}) : (\textsf{function}, \textsf{integer}, \textsf{range}, \textsf{function}, \textsf{constant}) $\rightarrow$ \textsf{function}\\
\textbf{remez}(\emph{f}, \emph{L}, \emph{range}, \emph{w}, \emph{quality}) : (\textsf{function}, \textsf{list}, \textsf{range}, \textsf{function}, \textsf{constant}) $\rightarrow$ \textsf{function}\\
\end{center}
Parameters: 
\begin{itemize}
\item \emph{f} is the function to be approximated
\item \emph{n} is the degree of the polynomial that must approximate \emph{f}
\item \emph{L} is a list of integers or a list of functions and indicates the basis for the approximation of \emph{f}
\item \emph{range} is the interval where the function must be approximated
\item \emph{w} (optional) is a weight function. Default is 1.
\item \emph{quality} (optional) is a parameter that controls the quality of the returned polynomial \emph{p}, with respect to the exact minimax $p^\star$. Default is 1e-5.
\end{itemize}
\noindent Description: \begin{itemize}

\item \textbf{remez} computes an approximation of the function $f$ with respect to
   the weight function $w$ on the interval \emph{range}. More precisely, it
   searches $p$ such that $\|pw-f\|_{\infty}$ is
   (almost) minimal among all $p$ of a certain form. The norm is
   the infinity norm, e.g. $\|g\|_{\infty} = \max \{|g(x)|, x \in \mathrm{range}\}.$

\item If $w=1$ (the default case), it consists in searching the best
   polynomial approximation of $f$ with respect to the absolute error.
   If $f=1$ and $w$ is of the form $1/g$, it consists in
   searching the best polynomial approximation of $g$ with respect to the
   relative error.

\item If $n$ is given, $p$ is searched among the polynomials with degree not
   greater than $n$.
   If \emph{L} is given and is a list of integers, $p$ is searched as a linear
   combination of monomials $X^k$ where $k$ belongs to \emph{L}.
   In the case when \emph{L} is a list of integers, it may contain ellipses but cannot
   be end-elliptic.
   If \emph{L} is given and is a list of functions $g_k$, $p$ is searched as a
   linear combination of the $g_k$. In that case \emph{L} cannot contain ellipses.
   It is the user responsability to check that the $g_k$ are linearly independent
   over the interval \emph{range}. Moreover, the functions $w\cdot g_k$ must be at least
   twice differentiable over \emph{range}. If these conditions are not fulfilled, the
   algorithm might fail or even silently return a result as if it successfully
   found the minimax, though the returned $p$ is not optimal.

\item The polynomial is obtained by a convergent iteration called Remez'
   algorithm (and an extension of this algorithm, due to Stiefel).
   The algorithm computes a sequence $p_1,\dots ,p_k,\dots$
   such that $e_k = \|p_k w-f\|_{\infty}$ converges towards
   the optimal value $e$. The algorithm is stopped when the relative error
   between $e_k$ and $e$ is less than \emph{quality}.
\end{itemize}
\noindent Example 1: 
\begin{center}\begin{minipage}{15cm}\begin{Verbatim}[frame=single]
> p = remez(exp(x),5,[0;1]);
> degree(p);
5
> dirtyinfnorm(p-exp(x),[0;1]);
1.12956981510961487071711938292660776072226345893629e-6
\end{Verbatim}
\end{minipage}\end{center}
\noindent Example 2: 
\begin{center}\begin{minipage}{15cm}\begin{Verbatim}[frame=single]
> p = remez(1,[|0,2,4,6,8|],[0,Pi/4],1/cos(x));
> canonical=on!;
> p;
0.99999999994393732180959690352543887130348096061124 + -0.4999999957155685776877
20530637215446709494672222587 * x^2 + 4.1666613233473633009941059480570275870113
220089059e-2 * x^4 + -1.3886529147145693651355523880319714051047635695061e-3 * x
^6 + 2.4372679177224179934800328511009205218114284220126e-5 * x^8
\end{Verbatim}
\end{minipage}\end{center}
\noindent Example 3: 
\begin{center}\begin{minipage}{15cm}\begin{Verbatim}[frame=single]
> p1 = remez(exp(x),5,[0;1],default,1e-5);
> p2 = remez(exp(x),5,[0;1],default,1e-10);
> p3 = remez(exp(x),5,[0;1],default,1e-15);
> dirtyinfnorm(p1-exp(x),[0;1]);
1.12956981510961487071711938292660776072226345893629e-6
> dirtyinfnorm(p2-exp(x),[0;1]);
1.12956980227478675612619255125474525171079325793124e-6
> dirtyinfnorm(p3-exp(x),[0;1]);
1.12956980227478675612619255125474525171079325793124e-6
\end{Verbatim}
\end{minipage}\end{center}
\noindent Example 4: 
\begin{center}\begin{minipage}{15cm}\begin{Verbatim}[frame=single]
> L = [|exp(x), sin(x), cos(x)-1, sin(x^3)|];
> g = (2^x-1)/x;
> p1 = remez(g, L, [-1/16;1/16]);
> p2 = remez(g, 3, [-1/16;1/16]);
> dirtyinfnorm(p1 - g, [-1/16;1/16]);
9.8841323829271038137685646777951687620288462194745e-8
> dirtyinfnorm(p2 - g, [-1/16;1/16]);
2.54337800593461418356437401152248866818783932027105e-9
\end{Verbatim}
\end{minipage}\end{center}
See also: \textbf{dirtyinfnorm} (\ref{labdirtyinfnorm}), \textbf{infnorm} (\ref{labinfnorm}), \textbf{fpminimax} (\ref{labfpminimax}), \textbf{guessdegree} (\ref{labguessdegree}), \textbf{taylorform} (\ref{labtaylorform}), \textbf{taylor} (\ref{labtaylor})

\subsection{rename}
\label{labrename}
\noindent Name: \textbf{rename}\\
rename the free variable.\\

\noindent Usage: 
\begin{center}
\textbf{rename}(\emph{ident1},\emph{ident2}) : \textsf{void}\\
\end{center}
Parameters: 
\begin{itemize}
\item \emph{ident1} is the current name of the free variable.
\item \emph{ident2} is a fresh name.
\end{itemize}
\noindent Description: \begin{itemize}

\item \textbf{rename} lets one change the name of the free variable. \sollya can handle only
   one free variable at a time. The first time in a session that an unbound name 
   is used in a context where it can be interpreted as a free variable, the name
   is used to represent the free variable of \sollya. In the following, this name
   can be changed using \textbf{rename}.

\item Be careful: if \emph{ident2} has been set before, its value will be lost. Use the 
   command \textbf{isbound} to know if \emph{ident2} is already used or not.

\item If \emph{ident1} is not the current name of the free variable, an error occurs.

\item If \textbf{rename} is used at a time when the name of the free variable has not been 
   defined, \emph{ident1} is just ignored and the name of the free variable is 
   set to \emph{ident2}.
\end{itemize}
\noindent Example 1: 
\begin{center}\begin{minipage}{15cm}\begin{Verbatim}[frame=single]
> f=sin(x);
> f;
sin(x)
> rename(x,y);
> f;
sin(y)
\end{Verbatim}
\end{minipage}\end{center}
\noindent Example 2: 
\begin{center}\begin{minipage}{15cm}\begin{Verbatim}[frame=single]
> a=1;
> f=sin(x);
> rename(x,a);
> a;
a
> f;
sin(a)
\end{Verbatim}
\end{minipage}\end{center}
\noindent Example 3: 
\begin{center}\begin{minipage}{15cm}\begin{Verbatim}[frame=single]
> verbosity=1!;
> f=sin(x);
> rename(y,z);
Warning: the current free variable is named "x" and not "y". Can only rename the
 free variable.
The last command will have no effect.
\end{Verbatim}
\end{minipage}\end{center}
\noindent Example 4: 
\begin{center}\begin{minipage}{15cm}\begin{Verbatim}[frame=single]
> rename(x,y);
> isbound(x);
false
> isbound(y);
true
\end{Verbatim}
\end{minipage}\end{center}
See also: \textbf{isbound} (\ref{labisbound})

\subsection{restart}
\label{labrestart}
\noindent Name: \textbf{restart}\\
\phantom{aaa}brings \sollya back to its initial state\\[0.2cm]
\noindent Usage: 
\begin{center}
\textbf{restart} : \textsf{void} $\rightarrow$ \textsf{void}\\
\end{center}
\noindent Description: \begin{itemize}

\item The command \textbf{restart} brings \sollya back to its initial state.  All
   current state is abandoned, all libraries unbound and all memory freed.
    
   The \textbf{restart} command has no effect when executed inside a \sollya
   script read into a main \sollya script using \textbf{execute}. It is executed
   in a \sollya script included by a $\#$include macro.
    
   Using the \textbf{restart} command in nested elements of imperative
   programming like for or while loops is possible. Since in most cases
   abandoning the current state of \sollya means altering a loop
   invariant, warnings for the impossibility of continuing a loop may
   follow unless the state is rebuilt.
\end{itemize}
\noindent Example 1: 
\begin{center}\begin{minipage}{15cm}\begin{Verbatim}[frame=single]
> print(exp(x));
exp(x)
> a = 3;
> restart;
The tool has been restarted.
> print(x);
x
> a;
Warning: the identifier "a" is neither assigned to, nor bound to a library funct
ion nor external procedure, nor equal to the current free variable.
Will interpret "a" as "x".
x
\end{Verbatim}
\end{minipage}\end{center}
\noindent Example 2: 
\begin{center}\begin{minipage}{15cm}\begin{Verbatim}[frame=single]
> print(exp(x));
exp(x)
> for i from 1 to 10 do {
      print(i);
      if (i == 5) then restart;
  };
1
2
3
4
5
The tool has been restarted.
Warning: the tool has been restarted inside a for loop.
The for loop will no longer be executed.
\end{Verbatim}
\end{minipage}\end{center}
\noindent Example 3: 
\begin{center}\begin{minipage}{15cm}\begin{Verbatim}[frame=single]
> print(exp(x));
exp(x)
> a = 3;
> for i from 1 to 10 do {
      print(i);
      if (i == 5) then {
          restart;
          i = 7;
      };
  };
1
2
3
4
5
The tool has been restarted.
8
9
10
> print(x);
x
> a;
Warning: the identifier "a" is neither assigned to, nor bound to a library funct
ion nor external procedure, nor equal to the current free variable.
Will interpret "a" as "x".
x
\end{Verbatim}
\end{minipage}\end{center}
See also: \textbf{quit} (\ref{labquit}), \textbf{execute} (\ref{labexecute})

\subsection{return}
\label{labreturn}
\noindent Name: \textbf{return}\\
indicates an expression to be returned in a procedure\\
\noindent Usage: 
\begin{center}
\textbf{return} \emph{expression} : \textsf{void}
\\ 
\end{center}
Parameters: 
\begin{itemize}
\item \emph{expression} represents the expression to be returned
\end{itemize}
\noindent Description: \begin{itemize}

\item The keyword \textbf{return} allows for returning the (evaluated) expression
   \emph{expression} at the end of a begin-end-block ({}-block) used as a
   \sollya procedure body. See \textbf{proc} for further details concerning
   \sollya procedure definitions.
     
   Statements for returning expressions using \textbf{return} are only possible
    at the end of a begin-end-block used as a \sollya procedure
    body. Only one \textbf{return} statement can be given per begin-end-block.

\item If at the end of a procedure definition using \textbf{proc} no \textbf{return}
   statement is given, a \textbf{return} \textbf{void} statement is implicitely
   added. Procedures, i.e. procedure objects, when printed out in \sollya
   defined with an implicit \textbf{return} \textbf{void} statement are displayed with
   this statement explicitly given.
\end{itemize}
\noindent Example 1: 
\begin{center}\begin{minipage}{15cm}\begin{Verbatim}[frame=single]
> succ = proc(n) { var res; res := n + 1; return res; };
> succ(5);
6
> succ;
proc(n)
begin
var res;
res := (n) + (1);
return res;
end
\end{Verbatim}
\end{minipage}\end{center}
\noindent Example 2: 
\begin{center}\begin{minipage}{15cm}\begin{Verbatim}[frame=single]
> hey = proc(s) { print("Hello",s); };
> hey("world");
Hello world
> hey;
proc(s)
begin
print("Hello", s);
return void;
end
\end{Verbatim}
\end{minipage}\end{center}
See also: \textbf{proc} (\ref{labproc}), \textbf{void} (\ref{labvoid})

\subsection{revert}
\label{labrevert}
\noindent Name: \textbf{revert}\\
\phantom{aaa}reverts a list.\\[0.2cm]
\noindent Library name:\\
\verb|   sollya_obj_t sollya_lib_revert(sollya_obj_t)|\\[0.2cm]
\noindent Usage: 
\begin{center}
\textbf{revert}(\emph{L}) : \textsf{list} $\rightarrow$ \textsf{list}\\
\end{center}
Parameters: 
\begin{itemize}
\item \emph{L} is a list.
\end{itemize}
\noindent Description: \begin{itemize}

\item \textbf{revert}(\emph{L}) returns the same list, but with its elements in reverse order.

\item If \emph{L} is an end-elliptic list, \textbf{revert} will fail with an error.
\end{itemize}
\noindent Example 1: 
\begin{center}\begin{minipage}{15cm}\begin{Verbatim}[frame=single]
> revert([| |]);
[| |]
\end{Verbatim}
\end{minipage}\end{center}
\noindent Example 2: 
\begin{center}\begin{minipage}{15cm}\begin{Verbatim}[frame=single]
> revert([|2,3,5,2,1,4|]);
[|4, 1, 2, 5, 3, 2|]
\end{Verbatim}
\end{minipage}\end{center}
See also: \textbf{sort} (\ref{labsort}), \textbf{head} (\ref{labhead}), \textbf{tail} (\ref{labtail})

\subsection{rn}
\label{labrn}
\noindent Name: \textbf{RN}\\
constant representing rounding-to-nearest mode.\\

\noindent Description: \begin{itemize}

\item \textbf{RN} is used in command \textbf{round} to specify that the value must be rounded
   to the nearest representable floating-point number.
\end{itemize}
\noindent Example 1: 
\begin{center}\begin{minipage}{15cm}\begin{Verbatim}[frame=single]
> display=binary!;
> round(Pi,20,RN);
1.100100100001111111_2 * 2^(1)
\end{Verbatim}
\end{minipage}\end{center}
See also: \textbf{RD} (\ref{labrd}), \textbf{RU} (\ref{labru}), \textbf{RZ} (\ref{labrz}), \textbf{round} (\ref{labround})

\subsection{roundcoefficients}
\label{labroundcoefficients}
\noindent Name: \textbf{roundcoefficients}\\
rounds the coefficients of a polynomial to classical formats.\\
\noindent Usage: 
\begin{center}
\textbf{roundcoefficients}(\emph{p},\emph{L}) : (\textsf{function}, \textsf{list}) $\rightarrow$ \textsf{function}\\
\end{center}
Parameters: 
\begin{itemize}
\item \emph{p} is a function. Usually a polynomial.
\item \emph{L} is a list of formats.
\end{itemize}
\noindent Description: \begin{itemize}

\item If \emph{p} is a polynomial and \emph{L} a list of floating-point formats, 
   \textbf{roundcoefficients}(\emph{p},\emph{L}) rounds each coefficient of \emph{p} to the corresponding format
   in \emph{L}.

\item If \emph{p} is not a polynomial, \textbf{roundcoefficients} does not do anything.

\item If \emph{L} contains other elements than \textbf{HP}, \textbf{halfprecision}, \textbf{SG}, \textbf{single}, \textbf{D}, \textbf{double}, 
   \textbf{DE}, \textbf{doubleextended}, \textbf{DD}, \textbf{doubledouble}, \textbf{QD}, \textbf{quad}, \textbf{TD} and \textbf{tripledouble},
   an error occurs.

\item The coefficients in \emph{p} corresponding to $X^i$ is rounded to the 
   format L[i]. If \emph{L} does not contain enough elements
   (e.g. if \textbf{length}(L) $<$ \textbf{degree}(p)+1), a warning is displayed. However, the
   coefficients corresponding to an element of \emph{L} are rounded. The trailing 
   coefficients (that do not have a corresponding element in \emph{L}) are kept with
   their own precision.
   If \emph{L} contains too much elements, the trailing useless elements are ignored.
   In particular \emph{L} may be end-elliptic in which case \textbf{roundcoefficients} has the 
   natural behavior.
\end{itemize}
\noindent Example 1: 
\begin{center}\begin{minipage}{15cm}\begin{Verbatim}[frame=single]
> p=exp(1) + x*(exp(2) + x*exp(3));
> display=binary!;
> roundcoefficients(p,[|DD,D,D|]);
1.010110111111000010101000101100010100010101110110100101010011010101011111101110
001010110001000000010011101_2 * 2^(1) + x * (1.110110001110011001001011100011010
100110111011010111_2 * 2^(2) + x * (1.010000010101111001011011111101101111101100
010000011_2 * 2^(4)))
> roundcoefficients(p,[|DD,D...|]);
1.010110111111000010101000101100010100010101110110100101010011010101011111101110
001010110001000000010011101_2 * 2^(1) + x * (1.110110001110011001001011100011010
100110111011010111_2 * 2^(2) + x * (1.010000010101111001011011111101101111101100
010000011_2 * 2^(4)))
\end{Verbatim}
\end{minipage}\end{center}
\noindent Example 2: 
\begin{center}\begin{minipage}{15cm}\begin{Verbatim}[frame=single]
> f=sin(exp(1)*x);
> display=binary!;
> f;
sin(x * (1.010110111111000010101000101100010100010101110110100101010011010101011
11110111000101011000100000001001110011110100111100111100011101100010111001110001
01100000111101_2 * 2^(1)))
> roundcoefficients(f,[|D...|]);
sin(x * (1.010110111111000010101000101100010100010101110110100101010011010101011
11110111000101011000100000001001110011110100111100111100011101100010111001110001
01100000111101_2 * 2^(1)))
\end{Verbatim}
\end{minipage}\end{center}
\noindent Example 3: 
\begin{center}\begin{minipage}{15cm}\begin{Verbatim}[frame=single]
> p=exp(1) + x*(exp(2) + x*exp(3));
> verbosity=1!;
> display=binary!;
> roundcoefficients(p,[|DD,D|]);
Warning: the number of the given formats does not correspond to the degree of th
e given polynomial.
Warning: the 0th coefficient of the given polynomial does not evaluate to a floa
ting-point constant without any rounding.
Will evaluate the coefficient in the current precision in floating-point before 
rounding to the target format.
Warning: the 1th coefficient of the given polynomial does not evaluate to a floa
ting-point constant without any rounding.
Will evaluate the coefficient in the current precision in floating-point before 
rounding to the target format.
Warning: rounding may have happened.
1.010110111111000010101000101100010100010101110110100101010011010101011111101110
001010110001000000010011101_2 * 2^(1) + x * (1.110110001110011001001011100011010
100110111011010111_2 * 2^(2) + x * (1.010000010101111001011011111101101111101100
01000001011111001011010100101111011111110001010011011101000100110000111010001110
010000010110000101100000111001011100101001_2 * 2^(4)))
\end{Verbatim}
\end{minipage}\end{center}
See also: \textbf{halfprecision} (\ref{labhalfprecision}), \textbf{single} (\ref{labsingle}), \textbf{double} (\ref{labdouble}), \textbf{doubleextended} (\ref{labdoubleextended}), \textbf{doubledouble} (\ref{labdoubledouble}), \textbf{quad} (\ref{labquad}), \textbf{tripledouble} (\ref{labtripledouble}), \textbf{fpminimax} (\ref{labfpminimax}), \textbf{remez} (\ref{labremez}), \textbf{implementpoly} (\ref{labimplementpoly}), \textbf{subpoly} (\ref{labsubpoly})

\subsection{ roundcorrectly }
\noindent Name: \textbf{roundcorrectly}\\
rounds an approximation range correctly to some precision\\

\noindent Usage: 
\begin{center}
\textbf{roundcorrectly}(\emph{range}) : \textsf{range} $\rightarrow$ \textsf{constant}\\
\end{center}
Parameters: 
\begin{itemize}
\item \emph{range} represents a range in which an exact value lies
\end{itemize}
\noindent Description: \begin{itemize}

\item Let \emph{range} be a range of values, determined by some approximation
   process, safely bounding an unknown value $v$. The command
   \textbf{roundcorrectly}(\emph{range}) determines a precision such that for this precision,
   rounding to the nearest any value in \emph{range} yields to the same
   result, i.e. to the correct rounding of $v$.
   If no such precision exists, a warning is displayed and \textbf{roundcorrectly}
   evaluates to NaN.
\end{itemize}
\noindent Example 1: 
\begin{center}\begin{minipage}{15cm}\begin{Verbatim}[frame=single]
> printbinary(roundcorrectly([1.010001_2; 1.0101_2]));
1.01_2 * 2^(0)
> printbinary(roundcorrectly([1.00001_2; 1.001_2]));
1._2 * 2^(0)
\end{Verbatim}
\end{minipage}\end{center}
\noindent Example 2: 
\begin{center}\begin{minipage}{15cm}\begin{Verbatim}[frame=single]
> roundcorrectly([-1; 1]);
@NaN@
\end{Verbatim}
\end{minipage}\end{center}
See also: \textbf{round}

\subsection{roundingwarnings}
\label{labroundingwarnings}
\noindent Name: \textbf{roundingwarnings}\\
\phantom{aaa}global variable controlling whether or not a warning is displayed when roundings occur.\\[0.2cm]
\noindent Library names:\\
\verb|   void sollya_lib_set_roundingwarnings_and_print(sollya_obj_t)|\\
\verb|   void sollya_lib_set_roundingwarnings(sollya_obj_t)|\\
\verb|   sollya_obj_t sollya_lib_get_roundingwarnings()|\\[0.2cm]
\noindent Usage: 
\begin{center}
\textbf{roundingwarnings} = \emph{activation value} : \textsf{on$|$off} $\rightarrow$ \textsf{void}\\
\textbf{roundingwarnings} = \emph{activation value} ! : \textsf{on$|$off} $\rightarrow$ \textsf{void}\\
\textbf{roundingwarnings} : \textsf{on$|$off}\\
\end{center}
Parameters: 
\begin{itemize}
\item \emph{activation value} controls if warnings should be shown or not
\end{itemize}
\noindent Description: \begin{itemize}

\item \textbf{roundingwarnings} is a global variable. When its value is \textbf{on}, warnings are
   emitted in appropriate verbosity modes (see \textbf{verbosity}) when roundings
   occur.  When its value is \textbf{off}, these warnings are suppressed.

\item This mode depends on a verbosity of at least 1. See
   \textbf{verbosity} for more details.

\item Default is \textbf{on} when the standard input is a terminal and
   \textbf{off} when \sollya input is read from a file.
\end{itemize}
\noindent Example 1: 
\begin{center}\begin{minipage}{15cm}\begin{Verbatim}[frame=single]
> verbosity=1!;
> roundingwarnings = on;
Rounding warning mode has been activated.
> exp(0.1);
Warning: Rounding occurred when converting the constant "0.1" to floating-point 
with 165 bits.
If safe computation is needed, try to increase the precision.
Warning: rounding has happened. The value displayed is a faithful rounding of th
e true result.
1.1051709180756476248117078264902466682245471947375
> roundingwarnings = off;
Rounding warning mode has been deactivated.
> exp(0.1);
1.1051709180756476248117078264902466682245471947375
\end{Verbatim}
\end{minipage}\end{center}
See also: \textbf{on} (\ref{labon}), \textbf{off} (\ref{laboff}), \textbf{verbosity} (\ref{labverbosity}), \textbf{midpointmode} (\ref{labmidpointmode}), \textbf{rationalmode} (\ref{labrationalmode}), \textbf{suppressmessage} (\ref{labsuppressmessage}), \textbf{unsuppressmessage} (\ref{labunsuppressmessage}), \textbf{showmessagenumbers} (\ref{labshowmessagenumbers}), \textbf{getsuppressedmessages} (\ref{labgetsuppressedmessages})

\subsection{round}
\label{labround}
\noindent Name: \textbf{round}\\
rounds a number to a floating-point format.\\
\noindent Usage: 
\begin{center}
\textbf{round}(\emph{x},\emph{n},\emph{mode}) : (\textsf{constant}, \textsf{integer}, \textbf{RD} $|$ \textbf{RU} $|$ \textbf{RN} $|$ \textbf{RZ}) $\rightarrow$ \textsf{constant}\\
\textbf{round}(\emph{x},\emph{format},\emph{mode}) : (\textsf{constant}, \textsf{D$|$double$|$DD$|$doubledouble$|$DE$|$doubleextended$|$TD$|$tripledouble}, \textbf{RD} $|$ \textbf{RU} $|$ \textbf{RN} $|$ \textbf{RZ}) $\rightarrow$ \textsf{constant}\\
\end{center}
Parameters: 
\begin{itemize}
\item \emph{x} is a constant to be rounded.
\item \emph{n} is the precision of the target format.
\item \emph{format} is the name of a supported floating-point format.
\item \emph{mode} is the desired rounding mode.
\end{itemize}
\noindent Description: \begin{itemize}

\item If used with an integer parameter \emph{n}, \textbf{round}(\emph{x},\emph{n},\emph{mode}) rounds \emph{x} to a floating-point number with 
   precision \emph{n}, according to rounding-mode \emph{mode}. 

\item If used with a format parameter \emph{format}, \textbf{round}(\emph{x},\emph{format},\emph{mode}) rounds \emph{x} to a floating-point number in the 
   floating-point format \emph{format}, according to rounding-mode \emph{mode}. 

\item Subnormal numbers are not handled are handled only if a \emph{format} parameter is given
   that is different from \textbf{doubleextended}. The range of possible exponents is the 
   range used for all numbers represented in \sollya (e.g. basically the range 
   used in the library MPFR). 
\end{itemize}
\noindent Example 1: 
\begin{center}\begin{minipage}{15cm}\begin{Verbatim}[frame=single]
> display=binary!;
> round(Pi,20,RN);
1.100100100001111111_2 * 2^(1)
\end{Verbatim}
\end{minipage}\end{center}
\noindent Example 2: 
\begin{center}\begin{minipage}{15cm}\begin{Verbatim}[frame=single]
> printhexa(round(exp(17),53,RU));
0x417709348c0ea4f9
> printhexa(D(exp(17)));
0x417709348c0ea4f9
\end{Verbatim}
\end{minipage}\end{center}
\noindent Example 3: 
\begin{center}\begin{minipage}{15cm}\begin{Verbatim}[frame=single]
> display=binary!;
> a=2^(-1100);
> round(a,53,RN);
1_2 * 2^(-1100)
> round(a,D,RN);
0
> double(a);
0
\end{Verbatim}
\end{minipage}\end{center}
See also: \textbf{RN} (\ref{labrn}), \textbf{RD} (\ref{labrd}), \textbf{RU} (\ref{labru}), \textbf{RZ} (\ref{labrz}), \textbf{double} (\ref{labdouble}), \textbf{doubleextended} (\ref{labdoubleextended}), \textbf{doubledouble} (\ref{labdoubledouble}), \textbf{tripledouble} (\ref{labtripledouble}), \textbf{roundcoefficients} (\ref{labroundcoefficients}), \textbf{roundcorrectly} (\ref{labroundcorrectly}), \textbf{printhexa} (\ref{labprinthexa}), \textbf{printfloat} (\ref{labprintfloat})

\subsection{RU}
\label{labru}
\noindent Name: \textbf{RU}\\
constant representing rounding-upwards mode.\\
\noindent Description: \begin{itemize}

\item \\textbf{RU} is used in command \\textbf{round} to specify that the value $x$ must be rounded\n   to the smallest floating-point number $y$ such that $x \\le y$.\n\end{itemize}
\noindent Example 1: 
\begin{center}\begin{minipage}{15cm}\begin{Verbatim}[frame=single]
\end{Verbatim}
\end{minipage}\end{center}
See also: \textbf{RZ} (\ref{labrz}), \textbf{RD} (\ref{labrd}), \textbf{RN} (\ref{labrn}), \textbf{round} (\ref{labround})

\subsection{RZ}
\label{labrz}
\noindent Name: \textbf{RZ}\\
constant representing rounding-to-zero mode.\\
\noindent Description: \begin{itemize}

\item \textbf{RZ} is used in command \textbf{round} to specify that the value must be rounded
   to the closest floating-point number towards zero. It just consists in 
   truncating the value to the desired format.
\end{itemize}
\noindent Example 1: 
\begin{center}\begin{minipage}{15cm}\begin{Verbatim}[frame=single]
> display=binary!;
> round(Pi,20,RZ);
1.1001001000011111101_2 * 2^(1)
\end{Verbatim}
\end{minipage}\end{center}
See also: \textbf{RD} (\ref{labrd}), \textbf{RU} (\ref{labru}), \textbf{RN} (\ref{labrn}), \textbf{round} (\ref{labround})

\subsection{searchgal}
\label{labsearchgal}
\noindent Name: \textbf{searchgal}\\
searches for a preimage of a function such that the rounding the image commits an error smaller than a constant\\

\noindent Usage: 
\begin{center}
\textbf{searchgal}(\emph{function}, \emph{start}, \emph{preimage precision}, \emph{steps}, \emph{format}, \emph{error bound}) : (\textsf{function}, \textsf{constant}, \textsf{integer}, \textsf{integer}, \textsf{D$|$double$|$DD$|$doubledouble$|$DE$|$doubleextended$|$TD$|$tripledouble}, \textsf{constant}) $\rightarrow$ \textsf{list}\\
\textbf{searchgal}(\emph{list of functions}, \emph{start}, \emph{preimage precision}, \emph{steps}, \emph{list of format}, \emph{list of error bounds}) : (\textsf{list}, \textsf{constant}, \textsf{integer}, \textsf{integer}, \textsf{list}, \textsf{list}) $\rightarrow$ \textsf{list}\\
\end{center}
Parameters: 
\begin{itemize}
\item \emph{function} represents the function to be considered
\item \emph{start} represents a value around which the search is to be performed
\item \emph{preimage precision} represents the precision (discretization) for the eligible preimage values
\item \emph{steps} represents the log2 of the number of search steps to be performed
\item \emph{format} represents the format the image of the function is to be rounded to
\item \emph{error bound} represents a upper bound on the relative rounding error when rounding the image
\item \emph{list of functions} represents the functions to be considered
\item \emph{list of formats} represents the respective formats the images of the functions are to be rounded to
\item \emph{list of error bounds} represents a upper bound on the relative rounding error when rounding the image
\end{itemize}
\noindent Description: \begin{itemize}

\item The command \textbf{searchgal} searches for a preimage $z$ of a function
   \emph{function} or a list of functions \emph{list of functions} such that
   $z$ is a floating-point number with \emph{preimage precision}
   significant mantissa bits and the image $y$ of the function,
   respectively each image $y_i$ of the functions, rounds to
   format \emph{format} respectively to the corresponding format in \emph{list of format} 
   with a relative rounding error less than \emph{error bound}
   respectively the corresponding value in \emph{list of error bounds}. During
   this search, at most 2 raised to \emph{steps} attempts are made. The search
   starts with a preimage value equal to \emph{start}. This value is then
   increased and decreased by $1$ ulp in precision \emph{preimage precision} 
   until a value is found or the step limit is reached.
    
   If the search finds an appropriate preimage $z$, \textbf{searchgal}
   evaluates to a list containing this value. Otherwise, \textbf{searchgal}
   evaluates to an empty list.
\end{itemize}
\noindent Example 1: 
\begin{center}\begin{minipage}{15cm}\begin{Verbatim}[frame=single]
> searchgal(log(x),2,53,15,DD,1b-112);
[| |]
> searchgal(log(x),2,53,18,DD,1b-112);
[|2.0000000000384972054234822280704975128173828125|]
\end{Verbatim}
\end{minipage}\end{center}
\noindent Example 2: 
\begin{center}\begin{minipage}{15cm}\begin{Verbatim}[frame=single]
> f = exp(x);
> s = searchgal(f,2,53,18,DD,1b-112);
> if (s != [||]) then {
>    v = s[0];
>    print("The rounding error is 2^(",evaluate(log2(abs(DD(f)/f - 1)),v),")");
> } else print("No value found");
The rounding error is 2^( -1.12106878438809380148206984258358542322113874177832e
2 )
\end{Verbatim}
\end{minipage}\end{center}
\noindent Example 3: 
\begin{center}\begin{minipage}{15cm}\begin{Verbatim}[frame=single]
> searchgal([|sin(x),cos(x)|],1,53,15,[|D,D|],[|1b-62,1b-60|]);
[|1.00000000000159494639717649988597258925437927246094|]
\end{Verbatim}
\end{minipage}\end{center}
See also: \textbf{round} (\ref{labround}), \textbf{double} (\ref{labdouble}), \textbf{doubledouble} (\ref{labdoubledouble}), \textbf{tripledouble} (\ref{labtripledouble}), \textbf{evaluate} (\ref{labevaluate}), \textbf{worstcase} (\ref{labworstcase})

\subsection{simplifysafe}
\label{labsimplifysafe}
\noindent Name: \textbf{simplifysafe}\\
simplifies an expression representing a function\\
\noindent Usage: 
\begin{center}
\textbf{simplifysafe}(\emph{function}) : \textsf{function} $\rightarrow$ \textsf{function}
\end{center}
Parameters: 
\begin{itemize}
\item \emph{function} represents the expression to be simplified
\end{itemize}
\noindent Description: \begin{itemize}

\item The command \textbf{simplifysafe} simplifies the expression given in argument
   representing the function \emph{function}.  The command \textbf{simplifysafe} does not
   endanger the safety of computations even in \sollya's floating-point
   environment: the function returned is mathematically equal to the
   function \emph{function}. 
    
   Remark that the simplification provided by \textbf{simplifysafe} is not perfect:
   they may exist simpler equivalent expressions for expressions returned
   by \textbf{simplifysafe}.
\end{itemize}
\noindent Example 1: 
\begin{center}\begin{minipage}{15cm}\begin{Verbatim}[frame=single]
> print(simplifysafe((6 + 2) + (5 + exp(0)) * x));
8 + 6 * x
\end{Verbatim}
\end{minipage}\end{center}
\noindent Example 2: 
\begin{center}\begin{minipage}{15cm}\begin{Verbatim}[frame=single]
> print(simplifysafe((log(x - x + 1) + asin(1))));
(pi) / 2
\end{Verbatim}
\end{minipage}\end{center}
\noindent Example 3: 
\begin{center}\begin{minipage}{15cm}\begin{Verbatim}[frame=single]
> print(simplifysafe((log(x - x + 1) + asin(1)) - (atan(1) * 2)));
(pi) / 2 - (pi) / 4 * 2
\end{Verbatim}
\end{minipage}\end{center}
See also: \textbf{simplify} (\ref{labsimplify}), \textbf{autosimplify} (\ref{labautosimplify})

\subsection{simplify}
\label{labsimplify}
\noindent Name: \textbf{simplify}\\
simplifies an expression representing a function\\
\noindent Usage: 
\begin{center}
\textbf{simplify}(\emph{function}) : \textsf{function} $\rightarrow$ \textsf{function}\\
\end{center}
Parameters: 
\begin{itemize}
\item \emph{function} represents the expression to be simplified
\end{itemize}
\noindent Description: \begin{itemize}

\item The command \textbf{simplify} simplifies constant subexpressions of the
   expression given in argument representing the function
   \emph{function}. Those constant subexpressions are evaluated using
   floating-point arithmetic with the global precision \textbf{prec}.
\end{itemize}
\noindent Example 1: 
\begin{center}\begin{minipage}{15cm}\begin{Verbatim}[frame=single]
> print(simplify(sin(pi * x)));
sin(3.14159265358979323846264338327950288419716939937508 * x)
> print(simplify(erf(exp(3) + x * log(4))));
erf(2.00855369231876677409285296545817178969879078385544e1 + x * 1.3862943611198
906188344642429163531361510002687205)
\end{Verbatim}
\end{minipage}\end{center}
\noindent Example 2: 
\begin{center}\begin{minipage}{15cm}\begin{Verbatim}[frame=single]
> prec = 20!;
> t = erf(0.5);
> s = simplify(erf(0.5));
> prec = 200!;
> t;
0.5204998778130465376827466538919645287364515757579637000588058
> s;
0.52050018310546875
\end{Verbatim}
\end{minipage}\end{center}
See also: \textbf{simplifysafe} (\ref{labsimplifysafe}), \textbf{autosimplify} (\ref{labautosimplify}), \textbf{prec} (\ref{labprec}), \textbf{evaluate} (\ref{labevaluate})

\subsection{single}
\label{labsingle}
\noindent Names: \textbf{single}, \textbf{SG}\\
\phantom{aaa}rounding to the nearest IEEE 754 single (binary32).\\[0.2cm]
\noindent Library names:\\
\verb|   sollya_obj_t sollya_lib_single(sollya_obj_t)|\\
\verb|   sollya_obj_t sollya_lib_single_obj()|\\
\verb|   int sollya_lib_is_single_obj(sollya_obj_t)|\\
\verb|   sollya_obj_t sollya_lib_build_function_single(sollya_obj_t)|\\
\verb|   #define SOLLYA_SG(x) sollya_lib_build_function_single(x)|\\[0.2cm]
\noindent Description: \begin{itemize}

\item \textbf{single} is both a function and a constant.

\item As a function, it rounds its argument to the nearest IEEE 754 single precision (i.e. IEEE754-2008 binary32) number.
   Subnormal numbers are supported as well as standard numbers: it is the real
   rounding described in the standard.

\item As a constant, it symbolizes the single precision format. It is used in 
   contexts when a precision format is necessary, e.g. in the commands 
   \textbf{round} and \textbf{roundcoefficients}. In is not supported for \textbf{implementpoly}.
   See the corresponding help pages for examples.
\end{itemize}
\noindent Example 1: 
\begin{center}\begin{minipage}{15cm}\begin{Verbatim}[frame=single,commandchars=\\\|\~]
> display=binary!;
> SG(0.1);
1.10011001100110011001101_2 * 2^(-4)
> SG(4.17);
1.000010101110000101001_2 * 2^(2)
> SG(1.011_2 * 2^(-1073));
0
\end{Verbatim}
\end{minipage}\end{center}
See also: \textbf{halfprecision} (\ref{labhalfprecision}), \textbf{double} (\ref{labdouble}), \textbf{doubleextended} (\ref{labdoubleextended}), \textbf{doubledouble} (\ref{labdoubledouble}), \textbf{quad} (\ref{labquad}), \textbf{tripledouble} (\ref{labtripledouble}), \textbf{roundcoefficients} (\ref{labroundcoefficients}), \textbf{implementpoly} (\ref{labimplementpoly}), \textbf{round} (\ref{labround}), \textbf{printsingle} (\ref{labprintsingle})

\subsection{sinh}
\label{labsinh}
\noindent Name: \textbf{sinh}\\
the hyperbolic sine function.\\
\noindent Description: \begin{itemize}

\item \textbf{sinh} is the usual hyperbolic sine function: $\sinh(x) = \frac{e^x - e^{-x}}{2}$.

\item It is defined for every real number $x$.
\end{itemize}
See also: \textbf{asinh} (\ref{labasinh}), \textbf{cosh} (\ref{labcosh}), \textbf{tanh} (\ref{labtanh})

\subsection{sin}
\label{labsin}
\noindent Name: \textbf{sin}\\
\phantom{aaa}the sine function.\\[0.2cm]
\noindent Library names:\\
\verb|   sollya_obj_t sollya_lib_sin(sollya_obj_t)|\\
\verb|   sollya_obj_t sollya_lib_build_function_sin(sollya_obj_t)|\\
\verb|   #define SOLLYA_SIN(x) sollya_lib_build_function_sin(x)|\\[0.2cm]
\noindent Description: \begin{itemize}

\item \textbf{sin} is the usual sine function.

\item It is defined for every real number $x$.
\end{itemize}
See also: \textbf{asin} (\ref{labasin}), \textbf{cos} (\ref{labcos}), \textbf{tan} (\ref{labtan})

\subsection{sort}
\label{labsort}
\noindent Name: \textbf{sort}\\
sorts a list of real numbers.\\
\noindent Usage: 
\begin{center}
\textbf{sort}(\emph{L}) : \textsf{list} $\rightarrow$ \textsf{list}\\
\end{center}
Parameters: 
\begin{itemize}
\item \emph{L} is a list.
\end{itemize}
\noindent Description: \begin{itemize}

\item If \emph{L} contains only constant values, \textbf{sort}(\emph{L}) returns the same list, but
   sorted in increasing order.

\item If \emph{L} contains at least one element that is not a constant, the command fails 
   with a type error.

\item If \emph{L} is an end-elliptic list, \textbf{sort} will fail with an error.
\end{itemize}
\noindent Example 1: 
\begin{center}\begin{minipage}{15cm}\begin{Verbatim}[frame=single]
> sort([| |]);
[| |]
> sort([|2,3,5,2,1,4|]);
[|1, 2, 2, 3, 4, 5|]
\end{Verbatim}
\end{minipage}\end{center}

\subsection{sqrt}
\label{labsqrt}
\noindent Name: \textbf{sqrt}\\
square root.\\
\noindent Description: \begin{itemize}

\item \textbf{sqrt} is the square root, e.g. the inverse of the function square: $\sqrt{y}$
   is the unique positive $x$ such that $x^2=y$.

\item It is defined only for $x$ in $[0;+\infty]$.
\end{itemize}

\subsection{string}
\label{labstring}
\noindent Name: \textbf{string}\\
\phantom{aaa}keyword representing a \textsf{string} type \\[0.2cm]
\noindent Library name:\\
\verb|   SOLLYA_EXTERNALPROC_TYPE_STRING|\\[0.2cm]
\noindent Usage: 
\begin{center}
\textbf{string} : \textsf{type type}\\
\end{center}
\noindent Description: \begin{itemize}

\item \textbf{string} represents the \textsf{string} type for declarations
   of external procedures by means of \textbf{externalproc}.
    
   Remark that in contrast to other indicators, type indicators like
   \textbf{string} cannot be handled outside the \textbf{externalproc} context.  In
   particular, they cannot be assigned to variables.
\end{itemize}
See also: \textbf{externalproc} (\ref{labexternalproc}), \textbf{boolean} (\ref{labboolean}), \textbf{constant} (\ref{labconstant}), \textbf{function} (\ref{labfunction}), \textbf{integer} (\ref{labinteger}), \textbf{list of} (\ref{lablistof}), \textbf{range} (\ref{labrange}), \textbf{object} (\ref{labobject})

\subsection{ subpoly }
\noindent Name: \textbf{subpoly}\\
restricts the monomial basis of a polynomial to a list of monomials\\

\noindent Usage: 
\begin{center}
\textbf{subpoly}(\emph{polynomial}, \emph{list}) : (\textsf{function}, \textsf{list}) $\rightarrow$ \textsf{function}\\
\end{center}
Parameters: 
\emph{polynomial} represents the polynomial the coefficients are taken from\\
\emph{list} represents the list of monomials to be taken\\

\noindent Description: \begin{itemize}

\item \textbf{subpoly} extracts the coefficients of a polynomial \emph{polynomial} and builds up a
   new polynomial out of those coefficients associated to monomial degrees figuring in
   the list \emph{list}. 
   If \emph{polynomial} represents a function that is not a polynomial, subpoly returns 0.
   If \emph{list} is a list that is end-elliptic, let be j the last value explicitely specified
   in the list. All coefficients of the polynomial associated to monomials greater or
   equal to j are taken.
\end{itemize}
\noindent Example 1: 
\begin{center}\begin{minipage}{14.8cm}\begin{Verbatim}[frame=single]
   > p = taylor(exp(x),5,0);
   > s = subpoly(p,[|1,3,5|]);
   > print(p);
   1 + x * (1 + x * (0.5 + x * (1 / 6 + x * (1 / 24 + x / 120))))
   > print(s);
   x * (1 + x^2 * (1 / 6 + x^2 / 120))
\end{Verbatim}
\end{minipage}\end{center}
\noindent Example 2: 
\begin{center}\begin{minipage}{14.8cm}\begin{Verbatim}[frame=single]
   > p = remez(atan(x),10,[-1,1]);
   > subpoly(p,[|1,3,5...|]);
   x * (0.999866329465927392192206568432088436991654470572188 + x^2 * ((-0.330304785504971132950658277728545438994810895546443) + x^2 * (0.180159294636895327241868940582645835027165398881204 + x * (0.184005739836594527600896857815217464340110063213703e-46 + x * ((-0.851563508341582145150897325769046842604973435036432e-1) + x * ((-0.204304173639774340598827812338387968056700475880779e-46) + x * (0.208451141756196733464162256100630144371647287176463e-1 + x * 0.788341690807029292048116493729741056214183302078353e-47)))))))
\end{Verbatim}
\end{minipage}\end{center}
\noindent Example 3: 
\begin{center}\begin{minipage}{14.8cm}\begin{Verbatim}[frame=single]
   > subpoly(exp(x),[|1,2,3|]);
   0
\end{Verbatim}
\end{minipage}\end{center}
See also: \textbf{roundcoefficients}, \textbf{taylor}, \textbf{remez}

\subsection{substitute}
\label{labsubstitute}
\noindent Name: \textbf{substitute}\\
replace the occurrences of the free variable in an expression.\\
\noindent Usage: 
\begin{center}
\textbf{substitute}(\emph{f},\emph{g}) : (\textsf{function}, \textsf{function}) $\rightarrow$ \textsf{function}\\
\textbf{substitute}(\emph{f},\emph{t}) : (\textsf{function}, \textsf{constant}) $\rightarrow$ \textsf{constant}\\
\end{center}
Parameters: 
\begin{itemize}
\item \emph{f} is a function.
\item \emph{g} is a function.
\item \emph{t} is a real number.
\end{itemize}
\noindent Description: \begin{itemize}

\item \\textbf{substitute}(\\emph{f}, \\emph{g}) produces the function $(f \\circ g) : x \\mapsto f(g(x))$.\n
\item \\textbf{substitute}(\\emph{f}, \\emph{t}) is the constant $f(t)$. Note that the constant is\n   represented by its expression until it has been evaluated (exactly the same\n   way as if you type the expression \\emph{f} replacing instances of the free variable \n   by \\emph{t}).\n
\item If \\emph{f} is stored in a variable \\emph{F}, the effect of the commands \\textbf{substitute}(\\emph{F},\\emph{g}) or \\textbf{substitute}(\\emph{F},\\emph{t}) is absolutely equivalent to \n   writing \\emph{F(g)} resp. \\emph{F(t)}.\n\end{itemize}
\noindent Example 1: 
\begin{center}\begin{minipage}{15cm}\begin{Verbatim}[frame=single]
\end{Verbatim}
\end{minipage}\end{center}
\noindent Example 2: 
\begin{center}\begin{minipage}{15cm}\begin{Verbatim}[frame=single]
\end{Verbatim}
\end{minipage}\end{center}

\subsection{sup}
\label{labsup}
\noindent Name: \textbf{sup}\\
gives the upper bound of an interval.\\

\noindent Usage: 
\begin{center}
\textbf{sup}(\emph{I}) : \textsf{range} $\rightarrow$ \textsf{constant}\\
\textbf{sup}(\emph{x}) : \textsf{constant} $\rightarrow$ \textsf{constant}\\
\end{center}
Parameters: 
\begin{itemize}
\item \emph{I} is an interval.
\item \emph{x} is a real number.
\end{itemize}
\noindent Description: \begin{itemize}

\item Returns the upper bound of the interval \emph{I}. Each bound of an interval has its 
   own precision, so this command is exact, even if the current precision is too 
   small to represent the bound.

\item When called on a real number \emph{x}, \textbf{sup} considers it as an interval formed
   of a single point: [x, x]. In other words, \textbf{sup} behaves like the identity.
\end{itemize}
\noindent Example 1: 
\begin{center}\begin{minipage}{15cm}\begin{Verbatim}[frame=single]
> sup([1;3]);
3
> sup(5);
5
\end{Verbatim}
\end{minipage}\end{center}
\noindent Example 2: 
\begin{center}\begin{minipage}{15cm}\begin{Verbatim}[frame=single]
> display=binary!;
> I=[0; 0.111110000011111_2];
> sup(I);
1.11110000011111_2 * 2^(-1)
> prec=12!;
> sup(I);
1.11110000011111_2 * 2^(-1)
\end{Verbatim}
\end{minipage}\end{center}
See also: \textbf{inf} (\ref{labinf}), \textbf{mid} (\ref{labmid})

\subsection{tail}
\label{labtail}
\noindent Name: \textbf{tail}\\
gives the tail of a list.\\
\noindent Usage: 
\begin{center}
\textbf{tail}(\emph{L}) : \textsf{list} $\rightarrow$ \textsf{list}\\
\end{center}
Parameters: 
\begin{itemize}
\item \emph{L} is a list.
\end{itemize}
\noindent Description: \begin{itemize}

\item \textbf{tail}(\emph{L}) returns the list \emph{L} without its first element.

\item If \emph{L} is empty, the command will fail with an error.

\item \textbf{tail} can also be used with end-elliptic lists. In this case, the result of
   \textbf{tail} is also an end-elliptic list.
\end{itemize}
\noindent Example 1: 
\begin{center}\begin{minipage}{15cm}\begin{Verbatim}[frame=single]
> tail([|1,2,3|]);
[|2, 3|]
> tail([|1,2...|]);
[|2...|]
\end{Verbatim}
\end{minipage}\end{center}
See also: \textbf{head} (\ref{labhead})

\subsection{tanh}
\label{labtanh}
\noindent Name: \textbf{tanh}\\
the hyperbolic tangent function.\\
\noindent Description: \begin{itemize}

\item \textbf{tanh} is the hyperbolic tangent function, defined by $\tanh(x) = \sinh(x)/\cosh(x)$.

\item It is defined for every real number x.
\end{itemize}
See also: \textbf{atanh} (\ref{labatanh}), \textbf{cosh} (\ref{labcosh}), \textbf{sinh} (\ref{labsinh})

\subsection{tan}
\label{labtan}
\noindent Name: \textbf{tan}\\
\phantom{aaa}the tangent function.\\[0.2cm]
\noindent Library names:\\
\verb|   sollya_obj_t sollya_lib_tan(sollya_obj_t)|\\
\verb|   sollya_obj_t sollya_lib_build_function_tan(sollya_obj_t)|\\
\verb|   #define SOLLYA_TAN(x) sollya_lib_build_function_tan(x)|\\[0.2cm]
\noindent Description: \begin{itemize}

\item \textbf{tan} is the tangent function, defined by $\tan(x) = \sin(x)/\cos(x)$.

\item It is defined for every real number $x$ that is not of the form $n\pi + \pi/2$ where $n$ is an integer.
\end{itemize}
See also: \textbf{atan} (\ref{labatan}), \textbf{cos} (\ref{labcos}), \textbf{sin} (\ref{labsin})

\subsection{taylorform}
\label{labtaylorform}
\noindent Name: \textbf{taylorform}\\
\phantom{aaa}computes a rigorous polynomial approximation (polynomial, interval error bound) for a function, based on Taylor expansions.\\[0.2cm]
\noindent Library names:\\
\verb|   sollya_obj_t sollya_lib_taylorform(sollya_obj_t, sollya_obj_t,|\\
\verb|                                      sollya_obj_t, ...)|\\
\verb|   sollya_obj_t sollya_lib_v_taylorform(sollya_obj_t, sollya_obj_t,|\\
\verb|                                        sollya_obj_t, va_list)|\\[0.2cm]
\noindent Usage: 
\begin{center}
\textbf{taylorform}(\emph{f}, \emph{n}, \emph{$x_0$}, \emph{I}, \emph{errorType}) : (\textsf{function}, \textsf{integer}, \textsf{constant}, \textsf{range}, \textsf{absolute$|$relative}) $\rightarrow$ \textsf{list}\\
\textbf{taylorform}(\emph{f}, \emph{n}, \emph{$x_0$}, \emph{I}, \emph{errorType}) : (\textsf{function}, \textsf{integer}, \textsf{range}, \textsf{range}, \textsf{absolute$|$relative}) $\rightarrow$ \textsf{list}\\
\textbf{taylorform}(\emph{f}, \emph{n}, \emph{$x_0$}, \emph{errorType}) : (\textsf{function}, \textsf{integer}, \textsf{constant}, \textsf{absolute$|$relative}) $\rightarrow$ \textsf{list}\\
\textbf{taylorform}(\emph{f}, \emph{n}, \emph{$x_0$}, \emph{errorType}) : (\textsf{function}, \textsf{integer}, \textsf{range}, \textsf{absolute$|$relative}) $\rightarrow$ \textsf{list}\\
\end{center}
Parameters: 
\begin{itemize}
\item \emph{f} is the function to be approximated.
\item \emph{n} is the degree of the polynomial that must approximate \emph{f}.
\item \emph{$x_0$} is the point (it can be a real number or an interval) where the Taylor exansion of the function is to be considered.
\item \emph{I} is the interval over which the function is to be approximated. If this parameter is omitted, the behavior is changed (see detailed description below).
\item \emph{errorType} (optional) is the type of error to be considered. See the detailed description below. Default is \textbf{absolute}.
\end{itemize}
\noindent Description: \begin{itemize}

\item \textbf{WARNING:} \textbf{taylorform} is a certified command, not difficult to use but not
   completely straightforward to use either. In order to be sure to use it
   correctly, the reader is invited to carefully read this documentation
   entirely.

\item \textbf{taylorform} computes an approximation polynomial and an interval error
   bound for function $f$. 
   More precisely, it returns a list
   $L = \left[p, \textrm{coeffErrors},\Delta \right]$ where:
   \begin{itemize}
   \item $p$ is an approximation polynomial of degree $n$ such that $p(x-x_0)$ is
   roughly speaking a numerical Taylor expansion of $f$ at the point $x_0$.
   \item coeffsErrors is a list of $n+1$ intervals. Each interval coeffsErrors[$i$]
   contains an enclosure of all the errors accumulated when computing the $i$-th
   coefficient of $p$.
   \item $\Delta$ is an interval that provides a bound for the approximation error
   between $p$ and $f$. Its significance depends on the \emph{errorType} considered.
   \end{itemize}

\item The polynomial $p$ and the bound  $\Delta$ are obtained using Taylor Models
   principles.

\item Please note that $x_0$ can be an interval. In general, it is meant to be a
   small interval approximating a non representable value. If $x_0$ is given as a
   constant expression, it is first numerically evaluated (leading to a faithful
   rounding $\tilde{x_0}$ at precision \textbf{prec}), and it is then replaced by the (exactly
   representable) point-interval $[\tilde{x_0},\,\tilde{x_0}]$. In particular, it is not
   the same to call \textbf{taylorform} with $x_0 = \textbf{pi}$ and with $x_0 = [\textbf{pi}]$, for instance.
   In general, if the point around which one desires to compute the polynomial
   is not exactly representable, one should preferably use a small interval
   for~$x_0$.

\item More formally, the mathematical property ensured by the algorithm may be
   stated as follows. For all $\csi_0$ in $x_0$, there exist (small) values
   $\varepsilon_i \in \textrm{coeffsErrors}[i]$ such that:
   \\
   If \emph{errorType} is \textbf{absolute}, $\forall x \in I, \exists \delta \in \Delta,\,$
   $f(x)-p(x-\csi_0) = \sum\limits_{i=0}^{n} \varepsilon_i\, (x-\csi_0)^i + \delta$.
   \\
   If \emph{errorType} is \textbf{relative}, $\forall x \in I, \exists \delta \in \Delta,\,$
   $f(x)-p(x-\csi_0) = \sum\limits_{i=0}^{n} \varepsilon_i\, (x-\csi_0)^i + \delta\,(x-\csi_0)^{n+1}$.

\item It is also possible to use a large interval for $x_0$, though it is not
   obvious to give an intuitive sense to the result of \textbf{taylorform} in that case.
   A particular case that might be interesting is when $x_0 = I$ in relative mode.
   In that case, denoting by $p_i$ the coefficient of $p$ of order $i$, the interval
   $p_i + \textrm{coeffsError}[i]$ gives an enclosure of $f^{(i)}(I)/i!$.
   However, the command \textbf{autodiff} is more convenient for computing such
   enclosures.

\item When the interval $I$ is not given, the approximated Taylor polynomial is
   computed but no remainder is produced. In that case the returned list
   is $L = \left[p, \textrm{coeffErrors}\right]$.

\item The relative case is especially useful when functions with removable
   singularities are considered. In such a case, this routine is able to compute
   a finite remainder bound, provided that the expansion point given is the
   problematic removable singularity point.

\item The algorithm does not guarantee that by increasing the degree of the
   approximation, the remainder bound will become smaller. Moreover, it may 
   even become larger due to the dependecy phenomenon present with interval
   arithmetic. In order to reduce this phenomenon, a possible solution is to
   split the definition domain $I$ into several smaller intervals. 

\item The command \textbf{taylor} also computes a Taylor polynomial of a function. However
   it does not provide a bound on the remainder. Besides, \textbf{taylor} is a somehow
   symbolic command: each coefficient of the Taylor polynomial is computed
   exactly and returned as an expression tree exactly equal to theoretical
   value. It is henceforth much more inefficient than \textbf{taylorform} and \textbf{taylorform}
   should be prefered if only numercial (yet safe) computations are required.
   The same difference exists between commands \textbf{diff} and \textbf{autodiff}.
\end{itemize}
\noindent Example 1: 
\begin{center}\begin{minipage}{15cm}\begin{Verbatim}[frame=single]
> TL=taylorform(sin(x)/x, 10, 0, [-1,1], relative);
> p=TL[0];
> Delta=TL[2];
> errors=TL[1];
> for epsi in errors do epsi;
[0;0]
[0;0]
[0;5.3455294201843912922810729343029637576303937602101e-51]
[0;0]
[-3.3409558876152445576756705839393523485189961001313e-52;3.34095588761524455767
56705839393523485189961001313e-52]
[0;0]
[-1.04404871487976392427364705748104760891218628129103e-53;1.0440487148797639242
7364705748104760891218628129103e-53]
[0;0]
[-1.63132611699963113167757352731413688892529106451724e-55;1.6313261169996311316
7757352731413688892529106451724e-55]
[0;0]
[-1.91171029335894273243465647732125416670932546623114e-57;1.9117102933589427324
3465647732125416670932546623114e-57]
> p; Delta;
1 + x^2 * (-0.16666666666666666666666666666666666666666666666667 + x^2 * (8.3333
333333333333333333333333333333333333333333333e-3 + x^2 * (-1.9841269841269841269
8412698412698412698412698412698e-4 + x^2 * (2.7557319223985890652557319223985890
6525573192239859e-6 + x^2 * (-2.505210838544171877505210838544171877505210838544
19e-8)))))
[-1.6135797443886066084999806203254010793747502812764e-10;1.61357974438860660849
99806203254010793747502812764e-10]
\end{Verbatim}
\end{minipage}\end{center}
\noindent Example 2: 
\begin{center}\begin{minipage}{15cm}\begin{Verbatim}[frame=single]
> TL=taylorform(exp(x), 10, 0, [-1,1], absolute);
> p=TL[0];
> Delta=TL[2];
> p; Delta;
1 + x * (1 + x * (0.5 + x * (0.1666666666666666666666666666666666666666666666666
7 + x * (4.1666666666666666666666666666666666666666666666667e-2 + x * (8.3333333
333333333333333333333333333333333333333333e-3 + x * (1.3888888888888888888888888
8888888888888888888888889e-3 + x * (1.984126984126984126984126984126984126984126
98412698e-4 + x * (2.4801587301587301587301587301587301587301587301587e-5 + x * 
(2.75573192239858906525573192239858906525573192239859e-6 + x * 2.755731922398589
0652557319223985890652557319223986e-7)))))))))
[-2.31142719641187619441242534182684745832539555102969e-8;2.73126607556424744202
06278018039434042553645532164e-8]
\end{Verbatim}
\end{minipage}\end{center}
\noindent Example 3: 
\begin{center}\begin{minipage}{15cm}\begin{Verbatim}[frame=single]
> TL1 = taylorform(exp(x), 10, log2(10), [-1,1], absolute);
> TL2 = taylorform(exp(x), 10, [log2(10)], [-1,1], absolute);
> TL1==TL2;
false
\end{Verbatim}
\end{minipage}\end{center}
\noindent Example 4: 
\begin{center}\begin{minipage}{15cm}\begin{Verbatim}[frame=single]
> TL1 = taylorform(exp(x), 3, 0, [0,1], relative);
> TL2 = taylorform(exp(x), 3, 0, relative);
> TL1[0]==TL2[0];
true
> TL1[1]==TL2[1];
true
> length(TL1);
3
> length(TL2);
2
\end{Verbatim}
\end{minipage}\end{center}
\noindent Example 5: 
\begin{center}\begin{minipage}{15cm}\begin{Verbatim}[frame=single]
> f = exp(cos(x)); x0 = 0;
> TL = taylorform(f, 3, x0);
> T1 = TL[0];
> T2 = taylor(f, 3, x0);
> print(coeff(T1, 2));
-1.35914091422952261768014373567633124887862354684999
> print(coeff(T2, 2));
0.5 * exp(1)
\end{Verbatim}
\end{minipage}\end{center}
See also: \textbf{diff} (\ref{labdiff}), \textbf{autodiff} (\ref{labautodiff}), \textbf{taylor} (\ref{labtaylor}), \textbf{remez} (\ref{labremez})

\subsection{taylorrecursions}
\label{labtaylorrecursions}
\noindent Name: \textbf{taylorrecursions}\\
\phantom{aaa}controls the number of recursion steps when applying Taylor's rule.\\[0.2cm]
\noindent Library names:\\
\verb|   void sollya_lib_set_taylorrecursions_and_print(sollya_obj_t)|\\
\verb|   void sollya_lib_set_taylorrecursions(sollya_obj_t)|\\
\verb|   sollya_obj_t sollya_lib_get_taylorrecursions()|\\[0.2cm]
\noindent Usage: 
\begin{center}
\textbf{taylorrecursions} = \emph{n} : \textsf{integer} $\rightarrow$ \textsf{void}\\
\textbf{taylorrecursions} = \emph{n} ! : \textsf{integer} $\rightarrow$ \textsf{void}\\
\textbf{taylorrecursions} : \textsf{integer}\\
\end{center}
Parameters: 
\begin{itemize}
\item \emph{n} represents the number of recursions
\end{itemize}
\noindent Description: \begin{itemize}

\item \textbf{taylorrecursions} is a global variable. Its value represents the number of steps
   of recursion that are used when applying Taylor's rule. This rule is applied
   by the interval evaluator present in the core of \sollya (and particularly
   visible in commands like \textbf{infnorm}).

\item To improve the quality of an interval evaluation of a function $f$, in 
   particular when there are problems of decorrelation), the evaluator of \sollya
   uses Taylor's rule:  $f([a,b]) \subseteq f(m) + [a-m,\,b-m] \cdot f'([a,\,b])$ where $m=\frac{a+b}{2}$.
   This rule can be applied recursively.
   The number of step in this recursion process is controlled by \textbf{taylorrecursions}.

\item Setting \textbf{taylorrecursions} to 0 makes \sollya use this rule only once;
   setting it to 1 makes \sollya use the rule twice, and so on.
   In particular: the rule is always applied at least once.
\end{itemize}
\noindent Example 1: 
\begin{center}\begin{minipage}{15cm}\begin{Verbatim}[frame=single]
> f=exp(x);
> p=remez(f,3,[0;1]);
> taylorrecursions=0;
The number of recursions for Taylor evaluation has been set to 0.
> evaluate(f-p, [0;1]);
[-0.46839364816268368775174657814112460243249079671039;0.46947781754646820647293
019728402934746974652584671]
> taylorrecursions=1;
The number of recursions for Taylor evaluation has been set to 1.
> evaluate(f-p, [0;1]);
[-0.138131114954063839905475752120786856031651747712954;0.1392152843378483586266
5937126369160106890747684927]
\end{Verbatim}
\end{minipage}\end{center}
See also: \textbf{hopitalrecursions} (\ref{labhopitalrecursions}), \textbf{evaluate} (\ref{labevaluate}), \textbf{infnorm} (\ref{labinfnorm})

\subsection{ taylor }
\noindent Name: \textbf{taylor}\\
computes a Taylor expansion of a function in a point\\

\noindent Usage: 
\begin{center}
\textbf{taylor}(\emph{function}, \emph{degree}, \emph{point}) : (\textsf{function}, \textsf{integer}, \textsf{constant}) $\rightarrow$ \textsf{function}\\
\end{center}
Parameters: 
\emph{function} represents the function to be expanded\\
\emph{degree} represents the degree of the expansion to be delivered\\
\emph{point} represents the point in which the function is to be developped\\

\noindent Description: \begin{itemize}

\item The command \textbf{taylor} returns an expression that is a Taylor expansion
   of function \emph{function} in point \emph{point} having the degree \emph{degree}.
   Let $f$ be the function \emph{function}, $t$ be the point \emph{point} and
   $n$ be the degree \emph{degree}. Then, \textbf{taylor}(\emph{function},\emph{degree},\emph{point}) 
   evaluates to an expression mathematically equal to 
   $$\sum\limits_{i=0}^n \frac{f^{(i)}}{i!} \left(x - t \right)^i$$
   Remark that \textbf{taylor} evaluates to $0$ if the degree \emph{degree} is negative.
\end{itemize}
\noindent Example 1: 
\begin{center}\begin{minipage}{14.8cm}\begin{Verbatim}[frame=single]
   > print(taylor(exp(x),5,0));
   1 + x * (1 + x * (0.5 + x * (1 / 6 + x * (1 / 24 + x / 120))))
\end{Verbatim}
\end{minipage}\end{center}
\noindent Example 2: 
\begin{center}\begin{minipage}{14.8cm}\begin{Verbatim}[frame=single]
   > print(taylor(asin(x),7,0));
   x * (1 + x^2 * (1 / 6 + x^2 * (9 / 120 + x^2 * 225 / 5040)))
\end{Verbatim}
\end{minipage}\end{center}
\noindent Example 3: 
\begin{center}\begin{minipage}{14.8cm}\begin{Verbatim}[frame=single]
   > print(taylor(erf(x),6,0));
   x * (1 / sqrt((pi) / 4) + x^2 * ((sqrt((pi) / 4) * 4 / (pi) * (-2)) / 6 + x^2 * (sqrt((pi) / 4) * 4 / (pi) * 12) / 120))
\end{Verbatim}
\end{minipage}\end{center}
See also: \textbf{remez}

\subsection{time}
\label{labtime}
\noindent Name: \textbf{time}\\
procedure for timing \sollya code.\\
\noindent Usage: 
\begin{center}
\textbf{time}(\emph{code}) : \textsf{code} $\rightarrow$ \textsf{constant}\\
\end{center}
Parameters: 
\begin{itemize}
\item \emph{code} is the code to be timed.
\end{itemize}
\noindent Description: \begin{itemize}

\item \textbf{time} permits timing a \sollya instruction, resp. a begin-end block
   of \sollya instructions. The timing value, measured in seconds, is returned
   as a \sollya constant (and not merely displayed as for \textbf{timing}). This 
   permits performing computations of the timing measurement value inside \sollya.

\item The extended \textbf{nop} command permits executing a defined number of
   useless instructions. Taking the ratio of the time needed to execute a
   certain \sollya instruction and the time for executing a \textbf{nop}
   therefore gives a way to abstract from the speed of a particular 
   machine when evaluating an algorithm's performance.
\end{itemize}
\noindent Example 1: 
\begin{center}\begin{minipage}{15cm}\begin{Verbatim}[frame=single]
> t = time(p=remez(sin(x),10,[-1;1]));
> write(t,"s were spent computing p = ",p,"\n");
0.22402199999999999999351907309375064869527705013752s were spent computing p = -
3.3426550293345171908513995127407122194691200059639e-17 + x * (0.999999999736283
59955372011464713121003442988167693 + x * (7.88027518773027866844993437990477324
95568873819693e-16 + x * (-0.166666661386013037032912982196741385680498698107285
 + x * (-5.3734444911159112186289355138557504839692987221233e-15 + x * (8.333303
7186548537651002133031675072810009327877148e-3 + x * (1.337972213892188158841123
41005509831429347230871284e-14 + x * (-1.983448630182774164932681551541589244220
04290239026e-4 + x * (-1.3789116451286674170531616441916183417598709732816e-14 +
 x * (2.6876259495430304684251822024896210963401672262005e-6 + x * 5.02823783500
10211058128384123578805586173782863605e-15)))))))))
\end{Verbatim}
\end{minipage}\end{center}
\noindent Example 2: 
\begin{center}\begin{minipage}{15cm}\begin{Verbatim}[frame=single]
> write(time({ p=remez(sin(x),10,[-1;1]); write("The error is 2^(", log2(dirtyin
fnorm(p-sin(x),[-1;1])), ")\n"); }), "were spent\n");
The error is 2^(log2(2.39602467695631727848641768186659313738474584992648e-11))
0.377647000000000000047295500849031668622046709060669were spent
\end{Verbatim}
\end{minipage}\end{center}
\noindent Example 3: 
\begin{center}\begin{minipage}{15cm}\begin{Verbatim}[frame=single]
> t = time(bashexecute("sleep 10"));
> write(~(t-10),"s of execution overhead.\n");
3.85399999999999978705922387689497554674744606018066e-3s of execution overhead.
\end{Verbatim}
\end{minipage}\end{center}
\noindent Example 4: 
\begin{center}\begin{minipage}{15cm}\begin{Verbatim}[frame=single]
> ratio := time(p=remez(sin(x),10,[-1;1]))/time(nop(10));
> write("This ratio = ", ratio, " should somehow be independent of the type of m
achine.\n");
This ratio = 6.1247296112247880732802836417531280232837566456005 should somehow 
be independent of the type of machine.
\end{Verbatim}
\end{minipage}\end{center}
See also: \textbf{timing} (\ref{labtiming}), \textbf{nop} (\ref{labnop})

\subsection{timing}
\label{labtiming}
\noindent Name: \textbf{timing}\\
\phantom{aaa}global variable controlling timing measures in \sollya.\\[0.2cm]
\noindent Library names:\\
\verb|   void sollya_lib_set_timing_and_print(sollya_obj_t)|\\
\verb|   void sollya_lib_set_timing(sollya_obj_t)|\\
\verb|   sollya_obj_t sollya_lib_get_timing()|\\[0.2cm]
\noindent Usage: 
\begin{center}
\textbf{timing} = \emph{activation value} : \textsf{on$|$off} $\rightarrow$ \textsf{void}\\
\textbf{timing} = \emph{activation value} ! : \textsf{on$|$off} $\rightarrow$ \textsf{void}\\
\textbf{timing} : \textsf{on$|$off}\\
\end{center}
Parameters: 
\begin{itemize}
\item \emph{activation value} controls if timing should be performed or not
\end{itemize}
\noindent Description: \begin{itemize}

\item \textbf{timing} is a global variable. When its value is \textbf{on}, the time spent in each 
   command is measured and displayed (for \textbf{verbosity} levels higher than 1).
\end{itemize}
\noindent Example 1: 
\begin{center}\begin{minipage}{15cm}\begin{Verbatim}[frame=single]
> verbosity=1!;
> timing=on;
Timing has been activated.
> p=remez(sin(x),10,[-1;1]);
Information: Remez: computing the quality of approximation spent 4 ms
Information: Remez: computing the quality of approximation spent 4 ms
Information: Remez: computing the quality of approximation spent 4 ms
Information: computing a minimax approximation spent 27 ms
Information: assignment spent 31 ms
Information: full execution of the last parse chunk spent 36 ms
\end{Verbatim}
\end{minipage}\end{center}
See also: \textbf{on} (\ref{labon}), \textbf{off} (\ref{laboff}), \textbf{time} (\ref{labtime})

\subsection{tripledouble}
\label{labtripledouble}
\noindent Names: \textbf{tripledouble}, \textbf{TD}\\
represents a number as the sum of three IEEE doubles.\\
\noindent Description: \begin{itemize}

\item \textbf{tripledouble} is both a function and a constant.

\item As a function, it rounds its argument to the nearest number that can be written
   as the sum of three double precision numbers.

\item The algorithm used to compute \textbf{tripledouble}($x$) is the following: let $x_h$ = \textbf{double}($x$),
   let $x_m$ = \textbf{doubledouble}($x-x_h$) and let $x_l$ = \textbf{doubledouble}($x-x_h-x_m$). 
   Return the number $x_h+x_m+x_l$. Note that if the
   current precision is not sufficient to represent exactly $x_h+x_m+x_l$, a rounding will
   occur and the result of \textbf{tripledouble}(x) will be useless.

\item As a constant, it symbolizes the triple-double precision format. It is used in 
   contexts when a precision format is necessary, e.g. in the commands 
   \textbf{roundcoefficients} and \textbf{implementpoly}.
   See the corresponding help pages for examples.
\end{itemize}
\noindent Example 1: 
\begin{center}\begin{minipage}{15cm}\begin{Verbatim}[frame=single]
> verbosity=1!;
> a = 1+ 2^(-55)+2^(-115);
> TD(a);
1.00000000000000002775557561562891353466491600711096
> prec=110!;
> TD(a);
Warning: double rounding occurred on invoking the triple-double rounding operato
r.
Try to increase the working precision.
1.000000000000000027755575615628913
\end{Verbatim}
\end{minipage}\end{center}
See also: \textbf{double} (\ref{labdouble}), \textbf{doubleextended} (\ref{labdoubleextended}), \textbf{doubledouble} (\ref{labdoubledouble}), \textbf{roundcoefficients} (\ref{labroundcoefficients}), \textbf{implementpoly} (\ref{labimplementpoly})

\subsection{true}
\label{labtrue}
\noindent Name: \textbf{true}\\
\phantom{aaa}the boolean value representing the truth.\\[0.2cm]
\noindent Library names:\\
\verb|   sollya_obj_t sollya_lib_true()|\\
\verb|   int sollya_lib_is_true(sollya_obj_t)|\\[0.2cm]
\noindent Description: \begin{itemize}

\item \textbf{true} is the usual boolean value.
\end{itemize}
\noindent Example 1: 
\begin{center}\begin{minipage}{15cm}\begin{Verbatim}[frame=single,commandchars=\\\|\~]
> true && false;
false
> 2>1;
true
\end{Verbatim}
\end{minipage}\end{center}
See also: \textbf{false} (\ref{labfalse}), \textbf{$\&\&$} (\ref{laband}), \textbf{$||$} (\ref{labor})

\subsection{ var }
\noindent Name: \textbf{var}\\
declaration of a local variable in a scope\\

\noindent Usage: 
\begin{center}
\textbf{var} \emph{identifier1}, \emph{identifier2},... , \emph{identifiern} : \textsf{void}\\
\end{center}
Parameters: 
\emph{identifier1}, \emph{identifier2},... , \emph{identifiern} represent variable identifiers\\

\noindent Description: \begin{itemize}

\item The keyword \textbf{var} allows for the declaration of local variables
   \emph{identifier1} through \emph{identifiern} in a begin-end-block ({}-block).
   Once declared as a local variable, an identifier will shadow
   identifiers declared in higher scopes and undeclared identifiers
   available at top-level.
   Variable declarations using \textbf{var} are only possible in the
   beginning of a begin-end-block. Several \textbf{var} statements can be
   given. Once another statement is given in a begin-end-block, no more
   \textbf{var} statements can be given.
   Variables declared by \textbf{var} statements are dereferenced as \textbf{error}
   until they are assigned a value. 
\end{itemize}
\noindent Example 1: 
\begin{center}\begin{minipage}{14.8cm}\begin{Verbatim}[frame=single]
   > exp(x); 
   exp(x)
   > a = 3; 
   > {var a, b; a=5; b=3; {var a; var b; b = true; a = 1; a; b;}; a; b; }; 
   1
   true
   5
   3
   > a;
   3
\end{Verbatim}
\end{minipage}\end{center}
See also: \textbf{error}

\subsection{verbosity}
\label{labverbosity}
\noindent Name: \textbf{verbosity}\\
global variable controlling the quantity of information displayed by commands.\\

\noindent Description: \begin{itemize}

\item \textbf{verbosity} accepts any integer value. At level 0, commands do not display anything
   on standard out. Note that very critical information may however be displayed on
   standard err.

\item Default level is 1. It displays important informations such as warnings when 
   roundings happen.

\item For higher levels more informations are displayed depending on the command.
\end{itemize}
\noindent Example 1: 
\begin{center}\begin{minipage}{15cm}\begin{Verbatim}[frame=single]
> verbosity=0!;
> 1.2+"toto";
error
> verbosity=1!;
> 1.2+"toto";
Warning: Rounding occurred when converting the constant "1.2" to floating-point 
with 165 bits.
If safe computation is needed, try to increase the precision.
Warning: at least one of the given expressions or a subexpression is not correct
ly typed
or its evaluation has failed because of some error on a side-effect.
error
> verbosity=2!;
> 1.2+"toto";
Warning: Rounding occurred when converting the constant "1.2" to floating-point 
with 165 bits.
If safe computation is needed, try to increase the precision.
Warning: at least one of the given expressions or a subexpression is not correct
ly typed
or its evaluation has failed because of some error on a side-effect.
Information: the expression or a partial evaluation of it has been the following
:
(0.119999999999999999999999999999999999999999999999999e1) + ("toto")
error
\end{Verbatim}
\end{minipage}\end{center}

\subsection{void}
\label{labvoid}
\noindent Name: \textbf{void}\\
the functional result of a side-effect or empty argument resp. the corresponding type\\
\noindent Usage: 
\begin{center}
\textbf{void} : \textsf{void} $|$ \textsf{type type}\\
\end{center}
\noindent Description: \begin{itemize}

\item The variable \textbf{void} represents the functional result of a
   side-effect or an empty argument.  It is used only in combination with
   the applications of procedures or identifiers bound through
   \textbf{externalproc} to external procedures.
    
   The \textbf{void} result produced by a procedure or an external procedure
   is not printed at the prompt. However, it is possible to print it out
   in a print statement or in complex data types such as lists.
    
   The \textbf{void} argument is implicit when giving no argument to a
   procedure or an external procedure when applied. It can nevertheless be given
   explicitly.  For example, suppose that foo is a procedure or an
   external procedure with a void argument. Then foo() and foo(void) are
   correct calls to foo. This even applies to functions with having an
   arbitrary number of arguments. In this case, any implicit or explicit \textbf{void}
   as the only parameter to a call of such a procedure gets converted into 
   an empty list of arguments.

\item \textbf{void} is used also as a type identifier for
   \textbf{externalproc}. Typically, an external procedure taking \textbf{void} as an
   argument or returning \textbf{void} is bound with a signature \textbf{void} $->$
   some type or some type $->$ \textbf{void}. See \textbf{externalproc} for more
   details.
\end{itemize}
\noindent Example 1: 
\begin{center}\begin{minipage}{15cm}\begin{Verbatim}[frame=single]
> print(void);
void
> void;
\end{Verbatim}
\end{minipage}\end{center}
\noindent Example 2: 
\begin{center}\begin{minipage}{15cm}\begin{Verbatim}[frame=single]
> hey = proc() { print("Hello world."); };
> hey;
proc()
begin
print("Hello world.");
return void;
end
> hey();
Hello world.
> hey(void);
Hello world.
> print(hey());
Hello world.
void
\end{Verbatim}
\end{minipage}\end{center}
\noindent Example 3: 
\begin{center}\begin{minipage}{15cm}\begin{Verbatim}[frame=single]
> bashexecute("gcc -fPIC -Wall -c externalprocvoidexample.c");
> bashexecute("gcc -fPIC -shared -o externalprocvoidexample externalprocvoidexam
ple.o");
> externalproc(foo, "./externalprocvoidexample", void -> void);
> foo;
foo(void) -> void
> foo();
Hello from the external world.
> foo(void);
Hello from the external world.
> print(foo());
Hello from the external world.
void
\end{Verbatim}
\end{minipage}\end{center}
\noindent Example 4: 
\begin{center}\begin{minipage}{15cm}\begin{Verbatim}[frame=single]
> procedure blub(l = ...) { print("Argument list:", l); };
> blub(1);
Argument list: [|1|]
> blub();
Argument list: [| |]
> blub(void); 
Argument list: [|void|]
\end{Verbatim}
\end{minipage}\end{center}
See also: \textbf{error} (\ref{laberror}), \textbf{proc} (\ref{labproc}), \textbf{externalproc} (\ref{labexternalproc})

\subsection{worstcase}
\label{labworstcase}
\noindent Name: \textbf{worstcase}\\
\phantom{aaa}searches for hard-to-round cases of a function\\[0.2cm]
\noindent Library names:\\
\verb|   void sollya_lib_worstcase(sollya_obj_t, sollya_obj_t, sollya_obj_t, sollya_obj_t, sollya_obj_t, ...)|\\
\verb|   void sollya_lib_v_worstcase(sollya_obj_t, sollya_obj_t, sollya_obj_t, sollya_obj_t, sollya_obj_t, va_list)|\\[0.2cm]
\noindent Usage: 
\begin{center}
\textbf{worstcase}(\emph{function}, \emph{preimage precision}, \emph{preimage exponent range}, \emph{image precision}, \emph{error bound}) : (\textsf{function}, \textsf{integer}, \textsf{range}, \textsf{integer}, \textsf{constant}) $\rightarrow$ \textsf{void}\\
\textbf{worstcase}(\emph{function}, \emph{preimage precision}, \emph{preimage exponent range}, \emph{image precision}, \emph{error bound}, \emph{filename}) : (\textsf{function}, \textsf{integer}, \textsf{range}, \textsf{integer}, \textsf{constant}, \textsf{string}) $\rightarrow$ \textsf{void}\\
\end{center}
Parameters: 
\begin{itemize}
\item \emph{function} represents the function to be considered
\item \emph{preimage precision} represents the precision of the preimages
\item \emph{preimage exponent range} represents the exponents in the preimage format
\item \emph{image precision} represents the precision of the format the images are to be rounded to
\item \emph{error bound} represents the upper bound for the search w.r.t. the relative rounding error
\item \emph{filename} represents a character sequence containing a filename
\end{itemize}
\noindent Description: \begin{itemize}

\item The \textbf{worstcase} command is deprecated. It searches for hard-to-round cases of
   a function. The command \textbf{searchgal} has a comparable functionality.
\end{itemize}
\noindent Example 1: 
\begin{center}\begin{minipage}{15cm}\begin{Verbatim}[frame=single]
> worstcase(exp(x),24,[1,2],24,1b-26);
prec = 165
x = 1.99999988079071044921875        f(x) = 7.3890552520751953125        eps = 4
.5998601423446695596184695493764120138001954979037e-9 = 2^(-27.695763) 
x = 2        f(x) = 7.38905620574951171875        eps = 1.4456360874967301812222
8379395533417878125150587072e-8 = 2^(-26.043720) 

\end{Verbatim}
\end{minipage}\end{center}
See also: \textbf{round} (\ref{labround}), \textbf{searchgal} (\ref{labsearchgal}), \textbf{evaluate} (\ref{labevaluate})

\subsection{write}
\label{labwrite}
\noindent Name: \textbf{write}\\
prints an expression without separators\\

\noindent Usage: 
\begin{center}
\textbf{write}(\emph{expr1},...,\emph{exprn}) : (\textsf{any type},..., \textsf{any type}) $\rightarrow$ \textsf{void}\\
\textbf{write}(\emph{expr1},...,\emph{exprn}) $>$ \emph{filename} : (\textsf{any type},..., \textsf{any type}, \textsf{string}) $\rightarrow$ \textsf{void}\\
\textbf{write}(\emph{expr1},...,\emph{exprn}) $>>$ \emph{filename} : (\textsf{any type},...,\textsf{any type}, \textsf{string}) $\rightarrow$ \textsf{void}\\
\end{center}
Parameters: 
\begin{itemize}
\item \emph{expr} represents an expression
\item \emph{filename} represents a character sequence indicating a file name
\end{itemize}
\noindent Description: \begin{itemize}

\item \textbf{write}(\emph{expr1},...,\emph{exprn}) prints the expressions \emph{expr1} through
   \emph{exprn}. The character sequences corresponding to the expressions are
   concatenated without any separator. No newline is displayed at the
   end.  In contrast to \textbf{print}, \textbf{write} expects the user to give all
   separators and newlines explicitely.
    
   If a second argument \emph{filename} is given after a single "$>$", the
   displaying is not output on the standard output of Sollya but if in
   the file \emph{filename} that get newly created or overwritten. If a double
    "$>>$" is given, the output will be appended to the file \emph{filename}.
    
   The global variables \textbf{display}, \textbf{midpointmode} and \textbf{fullparentheses} have
   some influence on the formatting of the output (see \textbf{display},
   \textbf{midpointmode} and \textbf{fullparentheses}).
    
   Remark that if one of the expressions \emph{expri} given in argument is of
   type \textsf{string}, the character sequence \emph{expri} evaluates to is
   displayed. However, if \emph{expri} is of type \textsf{list} and this list
   contains a variable of type \textsf{string}, the expression for the list
   is displayed, i.e.  all character sequences get displayed surrounded
   by quotes ('"'). Nevertheless, escape sequences used upon defining
   character sequences are interpreted immediately.
\end{itemize}
\noindent Example 1: 
\begin{center}\begin{minipage}{15cm}\begin{Verbatim}[frame=single]
> write(x + 2 + exp(sin(x))); 
> write("Hello\n");
x + 2 + exp(sin(x))Hello
> write("Hello","world\n");
Helloworld
> write("Hello","you", 4 + 3, "other persons.\n");
Helloyou7other persons.
\end{Verbatim}
\end{minipage}\end{center}
\noindent Example 2: 
\begin{center}\begin{minipage}{15cm}\begin{Verbatim}[frame=single]
> write("Hello","\n");
Hello
> write([|"Hello"|],"\n");
[|"Hello"|]
> s = "Hello";
> write(s,[|s|],"\n");
Hello[|"Hello"|]
> t = "Hello\tyou";
> write(t,[|t|],"\n");
Hello    you[|"Hello    you"|]
\end{Verbatim}
\end{minipage}\end{center}
\noindent Example 3: 
\begin{center}\begin{minipage}{15cm}\begin{Verbatim}[frame=single]
> write(x + 2 + exp(sin(x))) > "foo.sol";
> readfile("foo.sol");
x + 2 + exp(sin(x))
\end{Verbatim}
\end{minipage}\end{center}
\noindent Example 4: 
\begin{center}\begin{minipage}{15cm}\begin{Verbatim}[frame=single]
> write(x + 2 + exp(sin(x))) >> "foo.sol";
\end{Verbatim}
\end{minipage}\end{center}
See also: \textbf{print} (\ref{labprint}), \textbf{printexpansion} (\ref{labprintexpansion}), \textbf{printhexa} (\ref{labprinthexa}), \textbf{printfloat} (\ref{labprintfloat}), \textbf{printxml} (\ref{labprintxml}), \textbf{readfile} (\ref{labreadfile}), \textbf{autosimplify} (\ref{labautosimplify}), \textbf{display} (\ref{labdisplay}), \textbf{midpointmode} (\ref{labmidpointmode}), \textbf{fullparentheses} (\ref{labfullparentheses}), \textbf{evaluate} (\ref{labevaluate})




\end{document}