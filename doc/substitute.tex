\subsection{substitute}
\label{labsubstitute}
\noindent Name: \textbf{substitute}\\
replace the occurrences of the free variable in an expression.\\
\noindent Usage: 
\begin{center}
\textbf{substitute}(\emph{f},\emph{g}) : (\textsf{function}, \textsf{function}) $\rightarrow$ \textsf{function}\\
\textbf{substitute}(\emph{f},\emph{t}) : (\textsf{function}, \textsf{constant}) $\rightarrow$ \textsf{constant}\\
\end{center}
Parameters: 
\begin{itemize}
\item \emph{f} is a function.
\item \emph{g} is a function.
\item \emph{t} is a real number.
\end{itemize}
\noindent Description: \begin{itemize}

\item \\textbf{substitute}(\\emph{f}, \\emph{g}) produces the function $(f \\circ g) : x \\mapsto f(g(x))$.\n
\item \\textbf{substitute}(\\emph{f}, \\emph{t}) is the constant $f(t)$. Note that the constant is\n   represented by its expression until it has been evaluated (exactly the same\n   way as if you type the expression \\emph{f} replacing instances of the free variable \n   by \\emph{t}).\n
\item If \\emph{f} is stored in a variable \\emph{F}, the effect of the commands \\textbf{substitute}(\\emph{F},\\emph{g}) or \\textbf{substitute}(\\emph{F},\\emph{t}) is absolutely equivalent to \n   writing \\emph{F(g)} resp. \\emph{F(t)}.\n\end{itemize}
\noindent Example 1: 
\begin{center}\begin{minipage}{15cm}\begin{Verbatim}[frame=single]
\end{Verbatim}
\end{minipage}\end{center}
\noindent Example 2: 
\begin{center}\begin{minipage}{15cm}\begin{Verbatim}[frame=single]
\end{Verbatim}
\end{minipage}\end{center}
