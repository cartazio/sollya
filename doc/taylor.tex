\subsection{taylor}
\label{labtaylor}
\noindent Name: \textbf{taylor}\\
\phantom{aaa}computes a Taylor expansion of a function in a point\\[0.2cm]
\noindent Library name:\\
\verb|   sollya_obj_t sollya_lib_taylor(sollya_obj_t, sollya_obj_t, sollya_obj_t)|\\[0.2cm]
\noindent Usage: 
\begin{center}
\textbf{taylor}(\emph{function}, \emph{degree}, \emph{point}) : (\textsf{function}, \textsf{integer}, \textsf{constant}) $\rightarrow$ \textsf{function}\\
\end{center}
Parameters: 
\begin{itemize}
\item \emph{function} represents the function to be expanded
\item \emph{degree} represents the degree of the expansion to be delivered
\item \emph{point} represents the point in which the function is to be developped
\end{itemize}
\noindent Description: \begin{itemize}

\item The command \textbf{taylor} returns an expression that is a Taylor expansion
   of function \emph{function} in point \emph{point} having the degree \emph{degree}.
    
   Let $f$ be the function \emph{function}, $t$ be the point \emph{point} and
   $n$ be the degree \emph{degree}. Then, \textbf{taylor}(\emph{function},\emph{degree},\emph{point}) 
   evaluates to an expression mathematically equal to
   $$\sum\limits_{i=0}^n \frac{f^{(i)}\left(t\right)}{i!}\,x^i.$$
   In other words, if $p(x)$ denotes the polynomial returned by \textbf{taylor},
   $p(x-t)$ is the Taylor polynomial of degree $n$ of $f$ developped at point $t$.
    
   Remark that \textbf{taylor} evaluates to $0$ if the degree \emph{degree} is negative.
\end{itemize}
\noindent Example 1: 
\begin{center}\begin{minipage}{15cm}\begin{Verbatim}[frame=single]
> print(taylor(exp(x),3,1));
exp(1) + x * (exp(1) + x * (0.5 * exp(1) + x * exp(1) / 6))
\end{Verbatim}
\end{minipage}\end{center}
\noindent Example 2: 
\begin{center}\begin{minipage}{15cm}\begin{Verbatim}[frame=single]
> print(taylor(asin(x),7,0));
x * (1 + x^2 * (1 / 6 + x^2 * (3 / 40 + x^2 * 5 / 112)))
\end{Verbatim}
\end{minipage}\end{center}
\noindent Example 3: 
\begin{center}\begin{minipage}{15cm}\begin{Verbatim}[frame=single]
> print(taylor(erf(x),6,0));
x * (1 / sqrt((pi) / 4) + x^2 * ((sqrt((pi) / 4) * 4 / (pi) * (-2)) / 6 + x^2 * 
(sqrt((pi) / 4) * 4 / (pi) * 12) / 120))
\end{Verbatim}
\end{minipage}\end{center}
See also: \textbf{remez} (\ref{labremez}), \textbf{fpminimax} (\ref{labfpminimax}), \textbf{taylorform} (\ref{labtaylorform})
