\subsection{taylor}
\label{labtaylor}
\noindent Name: \textbf{taylor}\\
computes a Taylor expansion of a function in a point\\
\noindent Usage: 
\begin{center}
\textbf{taylor}(\emph{function}, \emph{degree}, \emph{point}) : (\textsf{function}, \textsf{integer}, \textsf{constant}) $\rightarrow$ \textsf{function}\\
\end{center}
Parameters: 
\begin{itemize}
\item \emph{function} represents the function to be expanded
\item \emph{degree} represents the degree of the expansion to be delivered
\item \emph{point} represents the point in which the function is to be developped
\end{itemize}
\noindent Description: \begin{itemize}

\item The command \\textbf{taylor} returns an expression that is a Taylor expansion\n   of function \\emph{function} in point \\emph{point} having the degree \\emph{degree}.\n    \n   Let $f$ be the function \\emph{function}, $t$ be the point \\emph{point} and\n   $n$ be the degree \\emph{degree}. Then, \\textbf{taylor}(\\emph{function},\\emph{degree},\\emph{point}) \n   evaluates to an expression mathematically equal to \n   $$\\sum\\limits_{i=0}^n \\frac{f^{(i)}\\left(t\\right)}{i!} \\left(x - t \\right)^i$$\n    \n   Remark that \\textbf{taylor} evaluates to $0$ if the degree \\emph{degree} is negative.\n\end{itemize}
\noindent Example 1: 
\begin{center}\begin{minipage}{15cm}\begin{Verbatim}[frame=single]
\end{Verbatim}
\end{minipage}\end{center}
\noindent Example 2: 
\begin{center}\begin{minipage}{15cm}\begin{Verbatim}[frame=single]
\end{Verbatim}
\end{minipage}\end{center}
\noindent Example 3: 
\begin{center}\begin{minipage}{15cm}\begin{Verbatim}[frame=single]
\end{Verbatim}
\end{minipage}\end{center}
See also: \textbf{remez} (\ref{labremez})
