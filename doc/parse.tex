\subsection{parse}
\label{labparse}
\noindent Name: \textbf{parse}\\
parses an expression contained in a string\\
\noindent Usage: 
\begin{center}
\textbf{parse}(\emph{string}) : \textsf{string} $\rightarrow$ \textsf{function} $|$ \textsf{error}\\
\end{center}
Parameters: 
\begin{itemize}
\item \emph{string} represents a character sequence
\end{itemize}
\noindent Description: \begin{itemize}

\item \textbf{parse}(\emph{string}) parses the character sequence \emph{string} containing
   an expression built on constants and base functions.
    
   If the character sequence does not contain a well-defined expression,
   a warning is displayed indicating a syntax error and \textbf{parse} returns
   a \textbf{error} of type \textsf{error}.

\item The character sequence to be parsed by \textbf{parse} may contain commands that 
   return expressions, including \textbf{parse} itself. Those commands get executed after the string has been parsed.
   \textbf{parse}(\emph{string}) will return the expression computed by the commands contained in the character 
   sequence \emph{string}.
\end{itemize}
\noindent Example 1: 
\begin{center}\begin{minipage}{15cm}\begin{Verbatim}[frame=single]
> parse("exp(x)");
exp(x)
\end{Verbatim}
\end{minipage}\end{center}
\noindent Example 2: 
\begin{center}\begin{minipage}{15cm}\begin{Verbatim}[frame=single]
> text = "remez(exp(x),5,[-1;1])";
> print("The string", text, "gives", parse(text));
The string remez(exp(x),5,[-1;1]) gives 1.00004475029055070643077052482053398765
426158966754 + x * (1.00003834652983970735244541124504033817544233075356 + x * (
0.49919698262882986492168824494240374771969012861297 + x * (0.166424656075155194
415920597322727380932279602909199 + x * (4.3793696387328047027125756620718349665
9575464236489e-2 + x * 8.7381910388065551140158420278330960479960476713376e-3)))
)
\end{Verbatim}
\end{minipage}\end{center}
\noindent Example 3: 
\begin{center}\begin{minipage}{15cm}\begin{Verbatim}[frame=single]
> verbosity = 1!;
> parse("5 + * 3");
Warning: syntax error, unexpected MULTOKEN. Will try to continue parsing (expect
ing ";"). May leak memory.
Warning: the string "5 + * 3" could not be parsed by the miniparser.
Warning: at least one of the given expressions or a subexpression is not correct
ly typed
or its evaluation has failed because of some error on a side-effect.
error
\end{Verbatim}
\end{minipage}\end{center}
See also: \textbf{execute} (\ref{labexecute}), \textbf{readfile} (\ref{labreadfile}), \textbf{print} (\ref{labprint}), \textbf{error} (\ref{laberror}), \textbf{dieonerrormode} (\ref{labdieonerrormode})
