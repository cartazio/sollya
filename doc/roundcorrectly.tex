\subsection{roundcorrectly}
\label{labroundcorrectly}
\noindent Name: \textbf{roundcorrectly}\\
rounds an approximation range correctly to some precision\\
\noindent Usage: 
\begin{center}
\textbf{roundcorrectly}(\emph{range}) : \textsf{range} $\rightarrow$ \textsf{constant}\\
\end{center}
Parameters: 
\begin{itemize}
\item \emph{range} represents a range in which an exact value lies
\end{itemize}
\noindent Description: \begin{itemize}

\item Let \emph{range} be a range of values, determined by some approximation
   process, safely bounding an unknown value $v$. The command
   \textbf{roundcorrectly}(\emph{range}) determines a precision such that for this precision,
   rounding to the nearest any value in \emph{range} yields to the same
   result, i.e. to the correct rounding of $v$.
    
   If no such precision exists, a warning is displayed and \textbf{roundcorrectly}
   evaluates to NaN.
\end{itemize}
\noindent Example 1: 
\begin{center}\begin{minipage}{15cm}\begin{Verbatim}[frame=single]
> printbinary(roundcorrectly([1.010001_2; 1.0101_2]));
1.01_2
> printbinary(roundcorrectly([1.00001_2; 1.001_2]));
1_2
\end{Verbatim}
\end{minipage}\end{center}
\noindent Example 2: 
\begin{center}\begin{minipage}{15cm}\begin{Verbatim}[frame=single]
> roundcorrectly([-1; 1]);
@NaN@
\end{Verbatim}
\end{minipage}\end{center}
See also: \textbf{round} (\ref{labround})
