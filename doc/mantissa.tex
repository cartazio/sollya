\subsection{mantissa}
\label{labmantissa}
\noindent Name: \textbf{mantissa}\\
\phantom{aaa}returns the integer mantissa of a number.\\[0.2cm]
\noindent Library name:\\
\verb|   sollya_obj_t sollya_lib_mantissa(sollya_obj_t)|\\[0.2cm]
\noindent Usage: 
\begin{center}
\textbf{mantissa}(\emph{x}) : \textsf{constant} $\rightarrow$ \textsf{integer}\\
\end{center}
Parameters: 
\begin{itemize}
\item \emph{x} is a dyadic number.
\end{itemize}
\noindent Description: \begin{itemize}

\item \textbf{mantissa}($x$) is by definition $x$ if $x$ equals 0, NaN, or Inf.

\item If \emph{x} is not zero, it can be uniquely written as $x = m \cdot 2^e$ where
   $m$ is an odd integer and $e$ is an integer. \textbf{mantissa}(x) returns $m$. 
\end{itemize}
\noindent Example 1: 
\begin{center}\begin{minipage}{15cm}\begin{Verbatim}[frame=single]
> a=round(Pi,20,RN);
> e=exponent(a);
> m=mantissa(a);
> m;
411775
> a-m*2^e;
0
\end{Verbatim}
\end{minipage}\end{center}
See also: \textbf{exponent} (\ref{labexponent}), \textbf{precision} (\ref{labprecision})
