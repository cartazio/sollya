\subsection{ doubledouble }
\noindent Names: \textbf{doubledouble}, \textbf{DD}\\
represents a number as the sum of two IEEE doubles.\\

\noindent Description: \begin{itemize}

\item \textbf{doubledouble} is both a function and a constant.

\item As a function, it rounds its argument to the nearest number that can be written
   as the sum of two double precision numbers.

\item The algorithm used to compute \textbf{doubledouble}(x) is the following: let xh = \textbf{double}(x)
   and let xl = \textbf{double}(x-xh). Return the number xh+xl. Note that if the current 
   precision is not sufficient to represent exactly xh+xl, a rounding will occur
   and the result of \textbf{doubledouble}(x) will be useless.

\item As a constant, it symbolizes the double-double precision format. It is used in 
   contexts when a precision format is necessary, e.g. in the commands 
   \textbf{roundcoefficients} and \textbf{implementpoly}.
   See the corresponding help pages for examples.
\end{itemize}
\noindent Example 1: 
\begin{center}\begin{minipage}{14.8cm}\begin{Verbatim}[frame=single]
   > verbosity=1!;
   > a = 1+ 2^(-100);
   > DD(a);
   0.100000000000000000000000000000078886090522101180541173e1
   > prec=50!;
   > DD(a);
   Warning: double rounding occurred on invoking the double-double rounding operator.
   Try to increase the working precision.
   1
\end{Verbatim}
\end{minipage}\end{center}
See also: \textbf{double}, \textbf{doubleextended}, \textbf{tripledouble}, \textbf{roundcoefficients}, \textbf{implementpoly}
