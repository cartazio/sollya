\subsection{hopitalrecursions}
\label{labhopitalrecursions}
\noindent Name: \textbf{hopitalrecursions}\\
controls the number of recursion steps when applying L'Hopital's rule.\\

\noindent Description: \begin{itemize}

\item \textbf{hopitalrecursions} is a global variable. Its value represents the number of steps of
   recursion that are tried when applying L'Hopital's rule. This rule is applied
   by the interval evaluator present in the core of \sollya (and particularly
   visible in commands like \textbf{infnorm}).

\item If an expression of the form $f/g$ has to be evaluated by interval 
   arithmetic on an interval $I$ and if $f$ and $g$ have a common zero
   in $I$, a direct evaluation leads to NaN.
   \sollya implements a safe heuristic to avoid this, based on L'Hopital's rule: in 
   such a case, it can be shown that $(f/g)(I) \subseteq (f'/g')(I)$. Since
   the same problem may hold for $f'/g'$, the rule is applied recursively.
   The number of step in this recursion process is controlled by \textbf{hopitalrecursions}.

\item Setting \textbf{hopitalrecursions} to 0 makes \sollya use this rule only one time ;
   setting it to 1 makes \sollya use the rule two times, and so on.
   In particular: the rule is always applied at least once, if necessary.
\end{itemize}
\noindent Example 1: 
\begin{center}\begin{minipage}{15cm}\begin{Verbatim}[frame=single]
> hopitalrecursions=0;
The number of recursions for Hopital's rule has been set to 0.
> evaluate(log(1+x)^2/x^2,[-1/2; 1]);
[-@Inf@;@Inf@]
> hopitalrecursions=1;
The number of recursions for Hopital's rule has been set to 1.
> evaluate(log(1+x)^2/x^2,[-1/2; 1]);
[-2.52258872223978123766892848583270627230200053744108;6.77258872223978123766892
84858327062723020005374411]
\end{Verbatim}
\end{minipage}\end{center}
