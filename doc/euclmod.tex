\subsection{mod}
\label{labeuclmod}
\noindent Name: \textbf{mod}\\
\phantom{aaa}Computes the euclidian division of polynomials or numbers and returns the rest\\[0.2cm]
\noindent Library name:\\
\verb|   sollya_obj_t sollya_lib_euclidian_mod(sollya_obj_t, sollya_obj_t)|\\[0.2cm]
\noindent Usage: 
\begin{center}
\textbf{mod}(\emph{p}, \emph{q}) : (\textsf{function}, \textsf{function}) $\rightarrow$ \textsf{function}\\
\end{center}
Parameters: 
\begin{itemize}
\item \emph{p} is a polynomial.
\item \emph{q} is a polynomial.
\end{itemize}
\noindent Description: \begin{itemize}

\item When both \emph{a} and \emph{b} are constants, \textbf{mod}(\emph{a},\emph{b}) computes \emph{a}
   less the product of \emph{b} and the largest integer less than or equal to
   \emph{a} divided by \emph{b}. In other words, it returns the rest of the
   Euclidian division of \emph{a} by \emph{b}.

\item When at least one of \emph{a} or \emph{b} is a polynomial of degree at least
   $1$, \textbf{mod}(\emph{a},\emph{b}) computes two polynomials \emph{q} and \emph{r} such
   that \emph{a} is equal to the product of \emph{q} and \emph{b} plus \emph{r}. The
   polynomial \emph{r} is of least degree possible. The \textbf{mod} command
   returns \emph{r}. In order to recover \emph{q}, use the \textbf{div} command.

\item When at least one of \emph{a} or \emph{b} is a function that is no polynomial,
   \textbf{mod}(\emph{a},\emph{b}) returns \emph{a}.
\end{itemize}
\noindent Example 1: 
\begin{center}\begin{minipage}{15cm}\begin{Verbatim}[frame=single]
> mod(1001, 231);
77
> mod(13, 17);
13
> mod(-14, 15);
1
> mod(-213, -5);
-3
> print(mod(23/13, 11/17));
105 / 221
> print(mod(exp(13),-sin(17)));
exp(13) + 460177 * sin(17)
\end{Verbatim}
\end{minipage}\end{center}
\noindent Example 2: 
\begin{center}\begin{minipage}{15cm}\begin{Verbatim}[frame=single]
> mod(24 + 68 * x + 74 * x^2 + 39 * x^3 + 10 * x^4 + x^5, 4 + 4 * x + x^2);
0
> mod(24 + 68 * x + 74 * x^2 + 39 * x^3 + 10 * x^4 + x^5, 2 * x^3);
24 + x * (68 + x * 74)
> mod(x^2, x^3);
x^2
\end{Verbatim}
\end{minipage}\end{center}
\noindent Example 3: 
\begin{center}\begin{minipage}{15cm}\begin{Verbatim}[frame=single]
> mod(exp(x), x^2);
exp(x)
> mod(x^3, sin(x));
x^3
\end{Verbatim}
\end{minipage}\end{center}
See also: \textbf{gcd} (\ref{labgcd}), \textbf{div} (\ref{labeucldiv}), \textbf{numberroots} (\ref{labnumberroots})
