\subsection{mod}
\label{labeuclmod}
\noindent Name: \textbf{mod}\\
\phantom{aaa}Computes the euclidian division of polynomials or numbers and returns the rest\\[0.2cm]
\noindent Library name:\\
\verb|   sollya_obj_t sollya_lib_euclidian_mod(sollya_obj_t, sollya_obj_t)|\\[0.2cm]
\noindent Usage: 
\begin{center}
\textbf{mod}(\emph{p}, \emph{q}) : (\textsf{function}, \textsf{function}) $\rightarrow$ \textsf{function}\\
\end{center}
Parameters: 
\begin{itemize}
\item \emph{p} is a polynomial.
\item \emph{q} is a polynomial.
\end{itemize}
\noindent Description: \begin{itemize}

\item TODO
\end{itemize}
\noindent Example 1: 
\begin{center}\begin{minipage}{15cm}\begin{Verbatim}[frame=single]
> mod(x^2 + 2 * x + 1, x + 2);
1
\end{Verbatim}
\end{minipage}\end{center}
See also: \textbf{gcd} (\ref{labgcd}), \textbf{div} (\ref{labeucldiv}), \textbf{numberroots} (\ref{labnumberroots})
