\subsection{implementpoly}
\label{labimplementpoly}
\noindent Name: \textbf{implementpoly}\\
\phantom{aaa}implements a polynomial using double, double-double and triple-double arithmetic and generates a Gappa proof\\[0.2cm]
\noindent Library names:\\
\verb|   sollya_obj_t sollya_lib_implementpoly(sollya_obj_t, sollya_obj_t,|\\
\verb|                                         sollya_obj_t, sollya_obj_t,|\\
\verb|                                         sollya_obj_t, sollya_obj_t, ...)|\\
\verb|   sollya_obj_t sollya_lib_v_implementpoly(sollya_obj_t, sollya_obj_t,|\\
\verb|                                           sollya_obj_t, sollya_obj_t,|\\
\verb|                                           sollya_obj_t, sollya_obj_t, va_list)|\\[0.2cm]
\noindent Usage: 
\begin{center}
\textbf{implementpoly}(\emph{polynomial}, \emph{range}, \emph{error bound}, \emph{format}, \emph{functionname}, \emph{filename}) : (\textsf{function}, \textsf{range}, \textsf{constant}, \textsf{D$|$double$|$DD$|$doubledouble$|$TD$|$tripledouble}, \textsf{string}, \textsf{string}) $\rightarrow$ \textsf{function}\\
\textbf{implementpoly}(\emph{polynomial}, \emph{range}, \emph{error bound}, \emph{format}, \emph{functionname}, \emph{filename}, \emph{honor coefficient precisions}) : (\textsf{function}, \textsf{range}, \textsf{constant}, \textsf{D$|$double$|$DD$|$doubledouble$|$TD$|$tripledouble}, \textsf{string}, \textsf{string}, \textsf{honorcoeffprec}) $\rightarrow$ \textsf{function}\\
\textbf{implementpoly}(\emph{polynomial}, \emph{range}, \emph{error bound}, \emph{format}, \emph{functionname}, \emph{filename}, \emph{proof filename}) : (\textsf{function}, \textsf{range}, \textsf{constant}, \textsf{D$|$double$|$DD$|$doubledouble$|$TD$|$tripledouble}, \textsf{string}, \textsf{string}, \textsf{string}) $\rightarrow$ \textsf{function}\\
\textbf{implementpoly}(\emph{polynomial}, \emph{range}, \emph{error bound}, \emph{format}, \emph{functionname}, \emph{filename}, \emph{honor coefficient precisions}, \emph{proof filename}) : (\textsf{function}, \textsf{range}, \textsf{constant}, \textsf{D$|$double$|$DD$|$doubledouble$|$TD$|$tripledouble}, \textsf{string}, \textsf{string}, \textsf{honorcoeffprec}, \textsf{string}) $\rightarrow$ \textsf{function}\\
\end{center}
\noindent Description: \begin{itemize}

\item The command \textbf{implementpoly} implements the polynomial \emph{polynomial} in range
   \emph{range} as a function called \emph{functionname} in \texttt{C} code
   using double, double-double and triple-double arithmetic in a way that
   the rounding error (estimated at its first order) is bounded by \emph{error bound}. 
   The produced code is output in a file named \emph{filename}. The
   argument \emph{format} indicates the double, double-double or triple-double
   format of the variable in which the polynomial varies, influencing
   also in the signature of the \texttt{C} function.
    
   If a seventh or eighth argument \emph{proof filename} is given and if this
   argument evaluates to a variable of type \textsf{string}, the command
   \textbf{implementpoly} will produce a \texttt{Gappa} proof that the
   rounding error is less than the given bound. This proof will be output
   in \texttt{Gappa} syntax in a file name \emph{proof filename}.
    
   The command \textbf{implementpoly} returns the polynomial that has been
   implemented. As the command \textbf{implementpoly} tries to adapt the precision
   needed in each evaluation step to its strict minimum and as it applies
   renormalization to double-double and triple-double precision
   coefficients to bring them to a round-to-nearest expansion form, the
   returned polynomial may differ from the polynomial
   \emph{polynomial}. Nevertheless the difference will be small enough that
   the rounding error bound with regard to the polynomial \emph{polynomial}
   (estimated at its first order) will be less than the given error
   bound.
    
   If a seventh argument \emph{honor coefficient precisions} is given and
   evaluates to a variable \textbf{honorcoeffprec} of type \textsf{honorcoeffprec},
   \textbf{implementpoly} will honor the precision of the given polynomial
   \emph{polynomials}. This means if a coefficient needs a double-double or a
   triple-double to be exactly stored, \textbf{implementpoly} will allocate appropriate
   space and use a double-double or triple-double operation even if the
   automatic (heuristic) determination implemented in command \textbf{implementpoly}
   indicates that the coefficient could be stored on less precision or,
   respectively, the operation could be performed with less
   precision. The use of \textbf{honorcoeffprec} has advantages and
   disadvantages. If the polynomial \emph{polynomial} given has not been
   determined by a process considering directly polynomials with
   floating-point coefficients, \textbf{honorcoeffprec} should not be
   indicated. The \textbf{implementpoly} command can then determine the needed
   precision using the same error estimation as used for the
   determination of the precisions of the operations. Generally, the
   coefficients will get rounded to double, double-double and
   triple-double precision in a way that minimizes their number and
   respects the rounding error bound \emph{error bound}.  Indicating
   \textbf{honorcoeffprec} may in this case short-circuit most precision
   estimations leading to sub-optimal code. On the other hand, if the
   polynomial \emph{polynomial} has been determined with floating-point
   precisions in mind, \textbf{honorcoeffprec} should be indicated because such
   polynomials often are very sensitive in terms of error propagation with
   regard to their coefficients' values. Indicating \textbf{honorcoeffprec}
   prevents the \textbf{implementpoly} command from rounding the coefficients and
   altering by many orders of magnitude the approximation error of the
   polynomial with regard to the function it approximates.
    
   The implementer behind the \textbf{implementpoly} command makes some assumptions on
   its input and verifies them. If some assumption cannot be verified,
   the implementation will not succeed and \textbf{implementpoly} will evaluate to a
   variable \textbf{error} of type \textsf{error}. The same behaviour is observed if
   some file is not writable or some other side-effect fails, e.g. if
   the implementer runs out of memory.
    
   As error estimation is performed only on the first order, the code
   produced by the \textbf{implementpoly} command should be considered valid iff a
   \texttt{Gappa} proof has been produced and successfully run
   in \texttt{Gappa}.
\end{itemize}
\noindent Example 1: 
\begin{center}\begin{minipage}{15cm}\begin{Verbatim}[frame=single]
> implementpoly(1 - 1/6 * x^2 + 1/120 * x^4, [-1b-10;1b-10], 1b-30, D, "p","impl
ementation.c");
1 + x^2 * (-0.166666666666666657414808128123695496469736099243164 + x^2 * 8.3333
333333333332176851016015461937058717012405395e-3)
> readfile("implementation.c");
/*
    This code was generated using non-trivial code generation commands of
    the Sollya software program.
    
    Before using, modifying and/or integrating this code into other
    software, review the copyright and license status of this generated
    code. In particular, see the exception below.
    
    Sollya is
    
    Copyright 2006-2013 by
    
    Laboratoire de l'Informatique du Parallelisme, UMR CNRS - ENS Lyon -
    UCB Lyon 1 - INRIA 5668,
    
    Laboratoire d'Informatique de Paris 6, equipe PEQUAN, UPMC Universite
    Paris 06 - CNRS - UMR 7606 - LIP6, Paris, France
    
    and by
    
    Centre de recherche INRIA Sophia-Antipolis Mediterranee, equipe APICS,
    Sophia Antipolis, France.
    
    Contributors Ch. Lauter, S. Chevillard, M. Joldes
    
    christoph.lauter@ens-lyon.org
    sylvain.chevillard@ens-lyon.org
    joldes@lass.fr
    
    The Sollya software is a computer program whose purpose is to provide
    an environment for safe floating-point code development. It is
    particularily targeted to the automatized implementation of
    mathematical floating-point libraries (libm). Amongst other features,
    it offers a certified infinity norm, an automatic polynomial
    implementer and a fast Remez algorithm.
    
    The Sollya software is governed by the CeCILL-C license under French
    law and abiding by the rules of distribution of free software.  You
    can use, modify and/ or redistribute the software under the terms of
    the CeCILL-C license as circulated by CEA, CNRS and INRIA at the
    following URL "http://www.cecill.info".
    
    As a counterpart to the access to the source code and rights to copy,
    modify and redistribute granted by the license, users are provided
    only with a limited warranty and the software's author, the holder of
    the economic rights, and the successive licensors have only limited
    liability.
    
    In this respect, the user's attention is drawn to the risks associated
    with loading, using, modifying and/or developing or reproducing the
    software by the user in light of its specific status of free software,
    that may mean that it is complicated to manipulate, and that also
    therefore means that it is reserved for developers and experienced
    professionals having in-depth computer knowledge. Users are therefore
    encouraged to load and test the software's suitability as regards
    their requirements in conditions enabling the security of their
    systems and/or data to be ensured and, more generally, to use and
    operate it in the same conditions as regards security.
    
    The fact that you are presently reading this means that you have had
    knowledge of the CeCILL-C license and that you accept its terms.
    
    The Sollya program is distributed WITHOUT ANY WARRANTY; without even
    the implied warranty of MERCHANTABILITY or FITNESS FOR A PARTICULAR
    PURPOSE.
    
    This generated program is distributed WITHOUT ANY WARRANTY; without
    even the implied warranty of MERCHANTABILITY or FITNESS FOR A
    PARTICULAR PURPOSE.
    
    As a special exception, you may create a larger work that contains
    part or all of this software generated using Sollya and distribute
    that work under terms of your choice, so long as that work isn't
    itself a numerical code generator using the skeleton of this code or a
    modified version thereof as a code skeleton.  Alternatively, if you
    modify or redistribute this generated code itself, or its skeleton,
    you may (at your option) remove this special exception, which will
    cause this generated code and its skeleton and the resulting Sollya
    output files to be licensed under the CeCILL-C licence without this
    special exception.
    
    This special exception was added by the Sollya copyright holders in
    version 4.1 of Sollya.
    
*/


#define p_coeff_0h 1.00000000000000000000000000000000000000000000000000000000000
000000000000000000000e+00
#define p_coeff_2h -1.6666666666666665741480812812369549646973609924316406250000
0000000000000000000000e-01
#define p_coeff_4h 8.33333333333333321768510160154619370587170124053955078125000
000000000000000000000e-03


void p(double *p_resh, double x) {
double p_x_0_pow2h;


p_x_0_pow2h = x * x;


double p_t_1_0h;
double p_t_2_0h;
double p_t_3_0h;
double p_t_4_0h;
double p_t_5_0h;
 


p_t_1_0h = p_coeff_4h;
p_t_2_0h = p_t_1_0h * p_x_0_pow2h;
p_t_3_0h = p_coeff_2h + p_t_2_0h;
p_t_4_0h = p_t_3_0h * p_x_0_pow2h;
p_t_5_0h = p_coeff_0h + p_t_4_0h;
*p_resh = p_t_5_0h;


}

\end{Verbatim}
\end{minipage}\end{center}
\noindent Example 2: 
\begin{center}\begin{minipage}{15cm}\begin{Verbatim}[frame=single]
> implementpoly(1 - 1/6 * x^2 + 1/120 * x^4, [-1b-10;1b-10], 1b-30, D, "p","impl
ementation.c","implementation.gappa");
1 + x^2 * (-0.166666666666666657414808128123695496469736099243164 + x^2 * 8.3333
333333333332176851016015461937058717012405395e-3)
\end{Verbatim}
\end{minipage}\end{center}
\noindent Example 3: 
\begin{center}\begin{minipage}{15cm}\begin{Verbatim}[frame=single]
> verbosity = 1!;
> q = implementpoly(1 - dirtysimplify(TD(1/6)) * x^2,[-1b-10;1b-10],1b-60,DD,"p"
,"implementation.c");
Warning: at least one of the coefficients of the given polynomial has been round
ed in a way
that the target precision can be achieved at lower cost. Nevertheless, the imple
mented polynomial
is different from the given one.
> printexpansion(q);
0x3ff0000000000000 + x^2 * 0xbfc5555555555555
> r = implementpoly(1 - dirtysimplify(TD(1/6)) * x^2,[-1b-10;1b-10],1b-60,DD,"p"
,"implementation.c",honorcoeffprec);
Warning: the infered precision of the 2th coefficient of the polynomial is great
er than
the necessary precision computed for this step. This may make the automatic dete
rmination
of precisions useless.
> printexpansion(r);
0x3ff0000000000000 + x^2 * (0xbfc5555555555555 + 0xbc65555555555555 + 0xb9055555
55555555)
\end{Verbatim}
\end{minipage}\end{center}
\noindent Example 4: 
\begin{center}\begin{minipage}{15cm}\begin{Verbatim}[frame=single]
> p = 0x3ff0000000000000 + x * (0x3ff0000000000000 + x * (0x3fe0000000000000 + x
 * (0x3fc5555555555559 + x * (0x3fa55555555555bd + x * (0x3f811111111106e2 + x
 * (0x3f56c16c16bf5eb7 + x * (0x3f2a01a01a292dcd + x * (0x3efa01a0218a016a + x
 * (0x3ec71de360331aad + x * (0x3e927e42e3823bf3 + x * (0x3e5ae6b2710c2c9a + x
 * (0x3e2203730c0a7c1d + x * 0x3de5da557e0781df))))))))))));
> q = implementpoly(p,[-1/2;1/2],1b-60,D,"p","implementation.c",honorcoeffprec,"
implementation.gappa");
> if (q != p) then print("During implementation, rounding has happened.") else p
rint("Polynomial implemented as given.");    
Polynomial implemented as given.
\end{Verbatim}
\end{minipage}\end{center}
See also: \textbf{honorcoeffprec} (\ref{labhonorcoeffprec}), \textbf{roundcoefficients} (\ref{labroundcoefficients}), \textbf{double} (\ref{labdouble}), \textbf{doubledouble} (\ref{labdoubledouble}), \textbf{tripledouble} (\ref{labtripledouble}), \textbf{readfile} (\ref{labreadfile}), \textbf{printexpansion} (\ref{labprintexpansion}), \textbf{error} (\ref{laberror}), \textbf{remez} (\ref{labremez}), \textbf{fpminimax} (\ref{labfpminimax}), \textbf{taylor} (\ref{labtaylor}), \textbf{implementconstant} (\ref{labimplementconstant})
