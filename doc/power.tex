\subsection{power}
\label{labpower}
\noindent Name: \textbf{\^}\\
power function\\

\noindent Usage: 
\begin{center}
\emph{function1} \textbf{\^} \emph{function2} : (\textsf{function}, \textsf{function}) $\rightarrow$ \textsf{function}\\
\end{center}
Parameters: 
\begin{itemize}
\item \emph{function1} and \emph{function2} represent functions
\end{itemize}
\noindent Description: \begin{itemize}

\item \textbf{\^} represents the power (function) on reals. 
   The expression \emph{function1} \textbf{\^} \emph{function2} stands for
   the function composed of the power function and the two
   functions \emph{function1} and \emph{function2}, where \emph{function1} is
   the base and \emph{function2} the exponent.
   If \emph{function2} is a constant integer, \textbf{\^} is defined
   on negative values of \emph{function1}. Otherwise \textbf{\^}
   is defined as $e^{y \cdot \ln x}$.
\end{itemize}
\noindent Example 1: 
\begin{center}\begin{minipage}{15cm}\begin{Verbatim}[frame=single]
> 5 ^ 2;
25
\end{Verbatim}
\end{minipage}\end{center}
\noindent Example 2: 
\begin{center}\begin{minipage}{15cm}\begin{Verbatim}[frame=single]
> x ^ 2;
x^2
\end{Verbatim}
\end{minipage}\end{center}
\noindent Example 3: 
\begin{center}\begin{minipage}{15cm}\begin{Verbatim}[frame=single]
> 3 ^ (-5);
0.41152263374485596707818930041152263374485596707818e-2
\end{Verbatim}
\end{minipage}\end{center}
\noindent Example 4: 
\begin{center}\begin{minipage}{15cm}\begin{Verbatim}[frame=single]
> (-3) ^ (-2.5);
@NaN@
\end{Verbatim}
\end{minipage}\end{center}
\noindent Example 5: 
\begin{center}\begin{minipage}{15cm}\begin{Verbatim}[frame=single]
> diff(sin(x) ^ exp(x));
sin(x)^exp(x) * ((cos(x) * exp(x)) / sin(x) + exp(x) * log(sin(x)))
\end{Verbatim}
\end{minipage}\end{center}
See also: \textbf{$+$} (\ref{labplus}), \textbf{$-$} (\ref{labminus}), \textbf{$*$} (\ref{labmult}), \textbf{/} (\ref{labdivide})
