\subsection{exponent}
\label{labexponent}
\noindent Name: \textbf{exponent}\\
returns the scaled binary exponent of a number.\\

\noindent Usage: 
\begin{center}
\textbf{exponent}(\emph{x}) : \textsf{constant} $\rightarrow$ \textsf{integer}\\
\end{center}
Parameters: 
\begin{itemize}
\item \emph{x} is a dyadic number.
\end{itemize}
\noindent Description: \begin{itemize}

\item \textbf{exponent}(x) is by definition 0 if x equals 0, NaN, or Inf.

\item If \emph{x} is not zero, it can be uniquely written as $x = m \cdot 2^e$ where
   $m$ is an odd integer and $e$ is an integer. \textbf{exponent}(x) returns $e$. 
\end{itemize}
\noindent Example 1: 
\begin{center}\begin{minipage}{15cm}\begin{Verbatim}[frame=single]
> a=round(Pi,20,RN);
> e=exponent(a);
> e;
-17
> m=mantissa(a);
> a-m*2^e;
0
\end{Verbatim}
\end{minipage}\end{center}
See also: \textbf{mantissa} (\ref{labmantissa}), \textbf{precision} (\ref{labprecision})
