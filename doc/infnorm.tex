\subsection{infnorm}
\label{labinfnorm}
\noindent Name: \textbf{infnorm}\\
computes an interval bounding the infinity norm of a function on an interval.\\
\noindent Usage: 
\begin{center}
\textbf{infnorm}(\emph{f},\emph{I},\emph{filename},\emph{Ilist}) : (\textsf{function}, \textsf{range}, \textsf{string}, \textsf{list}) $\rightarrow$ \textsf{range}\\
\end{center}
Parameters: 
\begin{itemize}
\item \emph{f} is a function.
\item \emph{I} is an interval.
\item \emph{filename} (optional) is the name of the file into a proof will be saved.
\item \emph{IList} (optional) is a list of intervals to be excluded.
\end{itemize}
\noindent Description: \begin{itemize}

\item \\textbf{infnorm}(\\emph{f},\\emph{range}) computes an interval bounding the infinity norm of the \n   given function $f$ on the interval $I$, e.g. computes an interval $J$\n   such that $\\max_{x \\in I} \\{|f(x)|\\} \\subseteq J$.\n
\item If \\emph{filename} is given, a proof in English will be produced (and stored in file\n   called \\emph{filename}) proving that  $\\max_{x \\in I} \\{|f(x)|\\} \\subseteq J$.\n
\item If a list \\emph{IList} of intervals $I_1, \\dots, I_n$ is given, the infinity norm will\n   be computed on $I \\backslash (I_1 \\cup \\dots \\cup I_n)$.\n
\item The function \\emph{f} is assumed to be at least twice continuous on \\emph{I}. More \n   generally, if \\emph{f} is $\\mathcal{C}^k$, global variables \\textbf{hopitalrecursions} and\n   \\textbf{taylorrecursions} must have values not greater than $k$.  \n
\item If the interval is reduced to a single point, the result of \\textbf{infnorm} is an \n   interval containing the exact absolute value of \\emph{f} at this point.\n
\item If the interval is not bound, the result will be $[0,\\,+\\infty]$ \n   which is correct but perfectly useless. \\textbf{infnorm} is not meant to be used with \n   infinite intervals.\n
\item The result of this command depends on the global variables \\textbf{prec}, \\textbf{diam},\n   \\textbf{taylorrecursions} and \\textbf{hopitalrecursions}. The contribution of each variable is \n   not easy even to analyse.\n   \\begin{itemize}\n   \\item  The algorithm uses interval arithmetic with precision \\textbf{prec}. The\n     precision should thus be set high enough to ensure that no critical\n     cancellation will occur.\n   \\item  When an evaluation is performed on an interval $[a,\\,b]$, if the result\n     is considered being too large, the interval is split into $[a,\\,\\frac{a+b}{2}]$\n     and $[\\frac{a+b}{2},\\,b]$ and so on recursively. This recursion step\n     is  not performed if the $(b-a) < \\delta \\cdot |I|$ where $\\delta$ is the value\n     of variable \\textbf{diam}. In other words, \\textbf{diam} controls the minimum length of an\n     interval during the algorithm.\n   \\item  To perform the evaluation of a function on an interval, Taylor's rule is\n     applied, e.g. $f([a,b]) \\subseteq f(m) + [a-m,\\,b-m] \\cdot f'([a,\\,b])$\n     where $m=\\frac{a+b}{2}$. This rule is recursively applied $n$ times\n     where $n$ is the value of variable \\textbf{taylorrecursions}. Roughly speaking,\n     the evaluations will avoid decorrelation up to order $n$.\n   \\item  When a function of the form $\\frac{g}{h}$ has to be evaluated on an\n     interval $[a,\\,b]$ and when $g$ and $h$ vanish at a same point\n     $z$ of the interval, the ratio may be defined even if the expression\n     $\\frac{g(z)}{h(z)}=\\frac{0}{0}$ does not make any sense. In this case, L'Hopital's rule\n     may be used and $\\left(\\frac{g}{h}\\right)([a,\\,b]) \\subseteq \\left(\\frac{g'}{h'}\\right)([a,\\,b])$.\n     Since the same can occur with the ratio $\\frac{g'}{h'}$, the rule is applied\n     recursively. The variable \\textbf{hopitalrecursions} controls the number of \n     recursion steps.\n   \\end{itemize}\n
\item The algorithm used for this command is quite complex to be explained here. \n   Please find a complete description in the following article:\\\\\n        S. Chevillard and C. Lauter\\\\\n        A certified infinity norm for the implementation of elementary functions\\\\\n        LIP Research Report number RR2007-26\\\\\n        http://prunel.ccsd.cnrs.fr/ensl-00119810\\\\\n\end{itemize}
\noindent Example 1: 
\begin{center}\begin{minipage}{15cm}\begin{Verbatim}[frame=single]
\end{Verbatim}
\end{minipage}\end{center}
\noindent Example 2: 
\begin{center}\begin{minipage}{15cm}\begin{Verbatim}[frame=single]
\end{Verbatim}
\end{minipage}\end{center}
\noindent Example 3: 
\begin{center}\begin{minipage}{15cm}\begin{Verbatim}[frame=single]
\end{Verbatim}
\end{minipage}\end{center}
\noindent Example 4: 
\begin{center}\begin{minipage}{15cm}\begin{Verbatim}[frame=single]
\end{Verbatim}
\end{minipage}\end{center}
\noindent Example 5: 
\begin{center}\begin{minipage}{15cm}\begin{Verbatim}[frame=single]
\end{Verbatim}
\end{minipage}\end{center}
\noindent Example 6: 
\begin{center}\begin{minipage}{15cm}\begin{Verbatim}[frame=single]
\end{Verbatim}
\end{minipage}\end{center}
See also: \textbf{prec} (\ref{labprec}), \textbf{diam} (\ref{labdiam}), \textbf{hopitalrecursions} (\ref{labhopitalrecursions}), \textbf{dirtyinfnorm} (\ref{labdirtyinfnorm}), \textbf{checkinfnorm} (\ref{labcheckinfnorm})
